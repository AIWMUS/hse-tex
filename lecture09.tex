\section{Лекция 9.11.2017}

\subsection{Поля}

\textbf{Определение.} \textit{Полем} называется множество F, на котором заданы две операции "сложение" ( $(a,b) \rightarrow a+b$) и "умножение" ($(a,b) \rightarrow ab$), причем $\forall \ a, b, c \in F$ и выполнены следующие условия:

(1) $a+b = b+a$ (коммутативность сложения)

(2) $(a+b)+c = a+(b+c)$ (ассициативность сложения)

(3) $\exists 0 \in F: 0 + a = a+ 0 = a \ \forall \ a \in F$ (нулевой элемент)

(4) $\exists -a \in F: a+(-a)=(-a)+a=0$ (противоположный элемент)

$\uparrow$ абелева группа $\uparrow$

(5) $a(b+c) = ab + ac$ (дистрибутивность)

(6) $ab=ba$ (коммутативность умножения)

(7) $(ab)c=a(bc)$ (ассоциативность умножения)

(8) $ \exists 1 \in F \backslash \{0\} : 1a=a1=a$ (единица)

(9) если $a \neq 0, \exists a^{-1} \in F: aa^{-1} = a^{-1} a = 1$ (обратный элемент)

\bigskip
Примеры. 

$\RR, \QQ$.

$\{0, 1\}$ сложение $mod \ 2$, умножение $mod \ 2$

\bigskip
Неформально: $\CC$ -- это наименьшее поле со следующими свойствами:

(c1) $\CC \supset \RR$

(c2) уравнение $x^2 + 1 = 0$ имеет решение в $\CC$, т.е. $\exists i \in \CC: i^2 = -1$.

\bigskip
\subsection{Формальная конструкция поля $\CC$}

\bigskip
$\CC := \{(a,b) | a, b \in \RR\}$

\bigskip
$(a_1, b_1) + (a_2, b_2) = (a_1 + a_2, b_1 + b_2)$

$(a_1, b_1) (a_2, b_2) = (a_1 a_2 - b_1 b_2, a_1 b_2 + b_1 a_2)$

$(a, b) \leftrightarrow a + bi$

$(a_1, b_1) + (a_2, b_2) \leftrightarrow (a_1 + b_1 i) + (a_2 + b_2 i) = (a_1 + a_2) + (b_1 + b_2)i$

$(a_1, b_1) (a_2, b_2) \leftrightarrow (a_1 + b_1 i) (a_2 + b_2 i) = a_1 a_2 + a_1 b_2 i + a_2 b_1 i + b_1 b_2 i^2 = (a_1 a_2 - b_1 b_2) + (a_1 b_2 + a_2 b_1)i$

\bigskip
Проверка свойств (1)-(9):

(1)-(4) выполнены в $\RR^2$.

$0 = (0,0), \ -(a, b) = (-a, -b)$

(5) $\textit{Упражнение на дом.}$

(6) Из явного вида формулы для умножения

(7) $((a_1, b_1) (a_2, b_2)) (a_3, b_3) = (a_1 a_2 - b_1 b_2, a_1 b_2 + b_1 a_2) (a_3, b_3) = (a_1 a_2 a_3 - b_1 b_2 a_3 - a_1 b_2 b_3 - b_1 a_2 b_3, a_1 a_2 b_3 - b_1 b_2 b_3 + a_1 b_2 a_3 + b_1 a_2 a_3) = (a_1, b_1)(a_2 a_3 - b_2 b_3, a_2 b_3 + b_2 a_3) = (a_1, b_1) ((a_2, b_2)(a_3, b_3))$.

(8) $1 = (1, 0)$

$(1,0)(a, b) = (a, b) = (a, b)(1, 0)$

(9) $(a, b) \neq (0,0) \Rightarrow a^2 + b^2 \neq 0$. Тогда $(a, b)^{-1} = (\frac{a}{a^2 + b^2}, \frac{-b}{a^2 + b^2})$

$(a, b) (\frac{a}{a^2 + b^2}, \frac{-b}{a^2 + b^2}) = (\frac{a^2}{a^2 + b^2} + \frac{b^2}{a^2 + b^2}, \frac{-ab}{a^2 + b^2} + \frac{ba}{a^2 + b^2}) = (1, 0)$.

\bigskip
Итак, $\CC$ - поле.

Проверка (c1):
$a \in R \leftrightarrow (a, 0)$

$ab \leftrightarrow (a, 0)(b, 0) = (ab, 0)$

$a+b \leftrightarrow (a, 0) + (b, 0) = (a+b, 0)$

$\Rightarrow \RR$ отождествляется $\CC$

$\{(a, 0) | a \in R \} \subseteq C$.

\bigskip
Проверка (c2): $i = (0, 1) \Rightarrow i^2 = (0, 1)(0, 1) = (-1, 0) = -1$

$\forall (a, b) \in C$

$(a, b) = (a, 0) + (0, b) = (a, 0) + (b, 0) (0, 1) i = a + bi, \ a, b \in \RR$

\bigskip
\textbf{Определение.} Представление числа $z \in \CC$ в виде  $z = a + bi$, где $a, b \in \RR$, называется его \textit{алгебраической формой}. Число $i$ называется \textit{мнимой единицей}.

$a =: Re(z)$ -- \textit{действительная часть} числа $z$

$b =: Im(z)$ -- \textit{мнимая часть} числа $z$

\bigskip
Числа вида $b_i$, где $b \in R$, называются \textit{чисто мнимыми}.

\bigskip
\textbf{Определение.} Число $\overline{z} := a - bi$ называется \textit{комплексно сопряженным к числу} $z=a+bi \in \CC$.

Операция $z \rightarrow \overline{z}$ называется \textit{комплексным сопряжением}.

\bigskip
\textbf{\textit{Свойства:}}

1) $\overline{\overline{z}} = z$

2) $\overline{z + w} = \overline{z} + \overline{w} \ \forall z, w \in C$

3) $\overline{z \cdot w} = \overline{z} \cdot \overline{w} \ \forall z, w \in C$

Доказательство -- прямая проверка (\textit{упражнение на дом}).

\bigskip
\textit{Геометрическая модель комплексных чисел} $z = a + bi \leftrightarrow$ точка (или вектор) на плоскости $\RR^2$ с координатами $(a, b)$. Сумма $z + w \leftrightarrow$ сумма соответствующих векторов. $\overline{z} \leftrightarrow$ отражение $z$ относительно действительной оси.

\bigskip
\textbf{Определение.} Число $|z| = \sqrt[]{a^2 + b^2}$ называется \textit{модулем числа} $z = a + bi \in \CC$ ($|z|$ - длина соответствующего вектора).

\bigskip
\textbf{\textit{Свойства.}} 

1) $|z| \geq 0$, причем $|z| = 0 \Leftrightarrow z = 0$.

2) $ |z + w| \leq |z| + |w|$ (неравенство треугольника)

3) $z \overline{z} = |z|^2$

4) $|zw| = |z||w| \ \ |zw|^2 = zw(\overline{zw}) = zw \overline{z} \overline{w} =  |z|^2 |w|^2$

\bigskip
\textbf{Замечание.} Из 3) следует, что для $z \neq 0$

$z^{-1} = \frac{\overline{z}}{|z|^2}$, т.е. $(a + bi)^{-1} = \frac{a-bi}{a^2 + b^2}$

\bigskip
$z = a + bi \in \CC, z \neq 0$

$z = |z| (\frac{a}{|z|} + \frac{b}{|z|}i)$

$(\frac{a}{|z|})^2 + (\frac{b}{|z|}i)^2 = 1$

\bigskip
\textbf{Определение.} \textit{Аргументом числа} $z = a + bi \in \CC \backslash \{0\}$ называется число $\varphi \in \RR$, такое что $\cos \varphi = \frac{a}{|z|} = \frac{a} {\sqrt[]{a^2 + b^2}}$, $\sin \varphi = \frac{b}{|z|} = \frac{b}{\sqrt[]{a^2 + b^2}}$.

В геометрических терминах, $\varphi$ есть угол между осью ОХ и соответсвующим вектором.

\bigskip
\textbf{Замечание.} 

1) Если $z = 0$, то аргумент не определен.

2) Если $z \neq 0$, то аргумент определен с точностью до $2 \pi k, k \in \ZZ$.

\bigskip
$Arg(z) :=$ множество всех аргументов числа $z$.

$arg(z)$ - это единственное значение из $Arg(z)$, лежащее в $[0;2 \pi]$.

$Arg(z) = \{ \varphi \in \RR \ | \ \cos \varphi = \frac{a}{|z|}, \\sin \varphi = \frac{b}{|z|} \}$

$Arg(z) = arg(z) + 2 \pi \ZZ$

$z = |z| (\frac{a}{|z|} + \frac{b}{|z|} i) = |z| (\cos \varphi + i \\sin \varphi)$, где $\varphi \in Arg(z)$.

\bigskip
\textbf{Определение.} Представление числа $z \in \CC \backslash \{0 \}$ в виде $z = |z|(\cos \varphi + i \sin \varphi)$ называется его \textit{тригонометрической формой}.

\bigskip
\textbf{Предложение.} $z_1 = |z_1| (\cos \varphi_1 + i \sin \varphi_1)$, $z_2 = |z_2| (\cos \varphi_2 + i \sin \varphi_2) \Rightarrow z_1 z_2 = |z_1| |z_2| (\cos (\varphi_1 + \varphi_2) + i \sin(\varphi_1 + \varphi_2)$.

\bigskip
\textbf{\textit{Доказательство.}} $\rhd \ z_1 z_2 = |z_1||z_2|(\cos \varphi_1 + i \sin \varphi_1)(\cos \varphi_2 + i \sin \varphi_2) = |z_1||z_2|((\cos \varphi_1 \cos \varphi_2 - \sin \varphi_1 \sin \varphi_2) + i(\cos \varphi_1 \sin \varphi_2 + \sin \varphi_1 \cos \varphi_2)) = |z_1| |z_2| (\cos (\varphi_1 + \varphi_2) + i \sin(\varphi_1 + \varphi_2) \ \lhd$.

