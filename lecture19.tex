\section{Лекция 8.02.2018}

$V$ -- векторное пространство

\textbf{Определение.} Линейной функцией (линейной формой, линейным функционалом) на $V$ называется всякое линейное отображение $\alpha : V \rightarrow F$.

Обозначение:$V^{*} := Hom(V, F)$ -- множество всех линейных функций на $V$

\vspace{\baselineskip}
Примеры:

1) $\alpha : F^n \rightarrow F, \begin{pmatrix} x_1 \\ \vdots \\ x_n \end{pmatrix} \rightarrow \alpha_1 x_1 + \dots + \alpha_n x_n$, где $\alpha_1, \dots, \alpha_n \in F$ -- фиксированные скаляры

2) $F(X, \mathbb{R})$ -- векторное пространство всех функций из множества $X$ в $\mathbb{R}, x_0 \in X \Rightarrow \alpha : F(X, \mathbb{R}) \rightarrow \mathbb{R}, f \rightarrow f(x_0)$

3) $\alpha : C[a, b]$(непрерывные функции на $[a, b]$) $\rightarrow \mathbb{R}, f \rightarrow \int^b_a f(x) dx$ 

4) $\alpha: M_n(F) \rightarrow F, A \rightarrow trA$

\vspace{\baselineskip}
Далее считаем, что $dimV = n < \infty$

Из общей теории линейных отображений:

1)$V^{*}$ -- векторное пространство (оно называется пространством, \textit{двойственным или сопряженным пространству} $V$)

2) Если $e = (e_1, \dots, e_n)$ -- фиксированный базис $V$, то имеется изоморфизм $V^{*} \iso Mat_{1 \times n}(F), \alpha \rightarrow (\alpha_1, \dots, \alpha_n)$, где $\alpha_i = \alpha(e_i)$ -- значение $\alpha$ на $e_i$ (коэффициенты линейной функции $\alpha$ в базисе $e$)

Если $x = x_1 e_1 + \dots + x_n e_n$, то $\alpha(x) = (\alpha_1, \dots, \alpha_n) \begin{pmatrix} x_1 \\ \vdots \\ x_n \end{pmatrix}$

\textbf{Следствие.} $dimV^* = dimV (\Rightarrow V^{*} \simeq V)$

\vspace{\baselineskip}
Фиксируем базис $e = (e_1, \dots, e_n)$ в $V$

$\forall i = 1, \dots, n$ рассмотрим линейную функцию $\varepsilon_i \in V^*$, такую что $\varepsilon_i(e_j) = \delta_{ij}$ (символ Кронекера) $\delta_{ij} = \begin{cases}
		1, i=j \\
		0, i \neq j
	\end{cases}$

\vspace{\baselineskip}
Соответствующая строка в $Mat_{1 \times n}(F)$ есть $
\bordermatrix{ 
& & & & i & & & &  \cr 
& 0 & \dots & 0 & 1 & 0 & \dots & 0} $

\vspace{\baselineskip}
Линейные функции $\varepsilon_i, i = 1, \dots, n$ образуют базис в $V^*$

\vspace{\baselineskip}
\textbf{Определение.} Базис $(\varepsilon_1, \dots, \varepsilon_n)$ пространства $V*$ называется \textit{двойственным к базису} $(e_1, \dots, e_n)$

Имеем: $\begin{pmatrix} \varepsilon_1 \\ \vdots \\ \varepsilon_n \end{pmatrix} (e_1, \dots, e_n) = (\varepsilon_i(e_j)) = E$

\vspace{\baselineskip}
\textbf{Предложение.} Всякий базис $V^*$ двойствен некоторому базису пространства $V$.

\vspace{\baselineskip}
\textbf{\textit{Доказательство.}} $\rhd$ Пусть $\varepsilon = (\varepsilon_1, \dots, \varepsilon_n)$ -- данный базис в $V^*$

Выберем в $V$ произвольный базис $e' = (e_1, \dots, e_n)$

Пусть $\varepsilon' = (\varepsilon'_1, \dots, \varepsilon'_n)$ -- двойственный ему базис в $V*$

\vspace{\baselineskip}
Тогда $\begin{pmatrix} \varepsilon_1 \\ \vdots \\ \varepsilon_n \end{pmatrix} = C \cdot \begin{pmatrix} \varepsilon'_1 \\ \vdots \\ \varepsilon'_n \end{pmatrix}$ для некоторой $C \in M_n(F), detC \neq 0$. Положим $e = (e'_1, \dots, e'_n) \cdot C^{-1}$. Тогда $\begin{pmatrix} \varepsilon_1 \\ \vdots \\ \varepsilon_n \end{pmatrix} (e_1, \dots, e_n) = C \cdot \begin{pmatrix} \varepsilon'_1 \\ \vdots \\ \varepsilon'_n \end{pmatrix} (e'_1, \dots, e'_n) \cdot C^{-1} = CEC^{-1} = E \ \lhd$

\vspace{\baselineskip}
\textbf{\textit{Упражнение.}} Базис $e$ на самом деле определен однозначно.

\vspace{\baselineskip}
\textbf{Определение.} \textit{Билинейной формой (или функцией) на} $V$ называется всякое отображение $\beta: V \times V \rightarrow F$, которое билинейно, т.е. линейно по каждому аргументу:

1) $\beta(x_1 + x_2, y) = \beta(x_1, y) + \beta(x_2, y) \ \forall x_1, x_2, y \in V$

2) $\beta(x, y_1 + y_2) = \beta(x, y_1) + \beta(x, y_2) \ \forall x, y_1, y_2 \in V$

3) $\beta(\lambda x, y) = \lambda \beta(x, y) = \beta(x, \lambda y) \ \forall x, y \in V, \lambda \in F$

\vspace{\baselineskip}
По факту: $\beta (\sum\limits_{i=1}^p \lambda_i v_i, \sum\limits_{j=1}^q \mu_j w_j) = \sum\limits_{i=1}^p \sum\limits_{j=1}^q \lambda_i \mu_j \beta(v_i, w_j) \ \forall v_i, w_j \in V, \lambda_i, m_j \in F$

\vspace{\baselineskip}
Примеры: 1) $V = F^n, \beta(x, y):= x_1 y_1 + \dots + x_n y_n$, где $x = \begin{pmatrix} x_1 \\ \vdots \\ x_n \end{pmatrix}, y = \begin{pmatrix} y_1 \\ \vdots \\ y_n \end{pmatrix}$

2) $V = F^2$, $\beta \left(\begin{pmatrix} x_1 \\ x_2 \end{pmatrix}, \begin{pmatrix} y_1 \\ y_2 \end{pmatrix} \right) = det \begin{pmatrix} x_1 & y_1 \\ x_2 & y_2 \end{pmatrix} = x_1 y_2 - x_2 y_1$

3) $V = C[a, b], \beta(f, g):= \int_a^b f(x) g(x) dx$

4) $V = Mat_{m \times n} (F), \beta (A, B) := tr(A^T \cdot B)$

\vspace{\baselineskip}
$dimV = n < \infty; \beta: V \times V \rightarrow F$ -- билинейная форма

Фиксируем базис в $V \ e = (e_1, \dots, e_n)$

\vspace{\baselineskip}
\textbf{Определение.} Матрица $B = (b_{ij})$, где $b_{ij} = \beta(e_i, e_j)$ , называется \textit{матрицей билинейной формы} $\beta$ в базисе $e$.

Обозначение: $B(\beta, e)$

\vspace{\baselineskip}
Пусть $x = x_1e_1 + \dots + x_n e_n, \ y = y_1 e_1 + \dots + y_n e_n$, $B = B(\beta, e)$

\vspace{\baselineskip}
$\beta(x, y) = \beta(\sum\limits_{i=1}^n x_i e_i, \sum\limits_{j=1}^n y_j e_j) = \sum\limits_{i=1}^n \sum\limits_{j=1}^n x_i y_j \beta(e_i, e_j) = \sum\limits_{i=1}^n \sum\limits_{j=1}^n x_i b_{ij} y_j = (x_1, \dots, x_n) B \begin{pmatrix} y_1 \\ \vdots \\ y_n \end{pmatrix}$ -- (*) формула вычисления значения билинейной формы на паре векторов

\vspace{\baselineskip}
\textbf{Предложение.} 1) Всякая билинейная форма однозначно определяется своей матрицей в базисе $e$

2) $\forall B \in M_n (F) \exists !$ билинейная форма $\beta$, такая что $B(\beta, e) = B$.

\vspace{\baselineskip}
\textbf{\textit{Доказательство.}} $\rhd$ 1) следует из (*)

2) единственность из 1)

существование: Определим $\beta$ по формуле (*), получится билинейная форма. $\lhd$

\vspace{\baselineskip}
Из предыдущих примеров.

1) $e$ -- стандартный базис $\Rightarrow B(\beta, e) = E$

2) $e$ -- стандартный базис $\Rightarrow B(\beta, e) = \begin{pmatrix} 0 & 1 \\ -1 & 0 \end{pmatrix} $

\vspace{\baselineskip}
Пусть $e = (e_1, \dots, e_n)$ и $e' = (e'_1, \dots, e'_n)$ -- два базиса $V$, $e' = e \cdot C$

$\beta : V \times V \rightarrow F$ -- билинейная форма на $V$

$B = B(\beta, e), B' = B(\beta, e')$.

\vspace{\baselineskip}
\textbf{Предложение.} $B' = C^{T} BC$

\vspace{\baselineskip}
\textbf{\textit{Доказательство.}} $x = x_1 e_1 + \dots x_n e_n = x'_1 e'_1 + \dots + x'_n e'_n, \begin{pmatrix} x_1 \\ \vdots \\ x_n \end{pmatrix} = C \begin{pmatrix} x'_1 \\ \vdots \\ x'_n \end{pmatrix}$

$y = y_1 e_1 + \dots y_n e_n = y'_1 e'_1 + \dots + y'_n e'_n, \begin{pmatrix} y_1 \\ \vdots \\ y_n \end{pmatrix} = C \begin{pmatrix} y'_1 \\ \vdots \\ y'_n \end{pmatrix}$

$\beta(x,y) = (x_1, \dots, x_n) B \begin{pmatrix} y_1 \\ \vdots \\ y_n \end{pmatrix} = (x'_1, \dots, x'_n) C^T B C \begin{pmatrix} y'_1 \\ \vdots \\ y'_n \end{pmatrix}$, а также $\beta(x,y) = (x'_1, \dots, x'_n) B' \begin{pmatrix} y'_1 \\ \vdots \\ y'_n \end{pmatrix} \Rightarrow C^{T}BC = B'$, т.к. $\forall P \in M_n \ p_{ij} = \bordermatrix{ 
& & & & i & & & &  \cr 
& 0 & \dots & 0 & 1 & 0 & \dots & 0}  P \bordermatrix{ &  \cr
& 0  \cr
& \dots  \cr
& 0  \cr 
& 1 & j  \cr
& 0  \cr
& \dots  \cr
& 0  \cr
} $

\vspace{\baselineskip}
\textbf{Следствие.} Число $rkB(\beta, e)$ не зависит от выбора базиса.

\vspace{\baselineskip}
\textbf{Определение.} Число $rk B(\beta, e)$ называется \textit{рангом билинейной формы} $\beta$.

\vspace{\baselineskip}
\textbf{Определение.} Билинейная форма называется симметричной, если $\beta(x, y) = \beta(y, x) \ \forall x, y \in V$.

\vspace{\baselineskip}
Пусть $e = (e_1, \dots, e_n)$ -- фиксированный базис $V$

$B = B(\beta, e)$

\vspace{\baselineskip}
\textbf{Предложение.} $\beta$ симметрична $\Leftrightarrow B$ симметрична (т.е. $B^T = B$).

\vspace{\baselineskip}
\textbf{\textit{Доказательство.}} $\rhd$  $\beta$ симметрична $\Rightarrow B$ симметрична: $b_{ij} = \beta(e_i, e_j) = \beta(e_j, e_i) = b_{ji} \Rightarrow B^T = B$.

$\beta$ симметрична $\Leftarrow B$ симметрична: $x = x_1 e_1 + \dots + x_n e_n, y = y_1 e_1 + \dots + y_n e_n$, $\beta(x, y) = (x_1, \dots, x_n) B \begin{pmatrix} y_1 \\ \dots \\ y_n \end{pmatrix} =$ $= \left[ (x_1, \dots, x_n) B \begin{pmatrix} y_1 \\ \dots \\ y_n \end{pmatrix} \right]^T = (y_1, \dots, y_n) B^T \begin{pmatrix} x_1 \\ \dots \\ x_n \end{pmatrix} = (y_1, \dots, y_n) B \begin{pmatrix} x_1 \\ \dots \\ x_n \end{pmatrix} = \beta(y, x) \ \lhd $

