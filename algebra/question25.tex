\section{Редуцируемость к нулю $S$-многочлена двух многочленов с взаимно простыми старшими членами}

\textbf{Предложение.} $f_1, f_2 \in R \setminus \{0\}, \text{НОД} (L(f_1), L(f_2)) = 1 \Rightarrow S(f_1, f_2) \overset{\{f_1, f_2\}}{\leadsto} 0$
\begin{proof}
Достаточно показать, что $f_1, f_2$ -- базис Грёбнера в идеале $(f_1, f_2).$ \\
Пусть $g \in (f_1, f_2)$ и $g = h_1f_1 + h_2f_2,$ где $h_1, h_2 \in R.$ Покажем, что $L(g) \vdots L(f_1)$ или  $L(g) \vdots L(f_2).$\\
Пусть это не так, тогда $L(h_1f_1) = -L(h_2f_2) \Rightarrow [\text{по лемме о старшем члене}]\Rightarrow L(h_1) = L(f_2) \cdot m, L(h_2) = -L(f_1) \cdot m, m \in M$.\\
Положим $h_1' = h_1- f_2m, h_2' = h_2 + f_1m; L(h_1') \prec L(h_1), L(h_2') \prec L(h_2).$ Имеем $g = (h_1' + f_2m)f_1+ (h_2' - f_1m)f_2 = h_1'f_1 + h_2'f_2$ и $L(h_1'f_1) = -L(_2'f_2).$ Повторяя процедуру, получим бесконечную цепочку равенств 
$g = h_1f_1 + h_2f_2 = h_1'f_1+h_2'f_2 = \dotsc = h_1^{(i)}f_1 + h_2^{(i)} f_2 = \dotsc ,$ причём $L(h_1) \succ L(h_1') \succ\dotsc \succ L(h_1^{(i)}) \succ \dotsc$ -- противоречие.
\end{proof}