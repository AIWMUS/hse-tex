\section{Смежные классы. Индекс подгруппы. Теорема Лагранжа}

\begin{definition}
    Пусть $G$ --- группа, $H \subseteq G$ --- подгруппа и $g \in G$. \textit{Левым смежным классом} элемента $g$ группы $G$ по подгруппе $H$ называется подмножество
    \begin{equation*}
        gH = \{gh \mid h \in H\}
    .\end{equation*}

    Наряду с левым смежным классом можно определить \textit{правый смежный класс} элемента $g$:
    \begin{equation*}
        Hg = \{hg \mid h \in H\}
    .\end{equation*}
\end{definition}

Все дальнейшие доказательства для правых смежный классов формулируются и доказываются аналогично.

\begin{lemma}
    Пусть $G$ --- группа, $H \subseteq G$ --- её подгруппа и $g_1, g_2 \in G$.

    Тогда либо $g_1 H = g_2 H$, либо $g_1 H \cap g_2 H = \varnothing$.
\end{lemma}

\begin{proof}
    Предположим, что $g_1 H \cap g_2 H \neq \varnothing$, то есть $g_1 h_1 = g_2 h_2$ для некоторых $h_1, h_2 \in H$. Нужно доказать, что $g_1 H = g_2 H$. Заметим, что $g_1 H = g_2 h_2 h_1^{-1} H \subseteq g_2 H$. Обратное включение доказывается аналогично.
\end{proof}

\begin{lemma}
    Пусть $G$ --- конечная группа и $H \subseteq G$ --- конечная подгруппа.

    Тогда $\left|gH\right| = \left|H\right|$ для любого $g \in G$.
\end{lemma}

\begin{proof}
    Поскольку $gH = \{gh \mid h \in H\}$, в $gH$ элементов не больше, чем в $H$.
    Если $g h_1 = gh_2$, то домножаем слева на $g^{-1}$ и получаем $h_1 = h_2$. Значит, все элементы вида $gh$, где $h \in H$, попарно различны, откуда $\left|gH\right| = \left|H\right|$.
\end{proof}

\begin{definition}
    Пусть $G$ --- группа и $H \subseteq G$ --- подгруппа. \textit{Индексом} подгруппы $H$ в группе $G$ называется число левых смежных классов $G$ по $H$.
\end{definition}

Индекс группы $G$ по подгруппе $H$ обозначается $\left[G : H\right]$.

\begin{theorem}[Теорема Лагранжа] Пусть $G$ --- конечная группа и $H \subseteq G$ --- подгруппа. Тогда
    \begin{equation*}
        |G| = |H| \cdot \left[G : H\right]
    .\end{equation*}
\end{theorem}

\begin{proof}
    Каждый элемент группы $G$ лежит в (своём) левом смежном классе по подгруппе $H$, разные смежные классы не пересекаются (лемма 1) и каждый из них содержит по $|H|$ элементов (лемма 2).
\end{proof}
