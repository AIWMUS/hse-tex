\section{Старший член многочлена от нескольких переменных. Элементарная редукция многочлена относительно другого многочлена. Лемма о конечности цепочек элементарных редукций относительно системы многочленов}

\textbf{Определение.} $f\in R\backslash\{0\}\Rightarrow$ \textbf{старший член} $L(f)$ -- это наибольший в лексикографическом порядке моном, присутствующий в f.\\
\\
$g\in R\leadsto M(g):=\{\text{все одночлены, входящие в g}\}$\\\\
\textbf{Лемма о старшем члене.} $f,g\in R\backslash\{0\}\Rightarrow L(fg)=L(f)\cdot L(g)$
\begin{proof}
    $u\in M(f), v\in M(g)\Rightarrow L(f)\succcurlyeq u, L(g)\succcurlyeq v$\\
    $uv\preccurlyeq L(f)v\preccurlyeq L(f)\cdot L(g)\Rightarrow$ Равенство только в случае $u=L(f), v=L(g)$. А значит, что $L(f)\cdot L(g)$ больше любого другого монома в $fg$.\\
    Итог: $L(fg)=L(f)L(g)$
\end{proof}
\noindent Пусть $g,f\in R\backslash\{0\}$, g содержит одночлен m, такой что $m\,\vdots\, L(f)\Rightarrow m=L(f)m'$, где $m'\in M$\\
\textbf{Элементарная редукция:} $g\overset{f}{\rightarrow}g':=g-m'f$\\
В g одночлен m заменяется суммой нескольких меньших одночленов.\\
$F\in R\backslash\{0\}$\\
\textbf{Определение.} g \textbf{редуцируется} к $g'$ при помощи F, если $\exists$ конечная цепочка элементарных редукций \\
$$g\overset{f_1}{\rightarrow}g_1\overset{f_2}{\rightarrow}g_2\overset{f_2}{\rightarrow}...\overset{f_k}{\rightarrow}g_k=g',\text{ где } f_i\in F $$
\textbf{Обозначение.} $g\overset{F}{\rightarrow}g'$\\\\
g \textbf{нередуцируем} относительно F, если $\forall \ m\in M(g)\ \forall f\in F\ m\!\!\not\vdots L(f)$\\\\
\textbf{Конечность цепочек элементарных редукций.}\\
\textbf{Лемма.} $F\subseteq R\backslash\{0\}\Rightarrow$ всякая последовательно элементарных редукций относительно F за конечное число шагов приводит к нередуцируемому многочлену.\\
\textbf{Обозначение.} $L_K(g)$ -- k-й по старшинству одночлен в $g\in R$.\\
\begin{proof}
    От противного. Пусть существует бесконечная цепочка элементарных редукций $g_1\overset{f_1}{\rightarrow}g_2\overset{f_2}{\rightarrow}g_3\overset{f_3}{\rightarrow}...$\\
    В силу того, что не существует бесконечно убывающих цепочек одночленов:\\
    $L(g_1)\succcurlyeq L(g_2)\succcurlyeq L(g_3)\succcurlyeq ...\Rightarrow \exists \ i_1\in \N : L(g_i)= L(g_{i_1})\ \forall i\geqslant i_1$\\
    $L_2(g_{i_1})\succcurlyeq L_2(g_{i_1+1})\succcurlyeq...\Rightarrow \exists i_2\geqslant i_1: L_2(g_i)=L_2(g_{i_2})\ \forall i\geqslant i_2$\\
    ... ... ... и так далее ... ... ...\\
    Итог: $L(g_{i_1})=L(g_{i_2})\succ L_2(g_{i_2})=L_2(g_{i_3})\succ L_3(g_{i_3})=L_3(g_{i_4})\succ...$\\
    $\Rightarrow L(g_{i_1})\succ L_2(g_{i_2})\succ L_3(g_{i_3})\succ...$ -- бесконечно убывающая цепочка одночленов -- противоречие.
\end{proof}