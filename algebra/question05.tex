\section{Нормальные подгруппы и факторгруппы}

\begin{definition}
    Подгруппа $H$ группы $G$ называется \textit{нормальной}, если $gH = Hg$ для любого $g \in G$.
\end{definition}

\begin{example}~
    \begin{enumerate}
    \item $G$ --- абелева. Тогда любая подгруппа $H$ нормальная.
    \item $G = S_3, G = \{\mathrm{Id}, (12)\}$. Тогда $H$ не является нормальной.
    \item Несобственные подгруппы $H = G$ и $H = \{0\}$ нормальны.
    \end{enumerate}
\end{example}

\begin{proposal}
    Для подгруппы $H \subseteq G$ следующие условия эквивалентны:
    \begin{enumerate}
    \item $H$ нормальна;
    \item $g H g^{-1} = H$ для любого $g \in G$;
    \item $g H g^{-1} \subseteq H$ для любого $g \in G$. 
    \end{enumerate}
\end{proposal}

\begin{proof}~
    \begin{description}
        \item[$(1) \implies (2)$] $gH = Hg \implies gHg^{-1} = H$.
        \item[$(2) \implies (3)$] Очев.
        \item[$(3) \implies (1)$] $gHg^{-1} \subseteq H \implies gH \subseteq Hg$. Теперь возьмем $g = g^{-1}$. Тогда $g^{-1} H g \subseteq H \implies Hg \subseteq gH \implies gh = Hg$.
            \qedhere
    \end{description}
\end{proof}

\bigskip
Рассмотрим множество смежных классов по нормальной подгруппе $G / H$.

Определим на $G / H$ бинарную операцию, полагая $(g_1 H)(g_2 H) = (g_1 g_2) H$.

\begin{description}
\item[Корректность] 
    Пусть $g'_1 H = g_1 H$ и $g_2' H = g_2 H$. 
    Тогда $g_1' = g_1 h_1$, $g_2' = g_2 h_2$, где $h_1, h_2 \in H$.
    \begin{equation*}
        (g_1'H)(g_2' H) = (g_1' g_2')H = (g_1 h_1 g_2 h_2) H = (g_1 g_2 (\underbrace{g_2^{-1} h_1 g_2}_{\in H}) h_2) H \subseteq (g_1 g_2) H \implies (g_1' g_2') H = (g_1 g_2) H
    .\end{equation*}
\end{description}

Структура группы $G / H$.
\begin{enumerate}[nosep]
\item Ассоциативность очевидна.
\item Нейтральный элемент --- $eH$.
\item Обратный к $gH$ --- $g^{-1}H$.
\end{enumerate}

\begin{definition}
    Множество $G / H$ с указанной операцией называется \textit{факторгруппой} группы $G$ по нормальной подгруппе $H$.
\end{definition}

\begin{example}
    Если $G = (\ZZ, +)$ и $H = n\ZZ$, то $G / H$ --- это в точности группа вычетов $(\ZZ_n, +)$.
\end{example}
