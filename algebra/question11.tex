\section{Криптография с открытым ключом. Задача дискретного логарифмирования. Система Диффи~Хеллмана обмена ключами. Криптосистема Эль-Гамаля}

Пусть $G$ --- конечная абелева группа (например, $G = (\ZZ_p \setminus \{0\}, \times)$, где $p$ --- большое простое число) и $g \in G$ --- элемент достаточно большого порядка.

\begin{problem}
    Задача дискретного логарифмирования.

    Дано $g \in G$, $\ord(g) \gg 0$, $h \in \left< g \right>$. Найти такое $k \in \NN$, что $g^{k} = h$.

    При этом задача возведения в степень имеет быстрый алгоритм --- повторное возведение в квадрат.
    \begin{equation*}
        g^{16} = \left(\left(\left(g^2\right)^2\right)^2\right)^2 \quad 
        g^{15} = \left(\left(g^2 \cdot g\right)^2 \cdot g\right)^2 \cdot g
    .\end{equation*}

    Сама же задача нахождения степени решается только переборными и близкими к перебору способами.
\end{problem}

\begin{problem}
    Система Диффи-Хеллмана обмена ключами (1976).

    $G$ и $g$ известны всем.

    Алиса фиксирует свое секретное $\alpha \in \NN$ и сообщает всем пользователям $g^{\alpha}$.

    Боб совершает аналогичные действия --- $\beta \in \NN, g^{\beta}$.

    Теперь Алиса и Боб возводят элемент другого в свою секретную степень, оба получают $(g^{\alpha})^{\beta} = \left(g^{\beta}\right)^{\alpha} = g^{\alpha\beta}$.

    Теперь по этому ключу можно устроить шифрованный канал связи, к которому никто не имеет доступа. При этом действительно в силу сложности задачи дискретного логарифмирования по $g^{\alpha}$ и $g^{\beta}$ нельзя быстро получить $g^{\alpha\beta}$.
\end{problem}

\begin{problem}
    Криптосистема Эль-Гамаля.

    Алиса фиксирует свое секретное $\alpha \in \NN$ и сообщает всем пользователям $g^{\alpha}$.

    Боб хочет передать Алисе элемент $h \in G$.

    Для этого Боб фиксирует какое-то $\beta \in \NN$ и объявляет пару $ \{g^{\beta}, h \cdot (g^{\alpha})^{\beta}\}$.

    Отсюда $h = \left(h \cdot (g^{\alpha})^{\beta}\right) \cdot ((g^{\beta})^{\alpha})^{-1} = (h \cdot (g^{\alpha})^{\beta}) \cdot (g^{\beta})^{|G|-\alpha}$, то есть зная $\alpha$ можно легко получить $h$.

    Следовательно, получить его может только Алиса, а всем остальным придется решать задачу дискретного логарифмирования.
\end{problem}
