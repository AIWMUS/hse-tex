\section{Лемма о конечности цепочек одночленов, в которых каждый следующий одночлен не делится ни на один из предыдущих. Алгоритм Бухбергера построения базиса Грёбнера идеала}

\textbf{Лемма.} $\not\exists$ бесконечных последовательностей одночленов $m_1, m_2, \dotsc$, таких что $m_i\!\not\vdots\, m_j\ \forall i > j.$
\begin{proof}
    Индукция по $n$: $n=1 \Rightarrow$ степени убывают $\Rightarrow$ цепочка конечна.\\
    Пусть доказано для $<n$, докажем для $n$. Пусть есть бесконечная последовательность $m_1, m_2, \dotsc,$ $m_i \!\!\not\vdots\, m_j\ \ \forall i > j:$ $m_i = a_i x_1^{k_1(i)}\cdot \dotsc \cdot x_n^{k_n(i)}$.
    Тогда $\forall j \geq 2$ $m_j \!\!\not\vdots\, m_1 \Rightarrow \exists i \in \{1, \dotsc, n\}$, такое что $k_i(j) < k_i(1)$ для \textbf{бесконечного числа значений $j$}. 
    \\Без ограничения общности считаем $i = n$. Перейдя к подпоследовательности, можем считать, что $k_n(j) < k_n(1), \ \forall j \geq 2$. Тогда $k_n(j)$ принимает лишь конечное число значений $\Rightarrow$ какое--то из этих значений встретится бесконечно много раз. 
    \\Снова перейдя к подпоследовательности, можем считать, что $k_n(1) = k_n(2) = \dotsc$, полагая $x_n = 1$, получим последовательность от $x_1, \dots, x_{n-1}$ с тем же свойством  -- противоречие.
\end{proof}
\noindent\textbf{Алгоритм Бухбергера построения базиса Грёбнера идеала.}
\\Дано: $I = (F), F={f_1, \dotsc, f_k}$\\
Перебираем все пары $i < j$. Если  $\exists\ i < j$, такое что $S(f_i, f_j) \overset{F}{\leadsto} r_1 \neq 0,$ $r_1$ - остаток, то добавляем $r_1$ в $F$ и повторяем процедуру для $F \cup \{r_1\}$. В итоге получаем $\forall i, j: S(f_i, f_j) \overset{F \cup \{r_n\}}{\leadsto} 0$. Полученное $F$ -- это система Грёбнера по критерию Бухбергера $\Rightarrow$ $F$ - базис Грёбнера в $I$. Если алгоритм не закончится за конечное число шагов, то получим бесконечную последовательность $r_1, r_2, r_3, \dotsc$, такую что $L(r_i) \!\!\not\vdots L(r_j)$ при $i > j$ -- противоречие с леммой.