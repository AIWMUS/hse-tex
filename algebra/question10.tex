\section{Экспонента конечной абелевой группы и критерий цикличности}

\begin{definition}
    \textit{Экспонентой} конечной абелевой группы $A$ называется число
    \begin{equation*}
        \exp A := \min \{m \in \NN \mid ma = 0 \ \forall a \in A\}
    .\end{equation*}
\end{definition}

\begin{comment}~
    \begin{enumerate}
        \item Так как $ma = 0 \iff m \divby \ord(a) \ \forall a \in A $ и $m \in \ZZ$, то определение экспоненты можно переписать ещё в виде $\exp A = \gcd \{\ord(a) \mid a \in A\}$.
        \item Так как $|A| \divby \ord(a) \ \forall a \in A$, то $|A|$ --- общее кратное множества $ \{\ord(a) \mid a \in A\}$, а значит, $|A| \divby \exp A$.

            В частности, $\exp A \leq |A|$.
    \end{enumerate}
\end{comment}


\begin{proposal}
    $\exp A = |A| \iff A $ --- циклическая группа.
\end{proposal}

\begin{proof}
    Пусть $|A| = n = p_1^{k_1} \cdot \ldots \cdot p_s^{k_s}$ --- разложение на простые множители, где $p_i$ --- простое и $k_s \in \NN$. ($p_i \neq p_j$ при $i \neq j$)

    \begin{description}
    \item[$\impliedby$] Если $A = \left< a \right>$, то $\ord a = n$, откуда сразу получаем $\exp A = n$.
    \item[$\implies$] 
        Если $\exp A = n$, то для $i = 1, \dots, s$ существует элемент $c_i \in A$, такой что $\ord c_i = p_i^{k_i}m_i$, где $m_i \in \NN$.
        Для каждого $i = 1, \dots, s$ положим $a_i = m_i c_i$, тогда $\ord(a_i) = p_i^{k_i}$.
        Теперь рассмотрим элемент $a = a_1 + \dots + a_s$ и покажем, что $\ord(a) = n$.
        Пусть $ma = 0$ для некоторого $m \in \NN$, то есть $ma_1 + \dots + ma_s = 0$.
        При фиксированном $i \in \{1, \dots, s\}$ умножим обе части последнего равенства на $n_i := n / p_i^{k_i}$.
        Легко видеть, что $mn_i a_j = 0$ при всех $i \neq j$, поэтому в левой части выживет только слагаемое $mn_i a_i$, откуда получаем $mn_i a_i = 0$.
        Следовательно, $mn_i \divby p_i^{k_i}$, а так как $n_i$ не делится на $p_i$, то $m \divby p_i^{k_i}$.
        В силу произвольности выбора $i$ отсюда вытекает, что $m \divby n$.
        Так как $na = 0$, то мы окончательно получаем $\ord(a) = n$.
        Значит, $A = \left< a \right>$ --- циклическая группа.
    \end{description}
\end{proof}


