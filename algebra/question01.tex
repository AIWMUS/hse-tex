\section{Бинарные операции. Полугруппы, моноиды и группы. Коммутативные группы. Примеры групп. Порядок группы. Подгруппы. Описание всех подгрупп в группе $(\ZZ, +)$}

\begin{definition}
    \textit{Множество с бинарной операцией} --- это множество $M$ с заданным отображением
    \begin{equation*}
        M \times M \to M, \quad (a, b) \mapsto a \circ b
    .\end{equation*}

    Множество с бинарной операцией обычно обозначают $(M, \circ)$.
\end{definition}

\begin{definition}
    Множество с бинарной операцией $(M, \circ)$ называется \textit{полугруппой}, если данная бинарная операция \textit{ассоциативна}, то есть
    \begin{equation*}
        a \circ (b \circ c) = (a \circ b) \circ c \quad \text{для всех } a, b, c \in M
    .\end{equation*}
\end{definition}

Не все естественно возникающие операции ассоциативны. Например, если $M = \NN$ и $a \circ b = a ^{b}$, то
\begin{equation*}
    2^{(1^{2})} = 2 \neq (2^{1})^{2} = 4
.\end{equation*}

Другой пример неассоциативной бинарной операции: $M = \ZZ$ и $a \circ b := a - b$.

Полугруппу обычно обозначают $(S, \circ)$.

\begin{definition}
    Полугруппа $(S, \circ)$ называется \textit{моноидом}, если в ней есть \textit{нейтральный элемент}, то есть такое элемент $e \in S$, что $e \circ a = a \circ e = a$ для любого $a \in S$.
\end{definition}

\begin{comment}
    Если в полугруппе есть нейтральный элемент, то он один. В самом деле, $e_1 \circ e_2 = e_1 = e_2$.
\end{comment}

\begin{definition}
    Моноид $(S, \circ)$ называется \textit{группой}, если для каждого элемента $a \in S$ найдется \textit{обратный элемент}, то есть такой $b \in S$, что $a \circ b = b \circ a = e$.
\end{definition}

Обратный элемент обозначается $a^{-1}$.

Группу принято обозначать $(G, \circ)$ или просто $G$, когда понятно, о какой операции идёт речь. Обычно символ $\circ$ обозначения операции опускают и пишут просто $ab$.

\begin{definition}
    Группа $G$ называется \textit{коммутативной} или \textit{абелевой}, если групповая операция \textit{коммутативна}, то есть $ab = ba$ для любых $a, b \in G$.
\end{definition}

Если в случае произвольной группы $G$ принято использовать мультипликативные обозначения для групповой операции --- $gh, e, g^{-1}$, то в теории абелевых групп чаще используют аддитивные обозначения, то есть $a + b, 0, -a$.

\begin{definition}
    \textit{Порядок} группы $G$ --- это число элементов в $G$. Группа называется \textit{конечной}, если её порядок конечен, и \textit{бесконечной} иначе.
\end{definition}

Порядок группы $G$ обозначается $|G|$.

\bigskip
Приведем несколько серий примеров групп.
\begin{enumerate}
\item Числовые аддитивные группы:

    $(\ZZ, +), (\QQ, +), (\RR, +), (\CC, +), (\ZZ_n, +)$.

\item Числовые мультипликативные группы:
    
    $(\QQ \setminus \{0\}, \times), (\RR \setminus \{0\}, \times), (\CC \setminus \{0\}, \times), (\ZZ_p \setminus \{\overline{0}\}, \times)$, $p$ --- простое.

\item Группы матриц:
    
    $\mathrm{GL}_n(\RR) = \{A \in \mathrm{Mat}_{n \times n}(\RR) \mid \det A \neq 0\}$ --- полная линейная группа;

    $\mathrm{SL}_n(\RR) = \{A \in \mathrm{Mat}_{n \times n}(\RR) \mid \det A = 1\}$ --- специальная линейная группа.

\item Группы перестановок (с операцией композиции):

    симметрическая группа $S_n$ --- все перестановки длины $n$, $|S_n| = n!$;

    знакопеременная группа $A_n$ --- чётные подстановки длины $n$, $|A_n| = \dfrac{n!}{2}$.

\item Группы преобразований: симметрия, движение.
\end{enumerate}


\begin{definition}
    Подмножество $H$ группы $G$ называется \textit{подгруппой}, если выполнены следующие три условия:
    \begin{enumerate}[nosep]
    \item $e \in H$;
    \item $ab \in H$ для любых $a, b \in H$;
    \item $a^{-1} \in H$ для любого $a \in H$.
    \end{enumerate}
\end{definition}

В каждой группе $G$ есть \textit{несобственные} подгруппы $H = \{e\}$ и $H = G$. Все прочие подгруппы называются \textit{собственными}. Например, чётные числа $2\ZZ$ образуют собственную подгруппу в $(\ZZ, +)$.

\begin{proposal}
    Всякая подгруппа в $(\ZZ, +)$ имеет вид $k\ZZ$ для некоторого целого неотрицательного $k$.
\end{proposal}

\begin{proof}
    Очевидно, что все подмножества вида $k\ZZ$ являются подгруппами в $\ZZ$.

    Пусть $H \subseteq \ZZ$ --- подгруппа. Если $H = \{0\}$, то $H = 0\ZZ$.
    
    Иначе положим $k = \min(H \cap \NN) \neq 0$. (это множество непусто, так как $\forall x \implies -x \in H$)

    Тогда $k\ZZ \subseteq H$.

    Покажем, что $k\ZZ = H$. Пусть $a \in H$ --- произвольный элемент. Поделим его на $k$ с остатком.

    $a = qk + r$, где $q \in H$, $0 \leq r < k \implies r = a - qk \in H$.

    В силу выбора $k$ получаем $r = 0 \implies a = qk \in k \ZZ$.
\end{proof}
