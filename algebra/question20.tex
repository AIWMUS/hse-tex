\section{Остаток многочлена относительно заданной системы многочленов. Системы Грёбнера. Характеризация систем Грёбнера в терминах цепочек элементарных редукций}

\textbf{Определение.} Если $g\overset{F}{\leadsto}r$ и r нередуцируем, r называется \textbf{остатком} многочлена g относительно F.\\
\textbf{Замечание.} Вообще говоря, остаток определён неоднозначно.\\\\
\textbf{Определение.} Множество F называется \textbf{системой Грёбнера}, если $\forall g \in R$, остаток $g$ относительно F определён однозначно, то есть не зависит от цепочки приводящих к нему элементарных редукций.\\\\
\textbf{Предложение.} Следующие условия эквивалентны:\\
\indent 1) F -- система Грёбнера.\\
\indent 2) $\forall g\in R$ обладает следующим свойством: если $g\overset{f_1}{\to}g_1$ и $g\overset{f_2}{\to}g_2$ -- две элементарные редукции, то $\exists \ g'\in R:\ \begin{cases} g_2\overset{F}{\leadsto}g'\\g_1\overset{F}{\leadsto}g' \end{cases}$
\begin{proof}
    $(1)\Rightarrow (2)$ В качестве $g'$ можно взять остаток $g$ относительно системы F.\\
    $(2)\Rightarrow (1)$\\ $B(F):=$ все многочлены из R, для которых остаток относительно F определён неоднозначно.\\
    $E_F(g)$ -- множество всех элементарных редукций многочлена g относительно F.\\
    Пусть $B(F)\neq \varnothing$ и $g\in B(F)$\\
    Если $E_F(g)\cap B(F)\neq\varnothing$, то возьмём $g_1\in E_F(g)\cap B(F)$\\
    Если $E_F(g_1)\cap B(F)\neq\varnothing$, то возьмём $g_2\in E_F(g_1)\cap B(F)$\\
    И так далее\\
    Из того, что цепочки элементарных редукций конечны, вытекает $\exists \ i\in \N: E_F(g_i)\cap B(F)=\varnothing$\\
    Тогда $\exists $ две цепочки элементарных редукций\\
    $\begin{cases} g_i\to h_1\to ...\to r_1\\g_i\to h_2\to ...\to r_2 \end{cases}$ -- неравные остатки.\\
    По условию $\exists r\in R$ -- нередуцируемый, такой что $h_1\leadsto r, h_2\leadsto r$.\\
    Так как $h_1, h_2\not\in B(F)$, то $r_1=r, r_2=r$\\
    $\Rightarrow r_1=r_2$ -- противоречие.
\end{proof}