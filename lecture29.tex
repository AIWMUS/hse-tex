\section{Лекция 26.04.2018}

$V$ -- векторное пространство над $F$, $dimV = n$

$\varphi \in L(V)$

\vspace{\baselineskip}
\textbf{Предложение.} Пусть $\{\lambda_1, \dots, \lambda_s\} \subseteq Spec(\varphi), \ \lambda_i \neq \lambda_j \ \forall \ i \neq j$, тогда подпространства $V_{\lambda_1} (\varphi), \dots, V_{\lambda_s} (\varphi)$ линейно независимы.

\vspace{\baselineskip}
\textbf{\textit{Доказательство.}} $\rhd$ Индукция по $s$.

$s = 1$ -- нечего доказывать

Пусть для $< s$ утверждение доказано, докажем для $s$.

Пусть $v_i \in V_{\lambda_i} (\varphi) \ \forall \ i = 1, \dots, s$ и пусть $v_1 + \dots + v_s = 0$ (*)

Применим $\varphi$: $\varphi(v_1 + \dots + v_s) = \varphi(0) = 0 \Rightarrow \varphi(v_1) + \dots + \varphi(v_s) = 0 \Rightarrow \lambda_1 v_1 + \dots + \lambda_s v_s = 0$

Вычтем отсюда (*)$\cdot \lambda_s$: $(\lambda_1 - \lambda_s) v_1 + \dots + (\lambda_{s-1} - \lambda_s) v_{s - 1} = 0$, по предположению индукции получаем отсюда $v_1 = \dots = v_s = 0$. Тогда $v_s = 0 \ \lhd$ 

\vspace{\baselineskip}
\textbf{Следствие.} Если $\chi_{\varphi}(t)$ имеет ровно $n$ различных корней, то $\varphi$ диагонализуем.

\vspace{\baselineskip}
\textbf{\textit{Доказательство.}} $\rhd$ Пусть $Spec (\varphi) = \{\lambda_1, \dots, \lambda_n\}, \lambda_i \neq \lambda_j \ \forall \ i \neq j$. $\forall \ i = 1, \dots, n$ выберем ненулевой вектор $v_i \in V_{\lambda_i} (\varphi)$. Тогда $v_1, \dots, v_n$ линейно независимы по предыдущему предложению $\Rightarrow (v_1, \dots, v_n)$ -- базис из сосбтвенных векторов $\Rightarrow \varphi$ диагонализуем $\lhd$

\vspace{\baselineskip}
\textbf{Теорема (критерий диагонализуемости).} Линейный оператор $\varphi \in L(V)$ диагонализуем $\Leftrightarrow$ выполнены следующие условия:

1) $\chi_{\varphi} (t)$ разлагается на линейные множители

2) $\forall \ \lambda \in Spec(\varphi)$ (геометрическая кратность $\lambda$) = (алгебраическая кратность $\lambda$)

\vspace{\baselineskip}
\textbf{\textit{Доказательство.}} $\rhd$ $(\Rightarrow)$ Пусть $(e_1, \dots, e_n)$ -- базис, такой что $A(\varphi, e) = \begin{pmatrix} \mu_1 & 0 & \dots & 0 \\ 0 & \mu_2 & \dots & 0 \\ \vdots & \vdots & \vdots & \vdots \\ 0 & 0 & \dots & \mu_n \end{pmatrix}$

Тогда $\chi_{\varphi} (t) = (-1)^n \begin{vmatrix} \mu_1 - t & 0 & \dots & 0 \\ 0 & \mu_2 - t & \dots & 0 \\ \vdots & \vdots & \vdots & \vdots \\ 0 & 0 & \dots & \mu_n - t \end{vmatrix} = (-1)^n (t - \mu_1) (t - \mu_2) \dots (t - \mu_n) \Rightarrow 1)$ выполнено

Перепишем $\chi_\varphi (t)$ в виде $\chi_\varphi (t) = (t - \lambda_1)^{k_1} \cdots (t - \lambda_s)^{k_s}$, где $\lambda_i \neq \lambda_j$ при $i \neq j$

$\forall \ i = 1, \dots, s$ имеем $V_{\lambda_i} (\varphi) \supseteq <e_j \ | \ \mu_j = \lambda_i > \Rightarrow dim V_{\lambda_i} (\varphi) \geq k_i$ (уже знаем $dim V_{\lambda_i} (\varphi) \leq k_i$) $\Rightarrow dim V_{\lambda_i} (\varphi) = k_i$

\vspace{\baselineskip}
$(\Leftarrow)$ Пусть $\chi_{\varphi} (t) = (t - \lambda_1)^{k_1} \cdots  (t - \lambda_s)^{k_s}, \lambda_i \neq \lambda_j$ при $i \neq j$

Т.к. $V_{\lambda_1} (\varphi), \dots, V_{\lambda_s} (\varphi)$ линейно независимы, то $dim (V_{\lambda_1} (\varphi) + \dots + V_{\lambda_s} (\varphi)) = dim V_{\lambda_1} (\varphi) + \dots + dim V_{\lambda_s} (\varphi) = k_1 + \dots + k_s = n \Rightarrow V = V_{\lambda_1} (\varphi) \oplus \dots \oplus V_{\lambda_s} (\varphi)$

$\forall i = 1, \dots, s$ выберем в $V_{\lambda_i} (\varphi)$ базис $e_i$ и положим $e = e_1 \cup e_2 \cup \dots \cup e_s$. Тогда $e$ -- базис $V$, состоящий из собственных векторов $\lhd$

\vspace{\baselineskip}
\textbf{Замечание.} Если выполнено 1), то линейный оператор $\varphi$ можно привести к \textit{жордановой нормальной форме}, т.е. $\exists$ базис $e$ в $V$, такой что $A(\varphi, e) = \left(
\begin{array}{c|c|c|c|c}
  J^{m_1}_{M_1} & 0 & 0 & \dots & 0  \\
  \hline
  0 & J^{m_2}_{M_2} & 0 & \dots & 0  \\
  \hline
  0 & 0 & J^{m_3}_{M_3} & \dots & 0 \\
  \hline
  \vdots & \vdots & \vdots & \vdots & \vdots \\
  \hline
  0 & 0 & 0 & \dots & J^{m_s}_{M_s} \\
\end{array}
\right)$,

где $J^{m_i}_{M_i} = \begin{pmatrix} M_i & 1 & 0 & \dots & 0 \\ 0 & M_i & 1 & \dots & 0 \\ 0 & 0 & M_i & \dots & 0
\\ \vdots & \vdots & \vdots & \vdots & \vdots \\ 0 & 0 & 0 & \dots & M_i \end{pmatrix}$ -- Жорданова клетка порядка $m_i$ с собственным значением $M_i$

\vspace{\baselineskip}
\textbf{Следствие.} Если $F = \mathbb{C}$, то $\forall \ \varphi \in L(V)$ можно привести к жордановой нормальной форме.

\vspace{\baselineskip}
\textbf{\textit{Доказательство.}} Условие 1) выполняется по основной теореме алгебры комплексных чисел.

\vspace{\baselineskip}
Примеры.

1) $\varphi = \lambda \cdot Id \Rightarrow \chi_{\varphi} (t) = (t - \lambda)^n$

$Spec(\varphi) = \{\lambda\}$, алгебраическая кратность $\lambda$ = геометрической кратности $\lambda$ = $n$

2) $\varphi: \mathbb{R}^2 \rightarrow \mathbb{R}^2$ -- ортогональная проекция на прямую $l \rightarrow 0$

$e_1 \in l \setminus \{0\}, e_2 \in l^{\bot} \setminus \{0\}, e = (e_1, e_2) \Rightarrow A(\varphi, e) = \begin{pmatrix} 1 & 0 \\ 0 & 0 \end{pmatrix}$

$\chi_{\varphi} (t) = t (t - 1)$

$Spec(\varphi) = \{0, 1\}, \lambda \in \{0, 1\} \Rightarrow$ алгебраическая кратность = геометрической кратности = 1

3) $\varphi: \mathbb{R}^2 \rightarrow \mathbb{R}^2$ -- поворот на угол $\alpha \neq \pi k$

$e = (e_1, e_2)$ -- положительно ориентированный ортогональный базис $\Rightarrow A(\varphi, e) = \begin{pmatrix} \cos \alpha & \ - \sin \alpha \\ \sin \alpha & \cos \alpha \end{pmatrix}$

$\chi_{\varphi} (t) = \begin{vmatrix} \cos \alpha - t & \ -\sin \alpha \\ \sin \alpha & \cos \alpha - t \end{vmatrix} = t^2 - 2 \cos \alpha t + 1$

$D/4 = \cos^2 \alpha - 1 = -\sin^2 \alpha < 0 \Rightarrow$ нет корней в $\mathbb{R} \Rightarrow 1)$ не выполняется $\Rightarrow \varphi$ не (над $\mathbb{C} \varphi$ диагонализуем) 

4) $V = F[x]_{\leq n}, \ \varphi: f \rightarrow f'$

$e = (1, x, x^2, \dots, x^n), A(\varphi, e) = \begin{pmatrix} 0 & 1 & 0 & \dots & 0 \\ 0 & 0 & 2 & \dots & 0 \\ \vdots & \vdots & \vdots & \vdots & \vdots \\ 0 & 0 & 0 & \dots & n \\ 0 & 0 & 0 & \dots & 0
\end{pmatrix}$

$\chi_{\varphi} (t) = t^{n+1} \Rightarrow 1)$ выполнено

$Spec (\varphi) = \{0\}$, алгебраическая кратность = $n + 1$

геометрическая кратность = 1 $\Rightarrow \varphi$ не диагонализуем, 2) не выполняется

$f = (1, x, \frac{x^2}{2}, \dots, \frac{x^n}{n!}) \Rightarrow A(\varphi, f) = \begin{pmatrix} 0 & 1 & 0 & \dots & 0 \\ 0 & 0 & 1 & \dots & 0 \\ \vdots & \vdots & \vdots & \vdots & \vdots \\ 0 & 0 & 0 & \dots & 1 \\ 0 & 0 & 0 & \dots & 0 \end{pmatrix} = J_0^{n+1}$

\vspace{\baselineskip}
\textbf{Предложение.} $F = \mathbb{R} \Rightarrow \ \forall \ \varphi \in L(V)$, $\exists$ либо одномерное, либо двумерное $\varphi$-инвариантное подпространство

\vspace{\baselineskip}
\textbf{\textit{Доказательство.}} $\rhd$ Если $\chi_{\varphi} (t)$ имеет действительный корень, то тогда у $\varphi$ есть собственный вектор $\Rightarrow$ есть 1-мерное инвариантное подпространство. Пусть $\chi_{\varphi} (t)$ не имеет корней в $\mathbb{R}$, возьмем какой-нибудь комплексный корень $\lambda + i \mu$, где $\lambda, \mu \in \mathbb{R}, \mu \neq 0$. Фиксируем базис $e = (e_1, \dots, e_n)$ в $V$, пусть $A = A(\varphi, e)$. 

Тогда для $A$ над $C$ $\exists$ собственный вектор с собстсвенным значением $\lambda + i \mu$. То есть $\exists \ u, v \in \mathbb{R}^n, u + i v \neq 0$, такой что $A \cdot (u + iv) = (\lambda + i \mu) (u + iv) = (\lambda u - \mu v) + i (\lambda v + \mu u) \Rightarrow \begin{cases}
        Au = \lambda u - \mu v \\
        Av = \lambda v + \mu u
\end{cases} \Rightarrow A <u, v> \subseteq <u, v> \Rightarrow$ векторы с коэффициентами $u, v$ порождают $\varphi$-инвариантное подпространство размерности $\leq 2 \ \lhd$

\subsection{Линейные отображения и линейные операторы в евклидовых пространствах}

$E$ -- евклидово пространство, $dim E = n, ( \cdot, \cdot)$ -- скалярное произедение

$E'$ -- евклидово пространство (другое), $dim E' = m, (\cdot, \cdot)'$ -- скалярное произведение

Пусть $\varphi: E \rightarrow E'$ -- линейное отображение

\textbf{Определение.} Линейное отображение $\psi: E' \rightarrow E$ называется \textit{сопряженным} к $\varphi$, если $\forall \ x \in E, \forall \ y \in E': \ (\varphi(x), y)' = (x, \psi(y)) \ (\#)$

Обозначение: $\varphi^*$

\vspace{\baselineskip}
\textbf{Предложение.} 1) $\psi$ существует и единственно

2) Если $e$ и $f$ -- ортонормированные базисы в $E$ и $E'$ соответственно, $A_{\varphi} = A(\varphi, e, f), A_{\psi} = A(\psi, f, e) \Rightarrow A_{\psi} = A_{\varphi}^T$

\vspace{\baselineskip}
\textbf{\textit{Доказательство.}} $\rhd$ 1) Фиксируем произвольный базис $e$ в $E$ и произвольный базис $f$ в $E'$, $A_{\varphi} = A(\varphi, e, f), A_{\psi} = A(\psi, f, e), G = G(e), G' = G(f)$

$x \in E, x = x_1 e_1 + \dots + x_n e_n$

$y \in E', y = y_1 f_1 + \dots + y_m f_m$

$(\varphi(x), y)' = \left(A_{\varphi} \begin{pmatrix} x_1 \\ \vdots \\ x_n \end{pmatrix} \right)^T G' \begin{pmatrix} y_1 \\ \vdots \\ y_m \end{pmatrix} = (x_1, \dots, x_n) A_{\varphi}^T G' \begin{pmatrix} y_1 \\ \vdots \\ y_m \end{pmatrix}$

$(x, \psi(y)) = (x_1, \dots, x_n) G A_{\psi} \begin{pmatrix} y_1 \\ \vdots \\ y_m \end{pmatrix}$

$\forall \ B \in Mat_{n \times m} (\mathbb{R}):$

$b_{ij} = (0 \dots 01^i0 \dots 0) B \begin{pmatrix} 0 \\ \vdots \\ 0 \\ 1_j \\ 0 \\ \vdots \\ 0 \end{pmatrix}$

$\Rightarrow (\#)$ выполняется $\Leftrightarrow A_{\varphi}^T G' = G A_{\psi} \Leftrightarrow G^{-1} A_{\varphi}^T G' \Rightarrow$ существование и единственность

2) $e$ и $f$ -- ортонормированные $\Rightarrow G = E$ и $G' = E \Rightarrow A_{\psi} = A_{\varphi}^T \lhd$

