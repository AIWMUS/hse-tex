\documentclass[a4paper]{article}
\usepackage{header}
\usepackage{float}

\newcommand\enumtocitem[3]{\item\textbf{#1}\addtocounter{#2}{1}\addcontentsline{toc}{#2}{\protect{\numberline{#3}} #1}}
\newcommand\defitem[1]{\enumtocitem{#1}{subsection}{\thesubsection}}
\newcommand\proofitem[1]{\enumtocitem{#1}{subsection}{\thesubsection}}

\newcommand{\italicbold}[1]{\emph{\textbf{#1}}}

\newlist{colloq}{enumerate}{1}
\setlist[colloq]{label=\textbf{\arabic*.}}

\everymath{\displaystyle}

\renewcommand*{\arraystretch}{1.5}

\title{\HugeМатематический анализ, Коллоквиум 2}
\author{
	Балюк Игорь \\
	\href{https://teleg.run/lodthe}{@lodthe},
    \href{https://github.com/LoDThe/hse-tex}{GitHub} \\
}
\date{2019 --- 2020}

\begin{document}
    \maketitle

    \tableofcontents

    \newpage

    \section{Вопросы предварительной части коллоквиума}

	Список вопросов предварительной части коллоквиума, ответ на которые	необходим для подготовки к основной части.

    \begin{colloq}
    
    \defitem{Определение непрерывности функции в точке.}

    	Пусть $f(x)$ --- функция, определенная на промежутке $I$ ($I$ --- это её область определения) и пусть $c$ --- произвольная точка из $I$. Предположим, что для любого $\eps > 0$ существует $\delta > 0$:

    	\begin{equation*}
    		\forall x \in I: \; |x - c| < \delta \implies |f(x) - f(c)| < \eps
    	\end{equation*}

    	Тогда функция $f(x)$ \italicbold{непрерывна} в точке $c$. 

    	Заметьте, если $c$ --- это левая граница $I$, то условие имеет вид (функция непрерывна в точке $c$ справа, аналогично для непрерывности слева).
    	\begin{equation*}
    		\forall x \in I: \; c < x < c + \delta \implies |f(x) - f(c)| < \eps
    	\end{equation*}

    	\begin{theorem*}
    		Также, функция $f(x)$ непрерывна в точке $a$. Тогда найдётся такое $\delta > 0$, что функция $f(x)$ ограничена окрестностью $U_{\delta}(a)$ точки $a$. 
    	\end{theorem*}

	\defitem{Точки разрыва, их классификация.}

		Пусть $f(x)$ определена в некоторой окрестности $U_{\delta}(a)$.

		\begin{itemize}
			\item
			\italicbold{Устранимый разрыв}: пределы справа и слева существуют и равны друг другу, но отличаются от значения функции в исследуемой точке:

			\begin{equation*}
				\lim_{x \to a - 0} f(x) = \lim_{x \to a + 0} f(x) \neq f(a)
			\end{equation*}

			\item
			\italicbold{Неустранимый разрыв первого рода}: пределы справа и слева существуют, но не равны друг другу

			\item
			\italicbold{Неустранимый разрыв второго рода}: хотя бы один из односторонних пределов не существует или равен бесконечности.
		\end{itemize}

	\defitem{Теорема о непрерывности сложной функции.}

		\begin{theorem*}
			Пусть функция $g(x)$ непрерывна в точке $a_0$ и функция $f(x)$ непрерывна в точке $b_0~=~g(a_0)$. Тогда функция $f(g(x))$ непрерывна в точке $a_0$.
		\end{theorem*}

	\defitem{Формулировки первой и второй теорем Вейерштрасса.}

		\begin{theorem*}[Первая теорема Вейерштрасса]
			Если функция $f(x)$ непрерывна на отрезке $[a, b]$, то она ограничена на этом отрезке.
		\end{theorem*}

		\begin{theorem*}[Вторая теорема Вейерштрасса]
			Непрерывная на отрезке $[a, b]$ функция $f$ достигает на нем своих нижней и верхней граней. Или, что тоже самое, достигает на отрезке своего минимума и максимума. То есть существуют такие точки $x_1, x_2 \in [a, b]$, так что для любого $x \in [a, b]$, выполняются неравенства:

			\begin{equation*}
				f(x_1) \leq f(x) \leq f(x_2)
			\end{equation*}
		\end{theorem*}

	\defitem{Понятие производной функции в точке.}

		Рассмотрим функцию, область определения которой содержит точку $x_0$. Тогда функция $f(x)$ является дифференцируемой в точке $x_0$, и ее производная определяется формулой

		\begin{equation*}
			f'(x_0) = \lim_{\Delta \to 0} \dfrac{f(x_0 + \Delta) - f(x_0)}{\Delta}
		\end{equation*}

		если предел существует.

	\defitem{Геометрический и физический смысл производной.}

		\textbf{Геометрический смысл производной}. Производная в точке $x_0$ равна угловому коэффициенту касательной к графику функции $y = f(x)$ в этой точке.

		\textbf{Физический смысл производной}. Если точка движется вдоль оси $OX$ и ее координата изменяется по закону  $x(t)$, то мгновенная скорость точки: $v(t) = x'(t)$.

	\defitem{Уравнение касательной к графику функции в точке.}

		Пусть дана функция $f$, которая в некоторой точке $x_0$ имеет конечную производную $f(x0)$. Тогда прямая, проходящая через точку $(x_0; f(x_0))$, имеющая угловой коэффициент $f’(x_0)$, называется касательной.

		Итак, пусть дана функция $y = f(x)$, которая имеет производную $y = f’(x)$ на отрезке $[a, b]$. Тогда в любой точке $x_0 \in (a; b)$ к графику этой функции можно провести касательную, которая задается уравнением:

		\begin{equation*}
			y = f'(x_0) \cdot (x - x_0) + f(x_0)
		\end{equation*}

	\defitem{Понятие дифференцируемости функции в точке.}

		Функция $f(x)$ является дифференциируемой в точке $x_0$ своей области определения $D[f]$, если существует такая константа A, что:

		\begin{equation*}
			f(x) = f(x_0) + A(x - x_0) + \os(x - x_0)
		\end{equation*}
		и
		\begin{equation*}
			A = f'(x_0) = \lim_{\Delta \to 0} \dfrac{f(x_0 + \Delta) - f(x_0)}{\Delta}
		\end{equation*}

	\defitem{Правила дифференцирования (производная суммы, произведения, частного).}

		Пусть функции $f(x)$ и $g(x)$ имеют производные в точке $x_0$. Тогда,

		\[\begin{gathered}
			(g + f)'(x_0) = g'(x_0) + f'(x_0) \\
			(g \cdot f)'(x_0) = g'(x_0) \cdot f(x_0) + g(x_0) \cdot f'(x_0) \\
		\end{gathered}\]

		Если $g(x_0) \neq 0$, то

		\begin{equation*}
			\left(\dfrac{f}{g}\right)'(x_0) = \dfrac{g'(x_0) \cdot f(x_0) - g(x_0) \cdot f'(x_0)}{g(x_0)^2}
		\end{equation*}

	\defitem{Формула вычисления производной сложной функции.}

		Если $g(x)$ дифференциируема в точке $x_0$ и $f(x)$ дифференциируема в точке $y_0 = g(x_0)$, тогда,

		\begin{equation*}
			(f \circ g)'(x_0) = (f(g(x_0)))' = f'(g(x_0)) \cdot g'(x_0)
		\end{equation*}

	\defitem{Таблица производных основных элементарных функций.}

		{
			\everymath{\textstyle}
			\begin{figure}[H]
				\centering
					\begin{tabular}{| c | c |} \hline
					$f(x) $&$ f'(x) $\\\hline
					$const $&$ 0 $\\\hline
					$x^a $&$ a \cdot x^{a - 1} $\\\hline
					$a^x $&$ a^x \cdot \ln a $\\\hline
					$e^x $&$ e^x $\\\hline
					$\ln_a x $&$ \frac{1}{\ln a \cdot x} $\\\hline
					$\ln x $&$ \frac{1}{x} $\\\hline
					$\sin x $&$ \cos x $\\\hline
					$\cos x $&$ -\sin x $\\\hline
					$\tg x $&$ \frac{1}{\cos^2 x} $\\\hline
					$\ctg x $&$ -\frac{1}{\sin^2 x} $\\\hline
				\end{tabular}\quad\quad\quad
				\begin{tabular}{| c | c |} \hline
					$f(x) $&$ f'(x) $\\\hline
					$\arcsin x $&$ \frac{1}{\sqrt{1 - x^2}} $\\\hline
					$\arccos x $&$ -\frac{1}{\sqrt{1 - x^2}} $\\\hline
					$\arctg x $&$ \frac{1}{1 + x^2} $\\\hline
					$\arcctg x $&$ -\frac{1}{1 + x^2} $\\\hline
				\end{tabular}
			\end{figure}

		}


	\defitem{Понятие дифференциала (первого) функции в точке.}

		Функция $f(x)$ является дифференциируемой в точке $x_0$ своей области определения $D[f]$, если существует такая константа A, что:

		\[\begin{gathered}
			f(x) = f(x_0) + A(x - x_0) + \os(x - x_0) \\
			A = f'(x_0) = \lim_{\Delta \to 0} \dfrac{f(x_0 + \Delta) - f(x_0)}{\Delta} \\
		\end{gathered}\]

		Тогда выражение $f'(x_0) dx$ называют дифференциалом функции $f(x)$ в точке $x_0$. Обозначение: $df~=~df(x_0, dx)$. Обратите внимание, что $df$ зависит и от точки, и от $dx$.

	\defitem{Геометрический смысл дифференциала.}

		Дифференциал функции численно равен приращению ординаты касательной, проведенной к графику функции $y = f(x)$ в данной точке, когда аргумент $x$ получает приращение $\Delta x$.

		\href{https://lms2.sseu.ru/courses/eresmat/metod/met1/razdmet1_4/parmet1_4_2.htm}{Подробнее тут}

	\defitem{Определение локального экстремума. Необходимое условие для внутреннего локального экстремума (теорема Ферма).}

		Точка $x_0$ называется точкой локального максимума (минимума) функции $f$, если существует такая окрестность $U_{\delta} (x_0)$ точки $x_0$, что 

		\begin{equation*}
			\forall x \in U_{\delta}(x_0) \implies f(x) \leq f(x_0) \text{ (для минимума соответственно $f(x) \geq f(x_0)$)}
		\end{equation*}

		$x_0$ называется точкой строгого локального максимума (минимума), если

		\begin{equation*}
			\forall x \in \overset{\circ}{U_{\delta}} (x_0) \implies f(x) < f(x_0) \text{ (для минимума соответственно $f(x) > f(x_0)$)}
		\end{equation*}

		\begin{theorem*}[Ферма]
			Если функция имеет в точке локального экстремума производную, то эта производная равна нулю.
		\end{theorem*}

	\defitem{Формулы Лагранжа и Коши.}

		\begin{theorem*}[Лагранж: о конечных приращениях]
			
		\end{theorem*}

	\defitem{Многочлен Тейлора и формула Тейлора для функций одной переменной.}
	\defitem{Формулы Маклорена для основных элементарных функций.}

		\begin{enumerate}[leftmargin=*]
			\item
			$e^x = 1 + \dfrac{x}{1!} + \dfrac{x^2}{2!} + \dots + \dfrac{x^n}{n!} + \os(x^n), x \to 0$

			\item
			$\ln(1 + x) = x - \dfrac{x^2}{2} + \dfrac{x^3}{3} - \dots + (-1)^{n + 1} \cdot \dfrac{x^n}{n} + \os(x^n), x \to 0$

			\item
			$\underset{\alpha \in \RR}{(1 + x)^{\alpha}} = 1 + \sum_{k = 1}^{n} \dbinom{\alpha}{k} x^k + \os(x^n)$

			Например $(1 + x)^{\frac{1}{3}} - 1 = \dbinom{\frac{1}{3}}{1}x + \dbinom{\frac{1}{3}}{2} x^2 + \os(x^2) = \dfrac{1}{3} x + \dfrac{\frac{1}{3}(\frac{1}{3} - 1)}{2} x^2 + \os(x^2)$

			\item
			$\sin (x) = x - \dfrac{x^3}{3!} + \dfrac{x^5}{5!} - \dots + (-1)^{2n - 1} \cdot \dfrac{x^{2n - 1}}{(2n - 1)!} + \os(x^{2n - 1})$

			\item
			$\cos (x) = 1 - \dfrac{x^2}{2!} + \dfrac{x^4}{4!} + \dots + (-1)^{2n - 2} \dfrac{x^{2n - 2}}{(2n - 2)!} + \os(x^{2n - 2})$

			\item
			$\tg (x) = x + \dfrac{x^3}{3} + \dfrac{2}{15} x^5 + \dots + \dfrac{B_{2n}(-4)^n(1 - 4^n)}{(2n)!} \cdot x^{2n - 1} + \os(x^{2n - 1})$, где $B_{2n}$ --- числа Бернулли

			Но достаточно помнить, что
			$\tg (x) = x + \dfrac{x^3}{3} + \dfrac{2}{15}x^5 + \os(x^5)$, т.е. общая формула для семинаров \underline{не} нужна

			\item
			$\arcsin (x) = x + \dfrac{x^3}{6} + \dfrac{3}{40} x^5 + \dots + \dfrac{(2n)!}{4^n (n!)^2 (2n + 1)} \cdot x^{2n + 1} + \os(x^{2n + 1})$

			Достаточно знать $\arcsin (x) = x + \dfrac{x^3}{6} + \dfrac{3}{40} x^5 + \os(x^5)$

			\item
			$\arccos (x) = \dfrac{\pi}{2} - \arcsin (x)$

			\item
			$\arctg(x) = x - \dfrac{x^3}{3} + \dfrac{x^5}{5} + \dots + (-1)^{n + 1} \dfrac{x^{2n - 1}}{2n - 1} + \os(x^{2n - 1})$
		\end{enumerate}

	\defitem{Правило Лопиталя.}

    \end{colloq}

    \section{Вопросы на знание доказательств}
    \begin{colloq}
    \setlength\parindent{20pt}

    \proofitem{Определения непрерывности функции в точке, их эквивалентность. Точки разрыва, их классификация.}
	\proofitem{Непрерывность основных элементарных функций.}
	\proofitem{Арифметические свойства непрерывных функций.}
	\proofitem{Теорема о непрерывности сложной функции.}
	\proofitem{Свойства функций, непрерывных на отрезке (первая и вторая теоремы Вейерштрасса).}
	\proofitem{Теорема Коши о прохождении непрерывной функции через промежуточные значения.}
	\proofitem{Понятие производной функции в точке.}
	\proofitem{Геометрический и физический смысл производной.}
	\proofitem{Уравнение касательной к графику функции в точке.}
	\proofitem{Понятие дифференцируемости функции в точке.}
	\proofitem{Необходимое условие дифференцируемости.}
	\proofitem{Правила дифференцирования.}
	\proofitem{Теорема о дифференцируемости и производной сложной функции.}
	\proofitem{Теорема о дифференцируемости обратной функции.}
	\proofitem{Таблица производных основных элементарных функций.}
	\proofitem{Производные функций, графики которых заданы параметрически.}
	\proofitem{Понятие дифференциала (первого) функции в точке.}
	\proofitem{Геометрический смысл дифференциала.}
	\proofitem{Инвариантность формы первого дифференциала.}
	\proofitem{Производные и дифференциалы высших порядков функции одной переменной в точке.}
	\proofitem{Понятие об экстремумах функции одной переменной.}
	\proofitem{Локальный экстремум. Необходимое условие для внутреннего локального экстремума (теорема Ферма).}
	\proofitem{Основные теоремы о дифференцируемых функций на отрезке (теорема Ролля, формулы Лагранжа и Коши).}
	\proofitem{Многочлен Тейлора и формула Тейлора для функций одной переменной с остаточным членом в форме Пеано и Лагранжа.}
	\proofitem{Формулы Маклорена для основных элементарных функций.}
	\proofitem{Правило Лопиталя.}
	\proofitem{Достаточное условие строгого возрастания (убывания) функции на промежутке.}
	\proofitem{Достаточные условия локального экстремума для функции одной переменной.}
	\proofitem{Выпуклые (вогнутые) функции одной переменной.}
	\proofitem{Достаточные условия выпуклости (вогнутости).}
	\proofitem{Точки перегиба.}
	\proofitem{Необходимые и достаточные условия для точки перегиба.}
	\proofitem{Асимптоты графика функции одной переменной.}

    \end{colloq}

\end{document}