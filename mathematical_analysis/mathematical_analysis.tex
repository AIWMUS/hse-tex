\documentclass[12pt, a4paper]{article}

\usepackage{../header}

\title{Математический анализ}
\author{
	Игорь Балюк, БПМИ193 \\ 
	\href{https://teleg.run/lodthe}{@lodthe}, \href{https://github.com/LoDThe/hse-tex}{github}
}
\date{2019 --- 2020}

\begin{document}
	\maketitle
	\tableofcontents
	\newpage
	\href{https://github.com/LoDThe/hse-tex}{fwfwq}
	\section{Пределы}

	\subsection{Следствия первого замечательного предела}
	\begin{align*}
		&\lim_{x \to 0} \dfrac{\sin x}{x} = 1 \\
		&\lim_{x \to 0} \dfrac{x}{\sin x} = 1 \\
		&\lim_{x \to 0} \dfrac{1 - \cos x}{x^2} = \dfrac{1}{2} \\
		&\lim_{x \to 0} \dfrac{\tg x}{x} = 1 \\
		&\lim_{x \to 0} \dfrac{\arcsin x}{x} = 1 \\
		&\lim_{x \to 0} \dfrac{\arctg x}{x} = 1 \\
	\end{align*}

	\subsection{Следствия второго замечательного предела}
	\begin{align*}
		&\lim_{x \to \infty} \left(1 + \frac{1}{x}\right)^x = e \\
		&\lim_{x \to 0} \left(1 + x\right)^{1/x} = e \\
		&\lim_{x \to \infty} \left(1 + \frac{a}{x}\right)^{bx} = e^{ab} \\
		&\lim_{x \to 0} \frac{\ln (1 + x)}{x} = 1 \\
		&\lim_{x \to 0} \frac{e^x - 1}{x} = 1 \\
		&\lim_{x \to 0} \frac{a^x - 1}{x \ln a} = 1, a > 0, a \neq 1 \\
		&\lim_{x \to 0} \frac{(1 + x)^a - 1}{x} = a
	\end{align*}

	\subsection{Тригонометрические формулы}
	\href{https://mnogoformul.ru/vse-formuly-po-trigonometrii}{Сайт с многими остальными формулами}
	\begin{align*}
		&\sin 0 = 0, \cos 0 = 1 \\
		&\sin (x)^2 + \cos (x)^2 = 1 \\
		&\sin 2x = 2 \sin x \cos x \\
		&\cos 2x = \cos (x)^2 - \sin (x)^2 \\
		&\sin(a \pm b) = \sin a \cos b \pm \cos a \sin b \\
		&\cos(a \pm b) = \cos a \cos b \mp \sin a \sin b \\
		&\sin a + \sin b = 2 \sin \dfrac{a + b}{2} \cos \dfrac{a - b}{2} \\
		&\sin a - \sin b = 2 \cos \dfrac{a + b}{2} \sin \dfrac{a - b}{2} \quad (\sin (-a) = - \sin (a)) \\
		&\cos a + \cos b = 2 \cos \dfrac{a + b}{2} \cos \dfrac{a - b}{2} \\
		&\cos a - \cos b = -2 \sin \dfrac{a + b}{2} \sin \dfrac{a - b}{2} \\
		&\sin a \cdot \sin b = \dfrac{1}{2} \cdot (\cos (a - b) - \cos (a + b)) \\
		&\sin a \cdot \cos b = \dfrac{1}{2} \cdot (\sin (a - b) + \sin (a + b)) \\
		&\cos a \cdot \cos b = \dfrac{1}{2} \cdot (\cos (a - b) + \cos (a + b)) \\
		&\cos a = 1 - 2 \sin^2 \left(\dfrac{a}{2}\right) \\
	\end{align*}

	\section{Коллоквиум 1}
	\href{https://docs.google.com/document/d/1e-29abaRHAqDMcxA154fgHdsHKb3Npta7beZLSCfcRs/edit?usp=sharing}{Информация о коллоквиуме}

	Ориентировочная дата проведения: 09.11.2019

	\subsection{В обязательный минимум входят}
	\subsubsection{Определение предела числовой последовательности}
	Число $a$ называется пределом числовой последовательности $\{x_n\}$, если
	\begin{equation*}
		\forall \eps > 0, \exists N(\eps): \forall n \geq N(\eps) \implies |x_n - a| < \eps
	\end{equation*}

	\subsubsection{Определение точной верхней и нижней грани}
	Верхняя (нижняя) грань числового множества $X$ --- число $a$ такое, что $\forall x \in X \implies x \leq (\geq) a$

	Точная верхняя грань (или супремум) ---	это наименьшая из всех верхних граней. Обозначается $\sup X$.

	Точная нижняя грань (или инфинум) --- это наибольшая из всех нижних граней. Обозначается $\inf X$.

	\[\begin{gathered}
		a = \sup X \iff (\forall x \in X \implies x \leq a) \land (\nexists b: \;b < a, \forall x \in X \implies x \leq b) \\
		a = \inf X \iff (\forall x \in X \implies x \geq a) \land (\nexists b: \;b > a, \forall x \in X \implies x \geq b) \\
	\end{gathered}\]

	\subsubsection{Определение бесконечно малой и бесконечно большой последовательности}
	Последовательность $\{x_n\}$ называется 
	\begin{itemize}
		\item
		бесконечно малой последовательностью, если 

		$\forall \eps > 0 \; \exists N(\eps): \; \forall n \geq N(\eps) \implies |x_n| < \eps$

		\item
		бесконечно большой последовательностью, если 

		$\forall A > 0 \; \exists N(A): \; \forall n \geq N(A) \implies |x_n| > A$
	\end{itemize}

	\subsubsection{Определение предела функции в точке и на бесконечности по Коши и по Гейне}
	\begin{itemize}
		\item
		По Коши: $A$ --- предел функции $f(x)$ в точке $a$ ($\lim_{x \to a} f(x) = A$), если 

		$\forall \eps > 0 \; \exists \delta > 0: \forall x: \; 0 < |x - a| < \delta \implies |f(x) - A| < \eps$

		\item
		По Гейне: $A$ называется пределом функции $f(x)$ в точке $a$, если $\forall \{x_n\} \to a, x_n \neq a$ (т.е. $\lim_{n \to \infty} x_n = a$), соответствующая последовательность значений $f(x_n) \to A$ (т.е. $\lim_{n \to \infty} f(x_n) = A$)
	\end{itemize}

	\subsubsection{Определение фундаментальной последовательности, критерий Коши сходимости последовательности}
	\underline{Критерий Коши}: Для того, чтобы последовательность $\{x_n\}$ сходилась, необходимо и достаточно, чтобы она была фундаментальной.

	Последовательность называется фундаментальной, если 

	$\forall \eps > 0 \; \exists N(\eps): \forall n, m > N(\eps): |x_n - x_m| < \eps$

	\subsubsection{Первый и второй замечательный пределы}
	\begin{enumerate}
		\item
		\begin{equation*}
			\lim_{x \to 0} \dfrac{\sin x}{x} = 1
		\end{equation*}

		\item
		\begin{equation*}
			\lim_{x \to \infty} \left(1 + \dfrac{1}{x}\right)^x = e
		\end{equation*}
	\end{enumerate}

	\subsubsection{Таблица производных элементарных функций}
	\begin{align*}
		&(C)' = 0 \\
		&(x^a)' = a \cdot x^{a - 1} \\
		&(a^x)' = a^x \ln a \\
		&(e^x)' = e^x \\ 
		&(\ln x)' = \dfrac{1}{x} \\
		&(\sin x)' = \cos x \\
		&(\cos x)' = - \sin x \\
		&(\tg x)' = \dfrac{1}{\cos^2 x} \\
		&(\ctg x)' = -\dfrac{1}{\sin^2 x} \\
	\end{align*}

	\subsection{Основные понятие и теоремы (с доказательствами)}
	\subsubsection{Числовые последовательности. Примеры.}
	\underline{Определение из википедии}: Пусть $X$ --- это либо множество вещественных чисел $\RR$, либо множество комплексных чисел $\CC$. Тогда последовательность $\{x_n\}_{n = 1}^{\infty}$ элементов множества $X$ называется \textbf{числовой последовательностью}.

	\underline{Определение из Ёжика}: Отображение $\NN \mapsto X$ будем называть последовательностью и записывать как $x_1, x_2, \dots, x_n$. Отображение $\NN \mapsto \RR$ будем называть \textbf{числовой последовательностью}.

	Примеры:
	\begin{itemize}
		\item
		Функция, сопоставляющая каждому натуральному числу $n \leq 12$ одно из слов «январь», «февраль», «март», «апрель», «май», «июнь», «июль», «август», «сентябрь», «октябрь», «ноябрь», «декабрь» (в порядке их следования здесь) представляет собой последовательность вида $\{x_n\}_{n=1}^{12}$. Например, пятым элементом $x_5$ этой последовательности является слово «май».

		\item
		$\{1/n\}_{n=1}^{\infty}$ является бесконечной последовательностью рациональных чисел. Элементы этой последовательности начиная с первого имеют вид $1,1/2,1/3,1/4,1/5,\ldots$.

		\item
		$\left((-1)^{n}\right)_{{n=1}}^{{\infty }}$ является бесконечной последовательностью целых чисел. Элементы этой последовательности начиная с первого имеют вид $-1,1,-1,1,-1,\ldots$.
	\end{itemize}

	\subsubsection{Понятие предела последовательности.}
	Число $a$ называется пределом числовой последовательности $\{x_n\}$, если
	\begin{equation*}
		\forall \eps > 0, \exists N(\eps): \forall n \geq N(\eps) \implies |x_n - a| < \eps
	\end{equation*}

	\subsubsection{Ограниченные и неограниченные последовательности.} 
	\begin{itemize}
		\item
		Ограниченная сверху последовательность — это последовательность элементов множества $\RR$, все члены которой не превышают некоторого элемента из этого множества. Этот элемент называется верхней гранью данной последовательности (говоря в общем, это верно и не только для $\RR$).
		\begin{equation*}
			\{x_n\} \text{ ограниченная сверху } \iff \exists M \in \RR: \; \forall n \implies x_n \leq M
		\end{equation*}

		\item
		Ограниченная снизу последовательность --- это последовательность элементов множества $X$, для которой в этом множестве найдётся элемент, не превышающий всех её членов. Этот элемент называется нижней гранью данной последовательности.
		\begin{equation*}
			\{x_n\} \text{ ограниченная снизу } \iff \exists m \in \RR: \; \forall n \implies x_n \geq m
		\end{equation*}

		\item
		Ограниченная последовательность (ограниченная с обеих сторон последовательность) --- это последовательность, ограниченная и сверху, и снизу.
		\begin{equation*}
			\{x_n\} \text{ ограниченная } \iff \exists M, m \in \RR: \; \forall n \implies m \leq x_n \leq M
		\end{equation*}

		\item
		Неограниченная последовательность — это последовательность, которая не является ограниченной.
		\begin{equation*}
			\{x_n\} \text{ неограниченная } \iff \forall M, m \in \RR: \; \exists N \implies (x_N < m) \lor (x_N > M)
		\end{equation*}

		\item
		\underline{Критерий ограниченности}: Числовая последовательность является ограниченной тогда и только тогда, когда существует такое число, что модули всех членов последовательности не превышают его.
		\begin{equation*}
			\{x_n\} \text{ ограниченная } \iff \exists A \in \RR: \; \forall N \implies |x_N| \leq A
		\end{equation*}
	\end{itemize}

	\subsubsection{Теорема об ограниченности сходящейся последовательности.}
	Всякая сходящаяся последовательность ограничена.
	\begin{proof}
		Все члены последовательности, кроме конечного их числа, принадлежат окрестности предела --- ограниченному множеству.

		Пусть последовательность $\{x_n\}$ сходится к $a$, т.е. $\lim\limits_{n \to \infty} x_n = a$. 
		\begin{equation*}
			\forall \eps > 0 \; \exists N: \; \forall n \geq N \implies |x_n - a| < \eps
		\end{equation*}
		Пусть $\eps = 1$, тогда $A = \max\{|x_1|, \dots, |x_N|, |a - \eps|, |a + \eps|\}$. Тогда, $\forall n \in \NN: \; |x_n| \leq A$.
	\end{proof}
		
	\subsubsection{Теорема о единственности предела сходящейся последовательности.}
	\begin{theorem*}
		Если предел числовой последовательности существует, то он единственный.
	\end{theorem*}
	\begin{proof}
		 Доказательство теоремы проведем «методом от противного». Предположим, что теорема неверна. Тогда, пусть $\lim_{n \to \infty} x_n = a = b$ и выполняется следующее:
		 \begin{equation*}
		 	\begin{cases}
		 		a < b, \\
		 		\forall \eps > 0, \exists N_1(\eps): \forall n \geq N_1(\eps) \implies |x_n - a| < \eps, \\
		 		\forall \eps > 0, \exists N_2(\eps): \forall n \geq N_2(\eps) \implies |x_n - b| < \eps, \\
		 	\end{cases}
		 \end{equation*}
		 Положим $\eps = \dfrac{b - a}{2}$ и $N = max\{N_1(\eps), N_2(\eps)\}$. Тогда, $\forall n \geq N \implies  |x_n - a| < \eps \; \land \; |x_n - b| < \eps$. Возьмём $n \geq N$, тогда,
		 \begin{equation*}
		 	b - a = |b - a| = |b - x_n + x_n - a| \leq |x_n - b| + |x_n - a| < \dfrac{b - a}{2} + \dfrac{b - a}{2} = b - a
		 \end{equation*}
		 Пришли к противоречию ($b - a < b - a$).
	\end{proof}

	\subsubsection{Теорема о переходе к пределу в неравенствах.}
	\begin{theorem*}
		Пусть $\lim\limits_{n \to \infty} x_n = a, \lim\limits_{n \to \infty} y_n = b$. Если $a < b$, то
		$\exists N: \; \forall n \geq N \implies x_n < y_n$.
	\end{theorem*}
	\begin{proof}
		Из аксиомы полноты $\exists c: \; a < c < b$. 
		\[\begin{gathered}
			\forall \eps_1 > 0 \; \exists N_1: \; \forall n \geq N_1 \implies |x_n - a| < \eps_1 \\
			\forall \eps_2 > 0 \; \exists N_2: \; \forall n \geq N_2 \implies |y_n - b| < \eps_2 \\
		\end{gathered}\]
		Тогда, $\forall n \geq N_1 \implies |x_n - a| < c - a$ и $\forall n \geq N_2 \implies |y_n - b| < b - c$. Отсюда $\forall n \geq \max \{N_1, N_2\} \implies x_n < c - a + a = c = c - b + b < y_n$.
	\end{proof}

	\subsubsection{Теорема о вынужденном пределе (Теорема о двух милиционерах).}
	\begin{theorem*}
		Если $\forall n \in \NN: x_n \leq y_n \leq z_n$ и $\exists \lim\limits_{n \to \infty} x_n = a = \lim\limits_{n \to \infty} z_n$, тогда $\lim\limits_{n \to \infty} y_n = a$.
	\end{theorem*}
	\begin{proof}
		Из определения предела $\{x_n\}$, $\forall \eps > 0 \; \exists N_1: \; \forall n \geq N_1 \implies |x_n - a| < \eps \iff a - \eps < x_n < a + \eps$. Аналогично для предела $\{z_n\}$, $\forall \eps > 0 \; \exists N_2: \; \forall n \geq N_2 \implies |z_n - a| < \eps \iff a - \eps < z_n < a + \eps$. 

		Тогда, 
		$\forall n \geq \max \{N_1, N_2\} \implies a - \eps < x_n \leq y_n \leq z_n < a + \eps \implies \lim\limits_{n \to \infty} y_n = a$.
	\end{proof}

	\subsubsection{Теорема о сходимости монотонных ограниченных последовательностей.}
	\begin{theorem*}
		Неубывающая числовая последовательность имеет предел, причём он в точности равен точной верхней границе (нижней границе, для ограниченной невозрастающей ч.п.).
	\end{theorem*}
	\begin{proof}
		Пусть $\{x_n\}$ --- ограниченная неубывающая числовая последовательность. Тогда множество $\{x_n\}_{n \in \NN}$ ограничено, следовательно, из определения супремума, имеет супремум. Обозначим его через $S$. Тогда $\lim\limits_{n \to \infty} x_n = S$. Действительно, так как $S = sup \{x_n\}_{n \in \NN}$, то
		\begin{equation*}
			\forall \eps > 0 \; \exists N: \; \forall n \geq N \implies S - \eps < x_N \leq x_n \leq S \implies |x_n - S| < \eps
		\end{equation*}
	\end{proof}
	Аналогичное доказательство для ограниченной невозрастающей ч.п.

	\subsubsection{Определение числа е.}
	\begin{equation*}
		e = \sum_{n = 0}^{\infty} \dfrac{1}{n!}
	\end{equation*}
	\begin{theorem*}
		Последовательность с общим членом $e_n = \left(1 + \dfrac{1}{n}\right)^n$ имеет конечный предел при $n \to \infty$. Для обозначение этого предела используется символ $e$.
	\end{theorem*}
	\begin{proof}
		Докажем сначала, что $\{e_n\}$ представляет собой монотонно возрастающую последовательность. Согласно биному Ньютона,
		\begin{align*}
			e_n = \left(1 + \frac{1}{n}\right)^n 
			&= 1 + n \cdot \frac{1}{n} + \frac{n (n - 1)}{2!} \cdot \frac{1}{n^2} + \frac{n(n - 1)(n - 2)}{3!} \cdot \frac{1}{n^3} + \dots + \frac{1}{n^n} \\
			&= 2 + \frac{1}{2!} \cdot \left(1 - \frac{1}{n}\right) + \frac{1}{3!} \cdot \left(1 - \frac{1}{n}\right) \cdot \left(1 - \frac{2}{n}\right) + \frac{1}{n!} \cdot \left(1 - \frac{1}{n}\right) \dots \left(1 - \frac{1}{n - 1}\right)
		\end{align*}
		\begin{align*}
			e_{n + 1} &= \left(1 + \frac{1}{n + 1}\right)^{n + 1} \\
			&= 2 + \frac{1}{2!} \cdot \left(1 - \frac{1}{n + 1}\right) + \frac{1}{3!} \cdot \left(1 - \frac{1}{n + 1}\right) \cdot \left(1 - \frac{2}{n + 1}\right) + \dots
		\end{align*}
		Сравним $e_n$ и $e_{n + 1}$:
		\begin{itemize}
			\item
			Оба выражения содержат только положительные слагаемые

			\item
			Начиная со второго слогаемого, каждый член в выражении $e_{n + 1}$ превышает соответствующий член в $e_n$, так как
			\begin{equation*}
				\left(1 - \frac{1}{n}\right) < \left(1 - \frac{1}{n + 1}\right),
				\left(1 - \frac{2}{n}\right) < \left(1 - \frac{2}{n + 1}\right) \dots
			\end{equation*}

			\item
			Выражение $e_{n + 1}$ состоит из большего числа слагаемых. Следовательно, $e_{n + 1} > e_n$.
		\end{itemize}

		Далее докажем, что последовательность $\{e_n\}$ является ограниченной. Действительно, первый член любой монотонно возрастающей последовательности является ее наибольшей нижней границей и, таким образом, $e_n \geq 2 \; \forall n \in \NN$.

		Перейдем к доказательству существования верхней границы. Очевидно, что
		\begin{align*}
			e_n 
			&= 2 + \frac{1}{2!} \cdot \left(1 - \frac{1}{n}\right) + \frac{1}{3!} \cdot \left(1 - \frac{1}{n}\right) \cdot \left(1 - \frac{2}{n}\right) + \frac{1}{n!} \cdot \left(1 - \frac{1}{n}\right) \dots \left(1 - \frac{1}{n - 1}\right) <\\
			&< 2 + \frac{1}{2!} + \frac{1}{3!} + \frac{1}{4!} + \frac{1}{5!} + \dots + \frac{1}{n!}
		\end{align*}

		Кроме того, $\dfrac{1}{k!} < \dfrac{1}{2^k} \; \forall k > 3$. Тогда,
		\begin{equation*}
			\frac{1}{4!} + \frac{1}{5!} + \dots + \frac{1}{n!} < 
			\frac{1}{2^4} + \frac{1}{2^5} + \dots + \frac{1}{2^n}
		\end{equation*}
		Правая часть этого неравенства представляет собой сумму убывающей геометрической прогрессии, которая равна $\dfrac{\frac{1}{16}}{1 - \frac{1}{2}} = \dfrac{1}{8}$. Таким образом, последовательность
		\begin{equation*}
			e_n < 2 + \frac{1}{2!} + \frac{1}{3!} + \frac{1}{4!} + \frac{1}{5!} + \dots + \frac{1}{n!}
			< 2 + \frac{1}{2} + \frac{1}{6} + \frac{1}{8} < 3
		\end{equation*}
		представляет собой ограниченную монотонно возрастающую последовательность и, следовательно, имеет конечный предел.
	\end{proof}

	\subsubsection{Бесконечно малые последовательности.}
	Последовательность $a_n$ называется бесконечно малой, если $\lim\limits_{n \to \infty} a_n = 0$.

	\subsubsection{Связь со сходящимися последовательностями.}
	Если предел последовательности равен 0, то это бесконечно малая последовательность. Бесконечно малые последовательности являются сходящимися последовательностями.

	Для того чтобы последовательность $\{x_n\}$ имела предел $b$, необходимо и достаточно, чтобы
	$x_n = b + \alpha_n$, где $\alpha_n$ --- бесконечно малая последовательность.

	\subsubsection{Арифметические свойства бесконечно малых и сходящихся последовательностей.} 
	Пусть $\{\alpha_n\}$ --- бесконечно малая числовая последвательность.
	\begin{itemize}
		\item
		$\{\alpha_n\}$ ограничена
		\begin{proof}
			Как известно, $\forall \eps > 0 \; \exists N: \; \forall n > N \implies |\alpha_n| < \eps$. Значит, для всех $n > N$ доказано. Но $\forall n < N \implies \alpha_n \leq \max\{|\alpha_1|, |\alpha_2|, \dots, |\alpha_{N - 1}|\}$. Тогда выберем $\eps = 1, A = \max\{|\alpha_1|, |\alpha_2|, \dots, |\alpha_{N - 1}|, 1\} \implies \forall n \in \NN, |\alpha_n| \leq A$.
		\end{proof}

		\item
 		Если $\{y_n\}$ ограничена, то $\{y_n \cdot \alpha_n\}$ --- бесконечно малая.
 		\begin{proof}
 			$\{\alpha_n\}$ --- бесконечно малая, поэтому $\forall \eps > 0 \; \exists N: \; \forall n \geq N \implies |\alpha_n| < \dfrac{\eps}{A}$. Ввиду ограниченности $\{y_n\}, \exists A: \; \forall n \in \NN \implies |y_n| \leq A$. Но тогда $\{y_n \cdot \alpha_n\}: \forall \eps > 0 \; \exists N: \; \forall n \geq N \implies |y_n \cdot \alpha_n| < \dfrac{\eps}{A} \cdot A = \eps$.
 		\end{proof}

 		\item
 		Если $\{\beta_n\}$ --- бесконечно малая, то $\{\alpha_n \pm \beta_n\}$ и $\{\alpha_n \cdot \beta_n\}$ --- бесконечно малые.
 		\begin{proof}
 			\[\begin{gathered}
 				\forall \eps > 0 \; \exists N_1: \forall n \geq N_1 \implies |\alpha_n| < \frac{\eps}{2}
 				\text{ и }
 				\forall \eps > 0 \; \exists N_2: \forall n \geq N_2 \implies |\beta_n| < \frac{\eps}{2} \\
 				\text{Тогда при } N = \max\{N_1, N_2\} \implies \forall n \geq N \implies
 				|\alpha_n \pm \beta_n| \leq |\alpha_n| + |\beta_n| < \frac{\eps}{2} + \frac{\eps}{2} = \eps
 			\end{gathered}\]
 			Аналогично для произведения:
 			\[\begin{gathered}
 				\forall \eps > 0 \; \exists N_1: \forall n \geq N_1 \implies |\alpha_n| < \frac{1}{\eps}
 				\text{ и }
 				\forall \eps > 0 \; \exists N_2: \forall n \geq N_2 \implies |\beta_n| < \eps^2 \\
 				\text{Тогда при } N = \max\{N_1, N_2\} \implies \forall n \geq N \implies
 				|\alpha_n \cdot \beta_n| \leq |\alpha_n| \cdot |\beta_n| < \frac{1}{\eps} \cdot \eps^2 = \eps
 			\end{gathered}\]
 		\end{proof}
	\end{itemize}

	\subsubsection{Бесконечно большие последовательности, их связь с бесконечно малыми.}
	\begin{itemize}
		\item
		Если $\{x_n\}$ --- бесконечно малая и $\forall n \in \NN \implies x_n \neq 0$, то $\left\{\dfrac{1}{x_n}\right\}$ --- бесконечно большая.
		\begin{proof}
			$\forall \eps > 0 \; \exists N: \; \forall n \geq N \implies |x_n| < \eps \iff \dfrac{1}{|x_n|} < \dfrac{1}{\eps} = A$
		\end{proof}

		\item
		Если $\{x_n\}$ --- бесконечно большая и $\forall n \in \NN \implies x_n \neq 0$, то $\left\{\dfrac{1}{x_n}\right\}$ --- бесконечно малая.
		\begin{proof}
			$\forall A > 0 \; \exists N: \; \forall n \geq N \implies |x_n| > A \iff \dfrac{1}{|x_n|} < \dfrac{1}{A} = \eps$
		\end{proof}
	\end{itemize}

	\subsubsection{Арифметические свойства для последовательностей, имеющих конечные и бесконечные пределы.} 
	Если $\exists \lim\limits_{n \to \infty} x_n = a, \lim\limits_{n \to \infty} y_n = b$, то $\exists \lim\limits_{n \to \infty} (x_n \pm y_n) = a \pm b, \lim\limits_{n \to \infty} (x_n \cdot y_n) = a \cdot b$, а также $\lim\limits_{n \to \infty} \dfrac{x_n}{y_n} = \dfrac{a}{b}$, если $b \neq 0$.
	\begin{proof}
		\[\begin{gathered}
			\lim_{n \to \infty} x_n = a, \lim_{n \to \infty} y_n \iff x_n = a + \alpha_n, y_n = b + \beta_n, \text{ где } \{\alpha_n\} \{\beta_n\} \text{--- бесконечно малые.} \\
			x_n \pm y_n = (a + \alpha_n) \pm (b + \beta_n) = (a \pm b) + \underbrace{(\alpha_n \pm \beta_n)}_{\text{б. м.}} \\
			x_n \cdot y_n = (a + \alpha_n) \cdot (b + \beta_n) = a \cdot b + \underbrace{(\alpha_n \cdot \beta_n + \alpha_n \cdot b + \beta_n \cdot a)}_{\text{б.м.}}
		\end{gathered}\]
		\begin{lemma}
			Пусть $\exists \lim\limits_{n \to \infty} y_n \neq 0$. Тогда $\exists r > 0: \exists N \in \NN: \forall n \geq N \implies |y_n| > r > 0$.
		\end{lemma}
		\begin{proof}
			$\forall \eps > 0 \exists N: \forall n \geq N \implies |y_n - b| < \eps \iff b - \eps < |y_n| < b + \eps$. Пусть $\eps = \dfrac{b}{2}$, тогда $r < \dfrac{b}{2} < y_n < \dfrac{3b}{2}$.
		\end{proof}
		Рассмотрим последовательность $\{\dfrac{x_n}{y_n} - \dfrac{a}{b}\}$ --- бесконечно малая.
		\[\begin{gathered}
			\frac{x_n}{y_n} - \frac{a}{b} = \frac{a + \alpha_n}{b + \beta_n} - \frac{a}{b}
			= \frac{b \cdot a + b \cdot \alpha_n - b \cdot a - \beta_n \cdot a}{y_n \cdot b}
			= (\alpha_n \cdot b - \beta_b \cdot a) \cdot \frac{1}{y_n \cdot b}
		\end{gathered}\]
		По лемме $\left|\dfrac{1}{y_n \cdot b}\right| \leq max\left\{\left|\dfrac{1}{y_1 \cdot b}\right|, \dots, \left|\dfrac{1}{y_N \cdot b}\right|, \dfrac{1}{rb} \right\} \implies \left\{\dfrac{1}{y_n \cdot b}\right\}$ ограничена. Но тогда имеем произведение бесконечно малой и ограниченной последовательностей, значит, $\{\dfrac{x_n}{y_n} - \dfrac{a}{b}\}$ --- бесконечно малая.
	\end{proof}

	\subsubsection{Неопределенности.} 
	\emph{Не очень понятно, что именно требуется в этом пункте}
	
	Основные виды неопределенностей: $\dfrac{0}{0}, \dfrac{\infty}{\infty}, 0 \cdot \infty, \infty - \infty, 1^{\infty}, 0^0, \infty^0$

	Раскрывать неопределенность помогает:
	\begin{itemize}
		\item
		упрощение вида функции (преобразование выражения с использованием формул сокращенного умножения, 

		\item
		тригонометрических формул, домножением на сопряженные выражения с последующим сокращением и т.п.);
		использование замечательных пределов;
	\end{itemize}	

	\subsubsection{Определение подпоследовательности.} 
	Подпоследоследовательность последовательности $\{x_n\}$ --- это последовательность $\{x_{n_k}\} = \{x_{n_1}, x_{n_2}, \dots, x_{n_k}\}$, полученная из $\{x_n\}$, удалением ряда её членов без изменения порядка следования членов. 

	То есть подпоследовательность состоит из членов исходной последовательности $\{x_n\}$ с номерами $n_k$, где $\{n_k\}$ --- строго монотонная последовательность натуральных чисел. 

	\begin{remark}
		Если $\lim\limits_{n \to \infty} a_n = a$, тогда $\forall \{a_{n_k}\}: \lim\limits_{k \to \infty} a_{n_k} = a$
	\end{remark}

	\subsubsection{Теорема Больц\'{а}но-В\'{е}йерштрасса.} \label{sss:bolzano}
	\begin{theorem*}
		Из любой ограниченной последовательности можно выделить сходящуюся подпоследовательность.
	\end{theorem*}
	\begin{proof}
		Для понимания происходящего следует ознакомиться с \ref{sss:segments} Системой стягивающихся отрезков (ССО)

		$\{x_n\}$ ограничена $\implies \exists [a, b]: \forall n \in N \implies a \leq x_n \leq b$. Поделим $[a; b]$ на две равные части. Хотя бы одна из частей (пусть это $[a_1; b_1]$) содержит бесконечно много элементов $\{x_n\}$.

		Выберем на $[a_1; b_1]$ произвольный элемент $\{x_n\}$. Назовем его $x_{n_1}$. Далее делим $[a_1; b_1]$ на
		две равные части. Хотя бы одна из этих частей содержит бесконечно много элементов $\{x_n\}$.
		Обозначим ее $[a_2; b_2]$. Выберем $x_{n_2} \in [a_2; b_2]$. Будем продолжать выполнять указанные действия.
		Обозначим за $x_{n_k}$ число, полученное на $k$-ом шаге, т.е. $x_{n_k} \in [a_k; b_k]$.

		$\{[a_k; b_k]\}$ --- система стягивающихся отрезков. Тогда, существует единственное $c: \forall k \implies c \in [a_k; b_k]$.

		$\lim\limits_{k \to \infty} a_k = \lim\limits_{k \to \infty} b_k = c \implies \exists \lim\limits_{k \to \infty} x_{n_k} = c$ (по теореме о двух милиционерах)
	\end{proof}

	\subsubsection{Критерий Коши сходимости последовательности.} 
	\underline{Критерий Коши}: Для того, чтобы последовательность $\{x_n\}$ сходилась, необходимо и достаточно, чтобы она была фундаментальной.

	Последовательность называется фундаментальной, если 

	$\forall \eps > 0 \; \exists N: \forall n, m \geq N: |x_n - x_m| < \eps$
	\begin{proof}
		Докажем необходимость и достаточность.
		\begin{itemize}
			\item
			\underline{Необходимость}:

			Пусть $\lim\limits_{n \to \infty} x_n = a$ по определению:
			\begin{equation*}
				\forall \eps > 0: \; \exists N: \; \forall p \geq N \implies |x_p - a| < \eps
			\end{equation*}
			Поскольку $\eps$ произвольное, можно взять вместо него $\dfrac{\eps}{2}$
			\[\begin{gathered}
				p = m \geq N \implies |x_m - a| < \frac{\eps}{2} \\
				p = n \geq N \implies |x_n - a| < \frac{\eps}{2} \\
				|x_n - x_m| = |x_n - a + a - x_m| \leq |x_n - a| + |a - x_m| < \frac{\eps}{2} + \frac{\eps}{2} = \eps
			\end{gathered}\]
			То есть $|x_n - x_m| < \eps$, а значит $\{x_n\}$ фундаментальная по определению. Необходимость доказана

			\item
			\underline{Достаточность}:

			Пусть $\{x_n\}$ --- фундаментальная последовательность, докажем, что она имеет предел. Сначала покажем, что $\{x_n\}$ --- ограничена. По определению фундаментальной последовательности
			\begin{equation*}
				\forall \eps > 0 \; \exists N: \forall n, m \geq N: |x_n - x_m| < \eps
			\end{equation*}
			Так как $\eps$ произвольное, возьмём $\eps = 1$. 
			\[\begin{gathered}
				|x_n| = |(x_n - x_N) + x_N| \leq \underbrace{|x_n - x_N|}_{\leq \eps} + |x_N| \leq 1 + |x_N| \\
				\forall n \geq N \implies |x_n| \leq (1 + |x_N|) \; = const = A \implies |x_n| \leq A \\
				A = \max\{1 + |x_N|; |x_1|; |x_2|; \dots; |x_N|\} \\
				\forall n \geq N \implies |x_n| \leq A
			\end{gathered}\]
			По теореме \ref{sss:bolzano} Больцано-Вейерштрасса, так как $\{x_n\}$ --- ограниченная, $\{x_n\}$ имеет сходящуюся подпоследовательность $\{x_{n_k}\}$.

			Пусть $\lim\limits_{k \to \infty} x_{n_k} = a$, покажем, что число $a$ и будет пределом всей последовательности $\{x_n\}$.

			Так как $\{x_n\}$ фундаментальная:
			\begin{equation*}
				\forall \eps > 0: \; \exists N_1: \; \forall p \geq N_1 \implies |x_p - a| < \frac{\eps}{2}
			\end{equation*}
			Так как $\{x_{n_k}\}$ сходящаяся:
			\[\begin{gathered}
				\lim_{k \to \infty} x_{n_k} = a: \; \forall \eps > 0 \; \exists N_2: \; \forall n_k \geq n_{N_2} \implies |x_{n_k} - a| < \frac{\eps}{2} \\
				\forall \eps > 0: \; |x_n - a| = |(x_n - x_{n_k}) + (x_{n_k} - a)| \leq |x_n - x_{n_k}| + |x_{n_k} - a| < \frac{\eps}{2} + \frac{\eps}{2} = \eps \\
				\text{Возьмём } N = \max\{N_1, N_2\}: \; \forall \eps > 0 \; \exists N: \; \forall n \geq N \implies |x_n - a| < \eps
			\end{gathered}\]
			Достаточность доказана.
		\end{itemize}
	\end{proof}

	\subsubsection{Определение предела функции в точке по Коши и по Гейне.} 
	\begin{itemize}
		\item
		\underline{По Коши (или на языке $\eps - \delta$)}: \\$A$ --- предел функции $f(x)$ в точке $a$ ($\lim_{x \to a} f(x) = A$), если 

		$\forall \eps > 0 \; \exists \delta > 0: \forall x: \; 0 < |x - a| < \delta \implies |f(x) - A| < \eps$

		\item
		\underline{По Гейне}: \\$A$ называется пределом функции $f(x)$ в точке $a$, если $\forall \{x_n\} \to a, x_n \neq a$ (т.е. $\lim_{n \to \infty} x_n = a$), соответствующая последовательность значений $f(x_n) \to A$ (т.е. $\lim_{n \to \infty} f(x_n) = A$)
	\end{itemize}

	\subsubsection{Теорема об эквивалентности этих определений.} 
	\begin{itemize}
		\item
		Из определения по Коши следует определение по Гейне:

		Выберем произвольную $\{x_n\} \to a, x_n \neq a$. По определению предела последовательности 
		\begin{equation*}
			\forall \delta > 0 \; \exists N: \; \forall n \geq N \implies |x_n - a| < \delta
		\end{equation*}
		Указанное неравенство выполняетс ядля любого $\delta > 0$. Тогда какое бы $\eps > 0$ мы бы ни выбрали, можно найти $\delta > 0$, такое, что по определению по Коши будет выполняться
		\begin{equation*}
			\forall x: \; 0 < |x - a| < \delta \implies |f(x) - A| < \eps
		\end{equation*}
		т.е. $\{f(x_n)\} \to A$, а значит из сходимости по Коши следует сходимость по Гейне.

		\item
		Из определения по Гейне следует определение по Коши:

		Пусть $\lim\limits_{n \to \infty} f(x_n) = A$ По Гейне. От противного: если $\lim\limits_{x \to a} f(x) = A$ по Гейне, то $\lim\limits_{x \to a} f(x) \neq A$ по Коши. Напишем отрицание определения по Коши:
		\begin{equation*}
			\exists \eps_0: \; \forall \delta > 0: \; \exists x: \; 0 < |x - a| < \delta: \; |f(x) - A| \geq \eps_0
		\end{equation*}
		Так как $\delta$ может быть любым, можно выбрать последовательность $\{\delta_n\} = \{\dfrac{1}{n}\}$, а соответствующие значения $x$ будем обозначать как $x_n$. Тогда $0 < |x_n - a| < \delta_n = \dfrac{1}{n}$, и $|f(x_n) - A| \geq \eps_0$. Отсюда следует, что последовательность $\{x_n\}$ является подходящей, но при этом число $A$ не является пределом функции $f(x)$ в точке $a$ (по Гейне). Пришли к противоречию.
	\end{itemize}

	\subsubsection{Односторонние пределы, их связь с двусторонними. Пределы функции в бесконечности.}
	Назовём число $A$ левым (правым) пределом $f$ по Коши, если:
	\begin{equation*}
		\forall \eps > 0 \; \exists \delta > 0: \; \forall x \in (a - \delta; a) (x \in (a; a + \delta)) \implies |f(x) - a| < A
	\end{equation*}

	Назовём число $A$ левым (правым) пределом $f$ по Гейне, если:
	\begin{equation*}
		\forall \{x_n\}: \; \forall n \in \NN, x_n \neq a, x_n < a \; (x_n > a) \text{ и } 
		\lim_{n \to \infty} x_n = a \implies \{f(x_n)\} \xrightarrow[n \to \infty]{} A
	\end{equation*}

	Обозначим односторонние пределы так: $\lim\limits_{x \to a - 0} f(x) = A = f(a - 0)$ и 
	$\lim\limits_{x \to a + 0} f(x) = A = f(a + 0)$. 
	Таким образом, когда мы можем «подойти» к предельному значению функции, двигаясь по
	$x$ к точке $a$ слева, говорят, что существует левый предел. Аналогично следует понимать и
	определение правого предела. Поэтому если мы можем подойти к $a$ и слева, и справа, то
	существует предел в точке $a$. В кванторах это значит следующее:
	\[\begin{gathered}
		\exists \lim_{x \to a} f(x) = A \iff \exists f(a - 0) = f(a + 0) = A \\
		(\text\{т. к. \} \forall x: \; a - \delta < x < a \; \text{ и } \forall x: \; a < x < a + \delta \iff
		\forall x: \; 0 < |x - a| < \delta)
	\end{gathered}\]

	Предел функции на бесконечности:
	\begin{equation*}
		\lim_{x \to \infty} f(x) = A \iff
		\forall \eps > 0 \; \exists \delta > 0: \; \forall x \in D(f): \; |x| > \delta \implies |f(x) - A| < \eps
	\end{equation*}

	\subsubsection{Неопределенности. Теоремы о переходе к пределу в неравенствах, о вынужденном пределе. [Не пройдено]}
	\subsubsection{Теорема о пределе сложной функции. [Не пройдено]} 
	\subsubsection{Первый и второй замечательные пределы. [Не пройдено]} 
	\subsubsection{Сравнение функций, о-символика, главная часть функции, порядок малости и порядок роста функции. [Не пройдено]} 
	\subsubsection{Критерий Коши существования конечного предела функции. [Не пройдено]} 
	\subsubsection{Определения непрерывности функции в точке, их эквивалентность. [Не пройдено]}
	\subsubsection{Точки разрыва, их классификация. Непрерывность основных элементарных функций. [Не пройдено]} 
	\subsubsection{Арифметические свойства непрерывных функций. [Не пройдено]}
	\subsubsection{Теорема о непрерывности сложной функции. [Не пройдено]}
	\subsubsection{Теоремы о локальной ограниченности и локальном сохранении знака для функций, непрерывных в точке. [Не пройдено]} 
	\subsubsection{Свойства функций, непрерывных на отрезке (первая и вторая теоремы  Вейерштрасса, теорема Коши). Критерий существования и непрерывности обратной функции на промежутке. [Не пройдено]} 
	\subsubsection{Понятие равномерной непрерывности функции на множестве. [Не пройдено]} 
	\subsubsection{Теорема Кантора [Не пройдено]}
	\subsubsection{Система стягивающихся отрезков [В списке вопросов к коллоквиуму отсутствует]} \label{sss:segments}
	Множество отрезков $\{[a_n; b_n]\}_{n=1}^{\infty}$ называется системой стягивающихся отрезков, если выполнено:
	\begin{enumerate}
		\item
		Каждый последовательный отрезок вложен в предыдущий, т.е. $a_1 \leq a_2 \leq ... \leq a_n \leq b_n \leq b_{n - 1} \leq \dots \leq b_1$

		\item
		$\lim\limits_{n \to \infty} (b_n - a_n) = 0$
	\end{enumerate}
	\begin{lemma}
		\textbf{(Коши-Кантора)} Для любой ССО существует, причем единственная, точка $c$, принадлежащая всем отрезками данной системы, т. е. $\exists! c: \forall n \in \NN \implies c \in [a_n, b_n]$
	\end{lemma}
	\begin{proof}
		\emph{Существование}. Используем аксиому полноты: если $a \leq b$, то $\exists c: a \leq c \leq b$.
		\begin{equation*}
			\exists c: \; \forall n \in \NN \implies c \in [a_n, b_n]
		\end{equation*}
		\emph{Единственность}. Предположим противное, пусть существуют две различные точки $c, c'$, принадлежащие всем отрезкам последовательности $\{[a_n; b_n]\}_{n=1}^{\infty}$. Не теряя общности, предположим, что $c > c'$.
		
		Тогда $\forall n \in \NN \implies a_n \leq c' < c \leq b_n \implies 0 \leq c - c' \leq b_n - a_n$. Т.к. $\lim\limits_{n \to \infty} (b_n - a_n) = 0 \implies 0 \leq c - c' \leq 0 \implies c - c' = 0 \implies c = c'$

		Пришли к противоречию.
	\end{proof}
\end{document}