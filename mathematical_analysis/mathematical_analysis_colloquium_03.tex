\documentclass[a4paper]{article}
\usepackage{header}
\usepackage{float}
\usepackage{cmap}

\newcommand\enumtocitem[3]{\item\textbf{#1}\addtocounter{#2}{1}\addcontentsline{toc}{#2}{\protect{\numberline{#3}} #1}}
\newcommand\defitem[1]{\enumtocitem{#1}{subsection}{\thesubsection}}
\newcommand\proofitem[1]{\enumtocitem{#1}{subsection}{\thesubsection}}

\newtheorem{theorem1*}{Theorem}

\newtheoremstyle{named}{}{}{}{}{\bfseries}{}{.5em}{Теорема \thmnote{#3}}
\theoremstyle{named}
\newtheorem*{namedtheorem}{Theorem}

\newcommand{\italicbold}[1]{\emph{\textbf{#1}}}

\newlist{colloq}{enumerate}{1}
\setlist[colloq]{label=\textbf{\arabic*.}}

\everymath{\displaystyle}

\renewcommand*{\arraystretch}{1.5}
\newcommand{\T}{\mathbb{T}}
\renewcommand{\int}{\intop}

\title{\HugeМатематический анализ, Коллоквиум 3}
\author{
	Балюк Игорь \\
	\href{https://teleg.run/lodthe}{@lodthe},
    \href{https://github.com/LoDThe/hse-tex}{GitHub} \\
}

\usepackage[yyyymmdd,hhmmss]{datetime}
\settimeformat{xxivtime}
\renewcommand{\dateseparator}{.}
\date{Дата изменения: \today \ в \currenttime}

\begin{document}
    \maketitle

    \tableofcontents

    \newpage

    Предварительная дата проведения коллоквиума --- $29$ февраля.

    \href{https://www.dropbox.com/s/3sdxlo11hnmmaxw/%D0%9F%D1%80%D0%BE%D0%B3%D1%80%D0%B0%D0%BC%D0%BC%D0%B0%20%D0%BA%D0%BE%D0%BB%D0%BB%D0%BE%D0%BA%D0%B2%D0%B8%D1%83%D0%BC%D0%B0%20%D0%9C%D0%90-2-1.pdf?dl=0}{Оригинальный список вопросов}

    Огромное спасибо Егору Косову: большая часть документа состоит из его материалов.

    \section{Выпуклые и вогнутые функции. Выпуклость в терминах производной. Неравенство Йенсена. Примеры.}

        \subsection{Выпуклые и вогнутые функции и их связь с производной}

        \begin{definition*}
            Функция $f$ на интервале $I$ называется \textbf{выпуклой}, если $\forall x, y \in I$ и для каждого $t \in [0; 1]$ выполнено $f(tx + (1 - t)y) \leq tf(x) + (1 - t)f(y)$.

            Функция $f$ на интервале $I$ называется \textbf{вогнутой}, если функция $-f$ --- выпуклая.
        \end{definition*}

        \begin{lemma*}
            Функция $f$ на интервале $I$ выпукла тогда и только тогда, когда для всех точек $x < z < y$ из этого интервала выполенно
            \begin{equation*}
                \dfrac{f(z) - f(x)}{z - x} \leq \dfrac{f(y) - f(z)}{y - z}
            \end{equation*}
        \end{lemma*}

        \begin{proof}
            Зафиксируем $t \in [0; 1]$. Пусть $z = tx + (1 - t)y$. Тогда $t = \dfrac{y - z}{y - x}$ и выпуклость $f$ равносильна выполнению неравенства:
            \begin{equation*}
                f(z) = f(tx + (1 - t)y) \leq tf(x) + (1 - t)f(y) = \dfrac{y - z}{y - x}f(x) + \dfrac{z - x}{y - x}f(y)
            \end{equation*}
            Так как $y - x = y - z + z - x$, полученное неравенство равносильно неравенству из формулировки леммы:
            \[\begin{gathered}
                f(z) \leq \dfrac{y - z}{y - x}f(x) + \dfrac{z - x}{y - x}f(y) \\
                f(z) \cdot (y - z + z - x) \leq (y - z)f(x) + (z - x) f(y) \\
                yf(z) - zf(z) + zf(z) - xf(z) \leq yf(x) - zf(x) + zf(y) - xf(y) \\
                yf(z) - zf(z) -yf(x) + zf(x) \leq zf(y) - xf(y) - zf(z) + xf(z) \\
                (f(z) - f(x)) \cdot (y - z) \leq (f(y) - f(z)) \cdot (z - x) \\
                \dfrac{f(z) - f(x)}{z - x} \leq \dfrac{f(y) - f(z)}{y - z}
            \end{gathered}\]
        \end{proof}

        \begin{theorem*}
            Дифференцируемая функция $f$ на интервале $I$ выпукла тогда и только тогда, когда $f'$ --- неубывает.
        \end{theorem*}

        \begin{proof}
            Если $f$ выпукла, то по предыдущей лемме для $x < y$ выполнено
            \begin{equation*}
                f'(x) \leq \dfrac{f(y) - f(x)}{y - x} \leq f'(y).
            \end{equation*}
            Первая часть неравенства выполняется, если в лемме приближать $z$ к $x$ справа. Вторая часть неравенства выполняется, если значение $z$ из леммы приближать к $y$ слева. Полученное неравенство означает неубывание $f'$.

            Наоборот, пусть теперь $f'$ неубывает. По теореме Лагранжа для всех точек $x < z < y$ найдутся точки $\xi_1 \in (x; z)$ и $\xi_2 \in (z; y)$ для которых
            \begin{equation*}
                \dfrac{f(z) - f(x)}{z - x} = f'(\xi_1), \quad
                \dfrac{f(y) - f(z)}{y - z} = f'(\xi_2)
            \end{equation*}.

            Так как $f'(\xi_1) \leq f'(\xi_2)$, то по предыдущей лемме получаем выпуклость $f$.
        \end{proof}

        Заметим, что дважды дифференцируемая функция $f$ на интервале $I$ выпукла тогда и только тогда, когда $f''(x) \geq 0 \forall x \in I$.

        \subsection{Неравенство Йенсена}

        \begin{namedtheorem}[(Неравенство Йенсена)]
            Пусть функция $f$ выпукла на интервале $I$. Тогда для всех точек $x_1, \dots, x_n \in I$ и для всех чисел $t_1 \geq 0, \dots, t_n \geq 0$, для которых $t_1 + \dots + t_n = 1$, выполнено $f(t_1x_1 + \dots + t_nx_n) \leq t_1f(x_1) + \dots + t_nf(x_n)$.
        \end{namedtheorem}

        \begin{proof}
            Докажем утверждение индукцией по $n$.

            База: $n = 2$, по определению выпуклости.

            Пусть утверждение выполнено для $n$ точек. Проверим, что оно выполнено для $n + 1$ точки. Пусть $t := t_1 + \dots + t_n$. Так как $\dfrac{t_1}{t}x_1 + \dots + \dfrac{t_n}{t}x_n \in I$ (проверяется подстановкой во все $x_i$ минимального/максимального из $x$), то
            \[\begin{gathered}
                f(t_1x_1 + \dots + t_nx_n + t_{n + 1}x_{n + 1}) \leq tf \left(\dfrac{t_1}{t}x_1 + \dots + \dfrac{t_n}{t}x_n\right) + t_{n + 1}f(x_{n + 1}) \\
                \leq t \left(\dfrac{t_1}{t}f(x_1) + \dots + \dfrac{t_n}{t}f(x_n)\right) + t_{n + 1}f(x_{n + 1}) = t_1f(x_1) + \dots + t_{n + 1}f(x_{n + 1})
            \end{gathered}\]

            Первое неравенство верно из определения выпуклости, второе --- воспользовались предположением индукции для $n$.
        \end{proof}

        \subsection{Пример}

        С помощью неравенства Йенсена докажем неравенство о средних. Пусть $x_1, \dots, x_n > 0$. Тогда $\sqrt[n]{x_1 \times \dots \times x_n} \leq \dfrac{x_1 + \dots + x_n}{n}$.

        \begin{proof}
            Действительно, рассмотрим функцию $f(x) = e^x$. Так как $f''(x) = e^x \geq 0$, то $f$ --- выпуклая функция. Теперь заметим, что
            \begin{equation*}
                \sqrt[n]{x_1 \times \dots \times x_n} = f \left(\dfrac{1}{n} \ln x_1 + \dots + \dfrac{1}{n} \ln x_n \right) \leq \dfrac{1}{n} f(\ln x_1) + \dots + \dfrac{1}{n}f(\ln x_n) = \dfrac{x_1 + \dots + x_n}{n}
            \end{equation*}
        \end{proof}


    \section{Первообразная и неопределенный интеграл. Линейность интеграла, формула интегрирования по частям и замены переменной.}

        \subsection{Первообразная и неопределенный интеграл}

        \begin{definition*}
            Функция $F$ называется \textbf{первообразной} функции $f$ на некотором интервале $I$, если $F$ дифференцируема на $I$ и $F'(x) = f(x) \forall x \in I$.
        \end{definition*}

        \begin{lemma*}
            Любые две первообразные $F_1$ и $F_2$ функции $f$ на интервале $I$ отличаются на константу.
        \end{lemma*}

        \begin{proof}
            По теореме Лагранжа, применимой к функции $F := F_1 - F_2$, для произвольных точек $x, y \in I$ выполнено $F(x) - F(y) = F'(\xi)(x - y) = 0$. Что означает, что для двух первообразных, для каждой пары точек из интервала, их разность равна. 

            $F'(\xi)(x - y) = 0$, так как $F'(\xi) = F_1'(\xi) - F_2'(\xi) = f(\xi) - f(\xi) = 0$.
        \end{proof}

        \begin{definition*}
            Множество всех первообразных функции $f$ на некотором заданном интервале $I$ называется \textbf{неопределенным интегралом} от $f$ и обозначается $\int f(x) \dd x$.
        \end{definition*}

        Если $F$ --- некоторая первообразная функции $f$ на некотором интервале $I$, то $\int f(x) \dd x = F + C$, где $C$ --- константа.

        \subsection{Линейность интеграла, формула интегрирования по частям и замены переменной}

        \begin{namedtheorem}[(Свойства неопределенного интеграла)]~
            \begin{enumerate}
            \item (\textbf{Линейность})
                $\int \left(\alpha f(x) + \beta g(x)\right) \dd x = \alpha \int f(x) \dd x + \beta \int g(x) \dd x + C$

            \item (\textbf{Формула интегрирования по частям})
                $\int f(x)g'(x) \dd x = f(x)g(x) - \int f'(x)g(x) \dd x$

            \item (\textbf{Формула замены переменной})
                $\int f(x) \dd x = [x = \phi(t)] = \int f(\phi(t))\phi'(t) \dd t$
            \end{enumerate}
        \end{namedtheorem}

        \begin{proof}~
            \begin{enumerate}
            \item
                Пусть $F$ и $G$ --- первообразные $f$ и $g$ соответственно. Тогда $\alpha F + \beta G$ --- первообразная функции $\alpha f + \beta g$, то есть $\int \left(\alpha f(x) + \beta g(x)\right) \dd x = \alpha F + \beta G + C$.

                В то же время
                \begin{equation*}
                    \alpha \int f(x) \dd x + \beta \int g(x) \dd x = \alpha F + \alpha C_1 + \beta G + \beta C_2 = \alpha F + \beta G + C
                \end{equation*}

            \item
                Так как $(fg)' = f'g + fg'$, то по линейности интеграла
                \begin{equation*}
                    \int f'(x)g(x) \dd x + \int f(x)g'(x) \dd x = f(x) \cdot g(x) + C.
                \end{equation*}

            \item
                Если $F$ --- первообразная для $f$, то $(F(\phi(t)))' = F'(\phi(t))\phi'(t)$.
            \end{enumerate}
        \end{proof}

    \section{Вычисление интеграла от рациональной функции. Примеры сведения интеграла к интегралу от рациональной функции.}

        \subsection{Представление интеграла от рациональной функции}

        \begin{theorem*}
            Пусть $P$ и $Q$ два многочлена. Тогда первообразная функции $\dfrac{P}{Q}$ выражается в элементарных функциях.
        \end{theorem*}

        \begin{proof}
            Пусть $Q(x) = (x - x_1)^{k_1} \cdot \dots \cdot (x - x_s)^{k_s} \cdot (x^2 + p_1x + q_1)^{m_1} \cdot \dots \cdot (x^2 + p_nx + q_n)^{m_n}$. Из курса алгебры известно (доказывать не требуется), что
            \begin{equation*}
                \dfrac{P(x)}{Q(x)} = p(x) + \sum_{j = 1}^s \sum_{k = 1}^{k_j} \dfrac{a_{j, k}}{(x - x_j)^k} + \sum_{j = 1}^n \sum_{k = 1}^{m_j} \dfrac{b_{j, k}x + c_{j, k}}{(x^2 + p_jx + q_j)^k},
            \end{equation*}
            где $p$ --- многочлен, а коэффициенты $a_{i, j}, b_{i, j}, c_{i, j}$ --- рациональные числа. То есть частное от деления рациональных многочленов представляется суммой неприводимых дробей (знаменатель имеет степень $1$ или $2$, числитель имеет степень на единицу меньше) и многочлена.

            По линейности нам надо научиться интегрировать каждое слагаемое отдельно. Выделяя у интеграла $\int \dfrac{bx + c}{(x^2 + px + q)^k} \dd x$ в знаменателе целую часть (выделяем полный квадрат) и делая линейную замену приводим его к виду $\int \dfrac{b'u + c'}{(u^2 + a^2)^k} \dd u$. 

            Более подробно:
            \begin{align*}
                \int \dfrac{bx + c}{(x^2 + px + q)^k} \dd x 
                &= \int \dfrac{bx + c}{\left(x^2 + 2 \cdot \frac{p}{2} \cdot x + \left(\frac{p^2}{4} - \frac{p^2}{4}\right) + q\right)^k} \dd x
                = \int \dfrac{bx + c}{\left(\left(x + \frac{p}{2}\right)^2 - \frac{p^2}{4} + q\right)^k} \dd x \\
                &= \left[\begin{aligned}
                    &u = x + \dfrac{p}{2} \\
                    &a = \sqrt{q - \dfrac{p^2}{4}}\\
                    &bx + c = b' \cdot \left(x + \frac{p}{2}\right) + c' = b'u + c' \\
                \end{aligned}\right]
                = \int \dfrac{b'u + c'}{(u^2 + a^2)^k} \dd u
            \end{align*}

            Заметим, что $q - \dfrac{p^2}{4} \geq 0$, так как если мы раскладываем такую дробь, то её знаменатель на раскладывается на произведениее двух многочленов степени $1$. Значит, квадратное уравнение имеет комплексные корни и дискриминант отрицательный, т.е. $p^4 - 4q < 0$ (дискриминант), что влечет $p^2 < 4q$.

            \subsection{Вычисление интеграла каждого типа}

            Перейдем к вычислению интеграла каждого типа.

            \begin{enumerate}
            \item
                \begin{equation*}
                    \int \dfrac{\dd x}{(x - a)^k} = \begin{cases}
                        \dfrac{1}{1 - k}(x - a)^{1 - k} + C, &k \neq 1, \\
                        \ln |x - a| + C, &k = 1.
                    \end{cases}
                \end{equation*}

            \item
                \begin{equation*}
                    \int \dfrac{u}{(u^2 + a^2)^k} \dd u
                    = \dfrac{1}{2} \int \dfrac{\dd (u^2 + a^2)}{(u^2 + a^2)^k}
                    = \begin{cases}
                        \dfrac{1}{2(1 - k)}(u^2 + a^2)^{1 - k} + C, &k \neq 1, \\
                        \dfrac{1}{2} \ln(u^2 + a^2) + C, &k = 1.
                    \end{cases}
                \end{equation*}

            \item
                \begin{align*}
                    I_k 
                    &= \int \dfrac{\dd u}{(u^2 + a^2)^k}
                    = \dfrac{u}{(u^2 + a^2)^k} + 2k \int \dfrac{u^2 \dd u}{(u^2 + a^2)^{k + 1}} \\
                    &= \dfrac{u}{(u^2 + a^2)^k} + 2k \int \dfrac{\dd u}{(u^2 + a^2)^k} - 2ka^2 \int \dfrac{\dd u}{(u^2 + a^2)^{k + 1}} \\
                    &= \dfrac{u}{(u^2 + a^2)^k} + 2kI_k - 2ka^2I_{k + 1}
                \end{align*}

                Решая рекуррентное уравнение, находим
                \begin{equation*}
                    I_{k + 1} = \dfrac{1}{2ka^2} \cdot \dfrac{u}{(u^2 + a^2)^k} + \dfrac{2k - 1}{2ka^2}I_k, \quad
                    I_1 = \int \dfrac{\dd u}{u^2 + a^2} = a^{-1} \arctg \dfrac{u}{a} + C
                \end{equation*}
            \end{enumerate}
        \end{proof}

        \subsection{Пример}

        \begin{example*}
            Пусть $t = \tg \dfrac{x}{2}$, $\dd x = \dfrac{2 \dd t}{1 + t^2}$. Заметим, что
            \begin{equation*}
                \cos x = \dfrac{1 - \tg^2 \dfrac{x}{2}}{1 + \tg^2 \dfrac{x}{2}} = \dfrac{1 - t^2}{1 + t^2}, \quad
                \sin x = \dfrac{2 \tg \dfrac{x}{2}}{1 + \tg^2 \dfrac{x}{2}} = \dfrac{2t}{1 + t^2}
            \end{equation*}

            Тем самым, интегралы от функций $R(\cos x, \sin x)$, где $R$ --- рациональная функция, сводятся заменой к интегралам от рациональных функций.
        \end{example*}

    \section{Интегралы Римана: определение, примеры интегрируемых и неинтегрируемых функций, линейность и монотонность интеграла, ограниченность интегрируемой функции.}

        \subsection{Определение интеграла по Риману}

        \begin{definition*}~

            \textbf{Разбиением} $\T$ отрезка $[a; b]$ называется набор точек $a = x_0 < x_1 < \dots < x_n = b$.

            Отрезки $\Delta_k := [x_{k - 1}; x_k]$ называются \textbf{отрезками разбиения}.

            Число $\lambda(\T) := \max\limits_{1 \leq k \leq n} |\Delta_k| := x_k - x_{k - 1}$, называется \textbf{масштабом} разбиения.

            \textbf{Отмеченным разбиением} $(\T, \xi)$ отрезка $[a; b]$ называется пара, состоящая из разбиения $\T$ отрезка $[a; b]$ и набора точек $\xi = (\xi_1, \dots, \xi_n)$, $\xi_k \in \Delta_k$.

            \textbf{Интегральной суммой} функции $f$, соответствующей отмеченному разбиению $(\T, \xi)$, называется выражение $\sigma(f, \T, \xi) := \sum_{k = 1}^n f(\xi_k) \cdot |\Delta_k|$.
        \end{definition*}

        \begin{definition*}
            Функция $f$ называется интегрируемой по Риману на отрезке $[a; b]$ и число $I$ называется её интегралом, если $\forall \eps > 0 \ \exists \delta > 0: \ \forall$ отмеченного разбиения $(\T, \xi)$ с $\lambda(\T) < \delta$ выполнено $|\sigma(f, \T, \xi) - I| < \eps$.

            Число $I$ обозначают $\int_a^b f(x) \dd x$.
        \end{definition*}

        \subsection{Примеры}

        \begin{example*}~

            \begin{enumerate}
            \item
                $\int_a^b 1 \dd x = b - a$

            \item
                Функция Дирихле не интегрируема по Риману:
                \begin{equation*}
                	f(x) = \begin{cases}
                		1 &x \in \QQ \cup [0; 1], \\
                		0 &otherwise,
                	\end{cases}
                \end{equation*}
                так как её верхняя и нижняя суммы Дарбу (будет в следующих параграфах) равны $1$ и $0$ соответственно.
            \end{enumerate}
        \end{example*}

        \subsection{Ограниченность интегрируемых функций}

        \begin{proposal*}
            Если функция $f$ интегрируема по Риману на отрезке $[a; b]$, то она ограничена на этом отрезке.
        \end{proposal*}

        \begin{proof}
            Так как $f$ интегрируема, то для некоторого разбиения $\T$ для произвольного выбора отмеченных точек $\xi = (\xi_1, \dots, \xi_n)$ выполнено
            \begin{equation*}
                \int_a^b f(x) \dd x - 1 < \sum_{k = 1}^n f(\xi_k) \cdot |\Delta_k| < \int_a^b f(x) \dd x + 1.
            \end{equation*}

            Если бы $f$ оказалась неограниченной на отрезке $[a; b]$, она была бы неограниченной на каком-то из отрезков разбиения $\Delta_{k_0}$, что в силу произвольности выбора $\xi_{k_0} \in \Delta_{k_0}$ и противоречит неравенству выше (<<зажали>> бесконечность с двух сторон)
        \end{proof}

        \subsection{Линейность интеграла}

        \begin{proposal*}
            \textbf{(Линейность интеграла)}. Пусть $f$ и $g$ интегрируемы по Риману на отрезке $[a; b]$. Тогда для произвольных чисел $\alpha, \beta$ функция $\alpha f + \beta g$ интегрируема по Риману на отрезке $[a; b]$ и $\int_a^b \left(\alpha f(x) + \beta g(x)\right) \dd x = \alpha \int_a^b f(x) \dd x + \beta \int_a^b g(x) \dd x$.
        \end{proposal*}

        \begin{proof}
            Заметим, что $\sigma(\alpha f + \beta g, \T, \xi) = \alpha \sigma(f, \T, \xi) + \beta \sigma(g, \T, \xi)$. Кроме того, для произвольного $\eps > 0$ найдется $\delta > 0$ для которого
            \begin{equation*}
                \left|\sigma(f, \T, \xi) - \int_a^b f(x) \dd x\right| < \eps; \quad
                \left|\sigma(g, \T, \xi) - \int_a^b g(x) \dd x\right| < \eps
            \end{equation*}

            для каждого отмеченного разбиения $(\T, \xi)$ с масштабом $\lambda(\T) < \delta$. Тем самым, для таких разбиений
            \begin{equation*}
                \left|\sigma(\alpha f + \beta g, \T, \xi) - \alpha \int_a^b f(x) \dd x - \beta \int_a^b g(x) \dd x\right| < (|\alpha| + |\beta|) \cdot \eps
            \end{equation*}
        \end{proof}

        \subsection{Монотонность интеграла} 

        \begin{proposal*}
            \textbf{(Монотонность интеграла)}. Пусть $f$ и $g$ интегрируемы по Риману на отрезке $[a; b]$. Если $f(x) \leq g(x) \forall x \in [a; b]$, то $\int_a^b f(x) \dd x \leq \int_a^b g(x) \dd x$.
        \end{proposal*}

        \begin{proof}
            В силу линейности достаточно доказать данное утверждение только для $f \equiv 0$ (иначе прибавим к обеим частям одинаковую функцию, знак неравенства не изменится). В этом случае $\sigma(g, \T, \xi) \geq 0$ для произвольного отмеченного разбиения $(\T, \xi)$. Так как интеграл приближается интегральными суммами с любой точностью, то и сам интеграл неотрицателен.
        \end{proof}

    \section{Нижние и верхние суммы Дарбу. Критерий Дарбу интегрируемости ограниченной функции.}

        \subsection{Нижние и верхние суммы Дарбу}

        \begin{definition*}
            Для ограниченной на отрезке $[a; b]$ функции $f$ и разбиения $\T$ определим \textbf{нижнюю}
            \begin{equation*}
                s(f, \T) := \sum_{k = 1}^n \inf\limits_{x \in \Delta_k} f(x) \cdot |\Delta_k|
            \end{equation*}
            и \textbf{верхнюю}
            \begin{equation*}
                S(f, \T) := \sum_{k = 1}^n \sup\limits_{x \in \Delta_k} f(x) \cdot |\Delta_k|
            \end{equation*}
            суммы Дарбу.

            \textbf{Нижним интегралом Дарбу} называется число $\underline{I} = \sup\limits_{\T} s(f, \T)$ (обратите внимание, что черта снизу), а \textbf{верхним интегралом Дарбу} называется число $\overline{I} = \inf\limits_{\T} S(f, \T)$. 
        \end{definition*}

        \begin{lemma*}~

            \begin{enumerate}
            \item
                $s(f, \T) \leq \inf_{\xi} \sigma(f, \T, \xi) \leq \sigma(f, \T, \xi) \leq \sup_{\xi} \sigma(f, \T, \xi) \leq S(f, \T)$

            \item
                Если $\T \subseteq \T'$, то $s(f, \T) \leq s(f, \T')$ и $S(f, \T') \leq S(f, \T)$

            \item
                $s(f, \T_1) \leq s(f, \T_1 \cup \T_2) \leq S(f, \T_1 \cup \T_2) \leq S(f, \T_2)$
            \end{enumerate}
        \end{lemma*}

        \begin{proof}~

            \begin{enumerate}
            \item
                Рассмотрим первое неравенство, так как последнее аналогично ему, а те, что между ними, следуют из определения. Логика здесь такая: пусть мы нашли инфинум справа и соответствующий ему $\xi$. Но тогда при подсчете слева мы могли брать те же $\xi$, поэтому получим сумму не больше правой.

            \item
                Рассмотрим на примере первого неравенства. Пусть между какими-то двумя точками из $\T$ появилось несколько точек из $\T'$. Значение, равное инфинуму функции на этом отрезке умноженному на длину отрезка, будет не больше сумме инфинумов на каждом из подотрезков умноженных на их длины.

            \item
                Рассмотрим первое неравенство. На самом деле, это верно из предыдущего пункта: пускай $\T = \T_1$, а $\T' = \T_1 \cup \T_2$.
            \end{enumerate}
        \end{proof}

        \begin{lemma*}
            $\forall \eps > 0 \ \exists \delta > 0: \ \forall \T: \lambda(\T) < \delta \implies \underline{I} \leq s(f, \T) + \eps$ и $\overline{I} \geq S(f, \T) - \eps$.
        \end{lemma*}

        \begin{proof}
            Докажем только первую часть. 

            Для каждого $\eps$ найдется такое разбиение $\T_{\eps}$, для которого $\underline{I} \leq s(f, \T_{\eps}) + \dfrac{\eps}{2} \leq s(f, \T_{\eps} \cup \T) + \dfrac{\eps}{2}$ для произвольного разбиения $\T$. Первое неравенство выполняется, так как можно в $\T_{\eps}$ подставить разбиение $\T$, которое было выбрано для супремума в $\underline{I}$. Второе неравенство выполняется по третьему пункту из предыдущей леммы.

            Заметим, что среди отрезков, порожденных разбиением $\T_{\eps} \cup \T$ не более чем $2|\T_{\eps}|$ отрезков, не порожденных разбиением $\T$ (худший случай, когда в каждый отрезок, порожденный $\T$, попадает одна точка из $\T_{\eps}$, тем самым порождая 2 новых отрезка). Поэтому $s(f, \T_{\eps} \cup \T) \leq s(f, \T) + 2|\T_{\eps}| \cdot 2 \sup\limits_{x \in [a; b]} |f(x)| \cdot \lambda(\T)$.

            Взяв теперь $\delta > 0$ так, чтобы $4|\T_{\eps}| \sup\limits_{x \in [a; b]} |f(x)| \cdot \delta < \dfrac{\eps}{2}$, получаем требуемую оценку.
        \end{proof}

        \subsection{Критерий Дарбу}

        \begin{theorem*}
            Ограниченная функция $f$ интегрируема по Риману на отрезке $[a; b]$ тогда и только тогда, когда $\underline{I} = \overline{I}$.
        \end{theorem*}

        \begin{proof}
            Если функция $f$ интегрируема, то $\forall \eps > 0 \ \exists \delta > 0:$ для любого отмеченного разбиения $(\T, \xi)$ с $\lambda(\T) < \delta$ выполнено $I - \eps \leq \sigma(f, \T, \xi) \leq I + \eps$, где $I := \int_a^b f(x) \dd x$.

            Тем самым, $I - \eps \leq s(f, \T) \leq \underline{I} \leq \overline{I} \leq S(f, \T) \leq I + \eps$. В силу произвольности $\eps$ выполнено равенство
            \begin{equation*}
                \underline{I} = \overline{I} = I = \int_a^b f(x) \dd x
            \end{equation*}

            Обратно: пусть $I = \underline{I} = \overline{I}$. По предыдущей лемме, $\forall \eps > 0 \ \exists \delta > 0:$ для любого отмеченного разбиения $(\T, \xi)$ с $\lambda(\T) < \delta$ выполнено
            \begin{equation*}
                I - \eps \leq s(f, \T) \leq \sigma(f, \T, \xi) \leq S(f, \T) \leq I + \eps
            \end{equation*}

            Это и означает, что $f$ интегрируема по Риману на $[a; b]$ и $I$ её интеграл.
        \end{proof}

    \section{Переформулировка критерия Дарбу в терминах колебаний. Интегрируемость модуля и произведения интегрируемых функций. Интегрируемость на подотрезке, аддитивность интеграла.}

        \subsection{Переформулировка критерия Дарбу в терминах колебаний}

        \begin{definition*}
            Назовём \textbf{колебанием} функции $f$ на отрезке $[a; b]$ число
            \begin{equation*}
                \omega(f, [a; b]) = \sup\limits_{\xi', \xi'' \in [a; b]} |f(\xi') - f(\xi'')| = \sup\limits_{[a; b]} f(x) - \inf\limits_{[a; b]} f(x)
            \end{equation*}
        \end{definition*}

        \begin{consequence*}
            Ограниченная функция $f$ интегрируема по Риману на отрезке $[a; b]$ тогда и только тогда, когда $\forall \eps > 0$ найдется разбиение $\T$, для которого $\sum_k \omega(f, \Delta_k) \cdot |\Delta_k| < \eps$.
        \end{consequence*}

        \begin{proof}
            Заметим, что $\overline{I} = \underline{I} \iff \forall \eps > 0$ найдется разбиение $\T$, для которого $S(f, \T) - s(f, \T) < \eps$.

            Требует пояснения только импликация $\implies$, так как обратное следует из выбора $\T$: для инфинума и супремума в $\overline{I}$ и $\underline{I}$ соответственно будет выбрано то самое $\T$.

            Если $\overline{I} = \underline{I}$, то $\forall \eps > 0$ найдутся разбиения $\T_1$ и $\T_2$: $S(f, \T_1) - s(f, \T_2) < \left(\overline{I} + \dfrac{\eps}{2}\right) - (\underline{I} - \dfrac{\eps}{2}) = \eps$ (так как можно взять $\T_1$ и $\T_2$ равные выбранным в $\overline{I}$ и $\underline{I}$ соответственно).

            Кроме того, $S(f, \T_1 \cup \T_2) - s(f, \T_1 \cup \T_2) \leq S(f, \T_1) - s(f, \T_2)$ (по свойствам для сумм Дарбу). 

            Остается лишь заметить, что верно равенство $S(f, \T) - s(f, \T) = \sum_k \omega(f, \Delta_k) \cdot |\Delta_k|$.
        \end{proof}

        \subsection{Интегрируемость модуля и произведения интегрируемых функций. Интегрируемость на подотрезке.}

        \begin{consequence*}
            Если $f$ интегрируема по Риману на отрезке $[a; b]$, то $|f|$ и $f^2$ интегрируемы по Риману на отрезке $[a; b]$ и для любого $[c; d] \subseteq [a; b]$ функция $f$ интегрируема по Риману на отрезке $[c; d]$.
        \end{consequence*}

        \begin{proof}
            Интегрируемость $|f|$ и $f^2$ следует из оценок
            \[\begin{gathered}
                \omega(|f|, \Delta) \leq \omega(f, \Delta) \\
                \omega(f^2, \Delta) \leq 2 \sup_{x \in \Delta} |f(x)| \cdot \omega(f, \Delta)
            \end{gathered}\]

            Интегрируемость на подотрезке доказывается следующим образом. Для каждого $\eps > 0$ найдутся разбиение $\T$ отрезка $[a; b]$ для которого $S_{[a; b]}(f, \T) - s_{[a; b]} < \eps$. Но
            \[\begin{gathered}
                S_{[c; d]}(f, (\T \cup \{c, d\}) \cap [c; d]) - s_{[c; d]}(f, (\T \cup \{c, d\}) \cap [c; d]) \\
                \leq S_{[a; b]}(f, \T \cup \{c, d\}) - s_{[a; b]}(f, \T \cup \{c, d\}) \leq S_{[a; b]}(f, \T) - s_{[a; b]}(f, \T),
            \end{gathered}\]
            где $S_{[c; d]}$, $s_{[c; d]}$, $S_{[a; b]}$ и $s_{[a; b]}$ обозначают верхние и нижние суммы Дарбу на отрезках $[c; d]$ и $[a; b]$ соответственно.
        \end{proof}

        \begin{consequence*}
            Если $f$ и $g$ интегрируемы на $[a; b]$, то и $f \cdot g$ интегрируема на $[a; b]$.
        \end{consequence*}

        \begin{proof}
            Действительно, $f \cdot g = \dfrac{1}{4} \cdot \left[(f + g)^2 - (f - g)^2\right]$
        \end{proof}

        \subsection{Аддитивность интеграла}

        \begin{consequence*}
            Если $f$ интегрируема по Риману на отрезке $[a; b]$, $c \in [a; b]$, то $f$ интегрируема на отрезках $[a; c]$ и $[c; b]$ и верно равенство
            \begin{equation*}
                \int_a^b f(x) \dd x = \int_a^c f(x) \dd x + \int_c^b f(x) \dd x
            \end{equation*}
        \end{consequence*}

        \begin{proof}
            Интегрируемость на подотрезках уже доказана. А равенство следует из того, что при вычислении интеграла можно использовать интегральные суммы, соответствующие разбиениям, содержащим точку $c$ (выберем $\T$, в котором будет точка $c$).
        \end{proof}

    \section{Интегрируемость монотонных функций. Равномерная непрерывность. Примеры. Равномерная непрерывность непрерывной на отрезке функции. Интегрируемость непрерывных функций.}

        \subsection{Интегрируемость монотонных функций}

        \begin{consequence*}
            Если $f$ монотонна на $[a; b]$, то $f$ интегрируема по Риману на $[a; b]$.
        \end{consequence*}

        \begin{proof}
            Заметим, что $\omega(f, \Delta_k) = |f(x_k) - f(x_{k - 1})|$. В таком случае,
            \begin{equation*}
                \sum_k \omega(f, \Delta_k) \cdot |\Delta_k| \leq \lambda(\T) \cdot \sum_k |f(x_k) - f(x_{k - 1})| = \lambda(\T) \cdot |f(b) - f(a)|.
            \end{equation*}

            Тем самым, при $\lambda(\T) < \dfrac{\eps}{|f(b) - f(a)|}$ получаем $\sum_k \omega(f, \Delta_k) \cdot |\Delta_k| < \eps$.
        \end{proof}

        \subsection{Равномерная непрерывность}

        \begin{definition*}
            Функция $f$ называется равномерно непрерывной на множестве $X$, если $\forall \eps > 0 \ \exists \delta  > 0$, для которого из неравенства $|x - y| < \delta$ следует $|f(x) - f(y)| < \eps$.
        \end{definition*}

        \subsection{Примеры}

        \begin{example*}~
            
            \begin{enumerate}
            \item
                Функция $f(x) := \sin x$ равномерно непрерывна на $\RR$, так как по теореме Лагранжа $|\sin x - \sin y| = |\cos \xi| \cdot |x - y| \leq |x - y|$.

            \item
                Функция $f(x) = \dfrac{1}{x}$ не равномерно непрерывна на $(0; 1)$, так как $f \left(\dfrac{1}{2n}\right) - f \left(\dfrac{1}{n}\right) = n$, а $\left|\dfrac{1}{n} - \dfrac{1}{2n}\right| = \dfrac{1}{2n} \to 0$.
            \end{enumerate}
        \end{example*}

        \subsection{Равномерная непрерывность непрерывной на отрезке функции}

        \begin{theorem*}
            Если функция $f$ непрерывна на отрезке $[a; b]$, то $f$ равномерно непрерывна на $[a; b]$.
        \end{theorem*}

        \begin{proof}
            Если $f$ не равномерно непрерывна, то найдется такое число $\eps > 0$, что для $\forall n \ \exists x_n, y_n \in [a; b]: \ |x_n - y_n| < \dfrac{1}{n}$ и $|f(x_n) - f(y_n)| \geq \eps$ (записали отрицание условия равномерной непрерывности). 

            В силу ограниченности последовательности $x_n$, у неё есть сходящаяся подпоследовательность $x_{n_k} \to x \in [a; b]$. Заметим, что $y_{n_k} \to x$. Но $f$ непрерывна в точке $x$ по условию, что противоречит оценке
            \begin{equation*}
                |f(x_{n_k}) - f(x)| + |f(x) - f(y_{n_k})| \geq |f(x_{n_k}) - f(y_{n_k})| \geq \eps.
            \end{equation*}
        \end{proof}

        \subsection{Интегрируемость непрерывных функций}

        \begin{consequence*}
            Пусть $f$ непрерывна на $[a; b]$. Тогда $f$ интегрируема по Риману на $[a; b]$.
        \end{consequence*}

        \begin{proof}
            В силу предыдущей теоремы $f$ равномерно непрерывна на $[a; b]$.

            Поэтому, $\forall \eps > 0 \ \exists \delta > 0: \omega(f, \Delta_k) < \eps \ \forall k$ для произвольного разбиения $\T$ с $\lambda(\T) < \delta$. Для такого разбиения
            \begin{equation*}
                \sum_k \omega(f, \Delta_k) \cdot |\Delta_k| < \eps \cdot |b - a|.
            \end{equation*}
        \end{proof}

    \section{Формула Ньютона-Лейбница и интегрирование по частям. Интеграл с переменным верхним пределом, его свойства. Наличие первообразной у непрерывной функции. Формула замены переменной.}

        \subsection{Формула Ньютона-Лейбница и интегрирование по частям}

        \begin{namedtheorem}[(Формула Ньютона-Лейбница)]
            Пусть $F$ дифференцируема на $[a; b]$ и ее производная $F'$ интегрируема по Риману на $[a; b]$. Тогда
            \begin{equation*}
                \left. \int_a^b F'(x) \dd x = F(b) - F(a) = F(x) \right|_a^b
            \end{equation*}
        \end{namedtheorem}

        \begin{proof}
            Для разбиения $\T$ по тореме Лагранжа $F(x_k) - F(x_{k - 1}) = F'(\xi_k) \cdot |\Delta_k|$. Поэтому
            \begin{equation*}
                F(b) - F(a) = \sum_{k = 1}^n (F(x_k) - F(x_{k - 1})) = \sum_{k = 1}^n F'(\xi_k) \cdot |\Delta_k| = \sigma(f, \T, \xi).
            \end{equation*}

            При достаточно малом масштабе $\lambda(\T)$ интегральная сумма близка к интегралу от $F'$.
        \end{proof}

        \begin{consequence*}
            Пусть $f$, $g$ --- непрерывно дифференцируемы на отрезке $[a; b]$ функции. Тогда
            \begin{equation*}
                \int_a^b f(x) g'(x) \dd x = f(b) g(b) - f(a) g(a) - \int_a^b f'(x) g(x) \dd x
            \end{equation*}
        \end{consequence*}

        \begin{proof}
            Применяем формулу Ньютона-Лейбница к $F = fg$:
            \begin{equation*}
                \int_a^b (f(x)g(x))' \dd x = f(b)g(b) - f(a)g(a)
            \end{equation*}
            и правило Лейбница: $(fg)' = f'g + fg'$.
        \end{proof}

        Далее будем использовать соглашение: при $b > a$ по определению $\int_b^a f(x) \dd x = - \int_a^b f(x) \dd x$.

        \subsection{Интеграл с переменным верхним пределом, его свойства}

        \begin{theorem*}
            Пусть $f$ интегрируема по Риману на $[a; b]$. Тогда функция
            \begin{equation*}
                F(x) := \int_a^x f(x) \dd x
            \end{equation*}
            непрерывна на $[a; b]$. Кроме того, если $f$ непрерывна в точке $x_0 \in [a; b]$, то $F$ дифференцируема в точке $x_0$ и $F'(x_0) = f(x_0)$.
        \end{theorem*}

        \begin{proof}
            Для $x, y \in [a; b]$ выполнено $|f(x) - f(y)| \leq \sup\limits_{z \in [a; b]} |f(z)| \cdot |x - y|$, что влечет непрерывность $F$. Пусть $f$ непрерывна в точке $x_0$. Тогда
            \begin{equation*}
                \forall \eps > 0 \ \exists \delta > 0: f(x_0) - \eps < f(x) < f(x_0) + \eps
            \end{equation*}
            для каждой точки $x$, для которой $|x - x_0| < \delta$.

            Так как $F(X) - F(x_0) = \int_{x_0}^x f(t) \dd t$, то для точек $x$, для которых $|x - x_0| < \delta$, выполнено
            \begin{equation*}
                f(x_0) - \eps < \dfrac{F(x) - F(x_0)}{x - x_0} < f(x_0) + \eps,
            \end{equation*}
            что и дает второе утверждение теоремы.  
        \end{proof}

        \begin{consequence*}
            Пусть $f$ непрерывна на $[a; b]$, тогда у $f$ существует первообразная.
        \end{consequence*}

        \subsection{Формула замены переменной}

        \begin{consequence*}
            Пусть $f$ --- непрерывна на $[a; b]$, $\phi: [\alpha, \beta] \to [a; b]$ --- непрерывно дифференцируемая функция. Тогда
            \begin{equation*}
                \int_{\phi(\alpha)}^{\phi(\beta)} f(x) \dd x = \int_{\alpha}^{\beta} f(\phi(t)) \phi'(t) \dd t
            \end{equation*}
        \end{consequence*}

        \begin{proof}
            Пусть $F$ --- первообразная $f$. Тогда $F(\phi(t))$ --- первообразная функция $f(\phi(t)) \phi'(t)$. 

            По формуле Ньютона-Лейбница:
            \begin{equation*}
                \int_{\phi(\alpha)}^{\phi(\beta)} f(x) \dd x = F(\phi(\beta)) - F(\phi(\alpha)) = \int_{\alpha}^{\beta} f(\phi(t)) \phi'(t) \dd t
            \end{equation*}
        \end{proof}

        \subsection{Площадь криволинейной трапеции и длина кривой}

        Пусть $f \geq 0$ на $[a; b]$. И пусть $S(\alpha, \beta)$ площадь под графиком функции $f$ на отрезке $[\alpha; \beta] \subseteq [a; b]$. Разумные требования на $S$ --- это

        \begin{enumerate}
        \item 
            аддитивность: $S(\alpha, \gamma) = S(\alpha, \beta) + S(\beta, \gamma)$ при $a \leq \alpha < \beta < \gamma \leq b$

        \item
            монотонность по включению: $\inf\limits_{x \in [\alpha; \beta]} f(x) \cdot (\beta - \alpha) \leq S(\alpha, \beta) \leq \sup\limits_{x \in [\alpha; \beta]} f(x) \cdot (\beta - \alpha)$ 
        \end{enumerate}

        \begin{proposal*}
            Пусть $f$ интегрируема по Риману на $[a; b]$. При выполнении выше описанных условий $S(a, b) = \int_a^b f(x) \dd x$.
        \end{proposal*}

        \begin{proof}
            Для произвольного разбиения $\T$ выполнено
            \begin{equation*}
                \sum_{k = 1}^n \inf\limits_{x \in \Delta_k} f(x) \cdot |\Delta_k| \leq S(a, b) \leq \sum_{k = 1}^n \sup_{x \in \Delta_k} f(x) \cdot |\Delta_k|.
            \end{equation*}

            Слева и справа стоят нижняя и верхняя суммы Дарбу, которые при малом масштабе разбиения близки к интегралу.
        \end{proof}

        Пусть $\gamma: [a; b] \to \RR^3$ --- гладкая кривая, те. $\gamma(t) = (x(t), y(t), z(t))$ и функции $x, y, z \in C^1([a; b])$ (принадлежность функции $C^1$ означает, что первая производная непрерывна). Пусть $l(a, b)$ --- длина пути, соответствующая отрезку $[a; b]$. Тогда естественными требованиями будут

        \begin{enumerate}
        \item
            $l(\alpha, \gamma) = l(\alpha, \beta) + l(\beta, \gamma)$ при $a \leq \alpha < \beta < \gamma \leq b$

        \item
            $\inf\limits_{t \in [\alpha; \beta]} |v(t)| \cdot (\beta - \alpha) \leq l(\alpha, \beta) \leq \sup\limits_{t \in [\alpha; \beta]} |v(t)| \cdot (\beta - \alpha)$, где $v(t) =  (x'(t), y'(t), z'(t))$
        \end{enumerate}

        Аналогично получаем, $l(a, b) = \int_a^b |v(t)| \dd t$, где $|v(t)| = \sqrt{(x'(t))^2 + (y'(t))^2 + (z'(t))^2}$.

    \section{Формула Тейлора с остаточным членом в интегральной форме. Ряд Тейлора для функций $e^x$, $\sin x$, $\cos x$, $\ln(1 + x)$, $(1 + x)^p$ (обоснование сходимости для $e^x$ и $\ln(1 + x)$). Площадь криволинейной трапеции и длина кривой.}

    	\subsection{Формула Тейлора с остаточным членом в интегральной форме}

    	\begin{theorem*}
    		Если $f$ непрерывно дифференцируема $m + 1$ раз на отрезке $[a; x]$, то
    		\begin{equation*}
    			f(x) = f(a) + f'(a)(x - a) + \dots + \dfrac{1}{m!} \cdot f^{(m)}(a) (x-  a)^m + \dfrac{1}{m!} \int_a^x (x - t)^m f^{(m + 1)}(t) \dd t
    		\end{equation*}
    	\end{theorem*}

    	\begin{proof}
    		Заметим, что
    		\begin{equation*}
    			\dfrac{1}{m!} \int_a^x (x - t)^m f^{(m + 1)}(t) \dd t
    			= -\dfrac{1}{m!}(x - a)^m \cdot f^{(m)}(a) - \dfrac{1}{m!} \int_a^x -m(x - t)^{m - 1}f^{(m)}(t) \dd t.
    		\end{equation*}

    		Чтобы получить интегральную формулу, воспользовались интегрированием по частям. Если начать раскрывать эти интегралы до конца ($m - 1$ раз), и сложить с посчитанными членами в ряде Тейлора, то получится
    		\begin{equation*}
    			f(x) = f(a) + \int_a^x f'(t) \dd t = f(a) + f(x) - f(a) = f(x)
    		\end{equation*}
    	\end{proof}

    	\subsection{Ряд Тейлора для функций $e^x$, $\sin x$, $\cos x$, $\ln(1 + x)$, $(1 + x)^p$ (обоснование сходимости для $e^x$ и $\ln(1 + x)$)}

    	\begin{consequence*}
    		Справедливы следующие равенства:

    		\begin{itemize}
    		\item
    			$e^x = \sum_{n = 0}^{\infty} \dfrac{x^n}{n!}$, $x \in \RR$

    		\item
    			$\sin x = \sum_{n = 0}^{\infty} \dfrac{(-1)^n x^{2n + 1}}{(2n + 1)!}$, $x \in \RR$

    		\item
    			$\cos x = \sum_{n = 0}^{\infty} \dfrac{(-1)^n x^{2n}}{(2n)!}$, $x \in \RR$

    		\item
    			$\ln(1 + x) = \sum_{n = 1}^{\infty} \dfrac{(-1)^{n - 1}x^n}{n}$, $x \in (-1; 1]$

    		\item
    			$(1 + x)^p = 1 + \sum_{n = 1}^{\infty} \binom{p}{n} x^n$, где $\binom{p}{n} := \dfrac{p(p - 1) \dots (p - n + 1)}{n!}$, $x \in (-1; 1)$
    		\end{itemize}
    	\end{consequence*}

    	\begin{proof}
    		Имеем
    		\begin{equation*}
    			\left|e^x - \sum_{n = 0}^m \dfrac{x^n}{n!}\right|
    			= \dfrac{1}{m!} \left|\int_0^x (x - t)^m e^t \dd t\right|
    			\leq \dfrac{|x|^m \cdot e^x}{m!} \xrightarrow[m \to \infty]{} 0
    		\end{equation*}

    		Функции $\sin$ и $\cos$ рассматриваются аналогично. Для $\ln(1 + x)$ имеем
    		\begin{equation*}
    			\left|\ln(1 + x) - \sum_{n = 1}^m \dfrac{(-1)^{n - 1}x^n}{n}\right| = \left|\int_0^x(x - t)^m (1 + t)^{-m - 1} \dd t\right|.
    		\end{equation*}

    		Если $x > 0$, то последнее выражение оценивается через $\dfrac{1}{m + 1}$, а если $x < 0$, то $|x - t| = |x| \cdot (1 - tx^{-1}) \leq |x| \cdot (1 - |t|) = |x| \cdot (1 + t)$, поэтому
    		\begin{equation*}
    			\left|\int_0^x (x - t)^m \cdot (1 + t)^{-m - 1} \dd t\right| \leq \dfrac{|x|^m}{1 + x} \xrightarrow[m \to \infty]{} 0
    		\end{equation*}

    		Ряд для функции $(1 + x)^p$ рассматривается похожим образом.
    	\end{proof}

    \section{Формула Стирлинга.}

   		\subsection{Формула Стирлинга}

   		\begin{theorem*}
   			Для некоторой числовой постоянной $c$ выполнено $N! \sim c \sqrt{N} (N / e)^N$
   		\end{theorem*}

   		\begin{proof}
   			Пусть $f(x) := \ln x$. В силу вогнутости $\forall a, b \geq 1$
   			\begin{equation*}
   				\dfrac{f(a) + f(b)}{2} \leq \dfrac{1}{b - a} \int_a^b f(x) \dd x \leq f\left(\dfrac{a + b}{2}\right).
   			\end{equation*}

   			Также отметим, что по Теореме Лагранжа найдутся точки $\xi_1 \in \left(a; \dfrac{a + b}{2}\right)$ и $\xi_2 \in \left(\dfrac{a + b}{2}; b\right)$, для которых
   			\[\begin{gathered}
   				f \left(\dfrac{a + b}{2}\right) - f(a) = f'(\xi_1) \cdot \left(\dfrac{a + b}{2} - a\right) \\
   				f \left(\dfrac{a + b}{2}\right) - f(b) = f'(\xi_2) \cdot \left(\dfrac{a + b}{2} - b\right) \\
   			\end{gathered}\]
   			Складываем оба равенства, делим на $2$:
   			\begin{equation*}
   				f \left(\dfrac{a + b}{2}\right) - \dfrac{f(a) + f(b)}{2} = \dfrac{b - a}{4} (f'(\xi_1) - f'(\xi_2)) = \dfrac{b - a}{4} f''(\xi) \cdot (\xi_1 - \xi_2) \leq \dfrac{(b - a)^2}{4} \max\limits_{x \in [a; b]} |f''(x)|.
   			\end{equation*}

   			В нашем случае,
   			\begin{equation*}
   				\dfrac{\ln(n - 1) + \ln n}{2} \leq \int_{n - 1}^n \ln x \dd x \leq \dfrac{\ln( n- 1) + \ln n}{2} + \dfrac{1}{4(n - 1)^2}.
   			\end{equation*}

   			Таким образом, последовательность
   			\begin{equation*}
   				S_N := \int_1^N \ln x \dd x - \sum_{n = 2}^N \ln n + \dfrac{1}{2} \ln N
   			\end{equation*}
   			монотонна и ограничена, а значит имеет предел. Тем самым, предел существует и у последовательности
   			\begin{equation*}
   				e^{-S_N} = \dfrac{N!}{(N / e)^N \sqrt{N}}
   			\end{equation*}
   		\end{proof}

   		\subsection{Вычисление константы в формуле Стирлинга}

   		\begin{theorem*}
   			Число $c$ в формуле Стирлинга равно $\sqrt{2\pi}$.
   		\end{theorem*}

   		\begin{proof}
   			Пусть
   			\begin{equation*}
   				I_k = \int_0^{\pi/2} \sin^k x \dd x.
   			\end{equation*}

   			Легко проверить, что
   			\begin{equation*}
   				I_{2n} = \dfrac{2n - 1}{2n}I_{2n - 2} = \dfrac{\pi}{2} \cdot \dfrac{(2n - 1)!!}{(2n)!!};
   				\quad
   				I_{2n + 1} = \dfrac{2n}{2n + 1}I_{2n - 1} = \dfrac{(2n)!!}{(2n + 1)!!}.
   			\end{equation*}

   			Кроме того,
   			\begin{equation*}
   				1 \geq \dfrac{I_{2n + 1}}{I_{2n}} \geq \dfrac{2n}{2n + 1}.
   			\end{equation*}

   			Тем самым,
   			\begin{equation*}
   				\lim_{n \to \infty} \dfrac{((2n)!!)^2}{(2n + 1)!!(2n - 1)!!} = \dfrac{\pi}{2}.
   			\end{equation*}

   			Но
   			\begin{equation*}
   				\dfrac{((2n)!!)^2}{(2n + 1)!!(2n - 1)!!} = \dfrac{((2n)!!)^4}{(2n + 1)((2n)!)^2} = \dfrac{(2^n n!)^4}{(2n + 1)((2n)!)^2} \sim \dfrac{2^{4n}c^4n^2(n / e)^{4n}}{c^2(2n + 1)2n((2n)/e)^{4n}}.
   			\end{equation*}

   			Таким образом,
   			\begin{equation*}
   				\lim_{n \to \infty} \dfrac{c^2n}{2(2n + 1)} = \dfrac{\pi}{2},
   			\end{equation*}
   			откуда следует нужное равенство.
   		\end{proof}

    \section{Несобственный интеграл Римана: определение и примеры. Регулярность и линейность несобственного интеграла, независимость сходимости интеграла от его <<начала>>. Формула интегрирования по частям и замены переменной для несобственного интеграла.}

    	\subsection{Несобственный интеграл Римана: определение}

    	\begin{definition*}
    		Пусть $f$ интегрируема на каждом отрезке $[a; x]$ при $x < b$ $(b \in (-\infty; +\infty])$.
    		Говорят, что несобственный интеграл
    		\begin{equation*}
    			\int_a^b f(t) \dd t
    		\end{equation*}
    		\textbf{сходится}, если существует предел
    		\begin{equation*}
    			\lim_{x \to b - 0} \int_a^x f(t) \dd t.
    		\end{equation*}

    		В этом случае значение несобственного интеграла полагают равным значению данного предела. В противном случае (если предела не существует) говорят, что несобственный интеграл \textbf{расходится}.

    		Аналогично определяется несобственный интеграл с особенностью в нижнем пределе интегрирования.
    	\end{definition*}

    	\subsection{Примеры}

    	Рассмотрим функцию $f_p(x) := \dfrac{1}{x^p}$. Тогда,
    	\begin{equation*}
    		\int_1^x f_p(t) \dd t = \begin{cases}
    			\dfrac{1}{1 - p} \cdot (x^{1 - p} - 1), &p \neq 1, \\
    			\ln x, &p = 1.
    		\end{cases}
    	\end{equation*}

    	Предел при $x \to \infty$ существует тогда и только тогда, когда $p > 1$.

    	С другой стороны,
    	\begin{equation*}
    		\int_x^1 f_p(t) \dd t = \begin{cases}
    			\dfrac{1}{1 - p} \cdot (1 - x^{1 - p}), &p \neq 1, \\
    			-\ln x, &p = 1.
    		\end{cases}
    	\end{equation*}
    	Предел при $x \to 0$ существует тогда и только тогда, когда $p < 1$ (т.к. иначе из-за $x^{1 - p}$ появляется произведение бесконечно большой последовательности на ограниченную).

    	\subsection{Регулярность и линейность несобственного интеграла, независимость сходимости интеграла от его <<начала>>}

    	\begin{theorem*}
    		Пусть $f$, $g$ интегрируемы на каждом отрезке $[a; x]$ при $x < b$ и пусть для них определены несобственные интегралы на промежутке $[a; b)$. Тогда
    		\begin{enumerate}
    		\item
    			если $b \in \RR$ и $f$ интегрируема на $[a; b]$, то значение несобственного интеграла на промежутке $[a; b)$ совпадает со значение обычного интеграла Римана по отрезку $[a; b]$.

    		\item
    			функция $\alpha f + \beta g$ интегриуема в несобственном смысле на промежутке $[a; b)$ и
    			\begin{equation*}
    				\int_a^b (\alpha f + \beta g) \dd x = \alpha \int_a^b f(x) \dd x + \beta \int_a^b g(x) \dd x
    			\end{equation*}

    		\item
    			если $c \in [a; b)$, то
    			\begin{equation*}
    				\int_a^b f(x) \dd x = \int_a^c f(x) \dd x + \int_c^b f(x) \dd x
    			\end{equation*}
    		\end{enumerate}
    	\end{theorem*}

    	\begin{remark*}
    		Последнее в частности означает, что интегралы
    		\begin{equation*}
    			\int_a^b f(x) \dd x 
    			\text{ и }
    			\int_c^b f(x) \dd x
    		\end{equation*}
    		сходятся или расходятся одновременно.
    	\end{remark*}

        \subsection{Формула интегрирования по частям и замены переменной для несобственного интеграла}

        \begin{theorem*}
            Пусть $f$ непрерывна на $[a; b)$, $\phi: [\alpha; \beta) \to [a; b)$ --- непрерывно дифференцируемое отображение, $\phi(\alpha) = a, \phi(t) \to b$ при $t \to \beta$. Тогда функция $t \mapsto f(\phi(t))\phi'(t)$ интегрируема в несобственном смысле на промежутке $[\alpha; \beta)$ и
            \begin{equation*}
                \int_a^b f(x) \dd x = \int_{\alpha}^{\beta} f(\phi(t))\phi'(t) \dd t
            \end{equation*}
        \end{theorem*}

        \begin{proof}
            По формуле замены переменной для произвольного $T < \beta$ выполнено
            \begin{equation*}
                \int_{\alpha}^T f(\phi(t))\phi'(t) \dd t = \int_a^{\phi(T)} f(x) \dd x
            \end{equation*}

            /\!/TODO: Добавить нормальное пояснение
        \end{proof}  

        \begin{theorem*}
            Пусть $f$, $g$ непрерывно дифференцируемы на $[a; b)$ и существует предел
            \begin{equation*}
                \lim_{x \to b - 0} f(x)g(x).
            \end{equation*}

            Тогда функции $f'g$ и $fg'$ одновременно интегрируемы или не интегрируемы в несобственном смысле на $[a; b)$, и в случае интегрируемости
            \begin{equation*}
                \int_a^b f(t)g'(t) \dd t = \lim_{x \to b - 0} f(x) g(x) - f(a)g(a) - \int_a^b f(t)g'(t) \dd t.
            \end{equation*}
        \end{theorem*}      

        \begin{proof}
            Утверждение следует из формулы интегрирования по частями для обычного интеграла Римана.

            Заметим, что сходимость интеграла от функции $f$ на промежутке $[a; b)$ равносильна существованию предела для функции
            \begin{equation*}
                F(X) := \int_a^x f(t) \dd t
            \end{equation*}
            при $x \to b - 0$.

            Тем самым, для сходимости интеграла верен критерий Коши: пусть $f$ интегрируема на каждом отрезке $[a; x]$ при $x < b$. Тогда интеграл
            \begin{equation*}
                \int_a^b f(t) \dd t
            \end{equation*}
            сходится тогда и только тогда, когда $\forall \eps > 0 \ \exists x_0: \forall x_1, x_2 \in (x_0, b)$ выполнено
            \begin{equation*}
                \left|\int_{x_1}^{x_2} f(t) \dd t\right| < \eps.
            \end{equation*}
        \end{proof}

    \section{Абсолютная и условная сходимость несобственных интегралов. Пример функции, интеграл от которой сходится условно. Исследования сходимости интеграла от неотрицательной функции с помощью неравенств и эквивалентности. Признаки Дирихле-Абеля сходимости несобственного интеграла.}

        \subsection{Абсолютная и условная сходимость несобственных интегралов}

        \begin{definition*}
            Говорят, что несобственный интеграл $\int_a^b f(t) \dd t$ сходится \textbf{абсолютно}, если сходится интеграл $\int_a^b |f(t)| \dd t$.
        \end{definition*}

        \begin{remark*}
            В силу критерия Коши ясно, что абсолютно сходящийся интеграл сходится.

            /\!/TODO: разобраться и пояснить.
        \end{remark*}

        \begin{remark*}
            Исследования абсолютной сходимости сводится к исследования сходимости интеграла от неотрицательной функции. В случае неотричательной функции $f$ функциия $F$ оказывается монотонной, поэтому сходимость интеграла от неотрицательной функции равносильна ограниченности $F$ на $[a; b)$.
        \end{remark*}

        \begin{proposal*}
            Пусть $f$, $g$ интегрируемы на каждом отрезке $[a; x]$ при $x < b$ и $0 \leq f(x) \leq g(x)$ при $x \in [a; b)$. Тогда из сходимости интеграла для $g$ следует сходимость интеграла для $f$, а из расходимости интеграла для $f$ следует расходимость интеграла для $g$.
        \end{proposal*}

        \begin{proof}
            Следует из оценки
            \begin{equation*}
                \int_a^x f(t) \dd t \leq \int_a^x g(t) \dd t
            \end{equation*}
            при $x \in [a; b)$ и неотрицательности обеих функций.
        \end{proof}

        \begin{consequence*}
            Если $f \geq 0$ и $f(x) \sim g(x)$ при $x \to b - 0$, то интегралы от функций $f$ и $g$ сходятся или расходятся одновременно.
        \end{consequence*}

        \begin{proof}
            Найдется такое число $x_0 \in [a; b)$, что $\dfrac{1}{2}g(x) \leq f(x) \leq \dfrac{3}{2}g(x)$ при $x \in (x_0; b)$.
        \end{proof}

        \begin{definition*}
            Говорят, что несобственный интеграл $\int_a^b f(t) \dd t$ сходится условно, если сам интеграл сходится, но не сходится абсолютно.
        \end{definition*}

        \subsection{Примеры}

        \begin{equation*}
            \left.\int_1^{\infty} \dfrac{\sin x}{x} \dd x = - \dfrac{\cos x}{x} \right|_1^{\infty} - \int_1^{\infty} \dfrac{\cos x}{x^2} \dd x,
        \end{equation*}
        последний интеграл сходится абсолютно. В то же время,
        \begin{equation*}
            \int_1^{\infty} \left|\dfrac{\sin x}{x}\right| \dd x \geq \int_1^{\infty} \dfrac{\sin^2 x}{x} \dd x = \int_1^{\infty} \left( \dfrac{1}{2x} - \dfrac{\cos 2x}{2x} \right)\dd x,
        \end{equation*}
        где первый интеграл не сходится, а второй сходится, а значит и интеграл от суммы сходиться не может.

        \subsection{Признак Дирихле-Абеля}

        \begin{theorem*}
            Пусть $f$ и $g$ интегрируемы на каждом отрезке $[a; x]$ при $x < g$, $f$ --- непрерывная функция, $g$ --- монотонная, непрерывно дифференцируемая функция. Пусть
            \begin{enumerate}
            \item
                функция $F(x) := \int_a^x f(t) \dd t$ ограничена, а $g(x) \to 0$ при $x \to b - 0$

            \textbf{или}

            \item
                интеграл $\int_a^b f(t) \dd t$ сходится, а $g$ --- ограниченная функция
            \end{enumerate}

            Тогда интеграл $\int_a^b f(x)g(x) \dd x$ сходится.
        \end{theorem*}

        \begin{proof}
            Заметим, что
            \begin{equation*}
                \int_{x_1}^{x_2} f(t) g(t) \dd t = (F(x_2) - F(x_1))g(x_2) - \int_{x_1}^{x_2} (F(t) - F(x_1))g'(t) \dd t.
            \end{equation*}

            Тогда
            \begin{equation*}
                \left|\int_{x_1}^{x_2} f(t) g(t) \dd t\right| \leq |g(x_2)| \cdot |F(x_2) - F(x_1)| + \sup\limits_{t \in [x_1; x_2]} |F(t) - F(x_1)| \cdot \int_{x_1}^{x_2} |g'(t)| \dd t.
            \end{equation*}

            Так как $\int_{x_1}^{x_2} |g'(t)| \dd t = |g(x_2) - g(x_1)|$, то
            \begin{equation*}
                \left|\int_{x_1}^{x_2} f(t) g(t) \dd t\right| \leq (2 \cdot |g(x_2)| + |g(x_1)|) \cdot \sup\limits_{t \in [x_1; x_2]} |F(t) - F(x_1)|.
            \end{equation*}

            При каждом из наших предположений, последнее выражение мало при больших $x_1$, $x_2$.
        \end{proof}
        
\end{document}