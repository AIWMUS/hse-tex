\section{Лекция 23.11.2017}

\subsection{Векторное пространство}

\vspace{\baselineskip}
Пусть $F$ -- поле (например, $F = \RR$ или $F = \CC$).

\vspace{\baselineskip}
\textbf{Определение.} Множество $V$ называется \textit{векторным (линейным) пространством над полем $F$}, если на $V$ заданы две операции: "сложение" ($V \times V \rightarrow V$, $(a,b) \rightarrow a+b$) и "умножение на скаляр" ($F \times V \rightarrow V$, $(\alpha, X) \rightarrow \alpha x$), причем $\forall \ x, y, z \in V$ и $\alpha, \beta \in F$ выполнены следующие свойства (называемые \textit{аксиомами векторного пространства}):

(1) $x + y = y + x$ (коммутативность сложения)

(2) $(x+y)+z = x+(y+z)$ (ассициативность сложения)

(3) $\exists \overrightarrow{0} \in V: \overrightarrow{0} + x = x + \overrightarrow{0} = x \ \forall \ x \in v$ (нулевой элемент)

(4) $\exists -x \in v: x+(-x)=(-x)+x=0$ (противоположный элемент)

$\uparrow$ абелева группа $\uparrow$

(5) $(\alpha + \beta) x = \alpha x + \beta x$

(6) $\alpha (x + y) = \alpha x + \alpha y$

(7) $(\alpha \beta) x  = \alpha (\beta x)$

(8) $ 1 \cdot x = x $ (единица)

\vspace{\baselineskip}
\textbf{Определение.} Элементы векторного пространства называются (абстрактными) \textit{вЕкторами}.

\vspace{\baselineskip}
Примеры. 

1) $\RR$ над $\RR$ (или $F$ над $F$)

2) $\RR^n$ над $\RR$ (или $\FF^n$ над $\FF$), реализованное как пространство столбцов (или иногда пространство строк)

3) $Mat_{m \times n} (F)$ над $F$

4) Пространство функций $f: M \rightarrow \RR$ (где $M$ -- некоторое множество): сложение $(f_1 + f_2)(x) := f_1(x) + f_2(x)$, умножение на скаляр $(\alpha f)(x) := \alpha f(x)$. Это векторное пространство над $\RR$

5) $F[x]$, $F[x]_{\leq n}$ -- векторное пространство над $F$

\vspace{\baselineskip}
\textbf{\textit{Простейшие следствия из аксиом:}}

1) элемент $\overrightarrow{0}$ единственный: если $\overrightarrow{0'}$ -- другой такой, что $0' = 0' + 0 = 0$. 

2) элемент $-x$ единственный: если (-x)' другой, такой что $(-x)' = (-x)' + 0 = (-x)' + (x + (-x)) = ((-x)' + x) + (-x) = 0 + (-x) = -x$

3) $\alpha 0 = 0 \ \forall \alpha \in F$

4) $\alpha (-x) = -(\alpha x)$

5) $0 \cdot x = 0 \ \forall x \in V$

6) $(-1)x = -x \ \forall x \in V$

\vspace{\baselineskip}
\textbf{Определение.} Подмножество $U$ векторного пространства $V$ называется \textit{подпространством}, если

1) $0 \in U$

2) $x, y \in U \Rightarrow x + y \in U$

3) $x \in U, \alpha \in F \Rightarrow \alpha x \in U$

\vspace{\baselineskip}
\textbf{Замечание.} Подпространство само является векторным пространством.

\vspace{\baselineskip}
Примеры.

1) $\{0\}$ и $V$ - всегда подпространства, они называются \textit{несобственными подпространствами}.

2) Множество всех диагональных (верхнетреугольных, нижнетреугольных) матриц в $M_n (F)$ -- подпространство

\vspace{\baselineskip}
\textbf{Предложение.} Множество решений однородной системы линейных уравнений $Ax = 0$ является подпространством в $F^n$.

\vspace{\baselineskip}
\textbf{\textit{Доказательство.}} $\rhd$ Пусть $S \subseteq F^n$ -- множество решений ОСЛУ $Ax = 0$

1) $0 = \begin{pmatrix} 0 \\ 0 \\ \dots \\ 0 \end{pmatrix}, A0 = 0 \Rightarrow 0 \in S$

2) $x, y \in S \Rightarrow Ax = 0$ и $Ay = 0$. $A(x+y) = Ax + Ay = 0 + 0 = 0 \Rightarrow x + y \in S$

3) $x \in S, \alpha \in F \Rightarrow A(\alpha x) = \alpha (Ax) = \alpha 0 = 0 \Rightarrow \alpha x \in S \ \lhd$.

\vspace{\baselineskip}
Пусть $V$ -- векторное пространство над полем $F$, $v_1, v_2, \dots, v_n$ --  набор векторов

\vspace{\baselineskip}
\textbf{Определение.} Всякий вектор $v \in V$ вида $v = \alpha_1 v_1 + \dots + \alpha_m x_m$ $(\alpha_i \in F)$ называется \textit{линейной комбинацией} векторов $v_1, v_2, \dots, v_n$.

\vspace{\baselineskip}
Пусть $S \subseteq V$ -- произвольное подмножество

\vspace{\baselineskip}
\textbf{Определение.} Множество всевозможных линейных комбинаций (конечных) векторов из $S$ называется \textit{линейной оболочкой} множества $S$.

\textit{Обозначение:} $<S>$

\vspace{\baselineskip}
Если $S = \{v_1, \dots, v_n\}$, то пишут просто $<v_1, \dots, v_n>$ и говорят "линейная оболочка векторов $v_1, \dots, v_n$".

\vspace{\baselineskip}
\textbf{Соглашение.} $<\varnothing> = \{0\}$

\vspace{\baselineskip}
Примеры.

0) $<\varnothing> = \{0\}$

1) $V \in \RR^2, v \neq 0 \Rightarrow <v>$ -- прямая, натянутая на $v$

2) $V = \RR^3, v_1, v_2$ не коллинеарны $\Rightarrow <v_1, v_2>$ -- плоскость, натянутая на $v_1, v_2$

3) $V = \RR^n, e_1 = \begin{pmatrix} 1 \\ 0 \\ \dots \\ 0 \end{pmatrix}, e_1 = \begin{pmatrix} 0 \\ 1 \\ \dots \\ 0 \end{pmatrix}, \dots, e_n = \begin{pmatrix} 0 \\ 0 \\ \dots \\ 1 \end{pmatrix}$, $<e_1, \dots, e_n> = \RR^n$

$x = \begin{pmatrix} x_1 \\ x_2 \\ \dots \\ x_n \end{pmatrix} \in \RR^n \Rightarrow x = x_1 e_1 + x_2 e_2 + \dots + x_n e_n$

\vspace{\baselineskip}
\textbf{Предложение.} $S \subseteq V$ -- подмножество $\Rightarrow <S>$ -- подпространство в $V$

1) Если $S = \varnothing$, то $<S> = \{0\} \Rightarrow 0 \in <S>$

Если $S \neq \varnothing$, то $\exists v \in S \Rightarrow 0 = 0 v \in <S>$

2) $x, y \in <S> \Rightarrow x = \alpha_1 v_1 + \dots + \alpha_m v_m$ и $y = \beta_1 w_1 + \dots + \beta_n w_n$, где $\alpha_i, \beta_i \in F$, $v_1, \dots, v_m, w_1, \dots, w_n \in S \Rightarrow x + y = \alpha_1 v_1 + \dots + \alpha_m v_m + \beta_1 w_1 + \dots + \beta_n w_n$ -- линейная комбинация векторов $v_1, \dots, v_m, w_1, \dots, w_n \in S$

3) $x \in <S>, \alpha \in F$

$x = \alpha_1 v_1 + \dots + \alpha_m v_m$, $\alpha_i \in F, v_i \in S$

$\alpha x = (\alpha \alpha_1)v_1 + \dots + (\alpha \alpha_m) v_m \in <S>$

\vspace{\baselineskip}
\textbf{Замечание.} Еще говорят, что $<S>$ -- подпространство, натянутое на $S$, или подпространство, порожденное множеством $<S>$.

\vspace{\baselineskip}
\textbf{Определение.} Линейная комбинация $\alpha_1 v_1 + \dots + \alpha_m v_m$ называется \textit{тривиальной}, если $\alpha_1 = \dots = \alpha_m = 0$, и \textit{нетривиальной} иначе (т.е. $\exists i: \alpha_i \neq 0$).

\vspace{\baselineskip}
\textbf{Определение.} Векторы $v_1, \dots, v_m \in V$ называются \textit{линейно зависимыми}, если $\exists$ их нетривиальная линейная комбинация, равная 0 (т.е. $\exists (\alpha_1, \dots, \alpha_m) \neq (0, \dots, 0)$, такая что $\alpha_1 v_1 + \dots + \alpha_m v_m = 0$), и \textit{линейно независимыми} иначе (т.е. из условия $\alpha_1 v_1 + \dots + \alpha_m v_m = 0$ следует $\alpha_1 = \dots = \alpha_m = 0$).

\vspace{\baselineskip}
Примеры.

0) $\overrightarrow{0}$ линейно зависим, $1 \cdot \overrightarrow{0} = \overrightarrow{0}$

1) $v \neq 0 \Rightarrow v$ линейно зависим

Пусть $\lambda v = 0$ при $\lambda \neq 0$

$0 = \lambda^{-1} 0 = \lambda^{-1} (\lambda v) = (\lambda^{-1} \lambda) v = 1 \cdot v = v$ -- противоречие

2) $v_1, v_2 \in V \Rightarrow v_1, v_2$ линейно зависимы $\Rightarrow v_1, v_2$ пропорциональны, т.е. либо $v_1 = \lambda v_2$, либо $v_2 = \mu v_1, \lambda, \mu \in F$

\textbf{\textit{Доказательство.}} $\rhd \ v_1 = \lambda v_2 \Rightarrow 1 \cdot v_1 - \lambda \cdot v_2 = \overrightarrow{0}$ (нетривиальная линейная комбинация), аналогично при $v_2 = \mu v_1$.

$\lambda_1 v_1 + \lambda_2 v_2 = \overrightarrow{0}$ (нетривиальная линейная комбинация). Если $\lambda_1 \neq 0$, то $v_1 = -\frac{\lambda_2}{\lambda_1} v_2$, аналогично при $\lambda_2 \neq 0 \ \lhd$.

\vspace{\baselineskip}
3) $\{v, \dots, -v\}$ -- линейно зависимы, т.к. $1 \cdot v + 1 \cdot (-v) = \overrightarrow{0}$.

4) $S' \subseteq S, S'$ линейно зависимо $\Rightarrow S$ тоже линейно зависимо.

Если $S$ линйно независимо, то $S'$ тоже линейно независимо.

