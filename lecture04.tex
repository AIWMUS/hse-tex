\section{Лекция 30.09.2017}

\subsection{Свойства умножения матриц}

\vspace{\baselineskip}
1) $A (B + C) = AB + AC$ (левая дистрибутивность)

\vspace{\baselineskip}
\textbf{\textit{Доказательство.}}

$\rhd$ размеры: $m \times n \ ( n \times p + n \times p) = m \times n \ n \times p + m \times n \ n \times p = m \times p$

$x_{ij} = A_{(i)} (B + C)^{(j)} = \sum\limits_{k=1}^n a_{ik} (b_{kj} + c_{kj}) = \sum\limits_{k=1}^n (a_{ik} b_{kj} + a_{ik} c_{kj}) = \sum\limits_{k=1}^n a_{ik} b_{kj} + \sum\limits_{k=1}^n a_{ik} c_{kj} = (AB)_{ij} + (AC)_{ij} = y_{ij} \Rightarrow X = Y$. $\lhd$

\vspace{\baselineskip}
1') $(A + B)C = AC + BC$ (правая дистрибутивность)

\vspace{\baselineskip}
2) $\lambda \in \RR \Rightarrow \lambda(AB) = (\lambda A) B = A(\lambda B)$

\vspace{\baselineskip}
3) A(BC) = (AB)C (ассоциативность)

\vspace{\baselineskip}
\textbf{\textit{Доказательство.}}

$\rhd$A(BC) = AU = X, (AB)C = VC = Y.

$x_{ij} = \sum\limits_{k=1}^n a_{ik} u_{kj} = \sum\limits_{k=1}^n a_{ik} (\sum\limits_{l=1}^p b_{kl} c_{lj}) = \sum\limits_{k=1}^n (\sum\limits_{l=1}^p a_{ik} b_{kl} c_{lj}) = \sum\limits_{l=1}^p (\sum\limits_{k=1}^n a_{ik} b_{kl} c_{lj}) = \sum\limits_{l=1}^p(\sum\limits_{k=1}^n (a_{ik} b_{kl}) c_{lj} = \sum\limits_{l=1}^p v_{il} c_{lj} = y_{ij} \Rightarrow X = Y. \lhd $

\vspace{\baselineskip}
4) $(AB)^T = B^T A^T$

\vspace{\baselineskip}
\textbf{\textit{Доказательство.}} 

$ \rhd (AB)^T = X, B^T = C, A^T = D, B^T A^T = Y$

размеры: $m \times n \ n \times p = p \times n \ n \times m = p \times m$

$x_{ij} = (AB)_{ji} = \sum\limits_{k=1}^n a_{jk} b_{ki} = \sum\limits_{k=1}^n d_{kj} c_{ik} = \sum\limits_{k=1}^n c_{ik} d{kj} = y_{ij} \Rightarrow X = Y \lhd$.

\begin{comment}
	Умножение матриц не обладает свойством коммутативности.
\end{comment}

Пример.
$A = \begin{pmatrix} 0 & 1 \\ 0 & 0 \end{pmatrix}, B = \begin{pmatrix} 0 & 0 \\ 1 & 0 \end{pmatrix}$

$AB = \begin{pmatrix} 1 & 0 \\ 0 & 0 \end{pmatrix}$

$BA = \begin{pmatrix} 0 & 0 \\ 0 & 1 \end{pmatrix}$
$\Rightarrow AB \neq BA$

\vspace{\baselineskip}
\subsection{Приложение к СЛУ}

СЛУ $Ax = b (*), A \in Mat_{m \times n}, x \in \RR^n$ -- столбец неизвестных, $b \in \RR^m$

\vspace{\baselineskip}
Если СЛУ (*) совместна, то ее \textit{частное решение} -- какое-либо одно ее решение.

\vspace{\baselineskip}
\textbf{Утверждение.} Пусть $Ax = b$ (*) -- совместная СЛУ, $Ax = \begin{pmatrix} 0 \\ 0 \\ \dots \\ 0 \end{pmatrix}$ (**) -- соответствующая ОСЛУ.

Пусть $L \subseteq  \RR^n$ -- множество решений СЛУ(*), $S \subseteq \RR^n$ -- частное решение ОСЛУ(**), $x_0 \in L$ -- частное решение СЛУ(*).

Тогда $L = x_0 + S$, где $x_0 + S = \{ x_0 + v \ | \ v \in S \}$.

\vspace{\baselineskip}
Пример.

\vspace{\baselineskip}
$\begin{pmatrix} 1 & 2 & 0 & -3 & \vline & 4 \\ 0 & 0 & 1 & 5 & \vline & 6 \end{pmatrix}$

\vspace{\baselineskip}
$x_1 = 4 - 2x_2 + 3x_4$

$x_3 = 6 - 5x_4$

$x_2 = t_1, x_4 = t_2$ -- параметры

\begin{equation*} \begin{pmatrix} x_1 \\ x_2 \\ x_3 \\ x_4 \end{pmatrix} = \begin{pmatrix} 4 - 2t_1 + 3t_2 \\ t_1 \\ 6 - 5t_2 \\ t_2 \end{pmatrix} = \begin{pmatrix} 4 \\ 0 \\ 6 \\ 0 \end{pmatrix} + t_1 \begin{pmatrix} -2 \\ 1 \\ 0 \\ 0 \end{pmatrix} + t_2 \begin{pmatrix} 3 \\ 0 \\ -5 \\ 1 \end{pmatrix} \end{equation*}, где $\begin{pmatrix} 4 \\ 0 \\ 6 \\ 0 \end{pmatrix}$ -- частное решение, а $ t_1 \begin{pmatrix} -2 \\ 1 \\ 0 \\ 0 \end{pmatrix} + t_2 \begin{pmatrix} 3 \\ 0 \\ -5 \\ 1 \end{pmatrix} $ -- решение ОСЛУ.

\vspace{\baselineskip}
\textbf{\textit{Доказательство.}}

$\rhd x_0$ -- частное решение $\Rightarrow Ax_0 = b$.

1) Пусть $u \in L$, тогда Au = b. Положим $v = u - x_0$. Тогда $u = x_0 + v$.

$Av = A(u - x_0) = Au - Ax_0 = b - b = \begin{pmatrix} 0 \\ \dots \\ 0  \end{pmatrix} \Rightarrow v \in S \Rightarrow L \subseteq x_0 + S$.

2) $v \in S \Rightarrow Av = \begin{pmatrix} 0 \\ \dots \\ 0  \end{pmatrix} \Rightarrow A(x_0 + v) = A(x_0 + v) = Ax_0 + Av = b + \begin{pmatrix} 0 \\ \dots \\ 0  \end{pmatrix} = b \Rightarrow x_0 + v \in L \Rightarrow x_0 + S \subseteq L$.

$1) \& 2) \Rightarrow L = x_0 + S$.

\subsection{}
\vspace{\baselineskip}
\textbf{Определение.} Матрица $A \in Mat_{m \times n}$ называется \textit{квадратной}, если m = n. При этом число n называется \textit{порядком этой матрицы}.

Обозначение: $Mat_{n \times m} =: M_n$

\vspace{\baselineskip}
$A \in M_n$ 

$ \begin{pmatrix} \Diamond & & & \\ & \Diamond & & \\ & & \Diamond & \\ & & & \Diamond \end{pmatrix} $ -- главная диагональ


$ \begin{pmatrix} & & & \Diamond \\ & & \Diamond & \\ & \Diamond & & \\ \Diamond & & & \end{pmatrix} $ -- побочная диагональ

\vspace{\baselineskip}
\textbf{Определение.} Матрица $A \in M_n$ называется \textit{диагональной}, если все элементы вне ее главной диагонали равны 0 (т.е. $a_{ij} = 0 \ \forall i \neq j$). 

\begin{equation*}A =  \begin{pmatrix} a_1 & 0 & \dots & 0 \\ 0 & a_2 & \dots & 0  \\ 0 & 0 & \dots & 0 \\ \dots & \dots & \dots & \dots \\ 0 & 0 & \dots & a_n \end{pmatrix} = diag(a_1, a_2, \dots , a_n) \end{equation*}

\begin{comment}
	В дальнейшем понятие \glqq диагональной матрицы\grqq \  будет использоваться также и для неквадратных матриц, условие то же: $a_{ij} = 0 \ \forall  i \neq j$
\end{comment}

\begin{equation*}A =  \begin{pmatrix} a_1 & 0 & \dots & \dots & 0 \\ 0 & a_2 & \dots & \dots & 0  \\ \dots & \dots & \dots & \dots & \dots \\ 0 & 0 & \dots & a_n & 0 \end{pmatrix} = diag(a_1, a_2, \dots , a_n) \end{equation*}

\begin{equation*}A =  \begin{pmatrix} a_1 & 0 & \dots & 0 \\ 0 & a_2 & \dots & 0  \\ \dots & \dots & \dots & \dots \\ 0 & 0 & \dots & a_n \\ 0 & 0 & 0 & 0 \end{pmatrix} = diag(a_1, a_2, \dots , a_n) \end{equation*}

\vspace{\baselineskip}
\textbf{Лемма.} Пусть $A = diag(a_1, a_2, \dots, a_n) \in M_n$.

\vspace{\baselineskip}
(1) Если $B \in Mat_{n \times p}$, то AB = $\begin{pmatrix} a_1 B_{(1)} \\ a_2 B_{(2)} \\ \dots \\ a_n B_{(n)} \end{pmatrix} $(т.е. i-я строка унмножается на $a_i$).

\vspace{\baselineskip}
(2) Если $B \in Mat_{m \times n}$, то $BA = (a_1 B^{(1)} \ a_2 B^{(2)} \dots a_n B^{(n)})$ (т.е. i-ый столбец умножается на $a_i$).

\textbf{\textit{Доказательство.}}

$\rhd$ (1)

\begin{equation*}\begin{pmatrix} a_1 & 0 & 0 & \dots & 0 \\
0 & a_2 & 0 & \dots & 0 \\
0 & 0 & a_3 & \dots & 0 \\
\dots & \dots & \dots & \dots & \dots \\
0 & 0 & 0 & \dots & a_n 
\end{pmatrix} \begin{pmatrix} b_{11} & b_{12} & b_{13} & \dots & b_{1p} \\
b_{21} & b_{22} & b_{23} & \dots & b_{2p} \\
\dots & \dots & \dots & \dots & \dots \\
b_{n1} & b_{n2} & b_{n3} & \dots & b_{np}
\end{pmatrix} = \begin{pmatrix} a_1 b_{11} & a_1 b_{12} & a_1 b_{13} & \dots & a_1 b_{1p} \\
a_2 b_{21} & a_2 b_{22} & a_2 b_{23} & \dots & a_2 b_{2p} \\
\dots & \dots & \dots & \dots & \dots \\
a_n b_{n1} & a_n b_{n2} & a_n b_{n3} & \dots & a_n b_{np}
\end{pmatrix} = \end{equation*}
\begin{equation*} = \begin{pmatrix} a_1 B_{(1)} \\ a_2 B_{(2)} \\ \dots \\ a_n B_{(n)} \end{pmatrix} \end{equation*}
(2) Аналогично. $\lhd$

\vspace{\baselineskip}
\textbf{Определение.} Матрица $E = E_n = diag(1, 1, \dots, 1)$ называется \textit{единичной матрицей порядка n}.
\begin{equation*} E = \begin{pmatrix} 1 & 0 & \dots & 0 \\ 
0 & 1 & \dots & 0 \\
\dots & \dots & \dots & \dots \\
0 & 0 & \dots & 1 
\end{pmatrix} \end{equation*}

\vspace{\baselineskip}
\textbf{Следствия из леммы}

1) $EA = A \ \forall \ A \in Mat_{n \times p}$

2) $AE = A \ \forall \ A \in Mat_{m \times n}$

3) $AE = EA = A \ \forall \ A \in Mat_n$

\vspace{\baselineskip}
\textbf{Утверждение.} Всякое элементарное преобразование строк матрицы $A \in Mat_{m \times n}$ представимо при помощи умножения слева на подходящую матрицу. А именно:

$Э_1(i, j, \lambda): A \rightarrow U_1(i, j, \lambda) A$, где
\begin{equation*} U_1(i, j, \lambda) = \bordermatrix{ 
    	 & & & & & & j & &  \cr
    	 & 1 & \cdots & 0 & 0 & \cdots & \cdots & 0 \cr 
          & 0 & 1 & 0 & 0 & \cdots & \cdots & 0 \cr
		i & 0 & \cdots & 1 & 0 & 0 & \lambda & 0  \cr
         & 0 & \cdots & 0 & 1 & 0 & 0 & 0  \cr
        & 0 & 0 & 0 & \cdots & 1 & 0 & 0  \cr
        & 0 & 0 & 0 & \cdots & 0 & 1 & 0  \cr
        & 0 & 0 & 0 & \cdots & 0 & 0 & 1 }
\end{equation*}
(на диагонали стоят единицы, на $i$--ом $j$--ом месте стоит $\lambda$, остальные элементы нули)

\vspace{\baselineskip}
$Э_2(i, j): A \rightarrow U_2(i, j) A$, где
\begin{equation*} U_2(i, j) = \bordermatrix{ 
    	 & & & i & & & j & &  \cr
    	 & 1 & \cdots & 0 & 0 & \cdots & \cdots & 0 \cr 
          & 0 & 1 & 0 & 0 & \cdots & \cdots & 0 \cr
		i & 0 & \cdots & 0 & 0 & 0 & 1 & 0  \cr
         & 0 & \cdots & 0 & 1 & 0 & 0 & 0  \cr
        & 0 & 0 & 0 & \cdots & 1 & 0 & 0  \cr
        j & 0 & 0 & 1 & \cdots & 0 & 0 & 0  \cr
        & 0 & 0 & 0 & \cdots & 0 & 0 & 1 }
\end{equation*}
(на диагонали стоят единицы, кроме $i$--ого и $j$-ого столбца (там нули), на $i$--ом $j$--ом и $j$--ом $i$--ом местах стоят 1, остальные элементы нули)

\vspace{\baselineskip}
$Э_1(i, \lambda): A \rightarrow U_3(i, \lambda) A$, где
\begin{equation*} U_3(i,\lambda) = \bordermatrix{ 
    	 & & & i & & & & &  \cr
    	 & 1 & \cdots & 0 & 0 & \cdots & \cdots & 0 \cr 
          & 0 & 1 & 0 & 0 & \cdots & \cdots & 0 \cr
		i & 0 & \cdots & \lambda & 0 & 0 & 0 & 0  \cr
         & 0 & \cdots & 0 & 1 & 0 & 0 & 0  \cr
        & 0 & 0 & 0 & \cdots & 1 & 0 & 0  \cr
        & 0 & 0 & 0 & \cdots & 0 & 1 & 0  \cr
        & 0 & 0 & 0 & \cdots & 0 & 0 & 1 } = diag 
        \bordermatrix{
        & & & & i & & \cr
        & 1, & \dots, & 1, & \lambda, & 1, & \dots, & 1}
\end{equation*}
(на диагонали стоят единицы, кроме $i$-ого столбца -- там $\lambda$, остальные элементы нули)

\vspace{\baselineskip}
\textbf{\textit{Упражнение на дом:}} Доказательство.

\textbf{\textit{Упражнение на дом:}} Сформулировать и доказать аналогичное утверждение для элементарных преобразований столбцов.

