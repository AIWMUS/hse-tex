\section{Лекция 18.01.2018}

$V$ -- векторное пространство над $F$, $dimV = n$

$(e_1, e_2, \dots, e_n)$ -- фиксированный базис

$(e'_1, e'_2, \dots, e'_n)$ -- какой-то набор из $n$ векторов

\vspace{\baselineskip}
$e'_1 = c_{11} e_1 + c_{21} e_2 + \dots + c_{n1} e_n$

$e'_2 = c_{12} e_1 + c_{22} e_2 + \dots + c_{n2} e_n$

$\vdots$

$e'_n = c_{1n} e_1 + c_{2n} e_2 + \dots + c_{nn} e_n$

\vspace{\baselineskip}
$(e'_1, \dots, e'_n) = (e_1, \dots, e_n) \cdot C$, где 

\begin{equation*} C = \begin{pmatrix} 
c_{11} & c_{12} & \dots & c_{1n} \\
c_{21} & c_{22} & \dots & c_{2n} \\
\dots & \dots & \dots & \dots \\
c_{n1} & c_{n2} & \dots & c_{nn} \end{pmatrix}
\end{equation*}

$C^{(i)}$ -- столбец координат вектора $e'_i$ в базисе $(e_1, \dots, e_n)$.

\vspace{\baselineskip}
\textbf{Предложение.} $(e'_1, \dots, e'_n)$ -- базис в $V \Leftrightarrow C$ невырожденна ($detC \neq 0$).

\vspace{\baselineskip}
\textbf{\textit{Доказательство.}} Пусть $(e'_1, \dots, e'_n)$ -- базис $\Rightarrow \exists C' \in M_n$, такие что $(e_1, \dots, e_n) = (e'_1, \dots, e'_n) \cdot C' = (e_1, \dots, e_n) \cdot C \cdot C'$. Т.к. $e_1, \dots, e_n$ линейно независимы, то отсюда следует $C \cdot C' = E \Rightarrow detC \neq 0$.

Пусть $detC \neq 0$. Покажем, что $(e'_1, \dots, e'_n)$ -- базис. Достаточно доказать, что $e'_1, \dots, e'_n$ линейно независимы. Пусть $\lambda_1, \dots, \lambda_n \in F$ таковы, что $\lambda_1 e'_1 + \dots + \lambda_n e'_n = 0$. Тогда $0 = (e'_1, \dots, e'_n) \begin{pmatrix} \lambda_1 \\ \vdots \\ \lambda_n \end{pmatrix} = (e_1, \dots, e_n) C \begin{pmatrix} \lambda_1 \\ \vdots \\ \lambda_n \end{pmatrix}$. Т.к. $e_1, \dots, e_n$ линейно независимы, то $C \cdot \begin{pmatrix} \lambda_1 \\ \vdots \\ \lambda_n \end{pmatrix} = 0$. Домножая слева на $C^{-1}$, получаем $\begin{pmatrix} \lambda_1 \\ \vdots \\ \lambda_n \end{pmatrix} = 0$.

\vspace{\baselineskip}
Пусть $e = (e_1, \dots, e_n)$ и $e' = (e'_1, \dots, e'_n)$ -- два базиса в $V$.

$(e'_1, \dots, e'_n) = (e_1, \dots, e_n) \cdot C$

\vspace{\baselineskip}
\textbf{Определение.} Матрица $C$ называется \textit{матрицей перехода от базиса $e$ к базису $e'$}.

\vspace{\baselineskip}
\textbf{Замечание.} Матрицей перехода от $e'$ к $e$ -- это $C^{-1}$. 

\vspace{\baselineskip}
$v \in V \rightarrow v = x_1 e_1 + \dots + x_n e_n = (e_1, \dots, e_n) \begin{pmatrix} x_1 \\ \vdots \\ x_n \end{pmatrix}$

$\rightarrow v = x'_1 e'_1 + \dots + x'_n e'_n = (e'_1, \dots, e'_n) \begin{pmatrix} x'_1 \\ \vdots \\ x'_n \end{pmatrix}$

\textbf{Предложение (формула замены координат вектора при смене базиса.} $\begin{pmatrix} x_1 \\ \vdots \\ x_n \end{pmatrix} = C \cdot \begin{pmatrix} x'_1 \\ \vdots \\ x'_n \end{pmatrix}$.

\textbf{\textit{Доказательство.}} $\rhd \ v = (e'_1, \dots, e'_n) \begin{pmatrix} x'_1 \\ \vdots \\ x'_n \end{pmatrix} = (e_1, \dots, e_n) \cdot C \cdot \begin{pmatrix} x'_1 \\ \vdots \\ x'_n \end{pmatrix}$

$v = (e_1, \dots, e_n) \begin{pmatrix} x_1 \\ \vdots \\ x_n \end{pmatrix}$

Отсюда $\begin{pmatrix} x_1 \\ \vdots \\ x_n \end{pmatrix} = C \cdot \begin{pmatrix} x'_1 \\ \vdots \\ x'_n \end{pmatrix} \lhd$.

\subsection{Линейные отображения}
$V, W$ -- векторные пространства

\vspace{\baselineskip}
\textbf{Определение.} Отображение $\varphi: V \rightarrow W$ называется \textit{линейным}, если 

(1) $\varphi(v_1 + v_2) = \varphi(v_1) + \varphi(v_2) \ \forall \ v_1, v_2 \in V$

(2) $\varphi(\lambda v) = \lambda \varphi(v) \ \forall \ v \in V, \lambda \in F$

\vspace{\baselineskip}
\textbf{Упражнение.} 1) \& 2) эквивалентно тому, что $\varphi(\lambda_1 v_1 + \lambda_2 v_2) = \lambda_1 \varphi(v_1) + \lambda_2 \varphi(v_2) \ \forall v_1, v_2 \in V, \lambda_1, \lambda_2 \in F$

\vspace{\baselineskip}
Примеры.

0) Нулевое отображение $\varphi (v) = 0 \ \forall \ v \in V$

1) Тождественное отображение $Id: V \rightarrow V, v \rightarrow v$

2) $\varphi: \mathbb{R}^2 \rightarrow \mathbb{R}^2$ -- поворот на угол $\alpha$ вокруг 0

3) $\varphi: \mathbb{R}^3 \rightarrow \mathbb{R}^2$ -- ортогональная проекция на $Oxy$

4) $P_n := R[x]_{\leq n}$ -- многочлены от $x$ с коэффициентами из $\mathbb{R}$ степени $\leq n$

$\triangle$ -- отображение дифференцирования

$\triangle: f \rightarrow f'$

$(f + g)' = f' + g'$

$(\lambda f)' = \lambda f'$

линейное отображение: $P_n \rightarrow P_{n-1}$

5) $V$ -- векторное пространство, $(e_1, \dots, e_n)$ -- базис

$\varphi: V \rightarrow F^n$

$v = x_1 e_1 + \dots + x_n e_n \rightarrow \begin{pmatrix} x_1 \\ \vdots \\ x_n \end{pmatrix}$

$w = y_1 e_1 + \dots + y_n e_n \rightarrow \begin{pmatrix} y_1 \\ \vdots \\ y_n \end{pmatrix}$

$v + w = (x_1 + y_1) e_1 + \dots + (x_n + y_n) e_n$

$\varphi(v + w) = \begin{pmatrix} x_1 + y_1 \\ \vdots \\ x_n + y_n \end{pmatrix} = \begin{pmatrix} x_1 \\ \vdots \\ x_n \end{pmatrix} + \begin{pmatrix} y_1 \\ \vdots \\ y_n \end{pmatrix} = \varphi(v) + \varphi(w)$

Аналогично, $\varphi(\lambda v) = \lambda \varphi (v) \Rightarrow \varphi$ -- линейное отображение, биективно!

\vspace{\baselineskip}
\textbf{\textit{Простейшие свойства линейного отображения}}

1) $\varphi(0_v) = 0_w$

$\varphi(0_v) = \varphi(0 \cdot 0_v) = 0 \cdot \varphi(0_v) = 0_w$

2) $\varphi(-v) = -\varphi(v)$

$\varphi(v) + \varphi(-v) = \varphi(v - v) = \varphi(0_v) = 0_w \Rightarrow \varphi(-v) = -\varphi(v)$

\vspace{\baselineskip}
\textbf{Определение.} Отображение $\varphi: V \rightarrow W$ называется \textit{изоморфизмом}, если оно линейно и биективно.

Обозначение: $\varphi: V \iso W$

\vspace{\baselineskip}
\textbf{Примеры}

0) Изоморфизм $\Leftrightarrow$ Оба $V, W$ есть $\{0\}$

1) да

2) да

3) нет

4) нет

5) да

\vspace{\baselineskip}
\textbf{Предложение 1.} $\varphi: V \rightarrow W$ -- изоморфизм $\Rightarrow$ обратное отображение $\varphi^{-1}: W \rightarrow V$ -- тоже изоморфизм

\vspace{\baselineskip}
\textbf{Доказательство.} $\rhd \ \varphi^{-1}$ биективно, остается проверить линейность.

$w_1, w_2 \in W \Rightarrow \exists ! v_1, v_2 \in V$, такой что $w_1 = \varphi (v_1), w_2 = \varphi(v_2) \Rightarrow v_1 = \varphi^{-1} (w_1), v_2 = \varphi^{-1} (w_2)$.

1) $\varphi^{-1} (w_1 + w_2) = \varphi^{-1} (\varphi(v_1) + \varphi(v_2)) = \varphi^{-1}(\varphi(v_1 + v_2)) = v_1 + v_2 = \varphi^{-1} (v_1) + \varphi^{-1} (v_2)$.

2) $\lambda \in F$

$\varphi^{-1} (\lambda w_1) = \varphi^{-1}( \lambda \varphi(v_1)) = \varphi^{-1} (\varphi (\lambda v)) = \lambda v_1 = \lambda \varphi^{-1}(w_1) \lhd$.

\vspace{\baselineskip}
$U \rightarrow^{\hspace{-4mm}^{\varphi}} V \rightarrow^{\hspace{-4mm}^{\psi}} W$ 

$\psi \circ \varphi(u)= \psi (\varphi (u))$ 

\vspace{\baselineskip}
\textbf{Предложение 2.} 1) Если $\varphi, \psi$ линейны, то $\psi \circ \varphi$ тоже линейна

2) Если $\varphi, \psi$ -- изоморфизмы, то $\psi \circ \varphi$ тоже изоморфизм

\vspace{\baselineskip}
\textbf{\textit{Доказательство.}} 1) $\rhd (\psi \circ \varphi) (u_1 + u_2) = \psi (\varphi(u_1 + u_2) = \psi(\varphi(u_1) + \varphi(u_2)) = \psi(\varphi(u_1) + \psi (\varphi(u_2)) = (\psi \circ \varphi) (u_1) + (\psi \circ \varphi) (u_2)$ 

2) $(\psi \circ \varphi) (\lambda u) = \psi (\varphi (\lambda u) = \psi (\lambda \varphi (u)) = \lambda \psi (\varphi(u)) = \lambda (\psi \circ \varphi) (u))$

\vspace{\baselineskip}
2) следует из 1) и того факта, что композиция биективных отображений биективна $\lhd$

\vspace{\baselineskip}
\textbf{Определение.} Два векторных пространства называются \textit{изоморфными}, если существует изоморфизм $\varphi : V \iso W$. 

\vspace{\baselineskip}
\textbf{Теорема.} Изоморфность -- это отношение эквивалентности на множестве всех векторных пространств.

\vspace{\baselineskip}
\textbf{\textit{Доказательство.}} $\rhd$ 1) Рефлексивность: $Id: V \iso V$

2) Симметричность: $V \simeq W \Rightarrow W \simeq V$ (предложение 1)

3) Транзитивность: $U \simeq V, V \simeq W \Rightarrow U \simeq W$ (предложение 2) $\lhd$

\vspace{\baselineskip}
\textbf{Определение.} Классы эквивалентности называются \textit{классами изоморфизма} векторных пространств.

\vspace{\baselineskip}
Пример. 

$F^n \simeq P_{n-1}$

$\begin{pmatrix} a_1 \\ \vdots \\ a_n \end{pmatrix} \rightarrow a_1 + a_2 x + \dots + a_n x^{n-1}$ изоморфизм

\vspace{\baselineskip}
