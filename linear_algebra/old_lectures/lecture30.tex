\section{Лекция 10.05.2018}

Пусть теперь $E' = E, \varphi: E \rightarrow E$ -- линейный оператор $\Rightarrow \ \exists!$ линейный оператор $\varphi^*: E \rightarrow E$, такой что $(\varphi(x), y) = (x, \varphi^*(y)) \ \forall \ x, y \in E$

\bigskip
\textbf{Определение.} $\varphi^*$ называется \textit{сопряженным} к $\varphi$ линейным оператором.

\bigskip
\textbf{Определение.} Линейный оператор $\varphi$ называется \textit{самомпряженным (или симметрическим)}, если $\varphi = \varphi^*$.

\bigskip
\textbf{Замечание.} $e$ -- ортогональный базис в $E, A_{\varphi} = A(\varphi, e), A_{\varphi^*} = A(\varphi^*, e) \Rightarrow A_{\varphi^*} = (A_{\varphi})^T$

$\varphi = \varphi^* \Leftrightarrow A_{\varphi} = (A_{\varphi})^T$

\bigskip
\textbf{Предложение.} $\varphi = \varphi^*, U \subseteq E$ -- $\varphi$-инвариантное подпространство $\Rightarrow U^{\bot}$ тоже $\varphi$-инвариантное

\bigskip
\textbf{\textit{Доказательство.}} $\rhd$ Имеем $\varphi(U) \subseteq U$, хотим $\varphi(U^{\bot}) \subseteq U^{\bot}$

$\forall x \in U^{\bot}, y \in U: (\varphi(x), y) = (x, \varphi^* (y)) = (\overbrace{x}^{\in U^{\bot}}, \overbrace{\varphi(y)}^{\in U}) = 0 \ \lhd$

\bigskip
\textbf{Предложение.} $\varphi = \varphi^* \Rightarrow \exists$ собственный вектор для $\varphi$

\bigskip
\textbf{\textit{Доказательство.}} $\rhd$ Два случая: 1) $\exists$ 1-мерное $\varphi$-инвариантное подпространство

2) $\exists$ 2-мерное $\varphi$-инвариантное подпространство

\bigskip
1) Очевидно

2) Пусть $U \subseteq E$ -- 2-мерное $\varphi$-инвариантное подпространство. Положим $\psi := \varphi|_U$

Т.к. $\varphi = \varphi^*$, то $\psi = \psi^*$

Пусть $(e_1, e_2)$ -- ортонормированный базис в $U \Rightarrow A(\psi, e) = \begin{pmatrix} a & b \\ b & c \end{pmatrix}$

$\chi_{\psi} (t) = (-1)^2 \begin{vmatrix}
a- t & b \\ b & c - t \end{vmatrix} = t^2 - (a + c) t + ac - b^2$

$D = (a+c)^2 - 4 (ac - b^2) = (a-c)^2 +4b^2 \geq 0 \Rightarrow \chi_{\psi} (t)$ имеет корни в $\RR \Rightarrow \psi$ имеет собственный вектор $\Rightarrow \varphi$ имеет собственный вектор $\lhd$

\bigskip
\textbf{Теорема.} $\varphi = \varphi^* \Rightarrow \exists$ ортонормированный базис, состоящий из собственных векторов для $\varphi$. В частности, $\varphi$ диагонализуем над $\RR$, $\chi_{\varphi} (t)$ разлагается на линейные множители над $\RR$.

\bigskip
\textbf{\textit{Доказательство.}} Индукция по $n = dim E$.

$n = 1 \Rightarrow$ очевидно

Пусть доказано для $<n$, докажем для $n$.

$\exists$ собственный вектор $v$ для $\varphi$.

Положим $e_1 = \frac{v}{|v|}, |e_1| = 1$.

$U = <e_1> \Rightarrow U$ $\varphi$-инвариантно $\Rightarrow U^{\bot}$ тоже $\varphi$-инвариантно

$dim U^{\bot} = n - 1 \Rightarrow$ по предположению индукции в $U^{\bot}$  $\exists$ ортонормированный базис $(e_2, \dots, e_n)$ из собственных векторов, $e_1 \bot (e_2, \dots, e_n)$. Тогда $(e_1, \dots, e_n)$ -- искомый базис $\lhd$

\bigskip
\textbf{Следствие.} $\varphi = \varphi^*, \ \lambda, \mu \in Spec(\varphi), \lambda \neq \mu \Rightarrow E_{\lambda} (\varphi) \bot E_{\mu} (\varphi)$

\bigskip
\textbf{\textit{Доказательство.}} \textbf{Первый способ.}

Пусть $e = (e_1, \dots, e_n)$ -- ортономированный базис из собственных векторов

$(\lambda_1, \dots, \lambda_n)$ -- соответствующий набор собственных значений (т.е. $\varphi (e_i) = \lambda_i e_i$)

$v = x_1 e_1 + \dots + x_n e_n \in E$

$\varphi(v) = x_1 \lambda_1 e_1 + \dots + x_n \lambda_n e_n$. Тогда $\varphi (v) = \lambda v \Leftrightarrow v \in <e_i \ | \ \lambda_i = \lambda> \Rightarrow E_{\lambda} (\varphi) = <e_i \ | \ \lambda_i = \lambda > \Rightarrow E_{\lambda} (\varphi) \bot E_{\mu} (\varphi)$ при $\lambda \neq \mu$

\bigskip
\textbf{Второй способ.}

$\lambda \neq \mu, x \in E_{\lambda} (\varphi), y \in E_{\mu} (\varphi)$

$\lambda(x, y) = (\lambda x, y) = (\varphi(x), y) = (x, \varphi(y)) = (x, \mu y) = \mu (x, y)$

Т.к. $\lambda \neq \mu$, то $(x, y) = 0 \ \lhd$

\bigskip
\textbf{Теорема (приведение квадратичной формы к главным осям).} $Q: E \rightarrow \RR$ -- квадратичная форма $\Rightarrow \exists$ ортономированный базис $e = (e_1, \dots, e_n)$, в котором $Q$ принимает канонический вид $Q(x) = \lambda_1 x_1^2 + \dots + \lambda_n x_n^2$, причем числа $\lambda_1, \dots, \lambda_n$ определены однозначно с точностью до перестановки. \textit{Главные оси} -- это $\RR e_1, \dots, \RR e_n$.

\bigskip
\textbf{\textit{Доказательство.}} $\rhd$ Пусть $f = (f_1, \dots, f_n)$ -- какой-то ортонормированный базис, $B = B(Q, f)$.

Пусть $\varphi: E \rightarrow E$ -- линейный оператор, такой что $A(\varphi, f) = B$

Т.к. $B = B^T$ и $f$ ортонормированный, то $\varphi = \varphi^*$

$\forall \ x = x_1 f_1 + \dots + x_n f_n \in E$ имеем $Q(x) = (x_1, \dots, x_n) B \begin{pmatrix} x_1 \\ \vdots \\ x_n \end{pmatrix} = (x, \varphi(x))$

Итого: $Q(x) = (x, \varphi(x)) \ \forall \ x \in E$ (*)

\bigskip
Знаем: $\exists$ ортонормированный базис $e= (e_1, \dots, e_n)$, состоящий из собственных векторов для $\varphi \Rightarrow A(\varphi, e) = diag (\lambda_1, \dots, \lambda_n)$

Но тогда $\forall \ x = x_1 e_1 + \dots + x_n e_n \ Q(x) = (x, \varphi(x)) = (x_1, \dots, x_n) D \begin{pmatrix}
x_1 \\ \vdots \\ x_n \end{pmatrix} = \lambda_1 x_1^2 + \dots + \lambda_n x_n^2$

\bigskip
Единственность: $\lambda_1, \dots, \lambda_n$ -- это в точности собственные значения линейного оператора $\varphi$

В свою очередь, $\varphi$ однозначно определяется из условия (*) $\lhd$

\bigskip
\textbf{Следствие.} $A \in M_n(\RR), A = A^T \Rightarrow \exists$ ортогональная матрица $C \in M_n$, такая что $C^{-1} A C = C^T A C = diag(\lambda_1, \dots, \lambda_n)$, причем $\lambda_1, \dots, \lambda_n$ -- это в точности собственные значения матрицы $A$.

\bigskip
\textbf{Определение.} Линейный оператор называется \textit{ортогональным}, если $\forall \ x, y \in E: (\varphi(x), \varphi(y)) = (x, y)$ (т.е. $\varphi$ сохраняет скалярное произведение)

\bigskip
\textbf{Предложение.} Для $\varphi \in L(E)$ следуюущие условия эквивалентны:

1) $\varphi$ ортогональный

2) $|\varphi(x)| = |x| \ \forall \ x \in E$ (т.е. $\varphi$ сохраняет длину)

3) $\exists \ \varphi^{-1}$ и $\varphi^{-1} = \varphi^*$

4) $\forall$ ортонормированного базиса $e$ матрица $A(\varphi, e)$ ортогональная

5) $\forall$ ортонормированного базиса $e$ $(\varphi(e_1), \dots, \varphi(e_n))$ -- тоже ортонормированный базис

\bigskip
\textbf{\textit{Доказательство.}} $\rhd$ $1) \Rightarrow 2) \ |\varphi(x)| = \sqrt[]{(\varphi(x), \varphi(x))} = \sqrt[]{(x, x)} = |x|$

\bigskip
$2) \Rightarrow 1) \ (\varphi(x), \varphi(y)) = \frac{1}{2} (|\varphi(x) + \varphi(y)|^2 - |\varphi(x)|^2 - |\varphi(y)|^2) = \frac{1}{2} (|x+y|^2 - |x|^2 - |y|^2) = (x, y)$

\bigskip
$1) \& 2) \Rightarrow 3) \ \varphi(x) = 0 \Rightarrow |\varphi(x)| = 0 \Rightarrow |x| = 0 \Rightarrow x = 0 \Rightarrow Ker \varphi = \{0\} \Rightarrow \varphi$ невырожден (т.е. $\exists \ \varphi^{-1}$)

$(\varphi^{-1}(x), y) = (\varphi(\varphi^{-1}(x)), \varphi(y)) = (x, \varphi(y)) \Rightarrow \varphi^{-1} = \varphi^*$

\bigskip
$3) \Rightarrow 4) \ e$ -- ортонормированный базис, $A = A(\varphi, e)$

$\begin{cases} A(\varphi^*, e) = A^T \\ A(\varphi^{-1}, e) = A^{-1} \end{cases} \Rightarrow A^T = A^{-1}$

\bigskip
$4) \Rightarrow 5) \ e$ -- ортонормированный базис, $A = A(\varphi, e) \Rightarrow (\varphi(e_1), \dots, \varphi(e_n)) = (e_1, \dots, e_n) A$ (где $A$ ортогональная) $\Rightarrow (\varphi(e_1), \dots, \varphi(e_n))$ -- ортонормированный базис

\bigskip
$5) \Rightarrow 1) \ e$ -- ортонормированный базис, $x = \sum\limits_{i=1}^n x_i e_i, y = \sum\limits_{j=1}^n y_j e_j$

$(\varphi(x), \varphi(y)) = (\varphi(\sum\limits_{i=1}^n x_i e_i), \varphi(\sum\limits_{j=1}^n y_j e_j)) = \sum\limits_{i=1}^n\sum\limits_{j=1}^n x_i y_j \overbrace{(\varphi(e_i), \varphi(e_j))}^{\delta_{ij}} = \sum\limits_{i=1}^n\sum\limits_{j=1}^n x_i y_j (e_i, e_j) = (\sum\limits_{i=1}^n x_i e_i, \sum\limits_{j=1}^n y_i e_j) = (x, y) \ \lhd$

