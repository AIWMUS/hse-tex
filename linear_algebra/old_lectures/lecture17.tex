\section{Лекция 25.01.2017}

\textbf{Теорема.} Пусть $V, W$ -- два конечномерных векторных пространства. Тогда $V \simeq W \Leftrightarrow dimV = dimW$.

\bigskip
\textbf{Лемма 1.} Пусть $dimV = n$. Тогда $V \simeq F^n$.

\bigskip
\textbf{\textit{Доказательство.}} $\rhd$ Фиксируем базис $(e_1, e_2, \dots, e_n)$ и рассмотрим отображение $\varphi: V \Rightarrow F^n$ из примера 5, т.е. $\varphi (x_1 e_1 + \dots + x_n e_n) = \begin{pmatrix} x_1 \\ \vdots \\ x_n \end{pmatrix}$. Знаем, что $\varphi$ -- изоморфизм $\lhd$.

\bigskip
\textbf{Лемма 2.} Пусть $\varphi : V \iso W$ -- изоморфизм, $(e_1, e_2, \dots, e_n)$ -- базис $V$. Тогда $(\varphi (e_1), \dots, \varphi (e_n))$ -- базис $W$.

\bigskip
\textbf{\textit{Доказательство.}} $\rhd \ w \in W$, пусть $v := \varphi^{-1} (w)$.

$v = x_1 e_1 + \dots + x_n e_n$, где $x_i \in F \Rightarrow w = \varphi (v) = x_1 \varphi(e_1) + \dots + x_n \varphi (e_n) \Rightarrow w = <\varphi(e_1), \dots, \varphi(e_n)>$.

Докажем, что $\varphi(e_1), \dots, \varphi(e_n)$ линейно независимы. Пусть $\alpha_1 \varphi(e_1) + \dots + \alpha_n \varphi(e_n) = 0$. Тогда $\varphi (\alpha_1 e_1 + \dots + \alpha_n e_n) = 0$. Применим $\varphi^{-1}$, получим $\alpha_1 e_1 + \dots + \alpha_n e_n = \varphi^{-1}(0) = 0$. Т.к. $(e_1, \dots, e_n)$ -- базис $V$, то $\alpha_1=\dots=\alpha_n = 0 \ \lhd$.

\bigskip
\textbf{\textit{Доказательство теоремы.}} $\rhd \ dimV = dimW = n \Rightarrow V \simeq F^n$ и $W \simeq F^n \Rightarrow V \simeq W$ (лемма 1).

$V \iso W$. Фиксируем изоморфизм $\varphi : V \iso W$ и базис $(e_1, \dots, e_n)$ в $V$. Тогда $(\varphi(e_1), \dots, \varphi(e_n))$ -- базис $W$ (лемма 2) $\Rightarrow dimW = n = dimV \ \lhd$.

\bigskip
$V$ -- векторное пространство

$(e_1, \dots, e_n)$ -- базис

\bigskip
\textbf{Предложение.} 1) Всякое линейное отображение $\varphi : V \rightarrow W$ однозначно определяется векторами $\varphi(e_1), \dots, \varphi(e_n)$.

2) $\forall$ набора векторов $w_1, \dots, w_n \in W \ \exists !$ линейное отображение $\varphi : V \rightarrow W$, т. что $\varphi(e_1) = w_1, \dots, \varphi(e_n) = w_n$.

\bigskip
\textbf{\textit{Доказательство.}} 1) $v \in V, v = x_1 e_1 + \dots + x_n e_n \Rightarrow \varphi(v) = x_1 \varphi (e_1) + \dots + x_n \varphi (e_n)$.

2) $v \in V, v = x_1 e_1 + \dots + x_n e_n$. Положим $\varphi (v) = x_1 w_1 + \dots + x_n w_n$. $\varphi$ линейно (упражнение), единственно по 1).

\bigskip
\textbf{Следствие 1.} Если $dimV = dimW = n$, то $\forall$ базиса $(f_1, \dots, f_n)$ в $W \ \exists !$ изоморфизм $\varphi : V \iso W$, такой что $\varphi (e_i) = f_i \ \forall i$.

\bigskip
\textbf{Следствие 2.} $\exists$ биекция между базисами $V$ и изоморфизмами $F^n \iso V$ (а тогда и изоморфизмами $V \iso F^n$) стандартный базис в $F^n \rightarrow$ данный базис в $V$.

\bigskip
$V, W$ -- векторные пространства

$dimV = n$, фиксированный базис $e = (e_1, \dots, e_n)$.

$dimW = m$, фиксированный базис $f = (f_1, \dots, f_m)$.

$\varphi : V \rightarrow W$ -- линейное отображение

$\varphi(e_j) = a_{1j} f_1 + \dots + a_{mj} f_m = (f_1, \dots, f_m) \begin{pmatrix} a_{1j} \\ \vdots \\ a_{mj} \end{pmatrix}$

\bigskip
\textbf{Определение.} Матрица $A = (a_{ij}) \in Mat_{n \times m}$ называется \textit{матрицей линейного отображения в базисах $e$ и $f$ (или по отношению к базисам)}.

Обозначение: $A = A (\varphi, e, f)$.

Обозначение: $Hom(V, W)$ -- множество всех линейных отображений из $V$ в $W$

\bigskip
\textbf{Следствие 3.} При фиксированном $e$ и $f$ отображение $Hom(V, W) \rightarrow Mat_{n \times m} (F) \ \varphi \rightarrow A(\varphi, e, f)$ является биекцией.

\bigskip
$(\varphi (e_1), \dots, \varphi(e_n)) = (f_1, \dots, f_m) \cdot A$, где $A = A(\varphi, e, f)$.

В $j$-- столбце этой матрице стоят координаты вектора $\varphi(e_j)$  в базисе $f$.

\bigskip
Примеры.

0) Нулевое отображение $\rightarrow$ нулевая матрица в любой паре базисов.

1) $\varphi : \RR^3 \rightarrow \RR^2$ -- проекция на $Oxy$

$e = (e_1, e_2, e_3), f = (e_1, e_2)$

$A(\varphi, e, f) = \begin{pmatrix} 1 & 0 & 0 \\ 0 & 1 & 0 \end{pmatrix}$

2) $\triangle : R[x]_{\leq n} \rightarrow R[x]_{\leq n-1}$

$f \rightarrow f'$

$e = (1, x, x^2, \dots, x^n), \ f = (1, x, x^2, \dots, x^{n-1})$

$A(\varphi, e, f) = \begin{pmatrix} 0 & 1 & 0 & 0 & \dots & 0 \\ 0 & 0 & 2 & 0 & \dots & 0 \\ 0 & 0 & 0 & 3 & \dots & 0 \\ \vdots & \vdots & \vdots & \vdots & \vdots & \vdots \\ 0 & 0 & 0 & 0 & \dots & n \end{pmatrix} \in Mat_{n \times (n-1)}$

3) $\varphi : F^n \rightarrow F^m$

$x \rightarrow Ax$, $A \in Man_{m \times n} (F)$

$e$ -- стандартный базис $F^n$

$f$ -- стандартный базис $F^m$

$A(\varphi, e, f) = A$

\bigskip
\textbf{Предложение.} $\varphi : V \rightarrow W$ -- линейное отображение. $e$ -- базис $V$, $f$ -- базис $W$, $A = A(\varphi, e, f)$.

$v \in V, v = x_1 e_1 + \dots + x_n e_n, \varphi(v) = y_1 f_1 + \dots + y_m f_m$

$\Rightarrow \begin{pmatrix} y_1 \\ \vdots \\ y_m \end{pmatrix} = A \cdot \begin{pmatrix} x_1 \\ \vdots \\ x_n \end{pmatrix}$.

\bigskip
\textbf{\textit{Доказательство.}} $\rhd \ v = (e_1, e_2, \dots, e_n) \begin{pmatrix} x_1 \\ x_2 \\ \vdots \\ x_n \end{pmatrix}$

$\varphi(v) = (\varphi(e_1), \dots, \varphi(e_n)) \begin{pmatrix} x_1 \\ x_2 \\ \vdots \\ x_n \end{pmatrix} = (f_1, \dots, f_n) \cdot A \cdot \begin{pmatrix} x_1 \\ x_2 \\ \vdots \\ x_n \end{pmatrix} = (f_1, \dots, f_n) \begin{pmatrix} y_1 \\ y_2 \\ \vdots \\ y_m \end{pmatrix} \Rightarrow \begin{pmatrix} y_1 \\ y_2 \\ \vdots \\ y_m \end{pmatrix} = A \cdot \begin{pmatrix} x_1 \\ x_2 \\ \vdots \\ x_n \end{pmatrix} \ \lhd$

\bigskip
Пусть $e'$ -- другой базис в $V$, $e' = e \cdot C$

$f'$ -- другой базис в $W$, $f' = f \cdot D$

$A = A(\varphi, e, f)$

$A' = A(\varphi, e', f')$

\bigskip
\textbf{Предложение.} $A' = D^{-1} \cdot A \cdot C$

\bigskip
\textbf{\textit{Доказательство.}} $\rhd \ (e'_1, \dots, e'_n) = (e_1, \dots, e_n) \cdot C \Rightarrow (\varphi(e'_1), \dots, \varphi(e'_n)) = (\varphi(e_1), \dots, \varphi(e_n)) \cdot C = (f_1, \dots, f_m) \cdot A \cdot C$ и $(\varphi(e'_1), \dots, \varphi(e'_n)) = (f'_1, \dots, f'_m) \cdot A' = (f_1, \dots, f_m) \cdot D \cdot A' \Rightarrow AC = DA' \Rightarrow A' = D^{-1} \cdot A \cdot C \ \lhd$ 

\bigskip
$\varphi, \psi \in Hom(V, W)$

\bigskip
\textbf{Определение.} \textit{Суммой линейных отображений} $\varphi$ и $\psi$ называется отображение $\varphi + \psi$, такое что $\forall \ v \in V \ (\varphi + \psi) (v) := \varphi(v) + \psi(v)$ 

\bigskip
\textbf{Определение.} $\lambda \in F \Rightarrow$ Произведение  $\varphi$ на $\lambda$ -- это отображение $\lambda \cdot \varphi$, такое что $(\lambda \cdot \varphi) (v) := \lambda \cdot \varphi (v)$

\bigskip
\textbf{\textit{Упражнение.}} $\varphi + \psi, \lambda \varphi \in Hom(V, W)$

\bigskip
\textbf{\textit{Упражнение.}} $Hom(V, W)$ -- это векторное пространство с такими операциями.

\bigskip
$\varphi, \psi \in Hom(V, W)$

\textbf{Предложение.} $e$ -- базис $V$, $f$ -- базис $W$

$A_{\varphi} = A(\varphi, e, f)$

$A_{\psi} = A(\psi, e, f)$

$A_{\varphi + \psi} = A(\varphi + \psi, e, f)$

$A_{\lambda \varphi} = A(\lambda \varphi, e, f)$

$\Rightarrow A_{\varphi + \psi} = A_{\varphi} + A_{\psi}$, $A_{\lambda \varphi} = \lambda A_{\varphi}$

\bigskip
\textbf{\textit{Доказательство.}} $\rhd \ ((\varphi + \psi)e_1, \dots, (\varphi + \psi)e_n) = (\varphi (e_1), \dots, \varphi (e_n)) + \\ + (\psi (e_1), \dots, \psi (e_n)) = (f_1, \dots, f_m) A_{\varphi} + (f_1, \dots, f_m) A_{\psi} = (f_1, \dots, f_m) (A_{\varphi} + A_{\psi}) $ 

$((\varphi + \psi)e_1, \dots, (\varphi + \psi)e_n) = (f_1, \dots, f_m) A_{\varphi + \psi}$

$\Rightarrow A_{\varphi + \psi} = A_{\varphi} + A_{\psi}$

$A_{\lambda \varphi} = \lambda A_{\varphi}$ аналогично $\lhd$

\bigskip
\textbf{Следствие 1.} При фиксированном $e$ и $f$ отображение $Hom(V, W) \rightarrow Mat_{m \times n} (F)$, $\varphi \rightarrow A(\varphi, e, f)$ является изоморфизмом векторных пространств.

\bigskip
\textbf{\textit{Доказательство.}} Это отображение биективно (уже знаем), линейно по предложению.

\bigskip
\textbf{Следствие 2.} $dimV = n, \ dimW = m \Rightarrow dim Hom(V, W) = nm$.

