\section{Лекция 11.01.2018}

$V$ -- векторное пространство над полем $F$, $dimV < \infty$

$U, W$ -- подпространства

$U \cap W$ -- тоже подпространство

\bigskip
\textbf{Определение.} \textit{Суммой подпространств} $U, W$ называется множество $U + W := \{ u + w \ | \ u \in U, w \in W \}$.

\bigskip
\textit{\textbf{Упражнение.}} $U + W$ -- подпространство в $V$.

\bigskip
\textbf{Замечание.} $U \cap W \subseteq U = U + 0 \subseteq U + W$, $dim(U \cap W) \leq dimU \leq dim(U + W)$.

\bigskip
\textbf{Теорема.} $dim(U \cap W) + dim(U + W) = dimU + dimW$

\bigskip
\textbf{\textit{Доказательство.}} $\rhd dim(U \cap W) = p, \ dimU = q, \ dimW = r$. 

Пусть $a = \{a_1, \dots, a_p\}$ -- базис в $U \cap W$. Т.к. $U \cap W \subseteq U$ и $U \cap W \subseteq W$, то $a$ можно дополнить до базиса $U$ и до базиса $W$.

$\Rightarrow \exists b = \{b_1, \dots, b_{q-p}\}$, такой что $a \cup b$ -- базис $U$, $\exists c = \{c_1, \dots, c_{r-p}\}$, такой что $a \cup c$ -- базис $W$.

Докажем, что $a \cup b \cup c$ -- базис в $U + W$.

1) $<a \cup b \cup c> = U + W$

$v \in U + W \Rightarrow \exists u \in U, w \in W$, такой что $v = u + w$

$u \in U = <a \cup b> \subseteq <a \cup b \cup c>$

$w \in W = <a \cup c> \subseteq <a \cup b \cup c>$

$\Rightarrow v = u + w \in <a \cup b \cup c>$

Доказано, что $U + W \subseteq <a \cup b \cup c>, \ a, b, c \subseteq U + W \Rightarrow <a \cup b \cup c> \subseteq U + W$.

\bigskip
2) система $a \cup b \cup c$ линейно независима. 

Пусть $\exists \alpha_i, \beta_j, \gamma_k \in F$, такие что (*) $\alpha_1 a_1 + \dots + \alpha_p a_p (=x) + \beta_1 b_1 + \dots + \beta_{q-p} b_{q-p} (=y) + \gamma_1 c_1 + \dots + \gamma_{r-p} c_{r-p} (=z) = 0$, $x + y + z = 0 \Rightarrow z = - x (\in U) - y (\in U) \in U$

Но мы знаем, что $z \in W \Rightarrow z \in U \cap W \Rightarrow \exists \lambda_1, \dots, \lambda_p \in F$, такие что $z = \lambda_1 a_1 + \dots + \lambda_p a_p \Rightarrow \lambda_1 a_1 + \dots + \lambda_p a_p - \gamma_1 c_1 - \dots - \gamma_{r-p} c_{r-p} = 0$.  

Это есть линейная комбинация векторов базиса $a \cup c$ пространства $W \Rightarrow$ все коэффициенты равны 0, т.е. $\lambda_i = \gamma_j = 0$ (и $z = 0$).

Значит, $x + y = 0 \Rightarrow \alpha_1 a_1 + \alpha_p a_p + \beta_1 b_1 + \dots + \beta_{q-p} b_{q-p} = 0$.

А это линейная комбинация векторов базиса $a \cup b$ пространства $U \Rightarrow \alpha_i = \beta_i = 0 \Rightarrow$ все коэффициенты в (*) равны 0.

$\Rightarrow a \cup b \cup c$ линейно независимы $\Rightarrow dim (U + W) = |a \cup b \cup c| = |a| + |b| + |c| = p + (q - p) + (r - p) = q + r - p = dimU + dim W - dim(U \cap W) \lhd.$

\bigskip
Пример. Любые две плоскости (содержащие 0) в $\RR^3$ имеют общую прямую 

$dimU = dimW = 2$

$U + W \subseteq \RR^3 \Rightarrow dim(U + W) \leq 3 \Rightarrow dim(U \cap W) \geq 2 + 2 - 3 = 1$.

\bigskip
$U_1, \dots, U_k \subseteq V$ -- подпространства

\bigskip
\textbf{Определение.} \textit{Суммой подпространств} $U_1, \dots, U_k$ называется множество $U_1 + \dots + U_k := \{u_1 + \dots + u_k \ | \ u_i \in U_i\}$.

\bigskip
\textbf{\textit{Упражнение.}} $U_1 + \dots + U_k$ -- подпространство в $V$

\bigskip
\textbf{Замечание.} $dim(U_1 + \dots + U_k) \leq dimU_1 + \dots dimU_k$.

\bigskip
\textbf{Определение.} Подпространства $U_1, \dots, U_k$ называются линейно независимыми, если $\forall u_1 \in U_1, \dots, u_k \in U_k$ из условия $u_1 + \dots + u_k = 0$ следует, что $u_1 = \dots = u_k = 0$.

\bigskip
Пример. $dim U_i = 1$ и $U_i = <u_i> \Rightarrow U_1, \dots, U_k$ линейно независимы $\Leftrightarrow u_1, \dots, u_k$ линейно независимы

\bigskip
\textbf{Теорема.} Следующие условия эквивалентны:

(1) $U_1, \dots, U_k$ линейно независимы

(2) $\forall v \in U_1 + \dots + U_k \ \exists! \ u_1 \in U_1, \dots, u_k \in U_k$, такие что $v = u_1 + \dots + u_k$

(3) Если $e_i$ -- базис $U_i$, то $e_1 \cup \dots \cup e_k$ -- базис $U_1 + \dots + U_k$

(4) $dim(U_1 + \dots + U_k) = dim U_1 + \dots + dim U_k$

(5) $\forall \ i \ U_i \cap (U_1 + \dots + U_{i-1} + U_{i+1} + \dots + U_k) = 0$

\bigskip
\textbf{\textit{Доказательство.}} $\rhd \ (1) \Rightarrow (2)$ 

Пусть $u_1 + \dots + u_k = u'_1 + \dots + u'_k$, где $u_i, u'_i \in U_i \Rightarrow (u_1 - u'_1) (\in U_1) + \dots + (u_k - u'_k) (\in U_k) = 0$

из (1) следует, что $(u_1 - u'_1) = \dots = (u_k - u'_k) = 0 \Rightarrow u_1 = u'_1, \dots, u_k = u'_k$.

\bigskip
$(2) \Rightarrow (3)$

$v \in U_1 + \dots + U_k \Rightarrow$ из (2) $\exists! \ u_1 \in U_1, \dots, u_k \in U_k$, такой что $v = u_1 + \dots + u_k$.

Но $\forall i \ u_i$ однозначно выражается через базис $e_i \Rightarrow v$ однозначно выражается через $e_1 \cup \dots \cup e_k \Rightarrow <e_1 \cup \dots \cup e_k> = U_1 + \dots + U_k$.

0 однозначно выражается через $e_1 \cup \dots \cup e_k \Rightarrow e_1 \cup \dots \cup e_k$ линейно независимы $\Rightarrow e_1 \cup \dots \cup e_k$ -- базис в $U_1 + \dots + U_k$.

\bigskip
$(3) \Rightarrow (4)$ Очевидно

\bigskip
$(4) \Rightarrow (5)$

$dim (U_i \cap (U_1 + \dots + U_{i-1} + U_{i+1} + \dots + U_k)) = dimU_i + dim(U_1 + \dots + U_{i-1} + U_{i+1} + \dots + U_k) - dim (U_1 + \dots + U_k) \leq dimU_i + dim U_1 + \dots + dimU_{i-1} + dimU_{i+1} + \dots + dimU_k - (dim U_1 + \dots + dim U_k) = 0 \Rightarrow dim(U_i \cap (U_1 + \dots + U_{i-1} + U_{i+1} + \dots + U_k)) = 0 \Rightarrow U_i \cap (U_1 + \dots + U_{i-1} + U_{i+1} + \dots + U_k) = 0$.

\bigskip
$(5) \Rightarrow (1)$

$u_1 + \dots + u_k = 0, u_i \in U_i \Rightarrow u_i (\in U_i) = -u_1 - \dots - u_{i-1} - u_{i+1} - \dots - u_k (\in U_1 + \dots + U_{i-1} + U_{i + 1} + \dots + U_k) \Rightarrow u_i \in U_i \cap (U_1 + \dots + U_{i-1} + U_{i+1} + \dots + U_k) \lhd$.

\bigskip
\textbf{Следствие.} Два подпространства $U, W \subseteq V$ линейно независимы $\Leftrightarrow U \cap W = 0$.

\bigskip
\textbf{Определение.} Говорят, что $V$ разлагается в прямую сумму подпространств $U_1, \dots, U_k$, если 

1) $V = U_1 + \dots + U_k$  

2) $U_1, \dots, U_k$ линейно независимы

\bigskip
Примеры. 

1) $e_1, \dots, e_n$ -- базис $V \Rightarrow V = <e_1 \oplus e_2 \oplus \dots \oplus e_n>$ 

2) $U, W \subseteq V$ -- подпространства

$V = U \oplus W \Leftrightarrow \begin{cases} V = U + W \\
U \cap W = 0 \end{cases}$

\bigskip
$V$ -- векторное пространство, $dimV = n, \ e_1, \dots, e_n$ -- базис $V$

$v \in V \Rightarrow \ \exists ! \ x_1, \dots, x_n \in F$, такой что $v = x_1 e_1 + \dots + x_n e_n$.

\bigskip
\textbf{Определение.} Коэффициенты $x_1, \dots, x_n$ называются \textit{координатами вектора} $v$ в базисе $(e_1, \dots, e_n)$.

