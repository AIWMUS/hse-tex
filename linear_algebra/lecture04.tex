\section{Лекция 19.09.2019}

Дана СЛУ с расширенной матрицей $(A \mid b)$.

Было: элементарные преобразования строк в $(A \mid b)$ сохраняют множество решений.

\subsection{Метод Гаусса решения систем линейных уравнений}

Прямой ход метода Гаусса.

Выполняя элементарные преобразования строк в $(A | b)$, приведем $A$ к ступенчатому виду:

\begin{equation*}
    \begin{amatrix}{7}{1}
        0 & \dots & 0 & a_{ij_1} & * & \dots & \dots & b_1 \\
        0 & \dots & 0 & 0 & a_{2j_2} & * & \dots & b_2 \\
        \vdots & \ddots & \vdots & \vdots & \vdots & \vdots & \vdots & \vdots \\
        0 & \dots & 0 & 0 & 0 & 0 & a_{rj_r} & b_r \\
        0 & \dots & 0 & 0 & 0 & 0 & 0 & b_{r + 1} \\
        0 & \dots & 0 & 0 & 0 & 0 & 0 & 0
    \end{amatrix}
.\end{equation*}

\begin{description}
\item[Случай 1]
    $\exists i \geq r + 1 : b_i \neq 0$ (в $A$ есть нулевая строка с $b_i \neq 0$)

    Тогда в новой СЛУ $i$-е уравнение $0 \cdot x_1 + \dots + 0 \cdot x_n = b_i$, т.е. $0 = b_i \implies $ СЛУ несовместна.

\item[Случай 2]
    либо $r = m$, либо $b_i = 0 \quad \forall i \geq r + 1$

    Выполняя элементарные преобразования строк приводим матрицу к улучшенному ступенчатому виду -- обратный ход метода Гаусса
    \begin{equation*}
        \begin{amatrix}{9}{1}
            0 & \dots & 0 & 1 & * & 0 & * & 0 & 0 & b_1 \\
            0 & \dots & 0 & 0 & \dots & 1 & * & 0 & 0 & b_2 \\
            0 & \dots & 0 & 0 & \dots & 0 & 0 & 1 & 0 & b_3 \\
            \vdots & \ddots & \vdots & \vdots & \ddots & \vdots & \vdots & \vdots & \vdots & \vdots \\
            0 & \dots & 0 & 0 & \dots & 0 & 0 & 0 & 1 & b_r \\
            0 & \dots & 0 & 0 & \dots & 0 & 0 & 0 & 0 & 0
        \end{amatrix}
    \end{equation*}

    Неизвестные $x_{j_1}$, $x_{j_2}$, $\dots$, $x_{j_r}$ называются \textit{главными}, а остальные \textit{свободными},
    где $j_i$ -- индексы столбцов с ведущими элементами.

    \begin{description}
    \item[Подслучай 2.1] $r = n$, т.е. все неизвестные -- главные

        \begin{equation*}
            \begin{pmatrix}
                1 & 0 & \dots & 0 & b_1 \\
                0 & 1 & \dots & 0 & b_2 \\
                \vdots & \vdots & \ddots & \vdots & \vdots \\
                0 & 0 & \dots & 1 & b_r \\
                0 & 0 & \dots & 0 & 0
            \end{pmatrix} \leftrightarrow \begin{cases}
                \begin{aligned}
                    x_1 &= b_1 \\
                    x_2 &= b_2 \\
                    \vdots \\
                    x_r &= b_r
                \end{aligned}
            \end{cases} \text{ --- единственное решение}
        .\end{equation*}
    \item[Подслучай 2.2] $r < n$, т.е. есть хотя бы одна свободная неизвестная

        Перенесем в каждом уравнении все члены со свободными неизвестными в правую часть, получаем выражения всех главных неизвестных через свободные, эти выражения называется \textit{общим решением исходной СЛУ}.
    \end{description}
\end{description}

\begin{example}
    Улучшенный ступенчатый вид:
    \begin{equation*}
        \begin{amatrix}{4}{1}
            1 & 3 & 0 & 1 & -1 \\
            0 & 0 & 1 & -2 & 4
        \end{amatrix}
    \end{equation*}

    Главные неизвестные: $x_1, x_3$.

    Свободные неизвестные: $x_2, x_4$.

    $x_2 = t_1, x_4 = t_2$ -- параметры.

    \begin{equation*}
        \begin{cases}
            x_1 = -1 - 3t_1 - t_2 \\
            x_2 = t1 \\
            x_3 = 4 + 2t_2 \\
            x_4 = t_2
        \end{cases}
        \iff
        \begin{pmatrix} x_1 \\ x_2 \\ x_3 \\ x_4 \end{pmatrix}
        =
        \begin{pmatrix} -1 - 3t_1 - t_2 \\ t_1 \\ 4 + 2t_2 \\ t_2 \end{pmatrix}
        =
        \begin{pmatrix} -1 \\ 0 \\ 4 \\ 0 \end{pmatrix}
        +
        t_1
        \begin{pmatrix} -3 \\ 1 \\ 0 \\ 0 \end{pmatrix}
        +
        t_2
        \begin{pmatrix} -1 \\ 0 \\ 2 \\ 1 \end{pmatrix}
    \end{equation*}

    Общее решение:
    \begin{equation*}
        \begin{cases}
            x_1 = -1 - 3x_2 - x_4 \\
            x_3 = 4 + 2x_4
        \end{cases}
    \end{equation*}
\end{example}

\setcounter{theorem}{1}
\begin{corollary}
    Всякая СЛУ с коэффициентами из $\RR$ имеет либо 0 решений, либо одно решение, либо бесконечно много решений.
\end{corollary}


\subsection{Однородные системы линейных уравнений}
\begin{definition}
    СЛУ называется однородной (ОСЛУ), если все её правые части равны 0. Расширенная матрица: $(A \mid 0)$
\end{definition}

\noindent
\textbf{Очевидный факт: } Всякая ОСЛУ имеет нулевое решение $(x_1 = x_2 = \dots = x_n = 0)$.

\begin{corollary}
    Всякая ОСЛУ либо имеет ровно 1 решение (нулевое), либо бесконечно много решений.
\end{corollary}

\begin{corollary}
    Всякая ОСЛУ, у которой число неизвестных больше числа уравнений, имеет ненулевое решение
\end{corollary}

\begin{proof}
    В ступенчатом виде будет хотя бы одна свободная неизвестная. Придавая ей ненулевое значение, получим ненулевое решение
\end{proof}


\subsection{Связь между множеством решений системы линейных уравнений и множеством решений соответствующей однородной системы.}

Пусть дана совместная СЛУ $Ax = b$

Частное решение СЛУ --- это какое-то одно её решение.

\begin{proposition}
    Пусть $Ax = b$ -- совместная СЛУ.

    $x_0$ -- частное решение $Ax + b$

    $S \subset \RR^n$ -- множество решений ОСЛУ $Ax = 0$

    $L \subset \RR^n $ -- множество решений $Ax = b$.

    Тогда, $L = x_0 + S$, где $x_0 + S = \{x_0 + v \mid v \in S\}$
\end{proposition}

\begin{proof}~
    \begin{enumerate}
    \item
        Пусть $u \in L$ ($u$ -- решение $Ax = b$), положим $v = u - x_0$

        Тогда, $Av = A(u - x_0) = Au - Ax_0 = b - b = 0 \implies v \in S \implies L \subseteq x_0 + S$

    \item
        Пусть $v \in S$ ($v$ -- решение $Ax = 0$), положим $u = x_0 + v$.

        Тогда, $Au = A(x_0 + v) = Ax_0 + Av = b + 0 = b \implies u \in L \implies x_0 + S \subseteq L$
    \end{enumerate}

    Значит, $x_0 + S = L$.
\end{proof}


\subsection{Матричные уравнения вида $AX = B$ и $XA = B$, общий метод их решения}

Два типа матричных уравнений:
\begin{enumerate}
\item $AX = B$

    A и B известны,
    X -- неизвестная матрица

\item $XA = C$

    A и C известны,
    X -- неизвестная матрица
\end{enumerate}

Из второго типа получается первый транспонирование матриц: $XA = C \iff A^T X = B^T$, то есть достаточно уметь решать только уравнения первого типа.


\subsubsection{Тип 1}

$\underset{n \times m}{A} \underset{m \times p}{X} = \underset{n \times p}{B}$ -- это уравнение равносильно системе
\begin{equation*}
    \begin{cases}
        \begin{aligned}
            AX^{(1)} &= B^{(1)} \\
            AX^{(2)} &= B^{(2)} \\
            &\vdots \\
            AX^{(p)} &= B^{(p)} \\
        \end{aligned}
    \end{cases}
\end{equation*}

Этот набор СЛУ надо решать одновременно методом Гаусса.

Записываем матрицу $(A \mid B)$ и элементарными преобразованиями строк с ней приводим $A$ к улучшенному ступенчатому виду.

Получаем $(A' \mid B')$, где $A'$ имеет улучшенный ступенчатый вид.

Остается выписать общее решение для каждой СЛУ
\begin{equation*}
    \begin{cases}
        \begin{aligned}
            A' x^{(1)} &= B'^{(1)} \\
            A' x^{(2)} &= B'^{(2)} \\
            &\vdots \\
            A' x^{(p)} &= B'^{(p)}
        \end{aligned}
    \end{cases}
\end{equation*}


\subsection{Обратные матрицы}

\begin{definition}
    Матрица $B \in M_n$ называется \textit{обратной}, к $A$, если $AB = BA = E$.

    Обозначение: $B = A^{-1}$
\end{definition}


% Используем следствие без теоремы/утверждения :/
\stepcounter{theorem}

Факты:
\begin{enumerate}
\item Если $\exists A^{-1}$, то она определена однозначно

    \begin{proof}
        Пусть $B, B'$ -- две матрицы, обратные к $A$. Тогда $B = B(AB') = (BA)B' = B'$.
    \end{proof}

\item Если $AB = E$ для некоторой $B \in M_n$, то $BA = E$ автоматически и тогда $B = A^{-1}$

    \begin{comment}
        Доказывается на Лекции 7 % TODO: добавить референс
    \end{comment}
\end{enumerate}

\begin{corollary}
    $A^{-1}$ является решение матричного уравнения $AX = E$ (если решение существует)
\end{corollary}

\subsection{Перестановки на множестве $\{1, 2, \dots, n\}$}

\begin{definition}
    \textit{Перестановкой множества} $\{1, 2, \dots, n\}$ называется упорядоченный набор $(i_1, i_2, \dots, i_n)$, в котором каждое число от 1 до $n$ встречается ровно один раз. 
\end{definition}

Обозначение: $P_n$ -- множество всех перестановок множества $\{1, 2, \dots, n\}$.

Например, $(4, 2, 1, 3) \in P_4$.

\begin{definition}
    \textit{Подстановкой} на множестве $\{1, 2, \dots, n\}$ называется всякое биективное (взаимно однозначное) отображение множества $\{1, 2, \dots, n\}$ в себя.

    \begin{equation*}
        \sigma : \{1, 2, \dots, n\} \to \{1, 2, \dots, n\}
    .\end{equation*}

    \begin{equation*}
        \begin{pmatrix}
            1 & 2 & 3 & \dots & n \\
            i_1 & i_2 & i_3 & \dots i_n
        \end{pmatrix}
    \end{equation*}
\end{definition}
