\documentclass[a4paper]{article}
\usepackage{header}


\newcommand\enumtocitem[3]{\item\textbf{#1}\addtocounter{#2}{1}\addcontentsline{toc}{#2}{\protect{\numberline{#3}} #1}}
\newcommand\defitem[1]{\enumtocitem{#1}{subsection}{\thesubsection}}
\newcommand\proofitem[1]{\enumtocitem{#1}{subsubsection}{\thesubsubsection}}

\newlist{colloq}{enumerate}{1}
\setlist[colloq]{label=\textbf{\arabic*.}}


\title{\HugeЛинейная алгебра, Коллоквиум \uppercase\expandafter{\romannumeral 2\relax}}
\author{
    Бобень Вячеслав \\
    \href{https://teleg.run/darkkeks}{@darkkeks},
    \href{https://github.com/LoDThe/hse-tex}{GitHub} \\
    Благодарность выражается Левину Александру (\href{https://teleg.run/azerty1234567890}{@azerty1234567890}) \\
    и Милько Андрею (\href{https://teleg.run/andrew_milko}{@andrew\_milko}) за видеозаписи лекций.
}

\usepackage[yyyymmdd,hhmmss]{datetime}
\settimeformat{xxivtime}
\renewcommand{\dateseparator}{.}
\date{Дата изменения: \today \ в \currenttime \\ 2019 --- 2020}

\begin{document}
    \maketitle

    \epigraph{
        ``К коллоку можете даже не готовиться''.
    }{\rightline{{\rm --- Роман Сергеевич Авдеев}}}

    \newpage

    \section{Определения и формулировки}

    \begin{colloq}
        \defitem{Сумма двух подпространств векторного пространства}
        \defitem{Теорема о связи размерности суммы двух подпространств с размерностью их пересечения}
        \defitem{Сумма нескольких подпространств векторного пространства}
        \defitem{Линейная независимость нескольких подпространств векторного пространства}
        \defitem{Разложение векторного пространства в прямую сумму подпространств}
        \defitem{При каких условиях на подпространства $U_1$, $U_2$ векторного пространства $V$ имеет место разложение $V = U_1 \oplus U_2$?}
        \defitem{Проекция вектора на подпространство вдоль дополнительного подпространства}
        \defitem{Матрица перехода от одного базиса векторного пространства к другому}
        \defitem{Формула преобразования координат вектора при замене базиса}
        \defitem{Линейное отображение векторных пространств, его простейшие свойства}
        \defitem{Изоморфизм векторных пространств, изоморфные векторные пространства}
        \defitem{Какими свойствами обладает отношение изоморфности на множестве всех векторных пространств?}
        \defitem{Критерий изоморфности двух конечномерных векторных пространств}
        \defitem{Матрица линейного отображения}
        \defitem{Связь между координатами вектора и его образа при линейном отображении}
        \defitem{Формула изменения матрицы линейного отображения при замене базисов}
        \defitem{Сумма двух линейных отображений и её матрица. Произведение линейного отображения на скаляр и его матрица}
        \defitem{Композиция двух линейных отображений и её матрица}
        \defitem{Ядро и образ линейного отображения. Являются ли они подпространствами в соответствующих векторных пространствах?}
        \defitem{Критерий инъективности линейного отображения в терминах его ядра}
        \defitem{Связь между рангом матрицы линейного отображения и размерностью его образа}
        \defitem{Каким свойством обладает набор векторов, дополняющих базис ядра линейного отображения до базиса всего пространства?}
        \defitem{Теорема о связи размерностей ядра и образа линейного отображения}
        \defitem{К какому простейшему виду можно привести матрицу линейного отображения путём замены базисов?}
        \defitem{Линейная функция на векторном пространстве}
        \defitem{Сопряжённое (двойственное) векторное пространство и его размерность}
        \defitem{Базис сопряжённого пространства, двойственный к данному базису исходного векторного пространства}
        \defitem{Билинейная форма на векторном пространстве}
        \defitem{Матрица билинейной формы}
        \defitem{Формула для вычисления значений билинейной формы в координатах}
        \defitem{Формула изменения матрицы билинейной формы при замене базисов}
        \defitem{Симметричная билинейная форма. Критерий симметричности билинейной формы в терминах её матрицы}
        \defitem{Квадратичная форма}
        \defitem{Соответствие между симметричными билинейными формами и квадратичными формами}
        \defitem{Симметризация билинейной формы}
        \defitem{Поляризация квадратичной формы}
        \defitem{Матрица квадратичной формы}
        \defitem{Канонический вид квадратичной формы}
        \defitem{Нормальный вид квадратичной формы над $\RR$}
        \defitem{Индексы инерции квадратичной формы над $\RR$}
        \defitem{Закон инерции для квадратичной формы над $\RR$}
        \defitem{Положительно/неотрицательно определённая квадратичная форма над $\RR$}
        \defitem{Отрицательно/неположительно определённая квадратичная форма над $\RR$}
        \defitem{Неопределённая квадратичная форма над $\RR$}
        \defitem{Способ нахождения индексов инерции квадратичной формы над $\RR$, вытекающий из метода Якоби}
        \defitem{Критерий Сильвестра положительной определённости квадратичной формы над $\RR$}
        \defitem{Критерий отрицательной определённости квадратичной формы над $\RR$}
        \defitem{Евклидово пространство}
        \defitem{Длина вектора в евклидовом пространстве}
        \defitem{Неравенство Коши--Буняковского}
        \defitem{Угол между ненулевыми векторами евклидова пространства}
        \defitem{Матрица Грама системы векторов евклидова пространства}
        \defitem{Свойства определителя матрицы Грама}
        \defitem{Ортогональное дополнение подмножества евклидова пространства}
        \defitem{Чему равна размерность ортогонального дополнения к подпространству евклидова пространства?}
        \defitem{Каким свойством обладают подпространство евклидова пространства и его ортогональное дополнение?}
        \defitem{Ортогональная проекция вектора на подпространство}
        \defitem{Ортогональная составляющая вектора относительно подпространства}
        \defitem{Формула для ортогональной проекции вектора на подпространство в $\RR^n$, заданное своим базисом}
        \defitem{Ортогональная система векторов евклидова пространства. Ортогональный базис}
        \defitem{Ортонормированная система векторов евклидова пространства. Ортонормированный базис}
        \defitem{Описание всех ортонормированных базисов евклидова пространства в терминах одного такого базиса и матриц перехода}
        \defitem{Ортогональная матрица}
        \defitem{Формула для координат вектора в ортогональном и ортонормированном базисах евклидова пространства}
        \defitem{Формула для ортогональной проекции вектора на подпространство в терминах его ортогонального базиса}
        \defitem{Метод ортогонализации Грама--Шмидта}
        \defitem{Теорема Пифагора в евклидовом пространстве}
        \defitem{Расстояние между векторами евклидова пространства}
        \defitem{Неравенство треугольника в евклидовом пространстве}
        \defitem{Теорема о расстоянии между вектором и подпространством в терминах ортогональной составляющей}
        \defitem{Псевдорешение несовместной системы линейных уравнений}
        \defitem{Формула для расстояния от вектора до подпространства в терминах матриц Грама}
        \defitem{$k$-мерный параллелепипед и его объём}
        \defitem{Формула для объёма $k$-мерного параллелепипеда в $n$-мерном евклидовом пространстве}
        \defitem{Формула для объёма $n$-мерного параллелепипеда в $n$-мерном евклидовом пространстве в терминах координат в ортонормированном базисе}
        \defitem{В каком случае два базиса евклидова пространства называются одинаково ориентированными?}
        \defitem{Ориентированный объём $n$-мерного параллелепипеда в $n$-мерном евклидовом пространстве}
        \defitem{Свойства ориентированного объёма $n$-мерного параллелепипеда в $n$-мерном евклидовом пространстве}
        \defitem{Связь векторного произведения со скалярным и ориентированным объёмом}
        \defitem{Формула для вычисления векторного произведения в терминах координат в положительно ориентированном ортонормированном базисе}
        \defitem{Смешанное произведение векторов трёхмерного евклидова пространства}
        \defitem{Формула для вычисления смешанного произведения в терминах координат в положительно ориентированном ортонормированном базисе}
        \defitem{Критерий компланарности трёх векторов трёхмерного евклидова пространства}
        \defitem{Критерий коллинеарности двух векторов трёхмерного евклидова пространства}
        \defitem{Геометрические свойства векторного произведения}
        \defitem{Линейное многообразие. Характеризация линейных многообразий как сдвигов подпространств}
        \defitem{Критерий равенства двух линейных многообразий. Направляющее подпространство и размерность линейного многообразия}
        \defitem{Теорема о плоскости, проходящей через $k + 1$ точку в $\RR^n$}
        \defitem{Три способа задания прямой в $\RR^2$. Уравнение прямой в $\RR^2$, проходящей через две различные точки}
        \defitem{Три способа задания плоскости в $\RR^3$}
        \defitem{Уравнение плоскости в $\RR^3$, проходящей через три точки, не лежащие на одной прямой}
        \defitem{Три способа задания прямой в $\RR^3$}
        \defitem{Уравнения прямой в $\RR^3$, проходящей через две различные точки}
        \defitem{Случаи взаимного расположения двух прямых в $\RR^3$}
        \defitem{Формула для расстояния от точки до прямой в $\RR^3$}
        \defitem{Формула для расстояния от точки до плоскости в $\RR^3$}
        \defitem{Формула для расстояния между двумя скрещивающимися прямыми в $\RR^3$}
        \defitem{Линейный оператор}
        \defitem{Матрица линейного оператора}
        \defitem{Связь между координатами вектора и его образа при действии линейного оператора}
        \defitem{Формула изменения матрицы линейного оператора при переходе к другому базису}
        \defitem{Подобные матрицы}
        \defitem{Подпространство, инвариантное относительно линейного оператора}
        \defitem{Вид матрицы линейного оператора в базисе, дополняющем базис инвариантного подпространства}
        \defitem{Вид матрицы линейного оператора в базисе, согласованном с разложением пространства в прямую сумму двух инвариантных подпространств}
        \defitem{Собственный вектор линейного оператора}
        \defitem{Собственное значение линейного оператора}
        \defitem{Спектр линейного оператора}
        \defitem{Диагонализуемый линейный оператор}
        \defitem{Критерий диагонализуемости линейного оператора в терминах собственных векторов}
        \defitem{Собственное подпространство линейного оператора}
        \defitem{Характеристический многочлен линейного оператора}
        \defitem{Связь спектра линейного оператора с его характеристическим многочленом}
        \defitem{Алгебраическая кратность собственного значения линейного оператора}
        \defitem{Геометрическая кратность собственного значения линейного оператора}
        \defitem{Связь между алгебраической и геометрической кратностями собственного значения линейного оператора}
        \defitem{Критерий диагонализуемости линейного оператора в терминах его характеристического многочлена и кратностей его собственных значений}
        \defitem{Линейное отображение евклидовых пространств, сопряжённое к данному}
        \defitem{Линейный оператор в евклидовом пространстве, сопряжённый к данному}
        \defitem{Самосопряжённый линейный оператор в евклидовом пространстве}
        \defitem{Теорема о каноническом виде самосопряжённого линейного оператора}
        \defitem{Каким свойством обладают собственные подпространства самосопряжённого линейного оператора, отвечающие попарно различным собственным значениям}
        \defitem{Приведение квадратичной формы к главным осям}
        \defitem{Ортогональный линейный оператор}
        \defitem{Теорема о пяти эквивалентных условиях, определяющих ортогональный линейный оператор}
        \defitem{Теорема о каноническом виде ортогонального линейного оператора}
        \defitem{Классификация ортогональных линейных операторов в трёхмерном евклидовом пространстве}
    \end{colloq}


    \newpage
    \section{Вопросы на доказательство}

    \subsection{Подпространства}
    \begin{colloq}
        \proofitem{Теорема о связи размерности суммы двух подпространств с размерностью их пересечения}
        \proofitem{Теорема о пяти эквивалентных условиях, определяющих линейно независимый набор подпространств векторного пространства}
    \end{colloq}

    \subsection{Линейные отображения}
    \begin{colloq}
        \proofitem{Свойства отношения изоморфности на множестве всех векторных пространств}
        \proofitem{Критерий изоморфности двух конечномерных векторных пространств}
        \proofitem{Существование и единственность линейного отображения с заданными образами базисных векторов}
        \proofitem{Связь между координатами вектора и его образа при линейном отображении}
        \proofitem{Формула изменения матрицы линейного отображения при замене базисов}
        \proofitem{Изоморфизм $\hom(V,W) \MapsTo \text{Mat}_{m \times n}(F)$ при фиксированных базисах $V$ и $W$}
        \proofitem{Матрица композиции двух линейных отображений}
        \proofitem{Утверждение о том, что ядро и образ --- подпространства в соответствующих векторных пространствах}
        \proofitem{Связь между рангом матрицы линейного отображения и размерностью его образа}
        \proofitem{Лемма о дополнении базиса ядра линейного отображения до базиса всего пространства}
        \proofitem{Теорема о связи размерностей ядра и образа линейного отображения, приведение матрицы линейного отображения к диагональному виду с единицами и нулями на диагонали}
    \end{colloq}

    \subsection{Линейные, билинейные и квадратичные формы}
    \begin{colloq}
        \proofitem{Свойство базиса сопряжённого векторного пространства}
        \proofitem{Формула для вычисления значений билинейной формы в координатах}
        \proofitem{Существование и единственность билинейной формы с заданной матрицей}
        \proofitem{Формула изменения матрицы билинейной формы при переходе к другому базису}
        \proofitem{Критерий симметричности билинейной формы в терминах её матрицы в каком-либо базисе}
        \proofitem{Соответствие между симметричными билинейными формами и квадратичными формами}
        \proofitem{Метод Лагранжа приведения квадратичной формы к каноническому виду}
        \proofitem{Метод Якоби приведения квадратичной формы к каноническому виду}
        \proofitem{Существование нормального вида для квадратичной формы над $\RR$}
        \proofitem{Закон инерции}
        \proofitem{Следствие метода Якоби о нахождении индексов инерции квадратичной формы над $\RR$}
        \proofitem{Критерий Сильвестра положительной определённости квадратичной формы, критерий отрицательной определёности квадратичной формы}
    \end{colloq}

    \subsection{Евклидовы пространства}
    \begin{colloq}
        \proofitem{Неравенство Коши--Буняковского}
        \proofitem{Свойства определителя матрицы Грама системы векторов евклидова пространства}
        \proofitem{Свойства ортогонального дополнения к подпространству в евклидовом пространстве: размерность, разложение в прямую сумму, ортогональное дополнение к ортогональному дополнению}
        \proofitem{Формула для ортогональной проекции вектора на подпространство в $\RR^n$ в терминах его произвольного базиса}
        \proofitem{Существование ортонормированного базиса в евклидовом пространстве, дополнение ортогональной (ортонормированной) системы векторов до ортогонального (ортонормированного) базиса}
        \proofitem{Описание всех ортонормированных базисов в терминах одного и матриц перехода}
        \proofitem{Формула для координат вектора в ортогональном (ортонормированном) базисе. Формула для ортогональной проекции вектора на подпространство в терминах его ортогонального (ортонормированного) базиса}
        \proofitem{Теорема Пифагора и неравенство треугольника в евклидовом пространстве}
        \proofitem{Теорема о расстоянии между вектором и подпространством в терминах ортогональной составляющей}
        \proofitem{Метод наименьших квадратов для несовместных систем линейных уравнений: постановка задачи и её решение. Единственность псевдорешения и явная формула для него в случае линейной независимости столбцов матрицы коэффициентов}
        \proofitem{Формула для расстояния между вектором и подпространством в терминах матриц Грама}
        \proofitem{Две формулы для объёма $k$-мерного параллелепипеда в евклидовом пространстве}
    \end{colloq}

    \subsection{Элементы аналитической геометрии и линейные многообразия}
    \begin{colloq}
        \proofitem{Теорема о векторном произведении и формуле для него в координатах в положительно ориентированном ортонормированном базисе}
        \proofitem{Критерий коллинеарности двух векторов трёхмерного евклидова пространства}
        \proofitem{Геометрические свойства векторного произведения}
        \proofitem{Антикоммутативность и билинейность векторного произведения}
        \proofitem{Линейные многообразия как сдвиги подпространств}
        \proofitem{Критерий равенства двух линейных многообразий}
        \proofitem{Теорема о плоскости, проходящей через $k+1$ точку в $\RR^n$}
    \end{colloq}


    \subsection{Линейные операторы}
    \begin{colloq}
        \proofitem{Критерий обратимости линейного оператора в терминах его ядра, образа и определителя}
        \proofitem{Критерий диагонализуемости линейного оператора в терминах собственных векторов}
        \proofitem{Связь спектра линейного оператора с его характеристическим многочленом}
        \proofitem{Связь между алгебраической и геометрической кратностями собственного значения линейного оператора}
        \proofitem{Линейная независимость собственных подпространств линейного оператора, отвечающих попарно различным собственным значениям}
        \proofitem{Диагонализуемость линейного оператора, у которого число корней характеристического многочлена равно размерности пространства}
        \proofitem{Критерий диагонализуемости линейного оператора в терминах его характеристического многочлена и кратностей собственных значений}
        \proofitem{Существование собственного вектора у линейного оператора в векторном пространстве над $\CC$. Существование одномерного или двумерного инвариантного подпространства для линейного оператора в векторном пространстве над $\RR$}
    \end{colloq}


    \subsection{Линейные отображения и операторы в евклидовых пространствах}
    \begin{colloq}
        \proofitem{Сопряжённое линейное отображение: определение, существование и единственность. Матрица сопряжённого отображения в паре произвольных и паре ортонормированных базисов}
        \proofitem{Инвариантность ортогонального дополнения к подпространству, инвариантному относительно самосопряжённого линейного оператора}
        \proofitem{Существование собственного вектора для самосопряжённого линейного оператора}
        \proofitem{Существование ортонормированного базиса из собственных векторов для самосопряжённого линейного оператора}
        \proofitem{Приведение квадратичной формы к главным осям}
        \proofitem{Теорема о пяти эквивалентных условиях, определяющих ортогональный линейный оператор}
        \proofitem{Инвариантность ортогонального дополнения к подпространству, инвариантному относительно ортогонального линейного оператора}
        \proofitem{Теорема о каноническом виде ортогонального линейного оператора}
    \end{colloq}

\end{document}
