\section{Лекция 7.11.2019}

\subsection{Модуль комплексного числа, его свойства}

\begin{definition}
    Число $|z| = \sqrt{a^2 + b^2}$ называется \textit{модулем числа} $z = a + bi \in \CC$ (то есть длина соответствующего вектора).
\end{definition}

\textbf{Свойства}
\begin{enumerate}
\item $|z| \geq 0$, причем $|z| = 0 \iff z = 0$.
\item $|z + w| \leq |z| + |w|$ (неравенство треугольника).

    Пусть $z = a + bi$, $w = c + di$.
    \begin{align*}
        |z + w| &\leq |z| + |w| \\
        \sqrt{(a + c)^2 + (b + d)^2} &\leq \sqrt{a^2 + b^2} + \sqrt{c^2 + d^2} \\
        (a + c)^2 + (b + d)^2 &\leq a^2 + b^2 + c^2 + d^2 + 2\sqrt{(a^2 + b^2)(c^2 + d^2)} \\
        ac + bd &\leq\sqrt{(a^2 + b^2)(c^2 + d^2)} \\
        ac + bd &\leq\sqrt{(ac)^2 + (ad)^2 + (bc)^2 + (bd)^2} \\
        (ac)^2 + (bd)^2 + 2acbd &\leq (ac)^2 + (ad)^2 + (bc)^2 + (bd)^2 \\
        2acbd &\leq (ad)^2 + (bc)^2 \\
        0 &\leq (ad)^2 + (bc)^2 - 2abcd \\
        0 &\leq (ad - bc)^2
    \end{align*}
\item $z \overline{z} = |z|^2$.

    $z \overline{z} = (a + bi)(a - bi) = a^2 - (bi)^2 = a^2 + b^2 = |z|^2$
\item $|zw| = |z||w|$.

    $|zw|^2 = (zw) \cdot (\overline{zw}) = z \cdot w \cdot \overline{z} \cdot \overline{w} = |z|^2 |w|^2$.
\end{enumerate}

\begin{comment}
    Из 3) следует, что для $\forall z \neq 0$, $z^{-1} = \frac{\overline{z}}{|z|^2}$, то есть $(a + bi)^{-1} = \frac{a-bi}{a^2 + b^2}$
\end{comment}


\subsection{Аргумент комплексного числа}

Пусть $z = a + bi \in \CC$, $z \neq 0$.

Тогда, $z = |z| \left(\frac{a}{|z|} + \frac{b}{|z|}i\right)$, при этом $\left(\frac{a}{|z|}\right)^2 + \left(\frac{b}{|z|}i\right)^2 = 1$

Значит, $\frac{a}{|z|}$ и $\frac{b}{|z|}$ являются синусом и косинусом некоторого угла.

\begin{definition}
    \textit{Аргументом числа} $z = a + bi \in \CC \setminus \{0\}$ называется число $\phi \in \RR$, такое что
    \begin{equation*}
        \cos \phi = \frac{a}{|z|} = \frac{a} {\sqrt{a^2 + b^2}}
    .\end{equation*}

    \begin{equation*}
        \sin \phi = \frac{b}{|z|} = \frac{b}{\sqrt{a^2 + b^2}}
    .\end{equation*}

    В геометрических терминах, $\phi$ есть угол между осью $Ox$ и соответствующим вектором.
\end{definition}

\begin{comment}
    При $z \neq 0$, аргумент определен с точностью до $2 \pi k$, $k \in \ZZ$.
\end{comment}

\begin{comment}
    При $z = 0$, удобно считать что любое $\phi$ является аргументом.
\end{comment}


\subsection{Тригонометрическая форма комплексного числа}

$Arg(z) :=$ множество всех аргументов числа $z$.

$arg(z) :=$ единственное значение из $Arg(z)$, лежащее в $[0; 2 \pi)$.

$Arg(z) = arg(z) + 2 \pi k$, $k \in \ZZ$

$Arg(z) = \{\phi \in \RR \mid \cos \phi = \frac{a}{|z|}, \sin \phi = \frac{b}{|z|}\}$

Тогда, $\forall z \in \CC$, $z = |z| \left(\frac{a}{|z|} + \frac{b}{|z|} i\right) = |z| \left(\cos \phi + i \sin \phi\right)$, где $\phi \in Arg(z)$.

\begin{definition}
    Представление числа $z \in \CC$ в виде $z = |z|(\cos \phi + i \sin \phi)$ называется его \textit{тригонометрической формой}.
\end{definition}


\subsection{Умножение и деление комплексных чисел в тригонометрической форме}

\begin{proposal}
    Пусть $z_1 = |z_1| (\cos \phi_1 + i \sin \phi_1)$ и $z_2 = |z_2| (\cos \phi_2 + i \sin \phi_2)$, тогда
    \begin{equation*}
        z_1 z_2 = |z_1| |z_2| (\cos (\phi_1 + \phi_2) + i \sin(\phi_1 + \phi_2))
    .\end{equation*}
\end{proposal}

\begin{proof}
    \begin{align*}
        z_1 z_2 &= |z_1||z_2|(\cos \phi_1 + i \sin \phi_1)(\cos \phi_2 + i \sin \phi_2)  \\
                &= |z_1||z_2|((\cos \phi_1 \cos \phi_2 - \sin \phi_1 \sin \phi_2) + i(\cos \phi_1 \sin \phi_2 + \sin \phi_1 \cos \phi_2)) \\
                &= |z_1||z_2|(\cos (\phi_1 + \phi_2) + i \sin(\phi_1 + \phi_2)) \qedhere
    .\end{align*}
\end{proof}

\begin{corollary}
    В условиях предложения, предположим, что $z_2 \neq 0$.

    Тогда $\frac{z_1}{z_2} = \frac{|z_1|}{|z_2|} (\cos (\phi_1 - \phi_2) + i \sin(\phi_1 - \phi_2))$.

    В частности, $\frac{1}{|z_2|}(\cos(- \phi_2) + i\sin(- \phi_2)) = \frac{1}{|z_2|}(\cos \phi_2 - i \sin \phi_2) = \frac{\overline{z}_2}{|z_2|^2}$.
\end{corollary}

\subsection{Возведение в степень комплексных чисел в тригонометрической форме, формула Муавра}

\begin{corollary}
    Пусть $z = |z|(\cos \phi + i \sin \phi)$. Тогда $\forall n \in \ZZ$,

    \begin{equation*}
        z^n = |z|^n (\cos(n \phi) + i \sin(n \phi)) \text{ -- формула Муавра}
    .\end{equation*}
\end{corollary}

\begin{comment}
    В комплексном анализе функция $\exp: \RR \to \RR$, $x \to e^x$, доопределяется до функции $\exp: \CC \to \CC$, $z \to e^z$ с сохранением всех привычных свойств.

    Доказывается $e^{i \phi} = \cos \phi + i \sin \phi$, $\forall \phi \in \CC$ -- формула Эйлера.

    Тогда $\forall z \in \CC$ представляется в виде $z = |z| e^{i \phi}$, где $\phi \in Arg(z)$ -- \textit{показательная форма}.
\end{comment}

\subsection{Извлечение корней из комплексных чисел}

Пусть $z \in \CC$, $n \in \NN$, $n \geq 2$.

\begin{definition}
    \textit{Корнем степени n} (или \textit{корнем n-й степени)} из числа $z$ называется всякое число $w \in \CC$, что $w^n = z$.
\end{definition}

Положим $\sqrt[n]{z} := \{w \in \CC \ | \ w^n = z \}$.

\bigskip
Опишем множество $\sqrt[n]{z}$.

$w = \sqrt[n]{z} \implies w^n = z \implies |w|^n = |z|$.

Если $z = 0$, то $|z| = 0 \implies |w| = 0 \implies w = 0 \implies \sqrt[n]{0} = \{0\}$.

\bigskip
Далее считаем, что $z \neq 0$.

$z = |z|(\cos \phi + i \sin \phi)$

$w = |w|(\cos \psi + i \sin \psi)$

$z = w^n = |w|^n (\cos (n \psi) + i \sin (n \psi))$

Отсюда,
\begin{equation*}
    z = w^n \iff
    \begin{cases}
        |z| = |w|^n  \\
        n \psi = \phi + 2 \pi k \text{, для некоторого } k \in \ZZ
    \end{cases}
    \iff
    \begin{cases}
        |w| = \sqrt[n]{|z|}  \\
        \psi = \frac{\phi + 2 \pi k}{n} \text{, для некоторого } k \in \ZZ
    \end{cases}
\end{equation*}

С точностью до $2 \pi l$, $l \in \ZZ$, получается ровно $n$ различных значений для $\psi$, при $k = 0, 1, \dots, n-1$.

В результате $\sqrt[n]{z} = \{w_0, w_1, \dots, w_{n-1} \}$, где $w_k = \sqrt[n]{|z|}\left(\cos \frac{\phi + 2 \pi k}{n} + i \sin \frac{\phi + 2 \pi k}{n}\right)$

\begin{comment}
    Числа $w_0, w_1, \dots, w_{n-1}$ лежат в вершинах правильного n-угольника с центром в начале координат.
\end{comment}

Примеры.

$\sqrt{1} = \{\pm 1\}$

$\sqrt{-1} = \{\pm i\}$

$\sqrt[3]{1} = \{1, -\frac{1}{2} \pm i \frac{\sqrt{3}}{2} \}$

$\sqrt[4]{1} = \{ \pm 1, \pm i \}$


\subsection{Основная теорема алгебры комплексных чисел (без доказательства)}

$\sqrt[n]{z} = \{$ корни многочлена $x^n - z \}$.

\begin{theorem}
    Всякий многочлен степени $\geq 1$ с комплексными коэффициентами имеет комплексный корень.
\end{theorem}

Пусть $f = a_n z^n + a_{n - 1} z^{n - 1} + \dots + a_1 z_1 + a_0$, $n \geq 1$, $a_n \neq 0$, $a_i \in \CC$, тогда $\exists c \in \CC : f(c) = 0$.

\begin{comment}
    Свойство поля $\CC$, сформулированное в теореме, называется \textit{алгебраической замкнутостью}.
\end{comment}

\subsection{Деление многочленов с остатком}

Пусть $\FF$ -- поле.

$\FF[x] := $ все многочлены от переменной $x$ с коэффициентами из $\FF$.

$f(x) = a_n x^n + \dots + a_1 x + a_0$, $a_n \neq 0 \implies \deg f = n$.

$\deg(f \cdot g) = \deg f + \deg g$.

\begin{definition}
    Многочлен $f(x) \in F[x]$ \textit{делится} на $g(x) \in F[x]$, если $\exists h(x) \in F[x]$, такой что $f(x) = g(x) h(x)$.
\end{definition}

Если $f(x)$ не делится на $g(x)$, то можно поделить с остатком.

\begin{proposal}[деление с остатком]
    Если $f(x), g(x) \in F[x]$, $g(x) \neq 0$, то $\exists! q(x), r(x) \in F[x]$, такие что
    \begin{equation*}
        \begin{cases}
            f(x) = q(x) g(x) + r(x) \\
            \text{либо } r(x) = 0 \text{, либо } \deg r(x) < \deg g(x)
        \end{cases}
    \end{equation*}
\end{proposal}

\begin{example}
    $f(x) = x^3 - 2x$, $g(x) = x + 1$.

    $f(x) = (x^2 - x - 1)(x + 1) + 1$, $q(x) = (x^2 - x - 1)$, $r(x) = 1$.
\end{example}


\subsection{Теорема Безу}

Частный случай деления многочлена $f(x)$ на многочлен $g(x)$ с остатком: $g(x) = x - c$, $\deg g(x) = 1$:

$f(x) = q(x) (x - c) + r(x)$, где либо $r(x) = 0$, либо $\deg r(x) < g(x) = 1$

Значит, $r(x) \equiv r = const \in F$.

\begin{theorem}
    $r = f(c)$.
\end{theorem}

\begin{proof}
    Подставить $x = c$ в $f(x) = (x - c)g(x) + r(x)$.
\end{proof}

\begin{corollary}
    Элемент $c \in F$ является корнем многочлена $f(x) \in F[x]$ тогда и только тогда, когда $f(x)$ делится на $(x - c)$.
\end{corollary}


\subsection{Кратность корня многочлена}

\begin{definition}
    \textit{Кратностью} корня $c \in F$ многочлена $f(x)$ называется наибольшее целое $k$ такое что, $f(x)$ делится на $(x - c)^k$.
\end{definition}

\subsection{Утверждение о том, что всякий многочлен степени n с комплексными коэффициентами имеет ровно n корней с учётом кратностей}

\begin{corollary}
    Пусть $f(z) \in F[z]$, $\deg f = n \geq 1$.

    $f(x) = a_n z^n + \dots + a_1 z + a_0$.

    $c_1, \dots c_s$ -- корни $f$, $k_1, \dots, k_s$ -- их кратности.

    \bigskip
    Любой многочлен с комплексными коэффициентами разлагается в произведение линейных множителей:
    \begin{equation*}
        f(x) = a_n (x - c_1)^{k_1} (x - c_2)^{k_2} \dots (x - c_s)^{k_s}
    .\end{equation*}

    Иными словами, $f(z)$ имеет ровно $n$ корней с учетом кратностей.
\end{corollary}
