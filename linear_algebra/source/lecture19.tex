\section{Лекция 6.02.2020}


\subsection{Утверждение о том, что всякий базис сопряжённого пространства двойствен некоторому базису исходного пространства}

$\epsilon_i (x_1 e_1 + \dots + x_n e_n) = x_i$, поэтому $\epsilon_i$ называется $i$-й координатной функцией в базисе $\E$.

\begin{proposal}
    Всякий базис пространства $V^*$ двойствен некоторому базису пространства $V$. 
\end{proposal}

\begin{proof}
    Пусть $\epsilon = \begin{pmatrix} \epsilon_1 \\ \dots \\ \epsilon_n \end{pmatrix}$ --- базис пространства $V^*$.
    Фиксируем какой-то базис $\E' = (e'_1, \dots, e'_n)$ пространства $V$, и пусть $\epsilon' = \begin{pmatrix} \epsilon'_1 \\ \dots \\ \epsilon'_n \end{pmatrix}$ --- соответствующий ему двойственный базис $V^*$.

    Тогда, 
    $\begin{pmatrix} \epsilon_1 \\ \dots \\ \epsilon_n \end{pmatrix} = C \cdot \begin{pmatrix} \epsilon'_1 \\ \dots \\ \epsilon'_n \end{pmatrix}$
    для некоторой матрицы $C \in M^0_n(F)$.

    \bigskip
    Положим $(e_1, \dots, e_n) = (e'_1, \dots, e'_n) \cdot C^{-1}$.

    Тогда, 
    \begin{equation*}
        \begin{pmatrix} \epsilon_1 \\ \dots \\ \epsilon_n \end{pmatrix} (e_1, \dots, e_n) = C \begin{pmatrix} \epsilon'_1 \\ \dots \\ \epsilon'_n \end{pmatrix} (e'_1, \dots, e'_n) C^{-1} = C E C^{-1} = E
    .\end{equation*}

    Значит, $\epsilon$ двойствен к $\E$.
\end{proof}

\begin{exercise}
    $\E$ определён однозначно.
\end{exercise}


\subsection{Билинейные формы на векторном пространстве}

Пусть $V$ --- векторное пространство над $F$.

\begin{definition}
    \textit{Биленейная форма} на $V$ --- это отображение $\beta : V \times V \to F$, линейная по каждому аргументу.
\end{definition}

\paragraph{Линейность по 1-му аргументу}
\begin{itemize}[nosep]
\item $\beta(x_1 + x_2, y) = \beta(x_1, y) + \beta(x_2, y) \quad \forall x_1, x_2, y \in V$,
\item $\beta(\lambda x, y) = \lambda\beta(x, y) \quad \forall x, y \in V, \ \lambda \in F$.
\end{itemize}

\paragraph{Линейность по 2-му аргументу}
\begin{itemize}[nosep]
\item $\beta(x, y_1 + y_2) = \beta(x, y_1) + \beta(x, y_2) \quad \forall x, y_1, y_2 \in V$,
\item $\beta(x, \lambda y) = \lambda\beta(x, y) \quad \forall x, y \in V, \ \lambda \in F$.
\end{itemize}


\subsection{Примеры}

\subsubsection{}

$V = F^n$, $x = \begin{pmatrix} x_1 \\ \dots \\ x_n \end{pmatrix}$, $y = \begin{pmatrix} y_1 \\ \dots \\ y_n \end{pmatrix}$;

$\beta(x, y) := x_1 y_1 + \dots + x_n y_n = (x_1 \dots x_n) \begin{pmatrix} y_1 \\ \dots \\ y_n \end{pmatrix} = x^{T} y$.

\subsubsection{}

$V = F^2$, $x = \begin{pmatrix} x_1 \\ x_2 \end{pmatrix}$, $y = \begin{pmatrix} y_1 \\ y_2 \end{pmatrix}$;

$\beta(x, y) := \det \begin{pmatrix} x_1 & y_1 \\ x_2 & y_2 \end{pmatrix}$.

\subsubsection{}

$V = C[a, b]; \quad \beta(f, g) := \displaystyle\int^b_a f(x) \mathop{}\!d x$.


\subsection{Матрица билинейной формы по отношению к фиксированному базису}

Далее считаем, что $\dim V = n < \infty$.

Пусть $\E = (e_1, \dots, e_n)$ --- базис $V$.

\begin{definition}
    Матрицей билинейной формы $\beta$ в базисе $\E$ называется такая матрица $B \in M_n$, что $b_{ij} = \beta(e_i, e_j)$.

    Обозначение: $B(\beta, \E)$.
\end{definition}

\paragraph{Примеры}
\begin{enumerate}
\item Пусть $\E$ --- стандартный базис, тогда $B(\beta, \E) = E$.
\item Пусть $\E$ --- стандартный базис, тогда $B(\beta, \E) = \begin{pmatrix} 0 & 1 \\ -1 & 0 \end{pmatrix}$.
\end{enumerate}

\paragraph{Формула вычисления значений билинейной формы в координатах}~\\

Пусть
\begin{math}
    \begin{aligned}[t]
        x &= x_1 e_1 + \dots + x_n e_n, \\
        y &= y_1 e_1 + \dots + y_n e_n.
    \end{aligned}
\end{math}

Тогда
\begin{align*}
    \beta(x, y)
    &= \beta\left(\sum_{i = 1}^{n} x_i e_i, \ \sum_{j = 1}^{n} y_j e_j\right)
    = \sum_{i = 1}^{n} x_i \cdot \beta\left(e_i, \ \sum_{j = 1}^{n} y_j e_j\right) \\
    &= \sum_{i = 1}^{n} \sum_{j = 1}^{n} x_i y_j \underbracket{\beta(e_i, e_j)}_{\beta_{ij}}
    = \sum_{i = 1}^{n} \sum_{j = 1}^{n} x_i \beta_{ij} y_j  \\
    &= (x_1, \dots, x_n) B \begin{pmatrix} y_1 \\ \dots \\ y_n \end{pmatrix}
.\end{align*}


\subsection{Существование и единственность билинейной формы с заданной матрицей}

\begin{proposal}
    Пусть $\E$ --- фиксированный базис $V$.
    \begin{enumerate}
    \item Всякая билинейная форма $\beta$ на $V$ однозначно определяется матрицей $B(\beta, \E)$.
    \item $\forall B \in M_n(F) \ \exists!$ билинейная форма $\beta$ на $V$, такая что $B(\beta, \E) = B$.
    \end{enumerate}
\end{proposal}

\begin{proof}~
    \begin{enumerate}
    \item Следует из формулы выше.
    \item Единственность следует из формулы выше. Докажем существование:

        Определим $\beta$ по формуле выше.

        Тогда $\beta$ --- билинейная форма на $V$ (упражнение).

        \begin{equation*}
            \beta(e_i, e_j) = \bordermatrix{
                  &   &   &   & i &   &   &  \cr
                  & 0 & \dots & 0 & 1 & 0 & \dots & 0
            } \cdot B \cdot \begin{pmatrix} 
                0 \\ \dots \\ 0 \\ 1 \\ 0 \\ \dots \\ 0
            \end{pmatrix} \ j = \beta_{ij}
        .\end{equation*}

        Действительно, $B(\beta, \E) = B$.
        \qedhere
    \end{enumerate}
\end{proof}


\subsection{Формула изменения матрицы билинейной формы при переходе к другому базису}

$B = B(\beta, \E)$.

Пусть $\E' = (e'_1, \dots e'_n)$ --- другой базис $V$.

$\E' = e \cdot C$.

$B' := B(\beta, \E')$.

\begin{proposal}
    $B' = C^{T} B C$.
\end{proposal}

\begin{proof}
    \begin{equation*}
        x = x_1 e_1 + \dots + x_n e_n = x'_1 e'_1 + \dots + x'_n e'_n
    ,\end{equation*}

    \begin{equation*}
        y = y_1 e_1 + \dots + y_n e_n = y'_1 e'_1 + \dots + y'_n e'_n.
    ,\end{equation*}

    \begin{equation*}
        \begin{pmatrix} x_1 \\ \dots \\ x_n \end{pmatrix} = C \cdot \begin{pmatrix} x'_1 \\ \dots \\ x'_n \end{pmatrix} \quad \begin{pmatrix} y_1 \\ \dots \\ y_n \end{pmatrix} = C \cdot \begin{pmatrix} y'_1 \\ \dots \\ y'_n \end{pmatrix}
    .\end{equation*}

    Тогда,
    \begin{align*}
        \beta(x, y) &= (x_1 \dots x_n) B \begin{pmatrix} y_1 \\ \dots \\ y_n \end{pmatrix} = (x'_1 \dots x'_n) C^{T} B C \begin{pmatrix} y'_1 \\ \dots \\ y'_n \end{pmatrix} \\
        \beta(x, y) &= (x'_1 \dots x'_n) B' \begin{pmatrix} y'_1 \\ \dots \\ y'_n \end{pmatrix}
    .\end{align*}

    Получаем, что $B' = C^{T} B C$.
\end{proof}

\begin{corollary}
    Величина $\rk B$ не зависит от выбора базисов. 
\end{corollary}


\subsection{Ранг билинейной формы}

\begin{definition}
    Число $\rk B := \rk B(\beta, \E)$ называется \textit{рангом} билинейной формы $\beta$.
\end{definition}


\subsection{Симметричные билинейные формы}

\begin{definition}
    Билинейная форма $\beta$ называется \textit{симметричной}, если $\beta(x, y) = \beta(y, x) \ \forall x, y \in V$.
\end{definition}

\subsection{Критерий симметричности билинейной формы в терминах её матрицы в каком-либо базисе}
\subsection{Квадратичные формы на векторном пространстве}
\subsection{Примеры}
\subsection{Соответствие между симметричными билинейными формами и квадратичными формами}

