\section{Лекция 30.09.2019}

\subsection{Определитель с углом нулей}

\begin{proposal}
    \begin{equation*}
        A = \left(
            \begin{array}{c|c}
                P & Q \\
                \hline
                0 & R
            \end{array}
        \right) \text{ или } A =
        \left(
            \begin{array}{c|c}
                P & 0 \\
                \hline
                Q & R
            \end{array}
        \right), \ P \in M_k, \ R \in M_{n-k} \implies \det A = \det P \det R
    .\end{equation*}
\end{proposal}

% TODO Форматирование матриц
Матрица с углом нулей:
\begin{equation*}
\left(
\begin{array}{c|ccc}
  * & * & * & * \\
  \hline
  0 & * & * & * \\
  0 & * & * & * \\
  0 & * & * & *
\end{array}
\right)
\end{equation*}

НЕ матрица с углом нулей:
\begin{equation*}
\left(
\begin{array}{c|ccc}
  * & * & * & * \\
  * & * & * & * \\
  \hline
  0 & * & * & * \\
  0 & * & * & *
\end{array}
\right)
\end{equation*}

\begin{proof}
    В силу свойства T достаточно доказать для строк.

    \begin{enumerate}
    \item
        Элементарными преобразованиями строк в $A$, приведем $(P \mid Q)$ к виду $(P' \mid Q')$, в котором $P'$ имеет ступенчатый вид. При этом $\det A$ и $\det P$ умножаются на один и тот же скаляр $\alpha \neq 0$.

    \item
        Элементарными преобразованиями строк в $A$, приведем $(0 \mid R)$ к виду $(0 \mid R')$, в котором $R'$ имеет ступенчатый вид. При этом $\det A$ и $\det R$ умножаются на один и тот же скаляр $\beta \neq 0$.
    \end{enumerate}

    $\begin{pmatrix} P' & Q' \\ 0 & R' \end{pmatrix} $ -- верхнетреугольная $\implies \det \begin{pmatrix} P' & Q' \\ 0 & R' \end{pmatrix} = \det P' \det R'$.

    \begin{equation*}
        \alpha \beta \det A = \det \begin{pmatrix} P' & Q' \\ 0 & R' \end{pmatrix} = \det P' \det R' = (\alpha \det P) (\beta \det R) = \alpha \beta \det P \det R
    .\end{equation*}
\end{proof}


\subsection{Определитель произведения матриц}
\begin{theorem}
    $A, B \in M_n \implies \det(AB) = \det A \det B$.
\end{theorem}

\begin{proof}
    Выполним с матрицей $A$ одно элементарное преобразование строк, получим матрицу $A'$.
    \begin{equation*}
        A \leadsto A' = U A
    .\end{equation*}

    Такое же преобразование строк с $AB$.
    \begin{equation*}
        AB \leadsto U(AB) = (UA)B = A'B
    .\end{equation*}

    Таким образом, сначала выполнив элементарное преобразование и домножив на матрицу $B$, либо домножив на $B$ и затем применив элементарное преобразование, получим тот же результат.

    Тогда, цепочка элементарных преобразований строк:
    \begin{equation*}
        A \leadsto C \text{ -- улучшенный ступенчатый вид}
    .\end{equation*}

    Так же цепочка для $AB$:
    \begin{equation*}
        AB \leadsto CB
    .\end{equation*}

    При этом, $\det A$ и $\det AB$ умножились на один и тот же скаляр $\alpha \neq 0$
    \begin{equation*}
        \det C = \alpha \det A
    .\end{equation*}
    \begin{equation*}
        \det CB = \alpha \det AB
    .\end{equation*}

    \begin{description}
    \item[Случай 1]
        Последняя строка состоит из нулей:
        \begin{align*}
            &C_{(n)} = (0 \dots 0) \\
            &\implies [CB]_{(n)} = C_{(n)} B = (0 \dots 0) \\
            &\implies \det CB = 0 = 0 \cdot \det B = \det C \det B
        .\end{align*}

    \item[Случай 2]
        Последняя строка ненулевая:
        \begin{equation*}
            \tag{$star$}
            C_{(n)} \implies C = E
        ,\end{equation*}
        так как матрица $C$ имеет улучшенный ступенчатый вид.

        Значит,
        \begin{equation*}
            \det CB = \det B = 1 \cdot \det B = \det C \cdot \det B
        .\end{equation*}
    \end{description}

    Из этих двух случаем следует, что $\det CB = \det C \det B$.

    Сокращая $\alpha$ получаем,
    \begin{equation*}
        \det CB = \det C \det B \implies \det AB = \det A \det B
    .\end{equation*}
\end{proof}

\begin{comment}
    Пусть $A \in M_n$, $A_{\text{ул}}$ -- её улучшенный ступенчатый вид.
    \begin{equation*}
        \det A \neq 0 \iff A_{\text{ул}} = E
    .\end{equation*}
\end{comment}


\subsection{Дополнительные миноры и алгебраические дополнения к элементам квадратной матрицы}
\begin{definition}
    \textit{Дополнительным минором} к элементу $a_{ij}$ называется определитель $(n-1) \times (n-1)$ матрицы, получающейся из $А$ вычеркиванием $i$-ой строки и $j$-го столбца.

    Обозначение: $\overline{M}_{ij}$.
\end{definition}

\begin{definition}
    \textit{Алгебраическим дополнением к} элементу $a_{ij}$ называется число $A_{ij} = (-1)^{i+j} \overline{M}_{ij}$.
\end{definition}


\subsection{Лемма об определителе матрицы, содержащей ровно один ненулевой элемент в некоторой строке}
\begin{lemma}
    Пусть $a_{ik} = 0$ при всех $k \neq j$. Тогда $\det A = a_{ij} \cdot A_{ij}$.
\end{lemma}

\begin{proof}
    \begin{equation*}
        A = \left(
            \begin{array}{c|c|c}
              P & U & Q \\
              \hline
              0 \dots 0 & a_{ij} & 0 \dots 0 \\
              \hline
              R & V & S
            \end{array}
        \right)
    .\end{equation*}

    Переставляя соседние строки $i - 1$ раз, вытолкнем $i$-ю строку наверх.

    \begin{equation*}
        A' = \left(
            \begin{array}{c|c|c}
                0 \dots 0 & a_{ij} & 0 \dots 0 \\
                \hline
                P & U & Q \\
                \hline
                R & V & S
            \end{array}
        \right)
    \end{equation*}

    Переставляя соседние столбцы $j - 1$ раз, переместим $j$-й столбец на первое место.

    \begin{equation*}
        A'' = \left(
            \begin{array}{c|c|c}
                a_{ij} & 0 \dots 0 & 0 \dots 0 \\
                \hline
                U & P & Q \\
                \hline
                V & R & S
            \end{array}
        \right)
    \end{equation*}

    $\det A'' = a_{ij} \det \left(
        \begin{array}{c|c}
            P & Q \\
            \hline
            R & S
        \end{array}
    \right) = a_{ij} \overline{M}_{ij}$.

    \bigskip
    $\implies det A = (-1)^{i - 1 + j - 1} \det A'' = (-1)^{i + j} a_{ij} \overline{M}_{ij} = a_{ij} A_{ij}$.

\end{proof}


\subsection{Разложение определителя по строке (столбцу)}
\begin{theorem}
    При любом фиксированном $i \in \{1, 2, \dots, n\}$,
    \begin{equation*}
        \det A = a_{i1} A_{i1} + a_{i2} A_{i2} + \dots + a_{in} A_{in} = \sum_{j = 1}^n a_{ij} A_{ij} \text{ -- разложение по i-й строке}
    .\end{equation*}

    Аналогично, для любого фиксированного $j \in \{1, 2, \dots, n\}$,
    \begin{equation*}
        \det A = a_{1j} A_{1j} + a_{2j} A_{2j} + \dots + a_{nj} A_{nj} = \sum_{i = 1}^{n} a_{ij} A_{ij} \text{ -- разложение по j-у столбцу}
    .\end{equation*}
\end{theorem}

\begin{proof}
    В силу свойства T достаточно доказать для строк.
    \begin{equation*}
        A_{(i)} = (a_{i1}, 0, \dots, 0) + (0, a_{i2}, 0, \dots, 0) + \dots + (0, \dots, 0, a_{in})
    .\end{equation*}

    Требуемое следует из \hyperref[det:prop_2]{свойства 2} определителей и леммы.
\end{proof}


\subsection{Лемма о фальшивом разложении определителя}
\begin{lemma}~
    \begin{enumerate}
    \item
        При любых $i, k \in \{1, 2, \dots, n\} : i \neq k \implies \sum_{j = 1}^n a_{ij} A_{kj} = 0$.
    \item
        При любых $j, k \in \{1, 2, \dots, n\} : j \neq k \implies \sum_{i = 1}^n a_{ij} A_{ik} = 0$
    \end{enumerate}
\end{lemma}

\begin{proof}
    В силу свойства T достаточно доказать для строк.

    Пусть $B \in M_n$ -- матрица, полученная из $A$ заменой $k$-й строки на $i$-ю.

    \begin{equation*}
        B = \begin{pmatrix}
            A_{(1)} \\ \dots \\ A_{(i)} \\ \dots \\ A_{(i)} \\ \dots \\ A_{(n)}
        \end{pmatrix}
    \end{equation*}

    В $B$ есть две одинаковые строки $\implies \det B = 0$.

    Разлагая $\det B$ по $k$-й строке, получаем
    \begin{equation*}
        \det B = \sum_{j = 1}^n b_{kj} B_{kj} = \sum_{j = 1}^n a_{ij} A_{kj}
    .\end{equation*}
\end{proof}


\subsection{Обратная матрица, её единственность}

Пусть дана $A \in M_n$.

\begin{definition}
    Матрица $B \in M_n$ называется \textit{обратной} к $A$, если $AB = BA = E$.

    Обозначение: $A^{-1}$.
\end{definition}


\begin{lemma}
    Если $\exists A^{-1}$, то она единственна.
\end{lemma}

\begin{proof}
    Пусть $B, C \in M_n$ такие, что $AB = BA = E$ и $AC = CA = E$. Тогда,
    \begin{equation*}
        B = BE = B(AC) = (BA)C = EC = C \Rightarrow B = B'
    .\end{equation*}
\end{proof}


\subsection{Невырожденные матрицы}
\begin{definition}
    Матрица $A \in M_n$ называется \textit{невырожденной}, если $\det A \neq 0$, и \textit{вырожденной} иначе (то есть $\det A = 0$).
\end{definition}


\subsection{Определитель обратной матрицы}
\begin{lemma}
    Если $\exists A^{-1}$, то $\det A \neq 0$.
\end{lemma}

\begin{proof}
    $AA^{-1} = E \implies \det(AA^{-1}) = \det E \implies \det A \det(A^{-1}) = 1$.
\end{proof}


\subsection{Присоединённая матрица}
\begin{definition}
    \textit{Присоединенной} к А матрицей называется матрица $\widehat{A} = (A_{ij})^T$.
\end{definition}

\begin{equation*}
    \widehat{A} = \begin{pmatrix}
        A_{11} & A_{21} & \dots & A_{n1} \\
        A_{12} & A_{22} & \dots & A_{n2} \\
        \vdots & \vdots & \ddots & \vdots \\
        A_{1n} & A_{2n} & \dots & A_{nn}
    \end{pmatrix}
\end{equation*}


\subsection{Критерий обратимости квадратной матрицы, явная формула для обратной матрицы}
\begin{theorem}
    $A$ обратима (то есть $\exists A^{-1}$) $\iff$ $A$ невырождена ($\det A \neq 0$), при этом $A^{-1} = \frac{1}{\det A} \widehat{A}$.
\end{theorem}

\begin{proof}
    Утверждение в одну сторону следует из леммы 2.

    \bigskip
    Пусть $\det A \neq 0$. Покажем, что $\frac{1}{detA} \widehat{A} = A^{-1}$.
    Для этого достаточно доказать, что $A \widehat{A} = \widehat{A} A = \det(A) \cdot E$.

    Для $X = A \widehat{A}$ имеем
    \begin{equation*}
        x_{ij} = \sum_{k = 1}^n a_{ik} [\widehat{A}]_{kj} = \sum_{k = 1}^n a_{ik} A_{jk} = \begin{cases}
            detA, &\text{при }  i = j  \\
            0, &\text{при } i \neq j \\
        \end{cases}
    .\end{equation*}

    Для $Y = \widehat{A} A$ имеем

    \begin{equation*}
        y_{ij} = \sum_{k = 1}^n [\widehat{A}]_{ik} a_{kj} = \sum_{k = 1}^n A_{ki} a_{kj} = \begin{cases}
            detA, &\text{при }  i = j  \\
            0, &\text{при } i \neq j \\
        \end{cases}
    .\end{equation*}
\end{proof}
