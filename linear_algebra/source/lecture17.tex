\section{Лекция 23.01.2020}

\subsection{Матрица линейного отображения}

Пусть $V, W$ --- векторные пространства над $F$.

$\E = (e_1, \dots, e_n)$ --- базис $V$,

$\F = (f_1, \dots, f_m)$ --- базис $W$.

\bigskip
Пусть $\phi\colon V \to W$ --- линейное отображение.

$\forall j = 1, \dots, n$

$\phi(e_j) = a_{1j} f_1 + a_{2j} f_2 + \dots + a_{mj} f_m = (f_1, \dots, f_m) \begin{pmatrix} a_{1j} \\ a_{2j} \\ \dots \\ a_{mj} \end{pmatrix}$.

Тогда, $(\phi(e_1), \dots, \phi(e_n)) = (f_1, \dots, f_m) \cdot A$, где $A = (a_{ij}) \in \text{Mat}_{m \times n} (F)$.

\begin{definition}
    $A$ называется матрицей линейного отображения $\phi$ в базисах $\E$ и $\F$.

    Обозначение: $A = A(\phi, \E, \F)$.
\end{definition}

В $j$-м столбце матрицы $A$ стоят координаты вектора $\phi(e_j)$ в базисе $\F$.

\begin{designation}
    $\hom(V, W) := $ множество всех линейных отображений из $V$ в $W$.
\end{designation}

\begin{corollary}[из предложения \ref{lec16:propopop}]
    При фиксированных базисах $\E$ и $\F$ отображение $\phi \mapsto A(\phi, \E, \F)$ является биекцией между $\hom(V, W)$ и $\text{Mat}_{m \times n}(F)$.
\end{corollary}


\subsection{Примеры}

\begin{enumerate}[start=0]
\item $\phi(v) = 0 \ \forall v \implies \forall \E, \F \ A(\phi, \E, \F) = \begin{pmatrix} 0 & \dots & 0 \\ \vdots & \ddots & \vdots \\ 0 & \dots & 0 \end{pmatrix}$.
\stepcounter{enumi}
\stepcounter{enumi}
\item $\RR^3 \to \RR^2$ --- проекция на $Oxy$.

    \begin{math}
        \left.
            \begin{aligned}
                \E \text{ --- стандартный базис в } \RR^3 \\
                \F \text{ --- стандартный базис в } \RR^2
            \end{aligned}
        \right\} \implies A(\phi, \E, \F) = \begin{pmatrix} 
            1 & 0 & 0 \\
            0 & 1 & 0
        \end{pmatrix}.
    \end{math}

\item $\Delta \colon \RR[x]_{\leq n} \to \RR[x]_{\leq n - 1}$, $f \to f'$.

    $\E = (1, x, \dots, x^n)$, $\F = (1, x, \dots, x^{n - 1})$.

    \begin{math}
        A(\phi, \E, \F) = \begin{pmatrix} 
            0 & 1 & 0 & 0 & \dots & 0 \\
            0 & 0 & 2 & 0 & \dots & 0 \\
            0 & 0 & 0 & 3 & \dots & 0 \\
            \vdots & \vdots & \vdots & \vdots & \ddots & \vdots \\
            0 & 0 & 0 & 0 & \dots & n
        \end{pmatrix}
    \end{math}

\item $x_1 e_1 + \dots + x_n e_n \mapsto \begin{pmatrix} x_1 \\ \dots \\ x_n \end{pmatrix}$.

    \begin{math}
        \left.
        \begin{aligned}
            \E &= (e_1, \dots, e_n) \\
            \F &= \text{стандартный базис}
        \end{aligned}
        \right\} \implies A(\phi, \E, \F) = E.
    \end{math}

\item $\phi\colon F^n \to F^m$.

    $\phi(x) = A \cdot x$, $A \in \text{Mat}_{m \times n}(F)$.

    $\E = $ стандартный базис,

    $\F = $ стандартный базис.

    $A(\phi, \E, \F) = A$.
\end{enumerate}


\subsection{Связь координат вектора и его образа при линейном отображении}

\begin{proposal}
    Пусть $\phi \colon V \to W$ --- линейное отображение,

    $\E = (e_1, \dots, e_n)$ --- базис $V$,

    $\F = (f_1, \dots, f_m)$ --- базис $W$,

    $A = A(\phi, \E, \F)$.

    \begin{math}
        \begin{aligned}
            v \in V \implies &v = x_1 e_1 + \dots + x_n e_n, \\
            &\phi(v) = y_1 f_1 + \dots + y_m f_m.
        \end{aligned}
    \end{math}

    Тогда,
    \begin{equation*}
        \begin{pmatrix} y_1 \\ \vdots \\ y_m \end{pmatrix} = A \begin{pmatrix} x_1 \\ \vdots \\ x_n \end{pmatrix}
    .\end{equation*}
\end{proposal}

\begin{proof}
    $v = (e_1, \dots, e_m) \begin{pmatrix} x_1 \\ \vdots \\ x_n \end{pmatrix}$.

    Значит, 
    \begin{align*}
        \phi(v) &= (\phi(e_1), \dots, \phi(e_n)) \begin{pmatrix} x_1 \\ \vdots \\ x_n \end{pmatrix} = (f_1, \dots, f_m) \cdot A \cdot \begin{pmatrix} x_1 \\ \vdots \\ x_n \end{pmatrix}
    .\end{align*}

    При этом, 
    \begin{align*}
        \phi(v) = (f_1, \dots, f_m) \begin{pmatrix} y_1 \\ \vdots \\ y_m \end{pmatrix}
    .\end{align*}

    Так как $f_1, \dots, f_m$ линейно независимы, то
    \begin{equation*}
        A \begin{pmatrix} x_1 \\ \vdots \\ x_n \end{pmatrix} = \begin{pmatrix} y_1 \\ \vdots \\ y_m \end{pmatrix}
    .\end{equation*}
\end{proof}


\subsection{Формула изменения матрицы линейного отображения между векторными пространствами $V$ и $W$ при замене их базисов}

Пусть теперь $\E'$ --- другой базис в $V$, $\F'$ --- другой базис в $W$.

$\E' = \E \cdot C_{\in M_n}$,

$\F' = \F \cdot D_{\in M_m}$.

$A = A(\phi, \E, \F)$,

$A' = A(\phi, \E', \F')$.

\begin{proposal}
    $A' = D^{-1} A C$.
\end{proposal}

\begin{proof}
    \begin{equation*}
        (e'_1, \dots e'_n) = (e_1, \dots, e_n) \cdot C
    .\end{equation*}
    
    Применим $\phi$, 
    \begin{align*}
        (\phi(e'_1), \dots, \phi(e'_n)) &= (\phi(e_1), \dots, \phi(e_n)) \cdot C = (f_1, \dots, f_m) \cdot A \cdot C
    \end{align*}
    
    При этом,
    \begin{align*}
        (\phi(e'_1), \dots, \phi(e'_n)) &= (f'_1, \dots, f'_m) \cdot A' = (f_1, \dots, f_m) \cdot D \cdot A'
    .\end{align*}

    Отсюда,
    \begin{equation*}
        A \cdot C = D \cdot A' \implies A' = D^{-1} \cdot A \cdot C
    .\qedhere\end{equation*}
\end{proof}


\subsection{Операции сложения и умножения на скаляр на множестве всех линейных отображений между двумя векторными пространствами}

Пусть $\phi, \psi \in \hom(V, W)$, $\lambda \in F$.

\begin{definition}~
    \begin{enumerate}
    \item \textit{Суммой} линейных отображений $\phi$ и $\psi$ называется линейное отображение $\phi + \psi \in \hom(V, W)$, \\ такое что ${(\phi + \psi)(v) := \phi(v) + \psi(v)}$.
    \item Произведение $\phi$ на $\lambda$ --- это линейное отображение $\lambda \phi \in \hom(V, W)$, такое что $(\lambda\phi)(v) := \lambda \phi(v)$.
    \end{enumerate}
\end{definition}

\begin{exercise}
    $\phi + \psi$ и $\lambda \phi$ --- действительно линейные отображения.
\end{exercise}

\begin{exercise}
    $\hom(V, W)$ с этими операциями является векторным пространством над $F$.
\end{exercise}


\subsection{Матрица суммы двух линейных отображений и произведения линейного отображения на скаляр}

Зафиксируем базисы $\E = (e_1, \dots, e_n)$ в $V$ и $\F = (f_1, \dots, f_m)$ в $W$.

\begin{proposal}~
    \begin{enumerate}
    \item
        \begin{math}
            \begin{aligned}[t]
                \phi, \psi \in \hom(V, W), \
                &A_\phi = A(\phi, \E, \F)& \\
                &A_\psi = A(\psi, \E, \F)& \\
                &A_{\phi + \psi} = A(\phi + \phi, \E, \F)& \implies A_{\phi + \psi} = A_\phi + A_\psi
            \end{aligned}
        \end{math}

    \item
        \begin{math}
            \begin{aligned}[t]
                \lambda \in F, \phi \in \hom(V, W), \
                &A_\phi = A(\phi, \E, \F)& \\
                &A_{\lambda \phi} = A(\lambda \phi, \E, \F)& \implies A_{\lambda \phi} = \lambda A_\phi
            \end{aligned}
        \end{math}
    \end{enumerate}
\end{proposal}

\begin{proof}~
    \begin{enumerate}
    \item 
        \begin{align*}
            (f_1, \dots, f_m) \cdot A_{\phi + \psi}
            &= ((\phi + \psi)(e_1), \dots, (\phi + \psi)(e_n)) \\
            &= (\phi(e_1), \dots, \phi(e_n)) + (\psi(e_1), \dots, \psi(e_n)) \\
            &= (f_1, \dots, f_m) A_\phi + (f_1, \dots, f_m) A_\psi \\
            &= (f_1, \dots, f_m) (A_\phi + A_\psi)
        .\end{align*}

        Следовательно, $A_{\phi + \psi} = A_\phi + A_\psi$.

    \item Аналогично. \qedhere
    \end{enumerate}
\end{proof}


\subsection{Изоморфизм между пространством $\hom(V,W)$ и пространством $(m \times n)$-матриц, где $n = \dim V$, $m = \dim W$}

\begin{corollary}
    При фиксированном $\E$ и $\F$ отображение $\phi \mapsto A(\phi, \E, \F)$ является изоморфизмом между $\hom(V, W)$ и $\text{Mat}_{m \times n}(F)$.
\end{corollary}

\begin{proof}~
    Биективность была выше.
    Линейность --- из предыдущего предложения.
\end{proof}

\begin{corollary}
    $\dim \hom(V, W) = m \cdot n$.
\end{corollary}


\subsection{Матрица композиции двух линейных отображений}

Пусть $U \xrightarrow{\psi} V \xrightarrow{\phi} W$ --- цепочка линейных отображений, а $\phi \circ \psi : U \to W$ --- их композиция,

$\E = (e_1, \dots, e_n)$ --- базис $V$,

$\F = (f_1, \dots, f_m)$ --- базис $W$,

$\G = (g_1, \dots, g_k)$ --- базис $U$.

$A_{\phi} = A(\phi, \E, \F)$,

$A_\psi = A(\psi, \G, \E)$,

$A_{\phi \circ \psi} = A(\phi \circ \psi, \G, \F)$.

Тогда, $A_{\phi \circ \psi} = A_\phi \cdot A_\psi$.

\begin{proof}
    $(\psi(g_1), \dots, \psi(g_k)) = (e_1, \dots, e_n) A_\psi$.
    Тогда применяя $\phi$,
    \begin{equation*}
        \left(\phi\left(\psi(g_1)\right), \dots, \phi\left(\psi(g_k)\right)\right) = \left(\phi(e_1), \dots, \phi(e_n)\right) A_\psi = (f_1, \dots, f_m) A_\phi A_\psi
    .\end{equation*}
    С другой стороны,
    \begin{equation*}
        \left(\phi\left(\psi(g_1)\right), \dots, \phi\left(\psi(g_k)\right)\right) = (f_1, \dots, f_m) A_{\phi \circ \psi}
    .\end{equation*}

    Значит, $A_\phi \cdot A_\psi = A_{\phi \circ \psi}$.
\end{proof}


\subsection{Ядро и образ линейного отображения; утверждение о том, что они являются подпространствами в соответствующих векторных пространствах}

Пусть $\phi \colon V \to W$.

\begin{definition}
    \textit{Ядро} линейной оболочки $\phi$ --- это $\ker \phi := \{v \in V \mid \phi(v) = 0\} \subseteq V$.

    \textit{Образ} линейного отображения $\phi$ --- это $\Im \phi := \phi(V) \subseteq W$.
\end{definition}

\begin{example}
    $\Delta \colon \RR[x]_{\leq n} \to \RR[x]_{\leq n}, \ f \mapsto f'$,

    $\ker \Delta = \{f \mid f = \text{const}\}$,

    $\Im \Delta = \RR[x]_{\leq n - 1}$.
\end{example}

\begin{proposal}~
    \begin{enumerate}
    \item Ядро --- подпространство в $V$.
    \item Образ --- подпространство в $W$.
    \end{enumerate}
\end{proposal}

\begin{proof}~
    \begin{enumerate}
    \item 
        \begin{enumerate}
        \item $\phi(0_V) = 0_W$,
        \item $v_1, v_2 \in \ker \phi \implies \phi(v_1 + v_2) = \phi(v_1) + \phi(v_2) = 0 + 0 = 0 \implies v_1 + v_2 \in \ker \phi$,
        \item $\lambda \in F, v \in V \implies \phi(\lambda v) = \lambda \phi(v) = \lambda 0 = 0 \implies \lambda v \in \ker \phi$.
        \end{enumerate}

    \item
        \begin{enumerate}
        \item $0_W = \phi(W) \in \Im \phi$,
        \item $w_1, w_2 \in \Im \phi \implies \exists v_1, v_2 : w_1 = \phi(v_1), w_2 = \phi(v_2) \implies w_1 + w_2 = \phi(v_1) + \phi(v_2) = \phi(v_1 + v_2) \in \Im \phi$,
        \item $\phi \in F, w \in \Im \phi \implies \exists v \in V : w = \phi(v) \implies \phi(w) = \lambda \phi(v) = \phi(\lambda v) \in \Im \phi$.
            \qedhere
        \end{enumerate}
    \end{enumerate}
\end{proof}
