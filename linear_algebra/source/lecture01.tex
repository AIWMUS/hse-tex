\section{Лекция 09.09.2019}
\subsection{Матрицы}
\begin{definition}
    \textit{Матрица размера $n \times m$} --- это прямоугольная таблица высоты $m$ и ширины $n$.
\end{definition}

\begin{equation*}
    A = \begin{pmatrix}
        a_{11}  & a_{12} & \dots & a_{1n} \\
        a_{21} & a_{22} & \dots & a_{2n} \\
        \vdots & \vdots & \ddots & \vdots \\
        a_{m1} & a_{m2} & \dots & a_{mn}
    \end{pmatrix}
\end{equation*}

$a_{ij}$ -- элемент на пересечении $i$-й строки и $j$-го столбца

Краткая запись -- $A = (a_{ij})$

Множество всех матриц размера $m \times n$ с коэффициентами из $\RR$ (множество всех действительных чисел) --- $\text{Mat}_{n \times m}(\RR)$ или $\text{Mat}_{n \times m}$

\begin{definition}
    Две матрицы $A \in \text{Mat}_{n \times m}$ и $B \in \text{Mat}_{p \times q}$ называются \textit{равными}, если $m = p$, $n = q$, и соответствующие элементы равны
\end{definition}

\begin{example}
    $\begin{pmatrix}
       \circ & \circ & \circ \\ \circ & \circ & \circ
    \end{pmatrix}
    \neq
    \begin{pmatrix}
        \circ & \circ \\ \circ & \circ \\ \circ & \circ
    \end{pmatrix}$
\end{example}

\subsection{Операции над матрицами}

Для любых $A, B \in \text{Mat}_{m \times n}$

\begin{itemize}
    \item \emph{Сложение} $A + B := (a_{ij} + b_{ij})$
    \item \emph{Умножение на скаляр} $\alpha \in \RR \implies \lambda A := (\lambda a_{ij})$
\end{itemize}

\textbf{Свойства суммы и произведения на скаляр}

$\forall A, B, C  \in \text{Mat}_{m \times n} \quad \forall \lambda, \mu \in \RR$
\begin{enumerate}[label=\arabic*), nosep]
    \item $A + B = B + A$ (коммутативность)

    \item $(A + B) + C = A + (B + C)$ (ассоциативность)

    \item $A + 0 = 0 + A = A$, где 
        \begin{equation*}
            0 = \begin{pmatrix}
                0 & 0 & \dots & 0 \\
                0 & 0 & \dots & 0 \\
                \vdots & \vdots & \ddots & \vdots \\
                0 & 0 & \dots & 0
            \end{pmatrix} \text{ --- нулевая матрица}
        .\end{equation*}

    \item $A + (-A) = 0$
    
        $-A = (-a_{ij})$ -- противоположная матрица

    \item $(\lambda + \mu) A = \lambda A + \mu A$

    \item $\lambda (A + B) = \lambda A + \lambda B$

    \item $\lambda (\mu A) = \lambda (\mu A)$

    \item $1 A = A$
\end{enumerate}

\bigskip
\textit{\textbf{Упражнение на дом.} Доказать эти свойства.}

\begin{comment}
    Из свойств 1) -- 8) следует, что $\text{Mat}_{n \times m}(\RR)$ является векторным пространством над \( \RR \)
\end{comment}


\subsection{Пространство $\RR^n$, его отождествление с матрицами-столбцами высоты $n$}

\noindent
$\RR^n := \{(x_1, \dots, x_n) \mid x_i \in \RR \ \forall i = 1, \dots, n \}$

$\RR$ -- числовая прямая

$\RR^2$ -- плоскость

$\RR^3$ -- трехмерное пространство

\bigskip
Договоримся отождествлять \( \RR^n \) со столбцами высоты \( n \)

$(x_1, \dots, x_n) \leftrightarrow \begin{pmatrix} x_1 \\ \vdots \\ x_n \end{pmatrix}$ -- \textit{вектор столбец}

$\RR^n = \left\{\begin{pmatrix} x_1 \\ x_2 \\ \vdots \\ x_n \end{pmatrix} \mid x \in \RR \ \forall i = 1, \dots, n \right\} = \text{Mat}_{n \times 1}(\RR)$

$\left[ x = \begin{pmatrix}
    x_1 \\ \vdots \\ x_n
\end{pmatrix} \in \RR^n, y = \begin{pmatrix}
    y_1 \\ \vdots \\ y_n
\end{pmatrix} \in \RR^n \right] \implies 
\left[ x = y \iff x_i = y_i \ \forall i \right]$

\( x + y := \begin{pmatrix}
    x_1 + y_1 \\ \vdots \\ x_n + y_n
\end{pmatrix} \)

$\lambda \in \RR \implies \lambda x_i := (\lambda x_1, \dots, \lambda x_n)$


\subsection{Транспонирование матриц, его простейшие свойства}

$A \in \text{Mat}_{m \times n} = \begin{pmatrix}
    a_{11} & a_{12} & \dots & a_{1n} \\
    a_{21} & a_{22} & \dots & a_{2n} \\
    \vdots & \vdots & \ddots & \vdots \\
    a_{m1} & a_{m2} & \dots & a_{mn}
\end{pmatrix} \leadsto A^T \in \text{Mat}_{n \times m} := \begin{pmatrix}
    a_{11} & a_{21} & \dots & a_{m1} \\
    a_{12} & a_{22} & \dots & a_{m2} \\
    \vdots & \vdots & \ddots & \vdots \\
    a_{1n} & a_{2n} & \dots & a_{mn}
\end{pmatrix}$

$A^T$ ---  \textit{транспонированная матрица}

\bigskip
Свойства:
\begin{enumerate}[label=\arabic*), nosep]
    \item $(A^T)^T = A$
    \item $(A + B)^T = A^T + B^T$
    \item $(\lambda A)^T = \lambda A^T$
\end{enumerate}

\begin{example}
    \( \begin{pmatrix}
        x_1 & \dots & x_n
    \end{pmatrix}^T = \begin{pmatrix}
        x_1 \\ \vdots \\ x_n
    \end{pmatrix} \)
\end{example}
\begin{example}
    \( \begin{pmatrix}
    x_1 \\ \vdots \\ x_n
    \end{pmatrix}^T = \begin{pmatrix}
    x_1 & \dots & x_n
    \end{pmatrix} \)
\end{example}
\begin{example}
    \( \begin{pmatrix}
        1 & 2 \\ 3 & 4 \\ 5 & 6
    \end{pmatrix}^T = \begin{pmatrix}
        1 & 3 & 5 \\
        2 & 4 & 6
    \end{pmatrix} \)
\end{example}

\subsection{Умножение матриц}

Пусть $A = (a_{ij}) \in \text{Mat}_{m \times n}$

\bigskip

$A_{(i)} = \begin{pmatrix} a_{i1}, a_{i2}, \dots, a_{in} \end{pmatrix} $ --- $i$-я строка матрицы $A$

$A^{(j)} = \begin{pmatrix} a_{1j} \\ a_{2j} \\ \vdots \\ a_{mn} \end{pmatrix} $ --- $j$-й  столбец матрицы $A$

\begin{enumerate}[label=\arabic*)]
    \item 
        Частный случай: умножение строки на столбец той же длинны
    
        $\underbrace{(x_1, \dots, x_n)}_{1 \times n} 
        \underbrace{\begin{pmatrix}
            y_1 \\ \vdots \\ y_n
        \end{pmatrix}}_{n \times 1} 
        = x_1 \cdot y_1 + \dots + x_n \cdot y_n$
        
    \item
        Общий случай:

        $A$ - матрица размера $m \times \underline{n}$
        
        $B$ - матрица размера $\underline{n} \times p$
        
        Количество строк матрицы $A$ равно количеству столбцов матрицы $B$ --- условие согласованности матриц
        
        $AB := C \in \text{Mat}_{m \times p}$, где $C_{ij} = A_{(i)} B^{(j)}$
\end{enumerate}

\begin{example}
    \( \begin{pmatrix}
        y_1 \\ \vdots \\ y_n
    \end{pmatrix} 
    \begin{pmatrix}
        x_1 & \dots & x_n
    \end{pmatrix}
    := 
    \begin{pmatrix}
        x_1 y_1 & x_2 y_1 & \dots & x_n y_1 \\
        x_1 y_2 & x_2 y_2 & \dots & x_n y_2 \\
        \vdots & \vdots & \ddots & \vdots \\
        x_1 y_n & x_2 y_m & \dots & x_n y_m 
    \end{pmatrix} \)
\end{example}

\begin{example}
    $\begin{pmatrix}
        1 & 0 & 2 \\
        0 & -1 & 3
    \end{pmatrix}
    \times
    \begin{pmatrix}
        2 & -1 \\
        0 & 5 \\
        1 & 1
    \end{pmatrix}
    =
    \begin{pmatrix}
        1 \cdot 2 + 0 \cdot 0 + 2 \cdot 1 & 1 \cdot (-1) + 0 \cdot 5 + 2 \cdot 1 \\
        0 \cdot 2 + (-1) \cdot 0 + 3 \cdot 1 & 0 \cdot (-1) + (-1) \cdot 5 + 3 \cdot 1
    \end{pmatrix}
    =
    \begin{pmatrix}
        4 & 1 \\
        3 & -2
    \end{pmatrix}$
\end{example}
