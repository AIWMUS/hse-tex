\section{Лекция 23.09.2019}

\subsection{Инверсии в перестановке}
Обозначение: $S_n$ -- множество всех перестановок из n элементов.

\bigskip
Пусть $\sigma \in S_n$, $i, j \in \{1, 2, \dots, n\}$, $i \neq j$

\begin{definition}
    Пара $\{i, j\}$ (неупорядоченная) образует \textit{инверсию} в $\sigma$, если числа $i - j$ и $\sigma(i) - \sigma(j)$ имеют разный знак (то есть либо $i < j$ и $\sigma(i) > \sigma(j)$, либо $i > j$ и $\sigma(i) < \sigma(j)$).
\end{definition}

\subsection{Знак и чётность перестановки}
\begin{definition}
    \textit{Знак} перестановки $\sigma$ -- это  число $\sgn(\sigma) = (-1)^{<\text{число инверсий в }\sigma>}$.
\end{definition}

\begin{definition}
    Перестановка $\sigma$ называется \textit{четной}, если $\sgn(\sigma)$ = 1 (четное количество инверсий), и \textit{нечетной} если $\sgn(\sigma) = -1$ (нечетное количество инверсий).
\end{definition}

\underline{Примеры}.
\begin{table}[!ht]
    \begin{tabular}{c|c|c}
        $\sigma$ & $\begin{pmatrix} 1 & 2 \\ 1 & 2 \end{pmatrix}$ & $\begin{pmatrix} 1 & 2 \\ 2 & 1 \end{pmatrix}$ \\
        \hline
        число инверсий & 0 & 1 \\
        \hline
        $\sgn(\sigma)$ & 1 & -1 \\
        \hline
        четность & четная & нечетная
    \end{tabular}
\end{table}

\begin{table}[!ht]
    \begin{tabular}{c|c|c|c|c|c|c}
        $\sigma$ & $\begin{pmatrix} 1 & 2 & 3 \\ 1 & 2 & 3 \end{pmatrix}$ & $\begin{pmatrix} 1 & 2 & 3 \\ 2 & 1 & 3 \end{pmatrix}$ & $\begin{pmatrix} 1 & 2 & 3 \\ 2 & 3 & 1 \end{pmatrix}$ & $\begin{pmatrix} 1 & 2 & 3 \\ 3 & 2 & 1 \end{pmatrix}$ & $\begin{pmatrix} 1 & 2 & 3 \\ 3 & 1 & 2 \end{pmatrix}$ & $\begin{pmatrix} 1 & 2 & 3 \\ 1 & 3 & 2 \end{pmatrix}$ \\
        \hline
        число инверсий & 0 & 1 & 2 & 3 & 2 & 1\\
        \hline
        $\sgn(\sigma)$ & 1 & -1 & 1 & -1 & 1 & -1 \\
        \hline
        четность & четная & нечетная & четная & нечетная & четная & нечетная
    \end{tabular}
\end{table}

\begin{comment}
    число инверсий в $\sigma \in S_n \leq \binom{n}{2} = \frac{n(n-1)}{2}$, равенство достигается при
    $\sigma = \begin{pmatrix}
        1 & 2 & \dots & n \\
        n & n-1 & \dots & 1
    \end{pmatrix}$
\end{comment}


\subsection{Произведение перестановок}
\begin{definition}
    \textit{Произведением} (или \textit{композицией}) двух перестановок $\sigma, \rho \in S_n$ называется такая перестановка $\sigma \rho \in S_n$, что $(\sigma \rho)(x) := \sigma (\rho (x)) \quad \forall x \in \{1, \dots, n\}$.
\end{definition}

\underline{Пример}.

$\sigma = \begin{pmatrix}
    1 & 2 & 3 & 4 \\
    4 & 3 & 2 & 1
\end{pmatrix}$, $\rho = \begin{pmatrix}
    1 & 2 & 3 & 4 \\
    3 & 4 & 1 & 2
\end{pmatrix}$

$\sigma \rho = \begin{pmatrix}
    1 & 2 & 3 & 4 \\
    2 & 1 & 4 & 3
\end{pmatrix}$

$\rho \sigma = \begin{pmatrix}
    1 & 2 & 3 & 4 \\
    2 & 1 & 3 & 4
\end{pmatrix}$

Видно, что $\sigma \rho \neq \rho \sigma \implies$ произведение перестановок не обладает свойством коммутативности.

\subsection{Ассоциативность произведения перестановок}
\begin{proposition}
    Умножение перестановок ассоциативно, то есть $\sigma (\tau \pi) = (\sigma \tau) \pi \ \forall \sigma, \tau, \pi \in S_n$.
\end{proposition}

\begin{proof}
    $\forall i \in \{1, 2, \dots, n\}$ имеем

    $[\sigma(\tau \pi)](i) = \sigma((\tau \pi)(i)) = \sigma(\tau(\pi(i)))$

    $[(\sigma \tau) \pi](i) = (\sigma \tau)(\pi(i)) = \sigma(\tau(\pi(i)))$
\end{proof}


\subsection{Тождественная перестановка}
\begin{definition}
    Перестановка $id = \begin{pmatrix}
        1 & 2 & \dots & n \\
        1 & 2 & \dots & n
    \end{pmatrix} \in S_n$ называется \textit{тождественной} перестановкой.
\end{definition}

\textbf{Свойства:}

$\forall \sigma \in S_n \quad id \cdot \sigma = \sigma \cdot id = \sigma$.

$\sgn(id) = 1$.


\subsection{Обратная перестановка и её знак}
\begin{definition}
    $\sigma \in S_n$, $\sigma = \begin{pmatrix}
        1 & 2 & \dots & n \\
        \sigma(1) & \sigma(2) & \dots & \sigma(n)
    \end{pmatrix} \implies$ подстановка $\sigma^{-1} := \begin{pmatrix}
        \sigma(1) & \sigma(2) & \dots & \sigma(n) \\
        1 & 2 & \dots & n
    \end{pmatrix}$ называется \textit{обратной} к $\sigma$ перестановкой.
\end{definition}

\textbf{Свойства:}
$\sigma \cdot \sigma^{-1} = id = \sigma^{-1} \cdot \sigma$

\subsection{Теорема о знаке произведения перестановок}
\begin{theorem}
    $\sigma, \rho \in S_n \implies \sgn(\sigma \rho) = \sgn \sigma \cdot \sgn \rho$.
\end{theorem}

\begin{proof}
    Для каждой пары $i < j$ введем следующие числа:

    \begin{equation*}
        \alpha(i,j) = \begin{cases}
            1, &\text{если } \{i, j\} \text{ образует инверсию в } \rho \\
            0, &\text{иначе}
        \end{cases}
    \end{equation*}
    \begin{equation*}
        \beta(i,j) = \begin{cases}
            1, &\text{если }  \{\rho(i), \rho(j)\} \text{ образует инверсию в } \sigma \\
            0, &\text{иначе}
        \end{cases}
    \end{equation*}
    \begin{equation*}
        \gamma(i,j) = \begin{cases}
            1, &\text{если }  \{i, j\} \text{ образует инверсию в } \sigma \rho \\
            0, &\text{иначе}
        \end{cases}
    \end{equation*}

    \everymath{\displaystyle}

    ``число инверсий в $\rho$'' $= \sum_{1 \leq i < j \leq n} \alpha(i, j) $

    ``число инверсий в $\sigma \rho$'' $= \sum_{1 \leq i < j \leq n} \gamma(i, j) $

    ``число инверсий в $\sigma$'' $= \sum_{1 \leq i < j \leq n} \beta(i, j)$ -- Почему?

    Когда $\{i, j\}$ пробегает все неупорядоченные пары в $\{1, 2, \dots, n\}$, пара $\{\rho(i), \rho(j)\}$ тоже пробегает все неупорядоченные пары в $\{1, 2, \dots, n\}$.

    \bigskip
    Зависимость $\gamma(i,j)$ от $\alpha(i,j)$ и $\beta(i,j)$:
    \begin{table}[!ht]
        \begin{center}
            \begin{tabular}{c|c|c|c|c}
                $\alpha(i,j)$ & 0 & 0 & 1 & 1 \\
                \hline
                $\beta(i,j)$ & 0 & 1 & 0 & 1 \\
                \hline
                $\gamma(i,j)$ & 0 & 1 & 1 & 0 \\
            \end{tabular}
        \end{center}
    \end{table}

    Вывод: $ \alpha(i, j) + \beta(i, j) \equiv \gamma(i,j) \pmod{2}$.

    \bigskip
    Тогда $\sgn(\sigma \rho) = (-1)^{\sum \gamma(i,j)} = (-1)^{\sum \beta(i,j) + \sum \alpha(i,j)} = (-1)^{\sum \alpha(i,j)} \cdot (-1)^{\sum \beta(i,j)} = \sgn \sigma \cdot \sgn \rho$.
\end{proof}

\begin{corollary}
    $\sigma \in S_n \implies \sgn (\sigma^{-1}) = \sgn(\sigma)$.
\end{corollary}

\begin{proof}
    $\sigma \sigma^{-1} = id \implies \sgn(\sigma \sigma^{-1}) = \sgn(id) \implies \sgn \sigma \sgn \sigma^{-1} = 1 \implies \sgn \sigma = \sgn \sigma^{-1}$.
\end{proof}

\textit{\textbf{Упражнение на дом:} Показать, что число инверсий в $\sigma^{-1}$ такое же, как в $\sigma$}.

\subsection{Транспозиции, знак транспозиции}

Пусть $i, j \in \{1, 2, \dots, n\}$, $i \neq j$.

Рассмотрим перестановку $\tau_{ij} \in S_n$, такую что

$\tau_{ij}(i) = j$.

$\tau_{ij}(j) = i$.

$\tau_{ij}(k) = k \quad \forall k \neq i, j$.

\begin{definition}
    Перестановки вида $\tau_{ij}$ называются \textit{транспозициями}.
\end{definition}

\begin{comment}
    $\tau$ -- траспозиция $\implies \tau^2 = id, \tau^{-1} = \tau$.
\end{comment}

\begin{definition}
    Перестановки вида $\tau_{i, i+1}$ называются \textit{элементарными траспозициями}.
\end{definition}

\begin{lemma}
    $\tau \in S_n$ -- транспозиция $\implies \sgn(\tau) = -1$.
\end{lemma}

\begin{proof}
    Пусть $\tau = \tau_{ij}$, можем считать, что $i < j$.
    \begin{equation*}
        \tau := \begin{pmatrix}
            1 & \dots & i-1 & i & i + 1 & \dots &
            j - 1 & j & j + 1 & \dots \ n
            \cr
            1 & \dots & i-1 & j & i + 1 & \dots &
            j - 1 & i & j + 1 & \dots \ n
        \end{pmatrix}
    \end{equation*}

    Посчитаем инверсии:

    $\{i, j\}$

    $\{i, k\}$ при $i + 1 \leq k \leq j -1$, всего = $j-i-1$

    $\{k, j\}$ при $i + 1 \leq k \leq j -1$, всего = $j-i-1$

    Значит, всего инверсий $2(j-i-1) + 1 \equiv 1 \pmod{2} \implies \sgn(\tau) = -1$.
\end{proof}

\begin{corollary}
    При $n \geq 2$ отображение $\sigma \rightarrow \sigma \tau_{12}$ является биекцией между множеством четных перестановок в $S_n$ и множеством нечетных перестановок в $S_n$.
\end{corollary}

\begin{corollary}
    При $n \geq 2$ количество нечетных перестановок в $S_n$ равно количеству четных перестановок в $S_n$ и равно $\frac{n!}{2}$.
\end{corollary}

\begin{theorem}
    Всякая перестановка $\sigma \in S_n$ может быть разложена в произведение конечного числа элементарных транспозиций.
\end{theorem}

\begin{proof}
    \begin{equation*}
        \sigma \in S_n :=
        \begin{pmatrix}
            1 & 2 & \dots & n \\
            \sigma (1) & \sigma (2) & \dots & \sigma (n)
        \end{pmatrix}
    \end{equation*}

    Тогда
    \begin{equation*}
        \sigma \tau_{i, i+1} = \begin{pmatrix}
            1 & 2 & \dots & i & i+1 & \dots & n \\
            \sigma (1) & \sigma (2) & \dots & \sigma(i+1) & \sigma(i) & \dots & \sigma (n)
        \end{pmatrix}
    \end{equation*}

    При умножении справа на $\tau_{i, i+1}$ в нижней строке меняются местами $i$-ый и $(i+1)$-ый элементы.

    Тогда, домножив $\sigma$ на подходящее произведение $\tau_1 \cdot \tau_2 \cdot \dots \cdot \tau_k$ элементарных траспозиций, можем добиться, что нижняя строка есть $(1, 2, \dots, n) \implies \sigma \tau_1 \tau_2 \dots \tau_k = id$.

    Теперь, домножая справа на $\tau_k \tau_{k-1} \dots \tau_1$, получим $\sigma = \tau_k \tau_{k-1} \dots \tau_1$.
\end{proof}

\subsection{Определитель квадратной матрицы}

\begin{definition}
    Определителем матрицы $A \in M_n$ называется число
    \begin{equation*}
        \det A = \sum_{\sigma \in S_n} \sgn(\sigma) a_{1\sigma(1)} a_{2 \sigma(2)} \dots a_{n\sigma(n)}
    .\end{equation*}

    ($\sum_{\sigma \in S_n}$ -- сумма по всем перестановкам)
\end{definition}

\bigskip
Другие обозначения: $|A|, \begin{vmatrix} a_{11} & a_{12} & \dots & a_{1n} \\ \dots & \dots & \dots & \dots \\ a_{n1} & a_{n2} & \dots & a_{nn} \end{vmatrix}$

\subsection{Определители порядков 2 и 3}
\begin{itemize}
\item
    $n = 2$

    $S_2 = \left\{ \begin{pmatrix} 1 & 2 \\ 1 & 2 \end{pmatrix}, \begin{pmatrix} 1 & 2 \\ 2 & 1 \end{pmatrix} \right\}$

    $\det A = \begin{vmatrix} a_{11} & a_{12} \\ a_{21} & a_{22} \end{vmatrix} = a_{11} a_{22} - a_{12} a_{21}$

\item
    $n = 3$

    $S_3 = \left\{
    \begin{pmatrix} 1 & 2 & 3 \\ 1 & 2 & 3 \end{pmatrix},
    \begin{pmatrix} 1 & 2 & 3 \\ 2 & 3 & 1 \end{pmatrix},
    \begin{pmatrix} 1 & 2 & 3 \\ 3 & 1 & 2 \end{pmatrix},
    \begin{pmatrix} 1 & 2 & 3 \\ 3 & 2 & 1 \end{pmatrix},
    \begin{pmatrix} 1 & 2 & 3 \\ 2 & 1 & 3 \end{pmatrix},
    \begin{pmatrix} 1 & 2 & 3 \\ 1 & 3 & 2 \end{pmatrix} \right\}$

    $\det A = \begin{vmatrix} a_{11} & a_{12} & a_{13} \\ a_{21} & a_{22} & a_{23} \\ a_{31} & a_{32} & a_{33} \end{vmatrix} = a_{11} a_{22} a_{33} + a_{12} a_{23} a_{31} + a_{13} a_{21} a_{32} - a_{13} a_{22} a_{31} - a_{12} a_{21} a_{33} - a_{11} a_{23} a_{32}$.
\end{itemize}
