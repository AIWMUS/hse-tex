\section{Лекция 5.04.2017}

$\RR^n \supseteq L \neq \varnothing$

$L$ -- линейное многообразие $\Leftrightarrow L = v_0 + S$, где $v_0 \in L, S \subseteq \RR^n$ -- подпространство

\bigskip
$L$ -- линейное многообразие

$S$ -- направляющее подпространство

Фиксируем $v_0 \in L$ и $(e_1, \dots, e_k)$ -- базис в $S$

\bigskip
\textbf{Определение.} Набор $(v_0; e_1, \dots, e_k)$ назывался \textit{репЕром} линейного многообразия $L$.

\bigskip
Всякий репер $(v_0; e_1, \dots, e_k)$ задает на $L$ \textit{афинную систему кординат} $\forall \ v \in L \exists! \ \alpha_1, \dots, \alpha_k \in \RR$, такие что $v = v_0 + \alpha_1 e_1 + \dots + \alpha_k e_k$

\bigskip
Числа $\alpha_1, \dots, \alpha_k$ называются координатами точки $v$ в данном репере.

\bigskip
\textbf{Теорема.} a) Через любые $k+1$ точек в $\RR^n$ проходит плоскость размерности $\leq k$

б) Если эти точки не содержатся в плоскости размерности $<k$, то тогда через них проходит единственная плоскость размерности $k$

\bigskip
\textbf{\textit{Доказательство.}} $\rhd$ Пусть $v_0, v_1, \dots, v_k$ -- заданные точки. 

a) Пусть $P = v_0 + <v_1 - v_0, \dots, v_k - v_0>$, тогда $P \ni v_0, v_1, \dots, v_k$ и $dimP \leq k$

б) Из условия следует, что $dimP = k \Rightarrow v_1 - v_0, \dots, v_k - v_0$ линейно независимы.

Пусть $P' = v_0 + S$ -- другая плоскость размерности $k$, содержащая $v_0, v_1, \dots, v_k$

Тогда $v_1 - v_0, 'dots, v_k - v_0 \in S \Rightarrow S = <v_1 - v_0, \dots, v_k - v_0> \Rightarrow P' = P \ \lhd$

\bigskip
\textbf{Следствие.} 1) Через любые две различные точки в $\RR^n$ проходит ровно одна прямая

2) Через любые три точки в $\RR^n$, не лежащих на одной прямой, проходит ровно одна плоскость

\subsection{Взаимное расположение двух линейных многообразий в $\RR^n$}

$L_1, L_2 \in \RR^n$ -- два линейных многообразия

$S_1, S_2$ -- направляющие подпространства

\bigskip
$L_1 \cap L_2 \neq \varnothing$

1) $L_1, L_2$ совпадают

2) $L_1 \supseteq L_2$ ($\Leftrightarrow S_1 \supseteq S_2$) или $L_1 \subseteq L_2$ ($\Leftrightarrow S_1 \subseteq S_2$)

3) Остальное

\bigskip
$L_1 \cap L_2 = \varnothing$

1) $L_1, L_2$ параллельны $\Leftrightarrow S_1 \subseteq S_2$ или $S_1 \supseteq S_2$

2) $L_1, L_2$ скрещиваются $\Leftrightarrow S_1 \cap S_2 = \{0\}$

3) Остальное

\subsection{Линейные многообразия в $\RR^2$}

Нетривиальный случай: $dim = 1$ (прямые)

\textbf{Способы задания:}

1) $Ax + By = C$, где $(A, B) \neq (0, 0)$, $(A, B)$ -- нормаль

2) $v_0 \in l$, $a$ -- направляющий вектор

$v \in l \Leftrightarrow [v - v_0, a] = 0$

Если $v_0 \in l, n$ -- нормаль, то $v \in l \Leftrightarrow (v - v_0, n) = 0$

3) Параметрическое уравнение

$v_0 \in l, a$ -- направляющий вектор

$v = v_0 + at, t \in \RR$

$\begin{cases} x = x_0 + a_1 t \\ y = y_0 + a_2 t \end{cases}$

\bigskip
\textbf{Уравнение прямой, проходящей через 2 различные точки $v_0 = (x_0, y_0)$ и $v_1 = (x_1, y_1)$}

$\begin{vmatrix} x - x_0 & y - y_0 \\ x_1 - x_0 & y_1 - y_0 \end{vmatrix} = 0 \Leftrightarrow \frac{x - x_0}{x_1 - x_0} = \frac{y - y_0}{y_1 - y_0}$

\bigskip
Если $x_1 = x_0$, то уравнение есть $x = x_0$

$y_1 = y_0 \Rightarrow y = y_0$

\subsection{Линейные многообразия в $\RR^3$}

Плоскости в $\RR^3$ ($dim = 2$)

\textbf{Способы задания:}

1) $Ax + By + Cz = D, (A, B, C) \neq (0, 0, 0), (A, B, C)$ -- нормаль

2) Векторное уравнение $(v - v_0, n) = 0, v_0$ -- точка, $n$ -- нормаль

3) Параметрическое уравнение

$v_0$ -- точка, $a, b$ -- базис в направляющем подпространстве

$v = v_0 + ta + sb, \ t, s \in \RR$

В координатах: $\begin{cases} x = x_0 + a_1 t + b_1 s \\ y = y_0 + a_2 t + b_2 s \\ z = z_0 + a_3 t + b_3 s \end{cases}$

Если $a_1 = 0 \Rightarrow \frac{x - x_0}{a_1}$ не пишут, вместо этого $x = x_0$

\bigskip
\textbf{Уравнение плоскости, проходящей через 3 точки $v_0 = (x_0, y_0, z_0), v_1 = (x_1, y_1, z_1)$ и $v_2 = (x_2, y_2, z_2)$}

$\begin{vmatrix} x - x_0 & y - y_0 & z - z_0 \\ x_1 - x_0 & y_1 - y_0 & z_1 - z_0 \\ x_2 - x_0 & y_2 - y_0 & z_2 - z_0 \end{vmatrix} = 0$

\bigskip
Прямые в $\RR^3 (dim = 1)$

\textbf{Способы задания:}

1) $\begin{cases} A_1 x + B_1 y + C_1 = D_1 \\ A_2 x + B_2 y + C_2 z = D_2 \end{cases} \Rightarrow rk \begin{pmatrix} A_1 & B_1 & C_1 \\ A_2 & B_2 & C_2 \end{pmatrix} = 2$

2) Векторное уравнение:

$[v - v_0, a] = 0, \ v_0$ -- точка, $a$ -- направляющий вектор

3) Параметрическое уравнение

$v = v_0 + at, \ t \in \RR, \ v_0$ -- точка, $a$ -- направляющий вектор

$\begin{cases} x = x_0 + a_1 t \\ y = y_0 + a_2 t \\ z = z_0 + a_3 t \end{cases}$

\bigskip
\textbf{Уравнение прямой, проходящей через 2 различные точки в $\RR^3$ $v_0 = (x_0, y_0, z_0)$ и $v_1 = (x_1, y_1, z_1)$}

$\frac{x - x_0}{x_1 - x_0} = \frac{y - y_0}{y_1 - y_0} = \frac{z - z_0}{z_1 - z_0}$

\subsection{Взаимное расположение двух линейных многообразий в $\RR^3$}

1) Две плоскости ($n_1, n_2$ -- нормали плоскостей):

(1) совпадают ($[n_1, n_2] = 0$)

(2) параллельны ($[n_1, n_2] = 0$)

(3) пересекаются по прямой

\bigskip
2) Две прямые($a_1, a_2$ -- направляющие векторы, $v_1, v_2$ -- точки):

(1) совпадают ($[a_1, a_2] = 0$) + ($l_1, l_2$ лежат в одной плоскости $\Leftrightarrow (a_1, a_2, v_2 - v_1)$)

(2) параллельны ($[a_1, a_2] = 0$) + ($l_1, l_2$ лежат в одной плоскости $\Leftrightarrow (a_1, a_2, v_2 - v_1)$)

(3) пересекаются ($l_1, l_2$ лежат в одной плоскости $\Leftrightarrow (a_1, a_2, v_2 - v_1)$)

(4) скрещиваются

\bigskip
3) Прямая и плоскость

$l$ -- прямая

$P$ -- плоскость

(1) $l \subseteq P$

(2) $l \ || \ P$

(3) $l \cap P$ -- точка

\bigskip
4) Три попарно различных плоскости в $\RR^3$

$P_1, P_2, P_3, \ P_i \neq P_j \ \forall \ i \neq j$

$n_i$ -- нормаль к $P_i$

(1) Все параллельны

(2) Две параллельны, третья их пересекает

(3) $P_1, P_2, P_3$ пересекаются по общей прямой

(4) Прямые пересечения параллельны

(5) Пересекаются в одной точке


