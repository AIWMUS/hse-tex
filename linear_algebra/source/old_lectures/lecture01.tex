\section{Лекция 7.09.2017}
\subsection{Системы линейных уравнений.}

\textit{Линейное уравнение}: $a_1 x_1 + \dots + a_n x_n = b$.

$a_1, a_2, \dots, a_n, b \in \RR$ -- коэффициенты.

$x_1, x_2, \dots, x_n$ -- неизвестные.

\bigskip
Система линейных уравнений (СЛУ):
\begin{equation*}
	\begin{cases}
        \begin{aligned}
            a_{11} x_1 + a_{12} x_2 + \cdots + a_{1n} x_n &= b_1 \\
            a_{21} x_1 + a_{22} x_2 + \cdots + a_{2n} x_n &= b_2 \\
            \hdotsfor{2} \\
            a_{m1} x_1 + a_{m2} x_2 + \cdots + a_{mn} x_n &= b_m \\
        \end{aligned}
	\end{cases}
\end{equation*}
m уравнений, n неизвестных

\begin{definition}~
    \begin{enumerate}
    \item \textit{Решение одного уравнения} -- это такой набор значений неизвестных $x_1, x_2, \dots, x_n$, при подстановке которого в уравнение получаем тождество.
    \item \textit{Решение СЛУ} -- такой набор значений неизвестных, который является решением каждого уравнения СЛУ.
    \end{enumerate}
\end{definition}

Основная задача: решить СЛУ, т.е. найти все решения.

\bigskip
\begin{example}

    $n = m = 1$

    $ax = b$, $a, b \in \RR$, x -- неизвестная

    \begin{enumerate}
    \item $a \neq 0 \Rightarrow x = \frac{b}{a}$ -- единственное

    \item $a = 0 \Rightarrow 0x = b$

        $b \neq 0 \Rightarrow$ решений нет.

        $b = 0 \Rightarrow$ x -- любое $\Rightarrow$ бесконечно много решений.
    \end{enumerate}
\end{example}

\begin{definition}
    СЛУ называется 

    -- \textit{совместной}, если у нее есть хотя бы одно решение

    -- \textit{несовмествной}, если решений нет
\end{definition}

\begin{definition}
    Две СЛУ от одних и тех же неизвестных называются \textit{эквивалетными}, если они имеют одинаковые множества решений.
\end{definition}

\bigskip
Примеры.

1) Любые две несовместные СЛУ эквивалентны, т.к. множество решений пусто.

2) 
\begin{equation*}
	\left\{
		\begin{aligned}
        x_1 + x_2 = 1 \\
        x_1 - x_2 = 0
		\end{aligned}
	\right. 
\end{equation*}
\begin{equation*}
    \left\{
		\begin{aligned}
        2x_1 = 1 \\
        2x_2 = 1
		\end{aligned}
	\right.
\end{equation*}
Эквивалентные множества решений: $ \left\{ \left( \frac{1}{2} ; \frac{1}{2} \right) \right\}$.

\bigskip
Как решить СЛУ?

\textit{Идея:} выполнить преобразования СЛУ, не меняющие множество решениq, и привести ее в итоге к виду, в котором она легко решается.

\begin{equation*}
	\begin{pmatrix}
		a_{11} & a_{12} & \cdots & a_{1n} & \vrule & b_1 \\
		a_{21} & a_{22} & \cdots & a_{2n} & \vrule & b_2 \\
       \vdots & \vdots & \vdots& \vdots & \vrule & \vdots\\ 
       a_{m1} & a_{m2} & \cdots & a_{mn} & \vrule & b_m
	\end{pmatrix}
    =
    \begin{pmatrix}
    	A & \vrule & b
	\end{pmatrix}
\end{equation*}

-- расширенная матрица СЛУ (*), она содержит в себе всю информацию о СЛУ(*), где 

\begin{equation*}
	\begin{pmatrix}
		a_{11} & a_{12} & \cdots & a_{1n} \\
		a_{21} & a_{22} & \cdots & a_{2n} \\
       \vdots & \vdots & \vdots& \vdots \\ 
       a_{m1} & a_{m2} & \cdots & a_{mn}
	\end{pmatrix}
    = A
\end{equation*}

-- матрица коэффициентов,

\begin{equation*}
	\begin{pmatrix}
		b_1 \\
        b_2 \\
        \vdots \\
        b_m
	\end{pmatrix}
    = b
\end{equation*}

-- столбец правых частей.

\bigskip
В примерах выше:

1) 
\begin{equation*}
	\begin{pmatrix}
		1 & 1 & \vrule & 1 \\
        1 & -1 & \vrule & 0 
	\end{pmatrix}
\end{equation*}

\begin{equation*}
	\begin{pmatrix}
		2 & 0 & \vrule & 1 \\
        0 & 2 & \vrule & 1 
	\end{pmatrix}
\end{equation*}

\bigskip
2) Пример простого вида:

\begin{equation*}
	\begin{pmatrix}
		1 & 0 & 0 & \cdots & 0 & \vrule & b_1 \\
		0 & 1 & 0 & \cdots & 0 & \vrule & b_2 \\
        0 & 0 & 1 & \cdots & 0 & \vrule & b_3 \\
        \cdots & \cdots & \cdots & \cdots & \cdots & \vrule & \cdots \\
        0 & 0 & 0 & \cdots & 1 & \vrule & b_m 
	\end{pmatrix}
\end{equation*}

соответсвующая СЛУ:

\begin{equation*}
    \left\{
		\begin{aligned}
        x_1 = b_1 \\
        x_2 = b_2 \\
        \vdots \\
        x_m = b_m
		\end{aligned}
	\right.
\end{equation*}

\subsection{Элементарные преобразования СЛУ и ее расширенная матрица.}

\bigskip
\begin{table}[!ht]
		\begin{tabular}{c|c|c}
    	тип & СЛУ & расширенная матрица \\
        \hline
        1. тип & К i-му уравнению прибавить j-ое, умноженное на $\lambda \in \RR (i \neq j)$ & $Э_1$(i, j, $\lambda$) \\
        2. тип & Переставить i-е и j-е уравнения $(i \neq j)$  & $Э_2$(i, j) \\
        3. тип & Умножить i-ое уравнение на $\lambda \neq 0$ & $Э_3$(i, $\lambda$) 
		\end{tabular}
\end{table}

\bigskip
1) $Э_1$(i, j, $\lambda$): к i-ой строке прибавить j-ую, умноженную на $\lambda$ (покомпонентно),

$a_{ik} \rightarrow a_ik + \lambda a_{jk} \forall k = 1, \cdots, n$,

$b_i \rightarrow b_i + \lambda b_j$.

2) $Э_2$(i, j): переставить i-ю и j-ю строки.

3) $Э_3$(i, $\lambda$): умножить i-ю строку на $\lambda$ (покомпоненто).

\bigskip
$Э_1$, $Э_2$, $Э_3$ называются \textit{элементарными преобразованиями строк расширенной матрицы}.

\bigskip
\textbf{Лемма.} Элементарные преобразования СЛУ не меняют множество решений.

\bigskip
\textbf{\textit{Доказательство.}} Пусть мы получили СЛУ(**) из СЛУ(*) путем элементарных преобразований.

$\rhd$ 1) всякое решение СЛУ(*) является решением СЛУ(**);

2) СЛУ(*) тоже получается из СЛУ(**) путем элементарных преобразований

\begin{table}[!ht]
	\begin{center}
		\begin{tabular}{c|c|c}
    	(*) $\rightarrow$ (**) & (**) $\rightarrow$ (*) \\
        \hline
        $Э_1$(i, j, $\lambda$) & $Э_1$(i, j, $-\lambda$)\\
        $Э_2$(i, j) & $Э_2$(i, j)\\
        $Э_3$(i, $\lambda$) & $Э_3$(i, $\frac{1}{\lambda}$)
		\end{tabular}
	\end{center}
\end{table}

\bigskip
1) $\Rightarrow$ всякое решение СЛУ(**) является решением СЛУ(*) $\Rightarrow$ (*) и (**) имеют одно и то же множество решений. $\lhd$

