\section{Лекция 19.03.2020} 

\href{https://www.dropbox.com/s/y3r4x7mjt7iv71a/%D0%9B%D0%90%D0%B8%D0%93_19-20_%D0%9B%D0%B5%D0%BA%D1%86%D0%B8%D1%8F_25.pdf?dl=0}{Записки с лекции}


\begin{lemma}
    Пусть $v_1, v_2 \in \EE$. Тогда, $(v_1, x) = (v_2, x) \ \forall x \in \EE \implies v_1 = v2$.
\end{lemma}

\begin{proof}
    Имеем $(v_1 - v_2, x) = 0 \ \forall x \in \EE$.
    Тогда, $v_1 - v_2 \in \EE^{\perp} = \{0\} \implies v_1 - v_2 = 0 \implies v_1 = v2$.
\end{proof}


\subsection{Трёхмерное евклидово пространство}

\begin{theorem}
    \label{lec25:t}
    Пусть $a, b \in \RR^3$. Тогда
    \begin{enumerate}
    \item $\exists! v \in \EE$, такой что $(v, x) = \Vol(a, b, x) \quad \forall x \in \RR^3$.
    \item Если $\E = (e_1, e_2, e_3)$ --- положительно ориентированный базис и
        \begin{math}
            \ \begin{aligned}[t]
                a &= a_1 e_1 + a_2 e_2 + a_3 e_3 \\
                b &= b_1 e_1 + b_2 e_2 + b_3 e_3
            \end{aligned},
        \end{math}
        то 
        \begin{equation}
            \tag{$\star$}
            \label{lec25:v}
            v = \begin{vmatrix}
                e_1 & e_2 & e_3 \\
                a_1 & a_2 & a_3 \\
                b_1 & b_2 & b_3
            \end{vmatrix}
            := \begin{vmatrix} 
                a_2 & a_3 \\
                b_2 & b_3
            \end{vmatrix} e_1 - \begin{vmatrix} 
                a_1 & a_3 \\
                b_1 & b_3
            \end{vmatrix} e_2 + \begin{vmatrix} 
                a_1 & a_2 \\
                b_1 & b_2
            \end{vmatrix} e_3
        .\end{equation}
    \end{enumerate}
\end{theorem}

\begin{proof}~
    \begin{description}
    \item[Единственность] если $v'$ --- другой такой вектор, то $(v, x) = (v', x) \ \forall x \in \RR^3$, а значит $v' = v$ по лемме.
    \item[Существование] Покажем, что $v$, заданный формулой \eqref{lec25:v} подойдёт.
        \begin{align*}
            x = x_1 e_2 + x_2 e_2 + x_3 e_3 \implies (v, x) &= \begin{vmatrix} 
                a_2 & a_3 \\
                b_2 & b_3
            \end{vmatrix} x_1 - \begin{vmatrix} 
                a_1 & a_3 \\
                b_1 & b_3
            \end{vmatrix} x_2 + \begin{vmatrix} 
                a_1 & a_2 \\
                b_1 & b_2
                \end{vmatrix} x_3 \\ &= \begin{vmatrix} 
                x_1 & x_2 & x_3 \\
                a_1 & a_2 & a_3 \\
                b_1 & b_2 & b_3
            \end{vmatrix} = \begin{vmatrix} 
                a_1 & a_2 & a_3 \\
                b_1 & b_2 & b_3 \\
                x_1 & x_2 & x_3
            \end{vmatrix} = \Vol(a, b, x)
        .\end{align*}
    \end{description}
\end{proof}


\subsection{Векторное произведение, его выражение в координатах}

\begin{definition}
    Вектор $v$ из теоремы выше называется \textit{векторным произведением} векторов $a$ и $b$.

    Обозначение: $[a, b]$ или $a \times b$.
\end{definition}


\subsection{Смешанное произведение трёх векторов, его свойства}

\begin{definition}
    $\forall a, b, c \in \EE$ число $(a, b, c) := ([a, b], c)$ называется \textit{смешанным произведением} векторов $a, b, c$.
\end{definition}

\begin{comment}
    Из \hyperref[lec25:t]{теоремы} видно, что $(a, b, c) = \Vol(a, b, c)$.
\end{comment}

\begin{proof}[Свойства смешанного произведения]~
    \begin{enumerate}[nosep]
    \item 
        $(a, b, c) > 0 \iff a, b, c$ --- положительно ориентированный ортонормированный базис,
        
        $(a, b, c) < 0 \iff a, b, c$ --- отрицательно ориентированный ортонормированный базис.

        \medskip
        Критерий компланарности ($= $ линейной зависимости)
        \begin{equation*}
            a, b, c\text{ компланарны} \iff (a, b, c) = 0
        .\end{equation*}

    \item Линейность по каждому аргументу.

    \item Кососимметричность (меняет знак при перестановке любых двух векторов).

    \item Если $e_1, e_2, e_3$ --- положительно ориентированный ортонормированный базис, то
        \begin{equation*}
            \left.\begin{aligned}
                a &= a_1 e_1 + a_2 e_2 + a_3 e_3 \\
                b &= b_1 e_1 + b_2 e_2 + b_3 e_3 \\
                c &= c_1 e_1 + c_2 e_2 + c_3 e_3
            \end{aligned} \right| \implies (a, b, c) = \begin{vmatrix} 
                a_1 & a_2 & a_3 \\
                b_1 & b_2 & b_3 \\
                c_1 & c_2 & c_3
            \end{vmatrix}
        \end{equation*}
    \end{enumerate}
\end{proof}


\subsection{Критерий коллинеарности двух векторов в терминах векторного произведения}

\begin{proposal}
    $a, b \in \EE$ коллинеарны $\iff [a, b] = 0$.
\end{proposal}

\begin{proof}~
    \begin{description}
        \item[$\implies$] 
            \begin{equation*}
                (a, b, x) = 0 \ \forall x \implies ([a, b], x) = 0 \ \forall x \implies [a, b] = 0
            .\end{equation*}

        \item[$\impliedby$]
            \begin{equation*}
                [a, b] = 0 \implies ([a, b], x) = 0 \ \forall x \implies (a, b, x) = 0 \ \forall x \in \RR^3
            .\end{equation*}

            Если $a$, $b$ линейно независимы, то можно взять $x$, который дополняет их до базиса в $\RR^3$.

            Тогда, $(a, b, x) \neq 0$ --- противоречие. Значит $a$, $b$ линейно зависимы $\implies$ коллинеарны.
            \qedhere
    \end{description}
\end{proof}


\subsection{Геометрические свойства векторного произведения}

\begin{proposal}~
    \begin{enumerate}[nosep]
    \item $[a, b] \perp \left< a, b \right>$.
    \item $\left|[a, b]\right| = \vol P(a, b)$.
    \item $\Vol(a, b, [a, b]) \geq 0$.
    \end{enumerate}
\end{proposal}

\begin{proof}~
    \begin{enumerate}
    \item $([a, b], a) = (a, b, a) = 0 = (a, b, b) = ([a, b], b)$.
    \item Если $a$, $b$ коллинеарны, то обе части равны 0.

        Пусть $[a, b] \neq 0$.
        \begin{equation*}
            \left|[a, b]\right|^2 = ([a, b], [a, b]) = (a, b, [a, b]) = (\#) > 0
        .\end{equation*}
        \begin{equation*}
            [a, b] \perp \left< a, b \right> \implies (\#) = \vol P(a, b, [a, b]) = \Vol(a, b, [a, b]) = \vol P(a, b,) \cdot \left|[a, b]\right|
        .\end{equation*}

        Сокращая на $|[a, b]| \neq 0$, получаем требуемое.

    \item
        $\Vol (a, b, [a, b]) = ([a, b], [a, b]) \geq 0$.
        \qedhere
    \end{enumerate}
\end{proof}

\begin{exercise}
    $[a, b]$ однозначно определяется свойствами $1)$ --- $3)$.
\end{exercise}

\subsection{Антикоммутативность и билинейность векторного произведения}
\subsection{Линейные многообразия в $\RR^n$}
\subsection{Характеризация линейных многообразий как сдвигов подпространств}
\subsection{Критерий равенства двух линейных многообразий}
\subsection{Направляющее подпространство и размерность линейного многообразия}
 
