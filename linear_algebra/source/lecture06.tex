\section{Лекция 26.09.2019}
% TODO: На лекции были другие пруфы. Некоторые -- сильно проще.

Напомним что такое определитель:
\begin{equation}
    \label{det:sum}
    \tag{$\star$}
    \sum_{\sigma \in S_n} \sgn(\sigma) a_{1\sigma(1)} a_{2\sigma(2)} \dots a_{n\sigma(n)}
.\end{equation}

\setcounter{theorem}{1}
\begin{comment}
    Каждое слагаемое содержит ровно 1 элемент из каждой строки и ровно 1 элемент из каждого столбца.
\end{comment}

\subsection{Свойства определителей}
\begin{description}
\item[Свойство T\label{det:prop_t}]
    $\det A = \det A^T$.

    \begin{proof}
        Пусть $B = A^T$, тогда $b_{ij} = a_{ji}$.
        \begin{align*}
            \det A^T
            = \det B
            &= \sum_{\sigma \in S_n} \sgn(\sigma) b_{1\sigma(1)} b_{2 \sigma(2)} \dots b_{n\sigma(n)}
            = \sum_{\sigma \in S_n} \sgn(\sigma) a_{\sigma(1) 1} a_{\sigma(2) 2} \dots a_{\sigma(n)} \\
            &= \sum_{\sigma \in S_n} \sgn(\sigma) a_{1\sigma(1)^{-1}} a_{2\sigma(2)^{-1}} \dots a_{n\sigma(n)^{-1}}
            \quad \text{ /\!/ замена } \sigma^{-1} = \rho \text{ /\!/} \\
            &= \sum_{\rho \in S_n} \sgn(\rho) a_{1\rho(1)} a_{2\rho(2)} \dots a_{n\rho(n)} = \det A
        .\qedhere\end{align*}
    \end{proof}

\item[Свойство 0\label{det:prop_0}]
    Если в $A$ есть нулевая строка или нулевой столбец, то $\det A = 0$.

    \begin{proof}
        В связи со \hyperref[det:prop_t]{свойством T} можно доказать только для строк.

        Так как в каждом слагаемом \eqref{det:sum} присутствует элемент из каждой строки, то все слагаемые в \eqref{det:sum} равны 0 $\implies detA = 0$.
    \end{proof}

\item[Свойство 1\label{det:prop_1}]
    Если в $A$ все элементы одной строки или одного столбца домножить на одно и то же число $\lambda$, то $\det A$ тоже умножается на $\lambda$.

    \begin{equation*}
        \begin{vmatrix} * & * & \dots & * \\ \dots & \dots & \dots & \dots \\ \lambda * & \lambda * & \lambda * & \lambda * \\ \dots & \dots & \dots & \dots \\ * & * & \dots & * \end{vmatrix} = \lambda \begin{vmatrix} * & * & \dots & * \\ \dots & \dots & \dots & \dots \\  * &  * & * & * \\ \dots & \dots & \dots & \dots \\ * & * & \dots & * \end{vmatrix}
    \end{equation*}

    \begin{proof}
        В связи со \hyperref[det:prop_t]{свойством T} можно доказать только для строк.

        $A_{(i)} \to \lambda A_{(i)} \implies a_{ij} \to \lambda a_{ij} \ \forall j \implies $ в \eqref{det:sum} каждое слагаемое умножается на $\lambda \implies \det A$ умножается на $\lambda$.
    \end{proof}

\item[Свойство 2\label{det:prop_2}]
    Если $A_{(i)} = A_{(i)}^1 + A_{(i)}^2$, то $\det A = \det \begin{pmatrix}
        A_{(1)} \\ \vdots \\ A_{(i)}^1 \\ \vdots \\ A_{(n)}
    \end{pmatrix} + \det \begin{pmatrix}
        A_{(1)} \\ \vdots \\ A_{(i)}^2 \\ \vdots \\ A_{(n)}
    \end{pmatrix}$.

    \bigskip
    Пример:
    \begin{equation*}
        \begin{vmatrix}
            a_1 & a_2 & a_3 \\
            b_1 + c_1 & b_2 + c_2 & b_3 + c_3 \\
            d_1 & d_2 & d_3
        \end{vmatrix} = \begin{vmatrix}
            a_1 & a_2 & a_3 \\
            b_1 & b_2 & b_3 \\
            d_1 & d_2 & d_3
        \end{vmatrix} + \begin{vmatrix}
            a_1 & a_2 & a_3 \\
            c_1 & c_2 & c_3 \\
            d_1 & d_2 & d_3
        \end{vmatrix}
    \end{equation*}

    Аналогично, если $A^{(j)} = A^{(j)}_1 + A^{(j)}_2$, то $\det A = \det (A^{(1)} \cdots A^{(j)}_1 \cdots A^{(n)}) + \det (A^{(1)} \cdots A^{(j)}_2 \cdots A^{(n)})$.

    \begin{proof}
        В связи со \hyperref[det:prop_t]{свойством T} можно доказать только для строк.

        Пусть $A_{(i)}^1 = (a_{i1}' a_{i2}' \cdots a_{in}')$, $A_{(i)}^2 = (a_{i1}'' a_{i2}'' \dots a_{in}'') \implies a_{ij} = a_{ij}' + a_{ij}''$.
        \begin{align*}
            \det A
            &= \sum_{\sigma \in S_n} \sgn(\sigma) a_{1\sigma(1)} a_{2\sigma(2)} \dots a_{n\sigma(n)} \\
            &= \sum_{\sigma \in S_n} \sgn(\sigma) a_{1\sigma(1)} a_{2\sigma(2)} \dots (a'_{i\sigma(i)} + a''_{i\sigma(i)}) \dots a_{n \sigma(n)} \\
            &= \sum_{\sigma \in S_n} \sgn(\sigma) a_{1\sigma(1)} a_{2\sigma(2)} \dots a'_{i\sigma(i)} \dots a_{n\sigma(n)} + \sum_{\sigma \in S_n} \sgn(\sigma) a_{1\sigma(1)} a_{2\sigma(2)} \dots a''_{i\sigma(i)} \dots a_{n\sigma(n)} \\
            &= \det A_1 + \det A_2
        .\qedhere\end{align*}
    \end{proof}

\item[Свойство 3\label{det:prop_3}]
    Если в $A$ поменять местами две строки или два столбца, то $\det A$ поменяет знак.

    \begin{proof}
        В связи со \hyperref[det:prop_t]{свойством T} можно доказать только для строк.

        Пусть $A = (a_{ij}) \in M_n$, $B = (b_{ij}) \in M_n $ -- матрица, полученная из А перестановкой $p$-ой и $q$-ой строк.

        Так же, $\tau = \tau_{pq}$.

        \begin{equation*}
            b_{ij} = a_{\tau(i)j} = \begin{cases}
                a_{ij}, &\text{если } i \neq p, q  \\
                a_{qj}, &\text{если } i = p \\
                a_{pj}, &\text{если } i = q
            \end{cases}
        \end{equation*}
        \begin{align*}
            b_{ij} = a_{\tau(i)j} \implies b_{i\sigma(i)} = a_{\tau(i)\sigma(i)} = a_{\tau(i),(\sigma\tau\tau)(i)}
        \end{align*}
        \begin{align*}
            \det B
            &= \sum_{\sigma \in S_n} \sgn(\sigma) \cdot b_{1\sigma(1)} \cdot b_{2\sigma(2)} \dots b_{n\sigma(n)} \\
            &= \sum_{\sigma \in S_n} \sgn(\sigma) \cdot a_{\tau(1), \sigma(1)} \cdot a_{\tau(2), \sigma(2)} \dots a_{\tau(n), \sigma(n)} \\
            &= \sum_{\sigma \in S_n} \sgn(\sigma) \cdot a_{\tau(1),(\sigma\tau\tau)(1)} \cdot a_{\tau(2),(\sigma\tau\tau)(2)} \dots a_{\tau(n),(\sigma\tau\tau)(n)} \\
            &\quad \text{ /\!/ уберем } \tau(i) \text{, переупорядочив элементы в произведении /\!/} \\
            &= \sum_{\sigma \in S_n} \sgn(\sigma) \cdot a_{1, (\sigma\tau)(1)} \cdot a_{2, (\sigma\tau)(2)} \dots a_{n,(\sigma\tau)(n)} \\
            &= -\sum_{\sigma \in S_n} \sgn(\sigma\tau) \cdot a_{1, (\sigma\tau)(1)} \cdot a_{2, (\sigma\tau)(2)} \dots a_{n,(\sigma\tau)(n)} \\
            &\quad \text{ /\!/ замена } \rho = \sigma\tau \text{ /\!/} \\
            &= -\sum_{\rho \in S_n} \sgn(\rho) \cdot a_{1, \rho(1)} \cdot a_{2, \rho(2)} \dots a_{n, \rho(n)} \\
            &= -\det A
        .\qedhere\end{align*}
    \end{proof}

\item[Свойство 4\label{det:prop_4}]
    Если к строке (столбцу) прибавить другую строку (столбец), умноженный на скаляр, то $\det A$ не изменится.

    \begin{proof}
        В связи со \hyperref[det:prop_t]{свойством T} можно доказать только для строк.

        \begin{equation*}
            A \to A' = \begin{pmatrix}
                \dots \\ A_{(i)} + \lambda A_{(j)} \\ \dots \\ A_{(j)} \\ \dots
            \end{pmatrix}
        \end{equation*}

        \begin{equation*}
            |A'| = \begin{vmatrix}
                \dots \\ A_{(i)} \\ \dots \\ A_{(j)} \\ \dots
            \end{vmatrix} + \begin{vmatrix}
                \dots \\ \lambda A_{(j)} \\ \dots \\ A_{(j)} \\ \dots
            \end{vmatrix} = |A| + \lambda \begin{vmatrix}
                \dots \\ A_{(j)} \\ \dots \\ A_{(j)} \\ \dots
            \end{vmatrix} = |A| + \lambda 0 = |A|
        .\end{equation*}
    \end{proof}

\item[Свойство 5\label{det:prop_5}]
    Если в А есть две одинаковые строки (столбца), то $\det A = 0$.
    \begin{proof}
        В связи со \hyperref[det:prop_t]{свойством T} можно доказать только для строк.

        При перестановке двух одинаковых строк (столбцов):

        -- А не изменится $\implies \det A$ не изменится

        -- по \hyperref[det:prop_3]{свойству 3}: $\det A$ меняет знак

        Значит, $\det A = -\det A \implies \det A = 0$.
    \end{proof}
\end{description}

\begin{definition}
    Матрица называется \textit{верхнетреугольной}, если $a_{ij} = 0$ при $i > j$, \textit{нижнетреугольной}, если $a_{ij} = 0$ $i < j$.
\end{definition}
\begin{equation*}
    \begin{pmatrix}
        a_{11} & a_{12} & a_{13} & \dots & a_{1n} \\
        0 & a_{22} & a_{23} & \dots & a_{2n} \\
        0 & 0 & a_{33} & \dots & a_{3n} \\
        \vdots & \vdots & \vdots & \ddots & \vdots \\
        0 & 0 & 0 & \dots & a_{mn}
    \end{pmatrix} \text{ -- верхнетреугольная}
\end{equation*}

\begin{equation*}
    \begin{pmatrix}
        a_{11} & 0 & 0 & \dots & 0 \\
        a_{21} & a_{22} & 0 & \cdots & 0 \\
        a_{31} & a_{32} & a_{33} & \cdots & 0 \\
        \vdots & \vdots & \vdots & \ddots & \vdots \\
        a_{m1} & a_{m2} & a_{m3} & \cdots & a_{mn}
    \end{pmatrix} \text{ -- нижнетреугольная}
\end{equation*}

\begin{comment}
    Всякая ступенчатая квадратная матрица верхнетреугольна.
\end{comment}

\begin{description}[resume]
\item[Свойство 6\label{det:prop_6}]
    Если $A$ верхнетреугольная или нижнетреугольная, то $\det A = a_{11} a_{22} \dots a_{nn}$.

    \begin{proof}
        В связи со \hyperref[det:prop_t]{свойством T} можно доказать только для строк.

        Выделим в \eqref{det:sum} слагаемые, которые могут быть отличны от нуля.
        \begin{equation*}
            a_{1, \sigma(1)} \dots a_{n - 1, \sigma(n - 1)} a_{n, \sigma(n)} \neq 0
        \end{equation*}
        \begin{equation*}
            \implies a_{n\sigma(n)} \neq 0 \implies \sigma(n) = n
        .\end{equation*}
        \begin{equation*}
            \implies a_{n - 1, \sigma(n - 1)} \neq 0 \implies \sigma(n - 1) \in \{n - 1, n\},
        \end{equation*}

        но $n$ уже занято, значит $\sigma(n - 1) = n - 1$, и так далее.

        Рассуждая аналогично, получаем $\sigma(k) = k \ \forall k \implies \sigma = id$ -- это единственное слагаемое в \eqref{det:sum}, которое может быть не равно 0.
        \begin{equation*}
            \sgn(id) = +1 \implies \det A = a_{11} a_{22} \dots a_{nn}
        .\end{equation*}
    \end{proof}

    \begin{corollary}
        $\det \text{diag}(a_1, a_2, \dots, a_n) = a_1 a_2 \dots a_4$.
    \end{corollary}

    \begin{corollary}
        $\det E = 1$.
    \end{corollary}
\end{description}


\subsection{Поведение определителя при элементарных преобразованиях строк (столбцов)}

$\text{Э}_1 (i, j, \lambda)$: $\det A$ не меняется.

$\text{Э}_2 (i, j)$: $\det A$ меняет знак.

$\text{Э}_3 (i, \lambda)$: $\det A$ умножается на $\lambda$.


\bigskip
\textit{Алгоритм.} Элементарными преобразованиями строк А приводится к ступенчатому ($\to$ верхнетреугольному) виду, в котором $\det A$ легко считается.
