\section{Лекция 19.04.2018}

$V$ --  векторное пространство над полем $F$

$dim V = n$

$\varphi \in L(V)$

\subsection{Наблюдения}

4) $A(\varphi, e) = \left(
\begin{array}{c|c|c|c|c}
  * & 0 & 0 & \dots & 0  \\
  \hline
  0 & * & 0 & \dots & 0  \\
  \hline
  0 & 0 & * & \dots & 0 \\
  \hline
  \vdots & \vdots & \vdots & \vdots & \vdots \\
  \hline
  0 & 0 & 0 & \dots & * \\
\end{array}
\right)$ -- блочно-диагональный вид (каждая звездочка размера $k_i \times k_i$ для $i \in 1, \dots, s$) $\Leftrightarrow$ все подпространства $U_1, \dots, U_s$ $\varphi$-инвариантны, где $U_1 = <e_1, \dots, e_{k_1}>, U_2 = <e_{k_1 + 1}, \dots, e_{k_1 + k_2}>, \dots, U_s = <e_{n-k_s+1}, \dots, e_n>$. При этом $V = U_1 \oplus \dots \oplus U_s$.

Предел мечтаний: найти такой базис $e$, в котором $A(\varphi, e)$ диагональна (увы, это не всегда возможно).

\vspace{\baselineskip}
\textbf{Определение.} Вектор $v \in V$ называется \textit{собственным вектором} линейного оператора $\varphi$, если $v \neq 0$ и $\varphi(v) = \lambda v$ для некотрого $\lambda$. 

\vspace{\baselineskip}
\textbf{Определение.} Элемент $\lambda \in F$ называется \textit{собственным значением} линейного оператора $\varphi$, если $\exists v \in V$, такой что $v \neq 0$ и $\varphi(v) = \lambda v$.

\vspace{\baselineskip}
\textbf{Определение.} Множество собственных значений линейного оператора $\varphi$ называется его \textit{спектром}.

Обозначение: $Spec(\varphi)$

\vspace{\baselineskip}
\textbf{Замечание.} В ситуации $\varphi(v) = \lambda v, \ v \neq 0$, говорят:

1) $v$ -- собственный вектор, отвечающий собственному значению $\lambda$

2) $\lambda$ -- собственное значение, отвечающее собственному вектору $v$

\vspace{\baselineskip}
\textbf{Предложение.} $0 \neq v \in V$ является собственным для линейного оператора $\varphi \Leftrightarrow <v>$ -- $\varphi$-инвариантное подпространство.

\vspace{\baselineskip}
\textbf{\textit{Доказательство.}} $\rhd \ (\Rightarrow) \varphi(v) = \lambda v \Rightarrow w \in <v> \Rightarrow w = \mu v, \mu in F$

$\varphi(w) = \varphi(\mu v) = \mu \varphi(v) = \mu \lambda v \in <v>$

$(\Leftarrow)$ Имеем $\varphi(v) \in <v> \Rightarrow \varphi(v) = \lambda v$ для некоторого $\lambda in F \ \lhd$

\vspace{\baselineskip}
Примеры.

1) $\varphi = \lambda \cdot Id$ -- скалярный оператор

$\varphi(v) = \lambda v \ \forall \ v \in V$

$\forall \ v \in V$ являются собственными с собственным значением $\lambda$

2) $V = \mathbb{R}^2, \ \varphi$ -- ортогональная проекция на прямую $l$, проходящую через 0

$v \in l \setminus \{0\} \Rightarrow \varphi(v) = v \Rightarrow v$ -- собственный вектор с собственным значением 1

$v \in l^{\bot} \setminus \{0\} \Rightarrow \varphi(v) = 0 \Rightarrow v$ -- собственный вектор с собственным значением 0

3) $\varphi: \mathbb{R}^2 \rightarrow \mathbb{R}^2$ -- поворот на угол $\alpha \neq \pi k$ ($\alpha = 2 \pi k \Rightarrow \varphi = Id$, $\alpha = \pi + 2 \pi k \Rightarrow \varphi = Id$)

Собственных векторов нет

4) $V = F[x]_{\leq n}$

$\varphi: f \rightarrow f'$ -- дифференцирование

$f$ -- собственный вектор $\Leftrightarrow f = const$

собственное значение 0

\vspace{\baselineskip}
\textbf{Определение.} Линейный оператор $\varphi$ называется \textit{диагонализуемым}, если $\exists$ базис $e$ пространства $V$, такой что матрица $A(\varphi, e)$ диагональна.

\vspace{\baselineskip}
\textbf{Предложение.} Линейный оператор $\varphi$ диагонализуем $\Leftrightarrow$ в $V \ \exists$ базис, состоящий из собственных векторов для $\varphi$.

\vspace{\baselineskip}
\textbf{\textit{Доказательство.}} $\rhd \ e = (e_1, \dots, e_n)$ -- базис $V \Rightarrow A(\varphi, e) = diag(\lambda_1, \dots, \lambda_n) = \begin{pmatrix} \lambda_1 & 0 & \dots & 0 \\ 0 & \lambda_2 & \dots & 0 \\ \vdots & \vdots & \vdots & \vdots \\ \ 0 & 0 & \dots & \lambda_n \end{pmatrix} \Leftrightarrow \varphi(e_1) = \lambda_1 e_1, \dots, \varphi(e_n) = \lambda_n e_n \Leftrightarrow e_1, \dots, e_n$ -- собственные векторы

\vspace{\baselineskip}
Примеры выше.

1) $\varphi$ уже диагонализован, т.к. $\forall$ вектор $\neq 0$ является собственным

2) $e_1 \in l, e_2 \in l^{\bot} \Rightarrow (e_1, e_2)$ -- базис из собственных векторов $\Rightarrow \varphi$ диагонализуем

$A(\varphi, e) = \begin{pmatrix} 1 & 0 \\ 0 & 0 \end{pmatrix}$

3) не диагонализуем, т.к. нет собственных векторов

4) $\varphi$ диагонализуем $\Rightarrow n = 0$

\vspace{\baselineskip}
$\varphi \in L(V), \lambda \in F$

$V_{\lambda} (\varphi) := \{v \in V \ | \ \varphi(v) = \lambda v \}$

\vspace{\baselineskip}
\textbf{\textit{Упражнение.}} $V_{\lambda} (\varphi)$ -- подпространство в $V$

\vspace{\baselineskip}
\textbf{\textit{Лемма.}} $V_{\lambda} (\varphi) \neq 0 \Leftrightarrow \lambda \in Spec(\varphi)$

\vspace{\baselineskip}
\textbf{\textit{Доказательство.}} Следует из определения.

\vspace{\baselineskip}
\textbf{Определение.} $\lambda \in Spec(\varphi) \Rightarrow V_{\lambda} (\varphi)$ называется \textit{собственным подпространством}, отвечающим собственному значению $\lambda$.

\vspace{\baselineskip}
\textbf{Замечание.} $V_{\lambda} (\varphi)$ -- $\varphi$-инвариантное подпространство

$\varphi |_{V_{\lambda} (\varphi)} = \lambda \cdot Id |_{V_{\lambda} (\varphi)}$

\vspace{\baselineskip}
\textbf{Предложение.} $\lambda \in F \Rightarrow V_{\lambda} (\varphi) = Ker(\varphi - \lambda \cdot Id)$

\vspace{\baselineskip}
\textbf{\textit{Доказательство.}} $\rhd \ v \in V_{\lambda} (\varphi) \Leftrightarrow \varphi (v) = \lambda v \Leftrightarrow \varphi(v) - \lambda v = 0 \Leftrightarrow (\varphi - \lambda \cdot Id) v = 0 \Leftrightarrow v \in Ker(\varphi - \lambda \cdot Id) \ \lhd$

\vspace{\baselineskip}
\textbf{Следствие.} $\lambda \in Spec(\varphi) \Leftrightarrow det(\varphi - \lambda \cdot Id) = 0$

\vspace{\baselineskip}
\textbf{\textit{Доказательство.}} $\rhd \ \lambda \in Spec(\varphi) \Leftrightarrow V_{\lambda} (\varphi) \neq \{0\} \Leftrightarrow Ker(\varphi - \lambda \cdot Id) \neq 0 \Leftrightarrow det (\varphi - \lambda \cdot Id) = 0 \ \lhd$

\vspace{\baselineskip}
\textbf{Определение.} Многочлен $\chi_{\varphi} (t) = (-1)^n det (\varphi - t \cdot Id)$ называется $\textit{характеристическим многочленом}$ линейного оператора $\varphi$.

\vspace{\baselineskip}
Если $e$ -- какой-либо базис $V$, $A = (a_{ij}) = A(\varphi, e)$, то

$\chi_{\varphi} (t) = (-1)^n det (A - t \cdot E) = (-1)^n \begin{vmatrix} a_{11} - t & a_{12} & a_{13} & \dots & a_{1n} \\
a_{21} & a_{22} - t & a_{23} & \dots & a_{2n} \\ 
a_{31} & a_{32} & a_{33} - t & \dots & a_{3n} \\
\vdots & \vdots & \vdots & \vdots & \vdots \\
a_{n1} & a_{n2} & a{n3} & \dots & a_{nn} - t \end{vmatrix}$

\vspace{\baselineskip}
$\chi_{\varphi} (t) = t^n + c_{n-1} t^{n-1} + \dots + c_1 t + c_0, \ c_0 = (-1)^n det \varphi, \ c_{n-1} = -tr \varphi$

\vspace{\baselineskip}
Вывод из разобранного выше:

\textbf{Утверждение.} $\lambda \in Spec(\varphi) \Leftrightarrow \chi_{\varphi} (\lambda) = 0$, т.е. $\lambda$ -- корень характеристического многочлена

\vspace{\baselineskip}
\textbf{Следствие.} $|Spec(\varphi)| \leq n$

\vspace{\baselineskip}
\textbf{Следствие.} $F = \mathbb{C} \Rightarrow \forall \ \varphi \in L(V) \ \exists$ собственный вектор

\vspace{\baselineskip}
\textbf{\textit{Доказательство.}} $\rhd \ \chi_{\varphi} (t)$ имеет корень по основной теореме алгебры комплексных чисел $\lhd$

\vspace{\baselineskip}
$\lambda \in Spec(\varphi)$

$k_\lambda$ -- кратность $\lambda$ как корня многочлена $\chi_{\varphi} (t)$, т.е. $\chi_{\varphi} (t)$ делится на $(t - \lambda)^{k_{\lambda}}$, но не делится на большую степень

\vspace{\baselineskip}
\textbf{Определение.} Число $k_{\lambda}$ называется \textit{алгебраической кратностью} собственного значения $\lambda$.

\vspace{\baselineskip}
\textbf{Определение.} Число $dim V_{\lambda} (\varphi)$ называется $\textit{геометрической кратностью}$ собственного значения $\lambda$.

\vspace{\baselineskip}
\textbf{Предложение.} $\lambda \in Spec(\varphi) \Rightarrow$ (геометрическая кратность $\lambda$) $\leq$ (алгебраическая кратность $\lambda$)

\vspace{\baselineskip}
\textbf{\textit{Доказательство.}} $\rhd$ Положим $m_{\lambda} = dim V_{\lambda} (\varphi)$, т.е. $m_{\lambda}$ -- геометрическая кратность для $\lambda$. Пусть $(e_1, \dots, e_{m_\lambda})$ -- базис $V_{\lambda} (\varphi)$. Дополним его до базиса $e = (e_1, \dots, e_n)$ всего $V$.

Тогда $A(\varphi, e) = \left( \begin{array}{c|c}
  A & B  \\
  \hline
  0 & D  \\
\end{array}
\right), A = \begin{pmatrix} \lambda & 0 & \dots & 0 \\ 0 & \lambda & \dots & 0 \\ \vdots & \vdots & \vdots & \vdots \\ 0 & 0 & \dots & \lambda \end{pmatrix} \in M_{m_{\lambda}}, B \in Mat_{m_{\lambda} \times (n - m_{\lambda})}, D \in M_{n - m_{\lambda}} \Rightarrow$

$\chi_{\varphi}(t) = (-1)^n \left| \begin{array}{c|c}
  A - t \cdot E & B  \\
  \hline
  0 & D - t \cdot E  \\
\end{array}
\right|  = (-1)^n (\lambda - t)^{m_\lambda} \cdot det (D - t \cdot E) = (t - \lambda)^{m_\lambda} \cdot (-1)^{n - m_{\lambda}} \cdot det (D - t \cdot E) \Rightarrow \chi_{\varphi} (t) \vdots (t - \lambda)^{m_{\lambda}} \Rightarrow$

(алгебраическая кратность $\lambda$) $\geq m_{\lambda} \ \lhd$

