\section{Лекция 19.10.2017}

\subsection{}

Свойство 4. Если в А есть две одинаковые строки (столбца), то $detA = 0$.

\bigskip
\textbf{\textit{Доказательство.}} $\rhd$ При перестановке двух одинаковых строк (столбцов):

-- А не изменится $\Rightarrow$ detA не изменится

-- по свойству 3: $detA$ меняет знак

$\Rightarrow detA = -detA \Rightarrow detA = 0 \ \lhd$.

\bigskip
Свойство 5. Если к строке (столбцу) прибавить другую строку (столбец), умноженный на скаляр, то $detA$ не изменится.

\bigskip
\textbf{\textit{Доказательство.}} $\rhd$ Свойство Т $\Rightarrow$ достаточно доказать для строк.

\begin{equation*}A \rightarrow A' = \begin{pmatrix} \dots \\ A_{(i)} + \lambda A_{(j)} \\ \dots \\ A_{(j)} \\ \dots \end{pmatrix} \end{equation*}

\begin{equation*}detA' = det \begin{pmatrix} \dots \\ A_{(i)} \\ \dots \\ A_{(j)} \\ \dots \end{pmatrix} + det \begin{pmatrix} \dots \\ \lambda A_{(j)} \\ \dots \\ A_{(j)} \\ \dots \end{pmatrix} = detA + \lambda det \begin{pmatrix} \dots \\ A_{(j)} \\ \dots \\ A_{(j)} \\ \dots \end{pmatrix} = det A + \lambda 0 = det A \ \lhd \end{equation*}.

\bigskip
\textbf{Определение.} Матрица называется \textit{верхнетреугольной}, если $a_{ij} = 0 \ при \ i > j$, \textit{нижнетреугольной}, если $a_{ij} = 0 \ при \ i < j$. 

\begin{equation*}\begin{pmatrix}
		a_{11} & a_{12} & \cdots & a_{1n} \\
		0 & a_{22} & \cdots & a_{2n} \\
        0 & 0 & \cdots & a_{3n} \\
       \vdots & \vdots & \vdots& \vdots \\ 
       0 & 0 & \cdots & a_{mn}
	\end{pmatrix}
\end{equation*} -- верхнетреугольная 

\begin{equation*}\begin{pmatrix}
		a_{11} & 0 & \cdots & 0 \\
		a_{21} & a_{22} & \cdots & 0 \\
        a_{31} & a_{32} & \cdots & 0 \\
       \vdots & \vdots & \vdots& \vdots \\ 
       a_{m1} & a_{m2} & \cdots & a_{mn}
	\end{pmatrix}
\end{equation*} -- нижнетреугольная 

\bigskip
\textbf{Предложение.} Если А -- верхнетреугольная (или нижнетреугольная) матрица, то 

$detA = a_{11} a_{22} \dots a_{nn}$.

\bigskip
\textbf{\textit{Доказательство.}} $\rhd$ Свойство Т $\Rightarrow$ достаточно доказать для верхнетреугольной.

Выясним, какие слагаемые в (*) могут быть отличны от нуля.

$a_{1, \sigma(1)} a_{2, \sigma(2)} \dots a_{n, \sigma(n)} \neq 0:$

$a_{n, \sigma(n)} \neq 0 \Rightarrow \sigma (n) = n$

$a_{n-1, \sigma(n - 1)} \neq 0 \Rightarrow \sigma (n - 1) \in \{n, n - 1 \}$ но $n$ уже занято $\Rightarrow \sigma (n - 1) = n - 1$.

Рассуждаем аналогично, получаем: $\sigma (k) = k \ \forall \ k = n, n-1, n-2, \dots, 1 \Rightarrow \sigma = id$.

Вывод: в (*) может быть не равно нулю ровно 1 слагаемое $a_{11} a_{22} \dots a_{nn}$, оно входит со знаком + $\Rightarrow detA  = a_{11} a_{22} \dots a_{nn} \ \lhd$.

\bigskip
\textbf{Следствие.} 1) $det(diag(a_1, a_2, \dots, a_n)) = a_1 a_2 \dots a_n$.

2) $detE = +1$.

\bigskip
\textbf{Замечание.} Всякая (квадратная) ступенчатая матрица верхнетреугольна.

\bigskip
\subsection{Вычисление определителей при помощи элементарных преобразований}

\bigskip
$Э_1 (i, j, \lambda): detA$ не меняется.

$Э_2 (i, j): detA$ меняет знак.

$Э_3 (i, \lambda): detA$ умножается на $\lambda$.

\bigskip
\textit{Алгоритм.} Элементарными преобразованиями строк А приводится к ступенчатому  ($\Rightarrow$ верхнетреугольному) виду, в котором $det$ легко считается.

\bigskip
\textbf{Предложение.} (Определитель с углом нулей) 

\begin{equation*} A = 
\left(
\begin{array}{c|c}
  P & Q \\
  \hline
  0 & R
\end{array}
\right) \ или \ A = 
\left(
\begin{array}{c|c}
  P & 0 \\
  \hline
  Q & R
\end{array}
\right),\ P \in M_k, \ R \in M_{n-k} \Rightarrow detA = detP detR
\end{equation*}

Матрица с углом нулей:
\begin{equation*} 
\left(
\begin{array}{c|ccc}
  * & * & * & * \\
  \hline
  0 & * & * & * \\
  0 & * & * & * \\
  0 & * & * & *
\end{array}
\right)
\end{equation*}

НЕ матрица с углом нулей:
\begin{equation*} 
\left(
\begin{array}{c|ccc}
  * & * & * & * \\
  * & * & * & * \\
  \hline
  0 & * & * & * \\
  0 & * & * & *
\end{array}
\right)
\end{equation*}

\bigskip
\textbf{\textit{Доказательство.}} $\rhd$ Свойство Т $\Rightarrow$ достаточно доказать для первого типа.

1) Элементарными преобразованиями строк приведем матрицу $(P|Q)$ к виду $(P'|Q')$, в котором P ступечатая ($\Rightarrow$ верхнетреугольная) $\Rightarrow$ detP и detA умножатся на одно и то же число $\alpha \neq 0$. 

2) Элементарными преобразованиями строк приведем матрицу $(0|R)$ к виду $(0|R')$, в котором R ступечатая ($\Rightarrow$ верхнетреугольная) $\Rightarrow$ detA и detR умножатся на одно и то же число $\beta \neq 0$. 

A' -- верхнетреугольная $\Rightarrow$ det A' = det P' det R'

$\alpha \beta det A = det A' = det P' det R' = (\alpha det P) (\beta det R) = \alpha \beta det P det R \Rightarrow det A = det P det R \ \lhd$.

\bigskip
\textbf{Теорема.} $A, B \in M_n \Rightarrow det(AB) = detA detB$.

\bigskip
\textbf{\textit{Доказательство.}} $\rhd$ 1) A, B верхнетреугольная $\Rightarrow$ AB -- верхнетреугольная.


$det(AB) = a_1 b_1 a_2 b_2 \dots a_n b_n = (a_1 a_2 \dots a_n)(b_1 b_2 \dots b_n) = det A det B$.

\bigskip
2) Элементраное преобразование строк А $\rightarrow$ то же элементарное преобразование в AB.

$A \rightarrow UA \Rightarrow AB \rightarrow UA(B) = U(AB)$

Элементарное преобразовнаие столбцов матрицы $B \rightarrow$ то же элементарное преобразование в AB.

$B \rightarrow BV \Rightarrow AB \rightarrow A(BV) = (AB)V$

\bigskip
Элементарным преобразованием строк приведем А к верхнетреугольному виду. При этом detA и det(AB) умножатся на одно и то же число $\alpha \neq 0$.

Элементарным преобразованием столбцов приведем B к верхнетрегольному виду. При этом detB и det(AB) умножатся на одно и то же число $\beta \neq 0$.


\bigskip
$A', B'$ -- верхнетреугольные $\Rightarrow det(A'B') = detA' detB' = \alpha \beta detAB = detA'B' = detA' det B' = (\alpha det A)(\beta bet B) = \alpha \beta det A det B \Rightarrow detAB = detA detB \ \lhd$.

\bigskip
Пусть $A \in M_n$.

\textbf{Определение.} \textit{Дополнительным минором к элементу} $a_{ij}$ называется определитель $(n-1) \times (n-1)$ -- матрицы, получающейся из А вычеркиванием i-ой строки и j-ого столбца. Обозначение: $\overline{M_{ij}}$.

\bigskip
\textbf{Определение.} \textit{Алгебраическим дополнением к элементу} $a_{ij}$ называется число $A_{ij} = (-1)^{i+j} \overline{M_{ij}}$.

\bigskip
\textbf{Лемма.} Пусть $a_{ik} = 0$ при всех $k \neq j$. Тогда $detA = a_{ij}A_{ij}$.

\textbf{\textit{Доказательство.}}
\begin{equation*} \rhd A = 
\left(
\begin{array}{ccc|c|ccc}
  & P & & U & & Q & \\
  \hline
  0 & \dots & 0 & a_{ij} & 0 & \dots & 0 \\
  \hline
  & R & & V & & S &
\end{array}
\right)
\end{equation*}

\bigskip
Переставляя соседние строки, вытолкнем i-ю строку наверх.

\begin{equation*} 
\left(
\begin{array}{ccc|c|ccc}
	0 & \dots & 0 & a_{ij} & 0 & \dots & 0 \\
  \hline
  & P & & U & & Q & \\
  \hline
  & R & & V & & S &
\end{array}
\right)
\end{equation*}

\bigskip
Переставляя соседние столбцы, переместим j-й столбец на первое место. 

\begin{equation*} A' = 
\left(
\begin{array}{ccc|c|ccc}
	 & a_{ij} &  & 0  & 0 & \dots & 0 \\
  \hline
  & U & & P & & Q & \\
  \hline
  & V & & R & & S &
\end{array}
\right)
\end{equation*}

\bigskip
$detA' = a_{ij}det \left(
\begin{array}{c|c}
  P & Q \\
  \hline
  R & S
\end{array}
\right) = a_{ij} \overline{M_{ij}}$.

\bigskip
$det A = (-1)^{i-1+j-1} det A' = (-1)^{i+j} a_{ij}\overline{M_{ij}} = a_{ij} A_{ij} \ \lhd$.

