\section{Лекция 21.09.2017}

\subsection{}
M - матрица.

\bigskip
\textbf{Определение.} Строка ($a_1, a_2, \dots , a_n$) называется \textit{нулевой}, если $a_1 = a_2 = \dots = a_n = 0$ , и \textit{ненулевой} иначе (т.е. $\exists i: a_i \neq 0$).

\bigskip
\textbf{Определение.} \textit{Ведущий элемент} ненулевой строки -- это первый ненулевой элемент этой строки.

\bigskip
\textbf{Определение.} Матрица М называется \textit{ступенчатой} (имеет \textit{ступенчатый вид}), если:

1) номера ведущих элементов ненулевых строк строго возратают;

2) все нулевые строки в конце (= внизу).

\begin{equation*}
    \begin{pmatrix}
    	0 & \cdots & 0 & \Diamond & * & * & * & * & * & * \\
        0 & \cdots & 0 & 0 & \cdots & \Diamond & * & * & * & * \\
        0 & \cdots & 0 & 0 & \cdots & 0 & 0 & \Diamond & * & * \\
        \vdots & \vdots & \vdots& \vdots & \vdots & \vdots & \vdots & \vdots & \vdots& \vdots \\
        0 & 0 & 0 & \cdots & \cdots & \cdots & \cdots & 0 & \Diamond & *\\
        0 & 0 & 0 & \cdots & 0 & 0 & 0 & 0 & 0 & 0
	\end{pmatrix}
\end{equation*}

$\Diamond \neq 0$

$*$ -- что угодно

\bigskip
\textbf{Определение.} Матрица М имеет \textit{улучшенный ступенчатый вид}, если:

1) М имеет ступенчатый вид;

2) все ведущие элементы $ = 1$;

3) в одном столбце с каждым ведущим элементом стоят только нули.

\begin{equation*}
	\begin{pmatrix}
    	0 & \cdots & 0 & 1 & * & 0 & * & 0 & 0 & * \\
        0 & \cdots & 0 & 0 & \cdots & 1 & * & 0 & 0 & * \\
        0 & \cdots & 0 & 0 & \cdots & 0 & 0 & 1 & 0 & * \\
        \vdots & \vdots & \vdots& \vdots & \vdots & \vdots & \vdots & \vdots & \vdots& \vdots \\
        0 & 0 & 0 & \cdots & \cdots & \cdots & \cdots & 0 & 1 & *\\
        0 & 0 & 0 & \cdots & 0 & 0 & 0 & 0 & 0 & 0
	\end{pmatrix}
\end{equation*}

\begin{theorem}[]\label{theorem1}
	1) Всякую матрицу элементарными преобразованиями строк можно привести к ступенчатому виду.

2) Всякую ступенчатую матрицу элементарными преобразованиями строк можно привести к улучшенному ступенчатому виду.
\end{theorem}

\begin{corollary}
	Всякую матрицу элементарными преобразованиями строк можно привести к ступенчатому виду.
\end{corollary}

\begin{comment}
	Улучшенный ступенчатый вид матрицы определен однозначно.
\end{comment}

\textbf{\textit{Доказательство теоремы.}} 

$\rhd$ Обозначения: m - число строк, n - число столбцов, $a_{ij}$ - элемент, стоящий на пересечении i-ой стhоки и j-ого столбца.

\bigskip
1) Алгоритм. Если М -- нулевая, то конец. Иначе:

\textit{Шаг 1.} Ищем первый ненулевой столбец, пусть j -- его номер.

\textit{Шаг 2.} Переставляя строки, если нужно, добиваемся того, что $a_{1j} \neq 0 $

\textit{Шаг 3.} Выполняем $Э_1$(2,1,$-\frac{a_{2j}}{a_{1j}}$), $\dots$, $Э_1$(m,1,$-\frac{a_{mj}}{a_{1j}}$). В результате $a_{ij} = 0$ при $i = 2,3, \dots, m$. 

Дальше все повторяем для меньшей матрицы М'.

\bigskip
2) Алгоритм. Пусть $a_{1j_1}, a_{2j_2}, \dots, a_{rj_r}$ -- ведущие элементы ступенчатой матрицы.

\textit{Шаг 1.} Выполняем $Э_3$(1,$\frac{1}{a_{1j_1}}$), \dots, $Э_3$(r,$\frac{1}{a_{rj_r}}$), в результате все ведущие элементы равны 1.

\textit{Шаг 2.} Выполнив $Э_1$(r-1, r, $-a_{{r-1}\ j_r}$), $Э_1$(r-2, r, $-a_{{r-2}\ j_r}$), $\dots$, $Э_1$(1, r, $-a_{1\ j_r}$). В результате все элементы над $a_{rj_r}$ равны 0.

Аналогично обнуляем элементы над всеми остальными ведущими. 

Итог: матрица имеет улучшеный ступенчатый вид. $\lhd$

\bigskip
\subsection{Метод Гаусса решения СЛУ (метод исключения неизвестных).}

Пусть есть СЛУ с расширенной матрицей (A|b): m уравнений, n неизвестных. Помним: элементарные преобразования строк расширенной матрицы не меняют множество решений.

Алгоритм. Приводим (A|b) к ступенчатому виду (прямой ход метода Гаусса). Получаем: 

\begin{equation*}
	\begin{pmatrix}
		0 & \cdots & 0 & a_{1j_1} & \cdots & \cdots & \cdots & \vrule & b_1 \\
		0 & \cdots & 0 & 0 & a_{2j_2} & \cdots & \cdots & \vrule & b_2 \\
       \vdots & \vdots & \vdots& \vdots & \vdots & \vdots & \vdots & \vrule & \vdots & \\ 
       0 & 0 & 0 & \cdots & 0 & 0 & a_{rj_r} & \vrule & b_r \\
       0 & 0 & 0 & \cdots & 0 & 0 & 0 & \vrule & b_{r+1} \\
       0 & 0 & 0 & \cdots & 0 & 0 & 0 & \vrule & 0
	\end{pmatrix}
\end{equation*}


\textit{Случай 1.} $b_{r+1} \neq 0$. Тогда новая СЛУ содержит уравнение $0*x_1 + \dots + 0*x_n = b_{r+1}$ ($\Leftrightarrow 0 = b_{r+1}$) $\Rightarrow$ СЛУ несовместна.

\textit{Случай 2.} Либо $r = m$, либо $b_{r+1} = 0$. Приводим матрицу к улучшенному ступенчатому виду (обратный ход метода Гаусса).
\begin{equation*}
	\bordermatrix{ 
    	 & & & & j_1 & j_2 & & j_r & \cr
    	 & 0 & \cdots & 0 & 1 & \cdots & \cdots & \cdots & \vrule & b_1 \cr
		 & 0 & \cdots & 0 & 0 & 1 & \cdots & \cdots & \vrule & b_2 \cr
        & \vdots & \vdots & \vdots& \vdots & \vdots & \vdots & \vdots & \vrule & \vdots & \cr 
        & 0 & 0 & 0 & \cdots & 0 & 0 & 1 & \vrule & b_r \cr
        & 0 & 0 & 0 & \cdots & 0 & 0 & 0 & \vrule & 0 \cr
        & 0 & 0 & 0 & \cdots & 0 & 0 & 0 & \vrule & 0}
\end{equation*}

Неизвестные $x_{j_1}, x_{j_2}, \dots , x_{j_r}$ называются \textit{главными}, а остальные -- \textit{свободными}.

\bigskip
\textit{Подслучай 2.1.} $r = n$, т. е. все неизвестные -- главные.

\begin{equation*}
	\begin{pmatrix}
		1 & \cdots & 0 & \vrule & b_1 \\
        \cdots & \ddots & \cdots & \vrule & \cdots \\
        0 & \cdots & 1 & \vrule & b_r \\
       0 & \cdots & 0 & \vrule & 0 \\
       0 & \cdots & 0 & \vrule & 0
	\end{pmatrix}
\end{equation*}

Тогда СЛУ имеет вид: 

\begin{equation*}
	\left\{
		\begin{aligned}
        x_1 = b_1 \\
        x_2 = b_2 \\
        \cdots \\
        x_n = b_n
		\end{aligned}
	\right.
\end{equation*}

-- единственное решение.

\bigskip
\textit{Подслучай 2.2.} $r < n$, т.е. хотя бы одна свободная неизвестная.

Перенося в каждом уравнении члены со свободными неизвестными в правую часть, получаем выражения всех главных неизвестных через свободные, эти выражения называются \textit{общим решением исходной СЛУ}.

Каждое решение исходной СЛУ получается подстановкой произвольных значений в свободные неизвестные и вычислением соответсвующих значений главных неизвестных.

\begin{comment}
	И тогда СЛУ имеет бесконечно много решений.
\end{comment}

Пример. Улучшенный ступенчатый вид: 

\begin{equation*}
	\begin{pmatrix}
		1 & 3 & 0 & 1 & \vrule & -1 \\
		0 & 0 & 1 & -2 & \vrule & 4
	\end{pmatrix}
\end{equation*}

Главные неизвестные: $x_1, x_3$

Свободные неизвестные: $x_2, x_4$.

$x_2 = t_1, x_4 = t2$ -- параметры.

\begin{equation*}
	\begin{pmatrix}
    x_1 \\
    x_2 \\
    x_3 \\
    x_4
	\end{pmatrix}
    =
    \begin{pmatrix}
    -1 - 3t_1 - t_2 \\
    t_1 \\
    4 + 2t_2 \\
    t_2
    \end{pmatrix}
    =
    \begin{pmatrix}
    -1\\
    0\\
    4\\
    0
    \end{pmatrix}
    + t_1
    \begin{pmatrix}
    -3\\
    1\\
    0\\
    0
    \end{pmatrix}
    +t_2
    \begin{pmatrix}
    -1\\
    0\\
    2\\
    1
    \end{pmatrix}
\end{equation*}

Общее решение:

\begin{equation*}
	\left\{
		\begin{aligned}
        x_1 = -1 - 3x_2 - x_4\\
        x_3 = 4 + 2x_4 \\
		\end{aligned}
	\right.
\end{equation*}

\begin{corollary}
Всякая СЛУ с коэффициентами из $\RR$ либо несовместна, либо имеет ровно 1 решение, либо имеет бесконечно много решений.
\end{corollary}

\textbf{Определение.} СЛУ называется \textit{однородной} (ОСЛУ), если все ее правые части равны 0. Расширенная: (A|0). 

\bigskip
\textbf{Очевидный факт:} всякая ОСЛУ имеет нулевое решение ($x_1 = x_2 = \cdots= x_n = 0$).

\begin{corollary}
Всякая ОСЛУ либо имеет ровно 1 решение (нулевое), либо бесконечно много решений.
\end{corollary}

\begin{corollary}
Всякая ОСЛУ, у которой число неизвестных больше числа уравнений, имеет ненулевое решение.
\end{corollary}

\textbf{\textit{Доказательство:}} 

$\rhd$ В ступенчатом виде расширенной матрице ступенек будет $\le m$ (m -- количество уравнений). Число ступенек = числу главных неизвестных $\Rightarrow$ главных неизвестных $\le m$ $\Rightarrow$ будет хотя бы одна свободная неизвестная. Подставляя в свободную неизвестную ненулевое значение, получим ненулевое решение. $\lhd$

