\documentclass[a4paper]{article}
\usepackage{header}


\newcommand\enumtocitem[3]{\item\textbf{#1}\addtocounter{#2}{1}\addcontentsline{toc}{#2}{\protect{\numberline{#3}} #1}}
\newcommand\defitem[1]{\enumtocitem{#1}{subsection}{\thesubsection}}
\newcommand\proofitem[1]{\enumtocitem{#1}{subsection}{\thesubsection}}

\newlist{colloq}{enumerate}{1}
\setlist[colloq]{label=\textbf{\arabic*.}}

\graphicspath{
    {img/}
}


\title{\HugeДискретная математика, Коллоквиум}
\author{
	Балюк Игорь \\
	\href{https://teleg.run/lodthe}{@lodthe},
    \href{https://github.com/LoDThe/hse-tex}{GitHub} \\
}
\date{2019 --- 2020}

\begin{document}
    \maketitle

    \tableofcontents

    \newpage

    \section{Определения}

    \begin{colloq}
    \defitem{Логические операции: конъюнкция, дизъюнкция и отрицание}

        \begin{tabular}{| c | c | c |}\hline
            Обозначение & Смысл & Название \\\hline
            $A \land B$ & $A$ и $B$ & Конъюнкция \\\hline
            $A \lor B$ & $A$ или $B$ & Дизъюнкция \\\hline
            $\neg A$ & не $A$ & Отрицание \\\hline
        \end{tabular}

        $\begin{tabular}{| c | c | c | c |}\hline
            $A$ & $B$ & $A \land B$ & $A \lor B$ \\\hline
            0 & 0 & 0 & 0 \\\hline
            0 & 1 & 0 & 1 \\\hline
            1 & 0 & 0 & 1 \\\hline
            1 & 1 & 1 & 1 \\\hline
        \end{tabular}$

        $\begin{tabular}{| c | c |}\hline
            $A$ & $\neg A$ \\\hline
            0 & 1 \\\hline
            1 & 0 \\\hline
        \end{tabular}$

    \defitem{Логические операции: импликация, XOR (исключающее или) и эквивалентность}

        \begin{tabular}{| c | c | c |}\hline
            Обозначение & Смысл & Название \\\hline
            $A \oplus B$ & либо $A$, либо $B$ & XOR \\\hline
            $A \to B$ & из $A$ следует $B$ & Импликация \\\hline
            $A \leftrightarrow B$ & $A$ равносильно $B$ & Эквивалентность \\\hline
        \end{tabular}

        $\begin{tabular}{| c | c | c | c | c |}\hline
            $A$ & $B$ & $A \oplus B$ & $A \to B$ & $A \leftrightarrow B$ \\\hline
            0 & 0 & 0 & 1 & 1 \\\hline
            0 & 1 & 1 & 1 & 0 \\\hline
            1 & 0 & 1 & 0 & 0 \\\hline
            1 & 1 & 0 & 1 & 1 \\\hline
        \end{tabular}$  

    \defitem{Булевы функции. Задание таблицей истинности и вектором значений}

        Логические связки --- это функции, которые зависят от набора переменных, принимающих значения 0 или 1 (от набора высказываний). Такие переменные называют булевыми переменными, а функции --- булевыми функциями. 

        Запись таблицей

        $\begin{tabular}{| c | c | c |}\hline
            $A$ & $B$ & $A \land B$ \\\hline
            0 & 0 & 0 \\\hline
            0 & 1 & 0 \\\hline
            1 & 0 & 0 \\\hline
            1 & 1 & 1 \\\hline
        \end{tabular}$

        Первым идёт набор из одних нулей, а дальше $i$-ый набор является двоичной записью числа $i - 1$. Таким образом, всего в таблице истинности $2^k$ строк (именно столько чисел имеют двоичную запись длины $k$).
        Благодаря стандартному порядку можно просто задать булеву функцию столбцом её значений:
        \begin{equation*}
            f(x_1) = 10 = \neg x_1, \quad g(x_1, x_2) = 0001 = x_1 \land x_2
        \end{equation*}
        Говорят, что функция задана \textbf{вектором значений}

    \defitem{Существенные и фиктивные переменные булевой функции}

        Если для булевой функции $f(x_1, x_2, \dots, x_n)$ справедливо равенство
        \begin{equation*}
            f(x_1, \dots, x_{i - 1}, 0, x_{i + 1}, \dots, x_n)
            =
            f(x_1, \dots, x_{i - 1}, 1, x_{i + 1}, \dots, x_n),
        \end{equation*}
        переменная $x_i$ называется \textbf{фиктивной}; в случае, если равенство не выполняется для переменной $x_i$, то она называется \textbf{существенной}.

    \defitem{Множество, подмножество, равенство множеств}

        Когда говорят, что задано множество $A$, под этим понимают, что $A$ представляет собой совокупность объектов, игнорируя при этом какие либо отношения между этими объектами, в частности порядок; кроме того, один объект не может входить в множество более одного раза.

        Два множества равны друг другу, если их элементы совпадают.

        $\forall x \in B \implies x \in A \implies B \subseteq A$ (каждый элемент из множества $B$ принадлежит множеству $A$ означает, что $B$ --- подмножество множества $A$)

    \defitem{Операции с множествами: объединение, пересечение, разность, симметрическая разность. Диаграммы Эйлера-Венна}

        \begin{itemize}
            \item
            Объединение
            \begin{equation*}
                A \cup B = \{x \mid (x \in A) \lor (x \in B)\}
            \end{equation*}

            \item
            Пересечение
            \begin{equation*}
                A \cap B = \{x \mid (x \in A) \land (x \in B)\}
            \end{equation*}

            \item
            Разность
            \begin{equation*}
                A \setminus B = \{x \mid (x \in A) \land (x \notin B)\}
            \end{equation*}

            \item
            Симметрическая разность
            \begin{equation*}
                A \triangle B = \{x \mid ((x \in A) \land (x \notin B)) \lor ((x \notin A) \land (x \in B))\}
            \end{equation*}

            \item
            Диаграмма Эйлера-Венна --- наглядное средство для работы со множествами. На этих диаграммах изображаются все возможные варианты пересечения множеств.
        \end{itemize}

    \defitem{Законы Моргана (с обобщением на произвольное семейство множеств)}

        С помощью диаграмм легко проверить, что $A \cap B = \overline{\overline{A} \cup \overline{B}}, 
        A \cup B = \overline{\overline{A} \cap \overline{B}}$. Из связи с таблицами истинности получаем, что $a\land b = \neg (\overline{a} \lor \overline{b})$ и $a \lor b = \neg (\overline{a} \land \overline{b})$

        Эти формулы  можно обобщить:
        \begin{equation*}
            A_1 \cup A_2 \cup \dots \cup A_n \cup \dots = \overline{\overline{A_1} \cap \overline{A_2} \cap \dots \cap \overline{A_n} \cap \dots}
        \end{equation*}

    \defitem{Закон контрапозиции}

        Логический закон контрапозиции $A \to B = \neg B \to \neg A$ при переводе на язык множеств гласит $A \subset B \iff \overline{B} \subseteq \overline{A}$

    \defitem{Метод математической индукции}

        Доказательство по индукции возможно только тогда, когда доказываемое утверждение зависит от натурального параметра. То есть доказывается утверждение
        \begin{equation*}
            \forall n \in N: \; A(n)
        \end{equation*}

        С помощью правил вывода схему доказательства по индукции можно описать так:
        \begin{equation*}
            \dfrac{A(0), \quad \forall n: \; A(n) \to A(n + 1)}{\forall n: \; A(n)}
        \end{equation*}

        Первая посылка называется базой, а вторая --- шагом индукции или переходом.

    \defitem{Графы. Основные определения: ребра, вершины, степени вершин.}

        Зафиксируем граф $G(V, E)$. Вершины $u$ и $v$ называются \textbf{смежными} или \textbf{соседями}, если они образуют ребро: ${u, v} \in E$. Рёбра $e$ и $f$ называются \textbf{смежными}, если они имеют общую вершину: $e \cap f \neq \varnothing$. Вершина $v$ \textbf{инцидента} ребру $e$, если $v \in e$. Вершины $u$ и $v$, инцидентные ребру $e$, называются его концами; говорят, что $e$ соединяет $u$ и $v$. Рёбра часто записывают сокращённо: $uv$ вместо $\{u, v\}$. \textbf{Степенью} вершины $v$ называется число смежных с $v$ рёбер и обозначается $d(v)$.

        \begin{equation*}
            \sum_{u \in V} d(u) = 2 |E|
        \end{equation*}

    \defitem{Базовые графы: граф-путь, граф-цикл, полный граф, граф-звезда}

        \begin{itemize}
            \item
            Граф-путь $P_n, n \geq 0$ состоит из вершин $\{v_0, v_1, \dots, v_n\}$ и рёбер $\{v_i, v_{i + 1}\}, i < n$.

            \item
            Граф-цикл $C_n, n \geq 3$ состоит из вершин $\{v_1, \dots, v_n\}$ и рёбер $\{v_i, v_{i + 1}\}, i < n$, а также $\{v_n, v_1\}$. Как и в случае пути, длина цикла --- количество рёбер в цикле.

            \item
            Полный граф $K_n(V, E), n \geq 1$ состоит из $n$ вершин и имеет всевозможные рёбра: $E = \binom{V}{2}$

            \item
            Граф-звезда состоит из выделенной вершины, соединённой рёбрами со всеми остальными вершинами (больше рёбер в этом графе нет).

        \end{itemize}

    \defitem{Подграфы. Путь, цикл, клика и независимое множество.}

        Граф $H(W, I)$ называется подграфом графа $G(V, E)$, если $W \subseteq V$ и $I \subseteq E$.
        Другими словами, граф $H$ получается из графа $G$ удалением рёбер и вершин (вместе со смежными рёбрами). Это обозначают $H \subseteq G$. 

        Подграф $H$ графа $G$ называется
        \begin{itemize}
            \item
            \textbf{путём} из вершины $u$ в вершину $v$, если $H$ --- это граф-путь $P_n$ с началом в $u$ и концом в $v$

            \item
            \textbf{циклом}, если $H$ --- это граф-цикл $C_n$

            \item
            \textbf{кликой}, если $H$ --- это полный граф $K_n$
        \end{itemize}

        Пусть $U \subseteq V$; подграф $H$ графа $G(V, E)$, состоящий из вершин $U$ и содержащий все рёбра, которые есть в $G$ называется \textbf{индуцированным} (множеством $U$); формально $H = (U, E \cap \binom{U}{2})$. Множество $U \subseteq V(G)$ называется \textbf{независимым}, если в индуцированном $U$ подграфе нет рёбер, т.е. никакие две вершины из множества $U$ в графе $G$ не соединены рёбрами.

    \defitem{Компонента связности. Индуцированный подграф.}

        Пусть $U \in V$; подграф $H$ графа $G(V, E)$, состоящий из вершин $U$ и содержащий все рёбра, которые есть в $G$ называется \textbf{индуцированным} (множеством $U$); формально $H = (U, E \cap \binom{U}{2})$.

        Вершина $u$ называется \textbf{достижимой} из $v$, если есть путь из $v$ в $u$. Граф $G$ называется \textbf{связным}, если любая его вершина достижима из любой другой.

        $H$ --- компонента связности графа $G$, если $H \in G$, $H$ --- связный граф и не существует связного подграфа $H' \in G$, такого что $H \subsetneq H'$.

    \defitem{Деревья. Полные бинарные деревья (см. ДЗ 7).}

        Будем называть граф деревом, если он удовлетворяет любому из следующих свойств:
        \begin{enumerate}[label=(\arabic*)]
            \item
            Минимально связный граф (т. е. при удалении любого ребра граф становится несвязным).
            
            \item
            Связный граф, в котором $|E| = |V| - 1$.

            \item
            Ациклический связный граф (связный граф без циклов).

            \item
            Граф, любая пара вершин которого связана единственным путём.
        \end{enumerate}

        Вершинами полного бинарного дерева ранга $n$ являются двоичные слова длины не больше
        $n$ (включая пустое слово длины 0). Два слова соединены ребром в полном бинарном дереве,
        если одно получается из другого приписыванием одной цифры справа (нуля или единицы).

    \defitem{Правильные раскраски графов. Формулировка критерия 2-раскрашиваемости.}

        \textbf{Раскраска графа} --- это функция $f$, которая ставит в соответствие каждой вершине графа некоторый цвет, т. е. $f(u) \in {1, . . . , k}$. Раскраска $f$ называется \textbf{правильной}, если концы всех рёбер покрашены в разные цвета, т. е. для каждого ребра $\{u, v\}$ справедливо $f(u) \neq f(v)$

        Минимальное число цветов, в который можно правильно раскрасить граф $G$ называется \textbf{хроматическим числом} и обозначается через $\chi(G)$.

        Граф $G$ является \textbf{двураскрашиваемым} тогда и только тогда, когда в нём нет циклов нечётной длины.

    \defitem{Двудольные графы. Двудольные и двураскрашиваемые графы.}

        Граф $G(V, E)$ называется \textbf{двудольным}, если существует разбиение множества $V$ на подмножества $L$ и $R$ ($V = L \cup R, L \cap R = \varnothing$), такие что у каждого ребра один конец лежит в $L$, а другой в $R$, т. е. между вершинами из $L$ нет рёбер, как и между вершинами из $R$. Множества $L$ и $R$ называют \textbf{долями} графа.

        Граф двудольный тогда и только тогда, когда он двураскрашиваемый.

    \defitem{Эйлеровы циклы.}

        \textbf{Маршрутом} в графе $G$ называется последовательность вершин $v_0, v_1, \dots, v_n$, такая что $n \geq 0$ и $\{v_i, v_{i + 1}\} \in E(G), 0 \leq i \leq n - 1$.

        Маршрут, который содержит все рёбра графа ровно один раз назовём \textbf{эйлеровым маршрутом}.

        Связный граф $G$ содержит замкнутый эйлеров маршрут (\textbf{эйлеров цикл}) тогда и только тогда, когда степень каждой вершины чётна.

    \defitem{Функции. Область определения и множество значений.}

        Неформально, \textbf{функция} --- это закон, который ставит в соответствие элементам множества $X$ элементы множества $Y$; каждому элементу $x \in X$ поставлен в соответствие не более, чем один элемент из множества $y$.

        Введём понятия степени вершин и (множества) соседей для ориентированного графа. Поскольку рёбра имеют направление, то мы разделяем исходящую степень $d_+(v)$ (число вершин достижимых из $v$ по одному ребру) и входящую степень $d_-(v)$ (числу вершин, из которых за один шаг по ребру можно добраться до $v$).

        Обозначим через $f$ множество рёбер графа, задающего функцию $f$ из $X$ в $Y$; тогда $(x, y) \in f$ означает, что $f(x) = y$. Пусть $G(X \cup Y, f)$ --- граф, для функции $f$.

        \begin{itemize}
            \item
            \textbf{Областью определения} $Dom(f) \in X$ называют подмножество вершин с исходящей степенью 1 (подмножество $X$, на котором определена функция $f$).

            \item
            \textbf{Множеством значений} $Range(f) \in Y$ называется подмножество вершин с входящей степенью больше 0 (подмножество $Y$ всевозможных значений $f$).

        \end{itemize}

    \defitem{Образ множества и полный прообраз.}

        \begin{itemize}
            \item
            \textbf{Образом} $f(A)$ множества $A \in X$ называют множество значений, которые принимает $f$ на подмножестве $A$; на языке графов --- это множество правых соседей $N_+(A)$
            \begin{equation*}
                f(A) = \{y \mid \exists x \in A: \; f(x) = y\} = N_+(A)
            \end{equation*}

            \item
            \textbf{Полным прообразом} $f^{-1}(B)$ множества $B \in Y$ называют множество элементов $X$, значение функции на которых лежит в $B$; на языке графов --- это множество левых соседей $N_-(B)$:
            \begin{equation*}
                f^{-1}(B) = \{x \mid \exists y \in B: \; f(x) = y\} = N_-(B)
            \end{equation*}
        \end{itemize}

        Рассмотрим на примере:
        \begin{equation*}
            f: 1 \mapsto a, \quad 2 \mapsto b, \quad 4 \mapsto b, \quad 5 \mapsto d, \quad 6 \mapsto d, \quad 7 \mapsto d
        \end{equation*}

        Тогда
        \begin{equation*}
            Dom(f) = \{1, 2, 4, 5, 6, 7\}, Range(f) = \{a, b, d\}
        \end{equation*}
        \begin{equation*}
            f(\{1, 3, 5, 7\}) = \{a, d\}, f^{-1}(\{a, b, c\}) = \{1, 2, 4\}
        \end{equation*}

    \defitem{Отображения (всюду определённые функции). Инъекции, сюръекции и биекции.}

        В случае $Dom(f) = X$, функция $f$ называется \textbf{всюду определённой} или \textbf{отображением} . Запись $f: X \mapsto Y$ означает, что $f$ всюду определена.

        \begin{itemize}
            \item
            Отображение $f: X \mapsto Y$ называется \textbf{инъекцией}, если $f(x) \neq f(x')$ при $x \neq x'$. В терминах графа, это означает, что входящая степень каждого $y \in Y$ не превосходит единицу.

            \item
            Отображение $f: X \mapsto Y$ называется \textbf{сюръекцией}, если у каждого элемента $y$ существует прообраз, т. е. $Range(f) = Y$ или что то же самое $\forall y \in Y \exists x \in X: \; f(x) = y$. В терминах графа, это означает, что входящая степень каждого $y \in Y$ больше нуля.

            \item
            Отображение $f: X \mapsto Y$ называется \textbf{биекцией}, если оно является инъекцией и сюръекцией.
        \end{itemize}

    \defitem{Правило суммы}

        \textbf{Правило суммы} гласит, что если конечные множества $A$ и $B$ не пересекаются, то мощность их объединения совпадает с суммой мощностей:
        \begin{equation*}
            |A \cup B| = |A| + |B|, \text{ если } A \cap B = \varnothing
        \end{equation*}

        В общем случае
        \begin{equation*}
            |A \cup B| = |A| + |B| - |A \cap B|
        \end{equation*}

    \defitem{Правило произведения}

        \textbf{Правило произведения} формулируется на естественном языке следующим образом. Если есть $n$ объектов первого типа и после выбора любого объекта первого типа можно выдрать $m$ объектов второго типа, то всего есть $n \times m $способов последовательно выбрать первый и второй объект.

        Правило произведения легко обобщается по индукции на $k$ последовательных выборов. Если объект первого типа можно выбрать $n_1$ способами, после чего второй объект можно выбрать $n_2$ способами и т. д. ($k$-ый объект можно выбрать $n_k$ способами), то выбрать последовательно $k$ объектов можно $n_1 \times n_2 \times \dots \times n_k$ способами

    \defitem{Комбинаторные числа. Число перестановок, число подмножеств размера k у n-элементного мно
    жества}

        \textbf{Слово} --- это конечная последовательность символов, которые в свою очередь определяются как элементы конечного множества --- \textbf{алфавита}. Под алфавитом из $k$ символов часто удобно понимать множество $[k]_0 = \{0, 1, \dots, k - 1\}$ или $[k]_1 = \{1, 2, \dots, k\}$.

        Слова над алфавитом $[n]_1$ длины $n$, в которых все символы разные называются \textbf{перестановками}. Число перестановок есть $n!$.

        Если $\displaystyle\binom{n}{k}$ число $k$-элементных подмножеств $n$-элементного множества, то $\displaystyle\binom{n}{k} \times k! = \dfrac{n!}{(n - k)!}$, отсюда получаем, что
        \begin{equation*}
            \binom{n}{k} = \dfrac{n!}{k! (n - k)!}
        \end{equation*}

        Число $\displaystyle\binom{n}{k}$ называют \textbf{числом сочетаний}

    \defitem{Характеристическая функция и её использование при подсчёте числа элементов множества.}

        Зафиксируем юнивёрсум $U$. Функция $\chi_A(x)$ называется характеристической функцией множества $A \subseteq U$, если
        \begin{equation*}
            \chi_A(x) = \begin{cases}
                1, \text{ если } x \in A; \\
                0, \text{ если } x \notin A.
            \end{cases}
        \end{equation*}

        С помощью характеристической функции легко выразить мощность множества:
        \begin{equation*}
            |A| = \sum_{x \in U} \chi_A(x)  
        \end{equation*}

    \defitem{Формула включений и исключений}

        Формула включений исключений устроена так (здесь $[n] = \{1, 2, \dots, n\}$)
        \begin{equation*}
            |A_1 \cup A_2 \cup \dots \cup A_n| = \sum_{i = 1}^n |A_i| - \sum_{1 \leq i < j \leq n} |A_i \cap A_j| + \dots + (-1)^{m + 1} \sum_{S \subseteq [n], \\|S| = m} \left|\bigcap\limits_{A \in S} A\right|
        \end{equation*}
        или более компактно
        \begin{equation*}
            \left|\bigcup\limits_{i=1}^n A_i\right| = \sum_{S \subseteq [n], S \neq \varnothing} (-1)^{|S| + 1} \left|\bigcap\limits_{A \in S} A\right| 
        \end{equation*}

    \defitem{Биномиальные коэффициенты, основные свойства. Бином Ньютона.}

        Число $\displaystyle\binom{n}{k}$ называют \textbf{числом сочетаний}

        \begin{equation*}
            (a + b)^n = \sum_{k = 0}^{n} \binom{n}{k} a^{n - k} b^k
        \end{equation*}
        Собственно говоря, сразу ясно, что после раскрытия скобок в $(a + b)(a + b)(a + b)\dots$ ($n$ раз) получатся произведения $n$ букв (сколько-то $a$, остальные $b$), и вопрос тольков том, какие будут коэффициенты при этих произведениях (сколько подобных членов). Так вот, формула бинома Ньютона и говорит, какие это будут коэффициенты: это числа сочетаний, и они написаны в $n$-й строке треугольника Паскаля. Поэтому числа сочетаний также называют биномиальными коэффициентами. Также число сочетаний из $n$ по $k$ соответствует количеству $k$-элементных подмножеств $n$-элементного множества.

        \begin{equation*}
            \binom{n}{k} = \binom{n - 1}{k - 1} + \binom{n - 1}{k}
        \end{equation*}

        \begin{equation*}
            \binom{n}{k} = \binom{n}{n - k}
        \end{equation*}

        \begin{equation*}
            (1 + 1)^n = \sum_{k = 0}^n \binom{n}{k} = 2^n
        \end{equation*}

        \begin{equation*}
            \sum_{k = 0}^n (-1)^k \binom{n}{k} = 0
        \end{equation*}

    \defitem{Треугольник Паскаля. Рекуррентное соотношение.}

        \begin{equation*}
            \begin{tabular}{>{$n=}l<{$\hspace{12pt}}*{13}{c}}
            0 &&&&&&&1&&&&&&\\
            1 &&&&&&1&&1&&&&&\\
            2 &&&&&1&&2&&1&&&&\\
            3 &&&&1&&3&&3&&1&&&\\
            4 &&&1&&4&&6&&4&&1&&\\
            5 &&1&&5&&10&&10&&5&&1&\\
            6 &1&&6&&15&&20&&15&&6&&1
            \end{tabular}
        \end{equation*}

        $k$-ый элемент в $n$-ой строке Паскаля равен $\binom{n}{k}$ и получается суммированием двух верхний соседних элементов.

        $k$-ый элемент в $n$-ой строке равен $(n - k)$-ому, а сумма элементов равна $2^n$. 

        Числа сначала возрастают (до середины), а потом начинают симметрично убывать. Последовательность биномиальных коэффициентов $\binom{n}{0}, \binom{n}{1}, \binom{n}{2}, \dots, \binom{n}{k}, \dots, \binom{n}{n}$ возрастает, если $k < \dfrac{n - 1}{2}$, и убывает, если $r > \dfrac{n - 1}{2}$.

        Справедлива оценка
        \begin{equation*}
            \binom{2n}{n} > \dfrac{2^{2n}}{2n + 1}
        \end{equation*}

    \defitem{Бинарные отношения. Транзитивность, симметричность, рефлексивность.}

        Формально \textbf{бинарное отношение} $R$ между множествами $A$ и $B$ --- это некоторое подмножество их декартова произведения:
        \begin{equation*}
            R \subseteq A \times B
        \end{equation*}

        Если $(a, b) \in R$, говорят, что элемент $a$ находится в отношении $R$ c элементом $b$. Запись $aRb$.

        В случае, когда $R \in A \times A$, говорят, что отношение $R$ задано на множестве $A$. Отношение $\subseteq A \times A$

        \begin{itemize}
            \item
            \textbf{рефлексивно}, если $\forall a \in A: \; aRa$

            \item
            \textbf{симметрично}, если $\forall a, b \in A: \; aRb \implies bRa$

            \item
            \textbf{транзитивно}, если $\forall a, b, c \in A: \; (aRb) \land (bRc) \implies aRc$
        \end{itemize}

    \defitem{Теоретико-множественные операции с отношениями. Операция обращения.}

        \textbf{Обратным отношением} к отношению $R \subseteq A \times B$ называют отношение
        \begin{equation*}
            R^{-1} = \{(y, x) \mid xRy\} \subseteq B \times A
        \end{equation*}

        Операция обращения отношений известна также как операция транспонирования, поскольку в случае отношений между конечными множествами, обратное отношение задаётся транспонированной матрицей исходного.

    \defitem{Композиция бинарных отношений}

        Пусть $P \subseteq X \times Y, Q \subseteq Y \times Z$.
        \begin{equation*}
            Q \circ P = \{(x, z) \mid \exists y \in Y: \; xPy \land yQz\}
        \end{equation*}

    \defitem{Отношения эквивалентности.}

        Рефлексивное, симметричное и транзитивное отношение называют \textbf{отношением эквивалентности}.

        Определим \textbf{класс эквивалентности} $[a]$ как множество всех таких элементов множества $A$, которые эквивалентны элементу a:
        \begin{equation*}
            [a] = \{x \mid x \in A, x \sim a\}
        \end{equation*}

        Классы эквивалентности $[a]$ и $[b]$ (по отношению эквивалентности $\sim$) либо не пересекаются1 , либо совпадают. Множество $A$ разбивается в объединение классов эквивалентности.

        Пример отношения эквивалентности из геометрии: отношение подобия треугольников.

    \defitem{Ориентированные графы, основные определения.}

         Формально, ориентированный граф задан парой $(V, E)$, где множество вершин $V$ как и раньше произвольное, а множество $E \subseteq V \times V$ --- состоит из упорядоченных пар вершин. По-умолчанию, мы считаем, что в каждой паре вершины различны:
         \begin{equation*}
             E \subseteq \{(u, v) \mid u, v \in V, u \neq v\}
         \end{equation*}
         рёбра вида $(u, u)$ называются петлями и они возникают естественным образом при описании бинарных отношений с помощью ориентированных графов. 

         Определения, введённые нами для неориентированных графов, с поправками переносятся на ориентированные. 

         \textbf{Исходящей степенью} вершины $d_+(v)$ называется число рёбер, исходящих из вершины $v$, \textbf{входящей степенью }$d_-(v)$ --- число рёбер, входящих в $v$. Вершины входящей степени 0 называют \textbf{источниками}, к таким отно сится вершина $s \; (d_-(s) = 0)$, а вершины с нулевой исходящей степенью называют стоками: $d_+(t) = 0$. Источники и стоки часто обозначают соответственно через $s$ и $t$, от слов $source$ и $target$, хотя стоки на английском и называются $sink$.

         \textbf{Маршрутом} в ориентированном графе называется последовательность вершин $\{v_0, v_1, \dots, v_n\}$ , такая что $n > 0$ и $(v_i, v_{i + 1}) \in E(G)$ для $0 \leq i \leq n - 1$. \textbf{Длина маршрута} --- это число рёбер, соединяющих вершины маршрута; оно совпадает с $n$. Маршрут называется 
         \textbf{замкнутым}, если $v_0 = v_n$. 

    \defitem{Компоненты сильной связности ориентированного графа.}

        Вершина $v$ достижима из $u$, если существует маршрут из $u$ в $v$; отношение достижимости между вершинами обозначим $u \rightsquigarrow v$. Определим отношение двусторонней достижимости $u \leftrightsquigarrow v = (u \rightsquigarrow v) \land (v \rightsquigarrow u)$. Отношение двусторонней достижимости ($u$ достижима из $v$ и наоборот) --- отношение эквивалентности.

       \textbf{Компонента сильной связности} ориентированного графа --- класс эквивалентности по отношению достижимости. То есть множество $U \subseteq V$ --- компонента сильной связности, если любые две вершины множества $U$ достижимы друг из друга и в $U$ нельзя добавить ещё вершины с сохранением этого
        свойства (множество $U$ --- максимальное по включению).

    \defitem{Отношения (частичного) порядка (строгие и нестрогие), линейные порядки.}

        Отношение называется \textbf{антисимметричным}, если из $uRv$ и $vRu$ следует, что $u = v$:
        \begin{equation*}
            \forall u, v \in V: \; (uRv) \land (vRu) \implies u = v
        \end{equation*}

        Отношение называется \textbf{антирефлексивным}, если не содержит ни одной пары $(v, v)$ (не содержит петлей). 

        Рассмотрим отношения, которые транзитивны и антисимметричны и либо рефлексивны, либо антирефлексивны. Такие отношения называют \textbf{отношениями (частичного) порядка} или (частичными) порядками. Например, отношения $\leq, <$ на множествах $\NN_0, \ZZ, \QQ, \RR$. Эти символы традиционно используют для отношений порядка.

        Рефлексивные отношение порядка традиционно обозначают символом $\leq$, быть может с индексом, такие порядки называют \textbf{нестрогими}, антирефлексивные отношения порядка обозначают символом $<$, их называют \textbf{строгими}. 

        Отношение порядка, в котором любые два элемента сравнимы называется \textbf{линейным}.

    \defitem{Отношение непосредственного следования (см. листок недели 11).}

        Каждому отношению порядка $< (\leq)$ ставят в соответствие отношение \textbf{непосредственного следования} $\prec$:
        \begin{equation*}
            (\prec_P) = \{(x, y) \mid (x <_P y) \land (\nexists z \in V: \; (x <_P z) \land (z <_P y))\}
        \end{equation*}

    \defitem{Изоморфизм графов и (частичных) порядков (см. листок недели 11).}

        Отношения частичного порядка $\leq_P \subseteq A \times A$ и $\leq_Q \subseteq B \times B$ называются \textbf{изоморфными}, если существует такая биекция $f: A \mapsto B$, что $x \leq_P y \iff f(x) \leq_Q f(y)$.
    \end{colloq}

    \section{Примерные задачи на понимание материала курса}
    \begin{colloq}
        \defitem{TODO()}
    \end{colloq}

    \section{Вопросы на знание доказательств}
    \begin{colloq}
    \setlength\parindent{20pt}
    \proofitem{Обобщённый закон Моргана}

        \begin{equation} \label{eq:1}
            A_1 \cap A_2 \cap \dots \cap A_n \cap \dots = 
            \overline{\overline{A_1} \cup \overline{A_2} \cup \dots \cup \overline{A_n} \cup \dots}
        \end{equation}
        \begin{equation} \label{eq:2}
            B_1 \cup B_2 \cup \dots \cup B_n \cup \dots = 
            \overline{\overline{B_1} \cap \overline{B_2} \cap \dots \cap \overline{B_n} \cap \dots}
        \end{equation}

        \begin{proof}
            Докажем обобщённую формулу \ref{eq:1}, обозначим левую часть за $X$, а правую за $Y$. Если $x \in X$, то $x$ принадлежит каждому множеству $A_i$, но тогда он не принадлежит ни одному дополнению $\overline{A_i}$, а значит и их объединению. Значит $x$ принадлежит дополнению от объединения дополнений, т. е. $Y$. Мы доказали, что $X \subseteq Y$. 

            Пусть теперь $y \in Y$, тогда $y \notin \overline{Y}$ и потому для каждого $i$ выполняется $y \notin \overline{A_i}$ . Но раз $y \notin \overline{A_i}$, то $y \in A_i$ (для каждого $i$), а потому $y \in X$. Отсюда $Y \subseteq X$; как и в первом случае включение справедливо в силу произвольности $y$. 

            Итак, мы доказали, что $X = Y$, что и требовалось. Обратим внимание, что при доказательстве равенства двух множеств требуется доказывать включения в обе стороны!

            Двойственный закон Моргана \ref{eq:2} можно доказать аналогично, но можно и вывести из первого закона. Поскольку тождество \ref{eq:1} справедливо для произвольных множеств, заменим в нём $A_i$ на $\overline{B_i}$, снимем двойное дополнение и возьмём дополнения от обеих частей равенств.
        \end{proof}

    \proofitem{Иррациональность числа $\sqrt{2}$. Существуют такие иррациональные числа $a$ и $b$, что число $a^b$ рационально.}

        Число $\sqrt{2}$ рационально.

        \begin{proof}
            Доказательство от противного. Положим, что $\sqrt{2} = \dfrac{m}{n}$, где $\dfrac{m}{n}$ --- несократимая дробь, $m \in \ZZ, n \in \NN_1$. Тогда $m^2 = 2n^2$, отсюда $m^2$ делится на 2, и $m$ делится на $2$, а значит $m^2$ делится на 4, и отсюда $n^2$ делится на 2 и $n$ делится на 2. Но тогда и $m$ делится на 2, и $n$ делится на 2, а значит дробь $\dfrac{m}{n}$ сократима, пришли к противоречию.
        \end{proof}

        Существуют такие иррациональные числа $a$ и $b$, что число $a^b$ рационально.

        \begin{proof}
            Положим, что $a = b = \sqrt{2}$. Если число $(\sqrt{2})^{\sqrt{2}}$ рационально, то утверждение доказано. Если нет, то возьмём $a = (\sqrt{2})^{\sqrt{2}}, b = \sqrt{2}$:
            \begin{equation*}
                \left((\sqrt{2})^{\sqrt{2}}\right)^{\sqrt{2}} = \left(\sqrt{2}\right)^{\sqrt{2} \times \sqrt{2}} = \left(\sqrt{2}\right)^2 = 2
            \end{equation*}

            То есть, либо подходит одна пара чисел, либо другая, а какая из --- мы не знаем.
        \end{proof}

    \proofitem{Нижняя оценка числа связных компонент в неориентированном графе.}

        \begin{theorem*}
            Обозначим через $C$ число компонент связности графа $G(V, E)$. Для него справедливо неравенство
            \begin{equation}\tag{1}\label{eq:components}
                C \geq |V| - |E|
            \end{equation}
        \end{theorem*}

        \begin{proof}
            Зафиксируем количество вершин в графе $|V|$ и докажем утверждение индукцией для графов с числом рёбер $|E|$ от $0$ до $|V|$.

            \begin{itemize}
                \item
                \emph{База}: При $|E| = 0$ число компонент связности совпадает с числом вершин: $C = |V|$.

                \item
                \emph{Шаг}: Пусть для $|E| = n$ утверждение доказано. Граф с $(n + 1)$-м ребром получается из некоторого графа с $n$ рёбрами добавлением ребра. Для графа на $n$ рёбрах неравенство выполняется, а добавленное ребро либо соединит две вершины из одной компоненты связности, что не уменьшит $C$, но уменьшит $|V| - |E|$, либо соединит вершины из разных компонент, что уменьшит и левую и правую часть неравенства \ref{eq:components} на 1. В каждом из случаев, верное неравенство перейдёт в верное.
            \end{itemize}
        \end{proof}

        \begin{consequence}
            Если граф связный, то $|E| > |V| - 1$.
        \end{consequence}

    \proofitem{Если $G$ --- минимально связный граф (удаление любого ребра приводит к несвязности), то $G$ не содержит циклов.}

        Пусть
        \begin{equation}\tag{1}\label{eq:minconnected}
            G \text{ --- минимально связный граф (удаление любого ребра приводит к несвязности)}
        \end{equation}
        \begin{equation}\tag{2}\label{eq:acyclic}
            G \text{ --- ациклический связный граф}
        \end{equation}

        \begin{proof}
            Установим импликацию \ref{eq:minconnected} $\implies$ \ref{eq:acyclic}, воспользовавшись контрапозицией, т. е. докажем $\neg$ \ref{eq:acyclic} $\implies$ $\neg$ \ref{eq:minconnected}. Отрицание условия \ref{eq:acyclic} означает, что граф несвязен или имеет цикл, а условия \ref{eq:minconnected}, что граф или несвязен или связен, но не минимиально. 

            Если граф несвязен, то импликация $\neg$ \ref{eq:acyclic} $\implies$ $\neg$ \ref{eq:minconnected} выполняется, поэтому сосредоточимся на случае связного графа, который содержит цикл. 

            В следующей лемме мы установили, что при удалении ребра из цикла в связном графе, граф остаётся связным, т. е. граф до удаления ребра был не минимально связным.

            \textbf{Лемма}. Пусть $G(V, E)$ --- связный граф и ребро $e$ лежит на цикле; тогда граф $G_0 = (V, E \setminus \{e\})$ связный. То есть, удаление ребра цикла не нарушает связность.

            Перед доказательством введём вспомогательные обозначения. Пусть $P$ и $Q$ --- пути в графе $G$, $x$, $y$ --- вершины, лежащие на пути $P$, а $y$ и $z$ --- вершины, лежащие на пути $Q$. Обозначим через $xPy$ --- подпуть пути $P$, начинающийся с вершины $x$ и заканчивающийся в вершине $y$.

            \begin{proof}
                Пусть ребро $e$ лежит в подграфе-цикле $C$. Обозначим через $Q \subseteq C$ подграф-путь, получающийся из $C$ удалением ребра $e$ (с сохранением его концов). 

                Зафиксируем все пути между всеми парами вершин перед удалением $e$ и рассмотрим путь $P$ с началом в вершине $w$ и концом в вершине $z$. Если ребро $e$ не лежит на пути $P$, то после удаления в графе ребра $e$ этот путь не пострадает. Если же $e$ лежит на пути, то превратим этот путь в другой путь с помощью пути $Q$. 

                Упорядочим вершины $P$; пусть вершина $x$ --- первая общая вершина путей $P$ и $Q$ (ближайшая к $w$, возможно сама $w$), а $y$ --- последняя общая вершина путей $P$ и $Q$ (ближайшая к $z$, быть может сама $z$). Вершины $x$ и $y$ определены, потому что пути $P$ и $Q$ имеют хотя бы две общие вершины --- концы ребра $e$.

                Докажем, что $wPxQyPz$ --- путь, соединяющий вершины $w$ и $z$, и не проходящий через ребро $e$. Действительно, пути $wPx$ и $xQy$ не имеют общих вершин, кроме $x$, поскольку иначе в пути $P$ нашлась бы вершина ближе к $w$, чем $x$, которая была бы общая с путём $Q$, что противоречит выбору $x$; симметрично пути $xQy$ и $yPz$ не имеют общих вершин, кроме $y$ (иначе нашлась бы общая вершина ближе к $z$, чем $y$); пути $wPx$ и $yPz$ не имеют общих вершин, поскольку это непересекающиеся подпути пути $P$. 

                Итак, мы доказали, что после удаления ребра $e$ в графе по-прежнему останутся пути между всеми парами вершин, т. е. граф останется связным.

            \end{proof}
        \end{proof}

    \proofitem{Если $G$ --- связный ациклический граф, то между любыми двумя вершинами $G$ существует единственный путь.}

        Пусть
        \begin{equation}\tag{1}\label{eq:acyclic1}
            G \text{ --- связный ациклический граф}
        \end{equation}
        \begin{equation}\tag{2}\label{eq:pathexistance}
            \text{Между любыми двумя вершинами $G$ существует единственный путь}
        \end{equation}

        \begin{proof}
            Докажем \ref{eq:acyclic1} $\implies$ \ref{eq:pathexistance}, доказав контрапозицию $\neg$ \ref{eq:pathexistance} $\implies$ $\neg$ \ref{eq:acyclic1}. Если выполнено условие $\neg$ \ref{eq:pathexistance} и между какой-то парой вершин нет ни одного пути, то граф несвязный и справедливо условие $\neg$ \ref{eq:acyclic1}. Осталось доказать следующую лемму.

            \textbf{Лемма}. Если между вершинами $w$ и $z$ графа есть два различных пути $P$ и $Q$, то граф содержит цикл.

            \begin{proof}
                Заметим, что $w \neq z$ (из вершины в себя ведёт единственный путь --- длины 0). Если $w$ и $z$ единственные общие вершины путей $P$ и $Q$, то склеив два пути получится цикл. 

                Если же нет, допустим, что у путей $P$ и $Q$ существуют общие вершины $x$ и $y$, такие что у путей $xPy$ и $xQy$ нет общих вершин, кроме концов, и один из путей не короче двух. В этом случае, при объединении путей $xPy$ и $xQy$, получится цикл.

                Чтобы найти $x$ будем двигаться вдоль путей $P$ и $Q$ от $w$ к $z$, пока не встретится первая несовпадающая вершина. Такое обязательно случится, иначе пути совпадают. Будем считать, что несовпадающая вершина $u$ лежит на пути $P$ и $u \neq z$ (иначе поменяем $P$ и $Q$ местами). Выберем вершину перед $u$ в качестве $x$. В качестве $y$ выберем первую после $x$ общую вершину путей $xPz$ и $xQz$. Таким образом получим, что пути $xQy$ и $xPy$ не имеют общих вершин кроме концов, и длина пути $P$ хотя бы $2$.
            \end{proof}
        \end{proof}

    \proofitem{Если между любыми двумя вершинами $G$ существует единственный путь, то $G$ --- связный граф с $|V| - 1$ ребром.}

        \begin{equation}\tag{1}\label{eq:pathexistance1}
            \text{Между любыми двумя вершинами $G$ существует единственный путь}
        \end{equation}
        \begin{equation}\tag{2}\label{eq:connectedtree}
            G \text{ --- связный граф с $|V| - 1$ ребром}
        \end{equation}

        \begin{proof}
            Осталось доказать импликацию \ref{eq:pathexistance1} $\implies$ \ref{eq:connectedtree}. Проведём доказательство индукцией по числу вершин в графе.

            \begin{itemize}
                \item
                \emph{База}: при $|V| = 1$ в графе нет рёбер и в вершину в себя есть единственный путь длины 0. 

                \item
                \emph{Шаг}: пусть утверждение верно для всех графов на $n$ вершинах и пусть $G$ --- произвольный граф, удовлетворяющий условию \ref{eq:pathexistance1}, в котором $V(G) = n + 1$. Выберем в $G$ самый длинный путь $P$, конец которого обозначим через $z$.

                Докажем от противного, что вершина $z$ имеет степень 1. Допустим, что у вершины $z$ есть ещё сосед $x$, кроме предшествующей ей вершины $y$ на пути $P$. Если вершина $x$ не лежит на пути $P$, то к пути $P$ можно добавить ребро $zx$ и сделать его длиннее --- противоречие с выбором $P$. 

                Если же $x$ лежит на пути $P$ и $x \neq y$, то в графе есть два простых пути, соединяющие вершины $x$ и $z$: $xPz$ и ребро $xz$, что противоречит условию \ref{eq:pathexistance1}.
                
                Удалив вершину $z$ из графа $G$ получим связный граф $G_0$ на $n$ вершинах, для которого справедливо предложение индукции: $|E(G_0)| = n - 1$, поскольку между любой парой вершин $G_0$ существует единственный путь. Вернув $z$ на место, получаем, что мы увеличили на единицу и число вершин и число рёбер графа $G_0$, а потому доказали, что $|E(G)| = |V(G)| - 1$; шаг индукции доказан.
            \end{itemize}
        \end{proof}

        При доказательстве импликации мы доказали следующее свойство деревьев:

        \begin{statement}
            В любом дереве, более с чем одной вершиной, есть хотя бы две вершины степени 1.
        \end{statement}

    \proofitem{Критерия 2-раскрашиваемости неориентированного графа.}

        \begin{theorem*}
            Граф $G$ является двураскрашиваемым тогда и только тогда, когда в нём нет циклов нечётной длины.
        \end{theorem*}

        \begin{proof}
            Докажем сначала, что в двураскрашиваемом графе нет циклов нечётной длины. По контрапозиции, это условие равносильно тому, что если в графе есть цикл нечётной длины, то его нельзя раскрасить в два цвета. Это утверждение легко проверить. Если правильная раскраска есть, то в силу симметрии можно считать, что первая вершина цикла покрашена в цвет 1, тогда вторая вершина покрашена в цвет 2 и так далее, то есть каждая нечётная вершина будет покрашена в цвет 1, а каждая чётная --- в цвет 2. Тогда последняя вершина цикла будет покрашена в тот же цвет, что и первая, что невозможно.
            

            Докажем теперь, что если в графе нет циклов нечётной длины, то он двураскрашиваемый. Для этого построим раскраску. Выберем в каждой компоненте связности по вершине $c$, которую назовём центром, и покрасим её в цвет 2; все вершины на расстоянии 1 от неё покрасим в цвет 1, все вершины на расстоянии 2 --- в цвет 2 и так далее: вершины на чётном расстоянии от центра покрасим в цвет 2, а на нечётном в цвет 1. 

            Предположим, что в результате этой процедуры получилась неправильная раскраска. Это означает, что у некоторого ребра $\{u, v\}$ концы были покрашены в один цвет, а это произошло, если расстояния от центра c некоторой компоненты до вершин $u$ и $v$ имеют одинаковую чётность. Пусть $P$ --- кратчайший путь от центра до $u$, а $Q$ --- кратчайший путь от центра до $v$ и $w$ самая дальняя от центра их общая вершина (быть может сам центр, если других общих вершин нет).

            Заметим, что $w$ не совпадает ни с $u$, ни с $v$: иначе мы получили бы, что расстояния до $u$ и $v$ отличаются на единицу; по этой же причине ребро $\{u, v\}$ не лежит ни на одном из этих путей. Пути $cPw$ и $cQw$ имеют одинаковую длину; в противном случае один из этих путей можно было бы заменить на более короткий другой и сократить длину пути до $u$ или $v$. Отсюда мы получаем, что пути $wPu$ и $wQv$ пересекаются только по вершине $w$ и их длины имеют одинаковые чётности (от длин одинаковой чётности отнимается расстояние от $c$ до $w$). Объединив эти пути и добавив к ним ребро $\{u, v\}$ получим цикл нечётной длины, что приводит нас к противоречию.
        \end{proof}

    \proofitem{Критерий существования эйлерова цикла в неориентированном графе.}

        \begin{theorem*}
            Связный граф $G$ содержит замкнутый эйлеров маршрут тогда и только тогда, когда степень каждой вершины чётна.
        \end{theorem*}

        \begin{proof}
            Докажем сначала, что если в графе есть вершина нечётной степени, то в нём нет замкнутого Эйлерова маршрута. Пусть $x$ --- вершина нечётной степени. Если в графе есть замкнутый эйлеров маршрут, то циклически сдвинув его вершины легко добиться, чтобы маршрут начинался и заканчивался в $x$. 

            Заметим, что каждый раз, когда вершина $x$ встречается в середине маршрута, в маршруте встречаются сразу два ребра $x$: в $x$ нужно сначала зайти по одному ребру, а потом выйти по другому. Поскольку $x$ является первой и последней вершиной маршрута, то на его концах также задействуется два ребра: первое на первом выходе из $x$, а второе --- при последнем возврате. Значит, что в во всём маршруте участвовало только какое-то чётное число рёбер, смежных $x$, а всего их нечётное число --- получается, что хотя бы одно ребро, смежное с $x$, в маршрут не попало. 

            Перейдём теперь к доказательству основной импликации: из чётности всех степеней связного графа следует существование замкнутого эйлерова маршрута. Пусть $R = r_0, r_1, \dots, r_m$ --- маршрут максимальной длины, в который каждое ребро графа $G$ входит не более одного раза. В случае, если таких маршрутов несколько, то выберем любой из них. Установим два свойства такого маршрута, которые и приведут к доказательству существования искомого маршрута.

            \begin{itemize}
                \item
                \textbf{Свойство 1}. $r_0 = r_{m}$. Предположим противное. Повторяя те же аргументы, что и в первом абзаце, получим, что между $r_0$ и $r_{m - 1}$ встречается чётное число рёбер (так как $r_0 \neq r_m$ и по определению маршрута $r_{m - 1} \neq r_m$, каждое вхождение $r_m$ дает 2 ребра), смежных с $r_m$, и ещё одно ребро, $r_{m-1} r_m$ встречается в конце маршрута. Итого на маршруте $R$ лежит нечётное число рёбер, смежных с $r_m$, а поскольку степень вершины $r_m$ чётна, то есть ещё хотя бы одно ребро $r_m x$, которое не лежит на маршруте $R$. Добавив вершину $x$ в конец маршрута получим маршрут большей длины, что противоречит выбору $R$.

                \item
                \textbf{Свойство 2}. Маршрут $R$ содержит все рёбра графа $G$. Допустим противное. Пусть некоторое ребро $xy$ графа $G$ не лежит на $R$. Рассмотрим путь из $r_0$ в $x$, который существует в силу связности $G$ и найдём на нём первое ребро, которое не лежит на $R$, если оно есть. Обозначим его через $r_i z$. Если такого ребра нет, то вершина $x$ лежит на $R$, а потому обозначим $r_i = x$ и $z = y$. 

                По \textbf{свойству 1}, маршрут $R$ замкнутый, а потому сдвинем его циклически так, чтобы он начинался и заканчивался в вершине $r_i$. Добавив в конец получившегося маршрута вершину $z$ получим маршрут, длиннее $R$, который также содержит каждое ребро не более одного раза, что приводит нас к противоречию выбора $R$.
            \end{itemize}
        \end{proof}

    \proofitem{Явная формула для числа сочетаний $C(n, k)$: числа $k$-элементных подмножеств $n$-элементного множества.}

        \begin{equation*}
            \binom{n}{k} = C_n^k = \dfrac{n!}{k! (n - k)!}
        \end{equation*}

        \begin{proof}
            Представим себе, что в классе из $n$ учеников нужно сформировать спортивную команду из $k$ учеников (внутри команды все равны, никаких ролей нет). Сколькими способами это можно сделать? 

            Один способ подсчёта такой. Будем выбирать игроков команды по очереди. Первого можно выбрать $n $способами, второго $n - 1$ способами (годятся все, кроме первого), третьего $n - 2$ способами, всего $(n)_k = n \cdot (n - 1) \cdot \dots \cdot (n - k + 1)$ способами (см. выше). 

            Но так мы подсчитали не число возможных команд, а число возможных упорядоченных списков команд (известно, какой игрок первый, какой второй, и так далее). Если не обращать внимание на номера игроков, то много разных списков соответствуют одной команде --- они получатся, если игроков переставлять в списке, а это можно сделать $k!$ способами. Значит, число групп равно $\dfrac{(n)_k}{k!}$; поскольку $(n)_k = \dfrac{n!}{(n - k)!}$, то можно переписать ответ в более симметричной форме:

            \begin{equation*}
                \binom{n}{k} = \dfrac{n!}{(n - k)! k!}
            \end{equation*}

            Это число (число способов выбрать $k$ элементов из $n$, без учёта порядка, или число $k$-элементных подмножеств $n$-элементного множества, если говорить научно) традиционно называют числом сочетаний из $n$ по $k$.
        \end{proof}

        \begin{statement}
            $\displaystyle\binom{n}{k}$ равно $k$-ому элементу в $n$-ой строке треугольника Паскаля.
        \end{statement}
        \begin{proof}
            Элемент в треугольнике Паскаля равен сумме двух верхний соседей, тогда требуется доказать
            \begin{equation*}
                \binom{n}{k} = \binom{n - 1}{k - 1} + \binom{n - 1}{k}
            \end{equation*}

            Пусть нас интересовало количество способов выбрать подмножество из $k$ элементов множества $\{a_1, a_2, \dots, a_n\}$. Тогда рассмотрим, что происходит с последним элементом $a_n$. 

            Количество подмножеств разбивается на сумму подмножеств, где есть $a_n$ элемент и где его нет.
            \begin{itemize}
                \item
                $a_n$ есть в подмножестве: то есть из оставшихся $n - 1$ элементов надо выбрать $k - 1$ элемент, чтобы собрать подмножество из $k$ элементов, то есть всего $\displaystyle\binom{n - 1}{k - 1}$ таких подмножеств .

                \item
                $a_n$ отсутствует в подмножестве: то есть из оставшихся $n - 1$ элементов надо выбрать $k$ элементов, чтобы собрать подмножество из $k$ элементов, то есть всего $\displaystyle\binom{n - 1}{k}$ интересующих нас подмножеств.
            \end{itemize}
        \end{proof}

    \proofitem{Бином Ньютона. Формула для биномиальных коэффициентов.}

        \begin{equation*}
            (a + b)^n = \sum_{k = 0}^n \binom{n}{k} a^k b^{n - k}
        \end{equation*}

        После раскрытия скобок в $(a+b)(a+b)\dots(a+b)$ ($n$ раз) получатся произведения $n$ букв (сколько-то $a$, остальные $b$), и вопрос только в том, какие будут коэффициенты при этих произведениях (сколько подобных членов). 

        Формула бинома Ньютона и говорит, какие это будут коэффициенты: это числа сочетаний, и они написаны в $n$-й строке треугольника Паскаля. Поэтому числа сочетаний также называют биномиальными коэффициентами. 

        \begin{proof}
            Если мы раскроем скобки в $(a+b)^n$ и не будем приводить подобные члены, переставляя сомножители, то получится $2^n$ слагаемых --- всевозможные слова длины $n$ из букв $a$ и $b$. (В каждой скобке можновзять $a$ или $b$ = на каждом месте в слове может стоять $a$ или $b$.) 

            Если сгруппироватьчлены по степеням, то соберутся все слова, в которых данное число букв $a$ и данное число букв $b$. А сколько их, мы знаем --- это как раз числа сочетаний. Слова, в которых $k$ букв $a$ и $n - k$ букв $b$ (из $n$ скобок мы выбираем $k$, из которых возьмём $a$), войдут в количестве $\displaystyle\binom{n}{k}$ штук в слагаемое $\displaystyle\binom{n}{k} a^k b^{n - k}$, как и предсказывает бином Ньютона.
        \end{proof}

        Как было доказано в предыдущем пункте про треугольник Паскаля,
        \begin{equation*}
            \binom{n}{k} = \binom{n - 1}{k - 1} + \binom{n - 1}{k}
        \end{equation*}

        Заметим, что знакочередующаяся сумма биномиальных коэффициентов в одной строке треугольника Паскаля равна 0:
        \begin{equation*}
            \binom{n}{0} - \binom{n}{1} + \binom{n}{2} - \dots + (-1)^n \binom{n}{n} = (-1)^k \sum_{k = 0}^n \binom{n}{k} = (1 - 1)^n = 0
        \end{equation*}

    \proofitem{Основные свойства треугольника Паскаля: симметричность строк, возрастание чисел в первой половине строки.}

        Так как $\displaystyle\binom{n}{k}$ в точности равен $k$-ому элементу в $n$-ой строке треугольника Паскаля, заметим, что $k$-ый элемент равен $(n - k)$-ому:
        \begin{equation*}
            \binom{n}{k} = \dfrac{n}{k! (n - k)!} = \dfrac{n!}{(n - k)! k!} = \binom{n}{n - k}
        \end{equation*}

        \begin{theorem*}
            $\displaystyle\binom{n}{k + 1} > \displaystyle\binom{n}{k}$ при $k < \dfrac{n + 1}{2}$
        \end{theorem*}

        \begin{proof}
            \[\begin{gathered}
                \dfrac{\binom{n}{k + 1}}{\binom{n}{k}}
                = \dfrac{\dfrac{n!}{(k + 1)! (n - k - 1)!}}{\dfrac{n!}{k! (n - k)!}}
                = \dfrac{k! (n - k)!}{(k + 1)! (n - k - 1)!}
                = \dfrac{n - k}{k + 1} > 1
            \end{gathered}\]
            \[\begin{gathered}
                k + 1 < n - k \implies
                2k < n + 1 \implies
                k < \dfrac{n + 1}{2}
            \end{gathered}\]
        \end{proof}

    \proofitem{Основные свойства треугольника Паскаля: формула для суммы чисел в строке, нижняя оценка на центральный коэффициент}

        \begin{theorem}
            Сумма элементов в $n$-ой строке треугольника Паскаля равна $2^n$.
        \end{theorem}

        \begin{proof}
            Так как $n$-ая строка представляет собой ряд биномиальных коэффициентов, сумма равна
            \begin{equation*}
                \sum_{k=0}^n \binom{n}{k} = 
                \sum_{k=0}^n \binom{n}{k} 1^k 1^{n - k} = 
                (1 + 1)^n = 2^n 
            \end{equation*}
        \end{proof}

        \begin{theorem*}
            Справедлива оценка
            \begin{equation*}
                \binom{2n}{n} > \dfrac{2^{2n}}{2n + 1}
            \end{equation*}
        \end{theorem*}

        \begin{proof}
            Уже известно, что максимальное число в строке треугольника Паскаля находится в середине и для $2n$-ой строки равно $\displaystyle\binom{2n}{n}$, а значит
            \begin{equation*}
                (2n + 1) \cdot \binom{2n}{n} > \sum_{k = 0}^{2n} \binom{2n}{k} = 2^n
            \end{equation*}
        \end{proof}

    \proofitem{Число решений уравнения $x_1 + x_2 + \dots + x_k = n$ в неотрицательных целых числах. (Задача Муавра.)}

        Число решений уравнения $x_1 + x_2 + \dots + x_k = n$ в неотрицательных целых числах равно $\displaystyle\binom{n + k - 1}{n}$.

        \begin{proof}
            Для простоты решим эквивалентную задачу. Необходимо раздать $n$ монет $k$ людям. 

            Представим себе, что наши $n$ монет разложены в ряд. Прежде чем раздавать эти монеты, давайте разделим их на $k$ групп перегородками, и договоримся, кому идёт самая левая группа, кому вторая слева и т. д. Заметим, что мы допускаем случай, когда две перегородки оказываются рядом --- это значит просто, что человеку, которому была назначена группа между ними, не повезло и в этом раскладе ему ни одной монеты не достанется.

            Каждому варианту раздачи (каждому решению уравнения в неотрицательных числах) соответствует последовательность из $n$ монет и $k - 1$ перегородки. (Число перегородок на единицу меньше числа группы: первая группа стоит слева от первой перегородки, а последняя --- справа от последней.)

            Наоборот, каждой последовательности из $n$ монет и $k - 1$ перегородки соответствует некоторый способ раздачи монет. Поэтому надо подсчитать число способов расставить перегородки.

            А это совсем просто --- каждый такой способ можно рассматривать как словов алфавите {монета, перегородка}, содержащее $n$ монет и $k - 1$ перегородку. Это количество, как мы знаем, равно $\displaystyle\binom{n + k - 1}{n}$ (из $(n - k + 1)$-ой позиции выбираем $k - 1$, куда поставим перегородку.)
        \end{proof}

    \proofitem{Формула включений и исключений}

        \begin{theorem*}
            \begin{equation}\tag{1}\label{eq:incl}
                \left|\bigcup_{i=1}^n A_i\right| = \sum_{S \subseteq [n], S \neq \varnothing} (-1)^{|S| + 1} \left|\bigcap_{A \in S} A\right| 
            \end{equation}
        \end{theorem*}

        \begin{proof}
            Основным ингредиентом доказательства является обобщённый закон де Моргана:
            \begin{equation*}
                \bigcup_{i = 1}^n A_i = \overline{\bigcap_{i = 1}^n \overline{A_i}}
            \end{equation*}
            который мы переведем на язык характеристических функций (для удобства опустим аргументы)
            \begin{equation*}
                \chi_{\bigcup\limits_{i = 1}^n A_i} = 1 - \chi_{\bigcap\limits_{i = 1}^n \overline{A_i}} = 1 - (1 - \chi_{A_1}) \times (1 - \chi_{A_2}) \times \dots \times (1 - \chi_{A_n})
            \end{equation*}

            Раскроем скобки в выражении $(1 - \chi_{A_1}) \times (1 - \chi_{A_2}) \times \dots \times (1 - \chi_{A_n})$ и проанализируем получившееся выражение аналогично анализу для бинома Ньютона. Ясно, что среди слагаемых есть 1; чтобы её получить нужно взять в качестве сомножителей 1 из каждой скобки; также для каждого i в выражение войдёт слагаемое $-\chi_{A_i}$: чтобы получить его нужно взять $-chi_{A_i}$ из $i$-ой скобки, а из каждой оставшейся скобки взять единицу. Продолжая рассуждения получаем, что чтобы получить слагаемое $\prod_{i \in S} \chi_{A_i} = \chi_{A_{i_1}} \times \chi_{A_{i_2}} \times \dots \times \chi_{A_{i_{|S|}}}$ нужно взять из каждой скобки с номером из $i \in S$ множества $S$ слагаемое $-\chi_{A_i}$ , а из остальных скобок взять 1; при этом коэффициент перед получившимся произведением будет $(-1)^{|S|}$. Итак, мы получили формулу

            \begin{equation*}
                \chi_{\bigcup\limits_{i = 1}^n A_i} = 1 - \left(
                1 + \sum\limits_{S \subseteq [n]} (-1)^{|S|} \prod_{i \in S} \chi_{A_i}\right)
                = \sum\limits_{S \subseteq [n]} (-1)^{|S| + 1} \chi_{\bigcap\limits_{i \in S} A_i}
            \end{equation*}

            (Здесь $[n] = \{1, 2, \dots, n\}$).

            В последнем переходе мы перешли от произведения характеристических функций к характеристической функции пересечения множеств. Просуммировав обе части по всем $x \in U$ получим требуемую формулу \ref{eq:incl}. Обратим внимание, что в результате суммирования левая часть формулы будет иметь вид

            \begin{equation*}
                \sum_{x \in U} \chi_{\bigcup\limits_{i = 1}^n A_i} (x) =
                \left|\bigcup_{i=1}^n A_i\right|,
            \end{equation*}

            а в правой части в результате суммирования отдельного слагаемого получится следующее (для каждого $x \in U$ сумма будет равна мощности пересечения)

            \begin{equation*}
                \sum\limits_{S \subseteq [n]} \sum\limits_{x \in U} (-1)^{|S| + 1} \chi_{\bigcap\limits_{i \in S} A_i}(x)
                = \sum\limits_{S \subseteq [n]} (-1)^{|S| + 1} \times \left|\bigcap_{i \in S} A_i\right|
            \end{equation*}
        \end{proof}

    \proofitem{Число отображений, функций, инъекций, биекций из $m$-элементного множества в $n$-элементное множество}

        Как и раньше, обозначим через $[m]$ --- $m$-элементное множество; для определённости $[m] = \{1, 2, \dots, m\}$. Найдём число отображений $f : [m] \mapsto [n]$ (из $m$-элементного множества в $n$-элементное множество). 

        Напомним, что отображения --- это всюду определённые функции, этот факт обозначается с помощью стрелочки между множествами. Для значения $f(1)$ годится любое из $n$ чисел, равно как для $f(2)$ и так вплоть до $f(m)$; по правилу произведения получаем, что число отображений есть $n^m$. Каждое отображение можно закодировать как слово длины $m$ над алфавитом из $n$ символов. Эта кодировка пригодится нам дальше.

        Найдём теперь число функций из $[m]$ в $[n]$. В отличие от отображений функции не обязательно всюду определены, поэтому возможно, что у каких-то элементов множества $[m]$ нет образов. Эту задачу легко свести к предыдущей, добавив к множеству $[n]$ элемент $n + 1$ и построить для каждой функции $f$ из $[m]$ в $[n]$ эквивалентное отображение $g : [m] \mapsto [n + 1]$, которое принимает значение $n + 1$ во всех точках, в которых функция $f$ не определена, а в остальных точках принимает то же значение, что и $f$. Итак, мы установили биекцию между множеством функций из $[m]$ в $[n]$ и множеством отображений из $[m]$ в $[n+1]$, таким образом число функций есть $(n + 1)^m$. 

        Перейдём теперь к подсчёту инъекций из $[m]$ в $[n]$. Вспомним, что инъекция --- отображение, которое ставит в соответствие разным элементам разные значения. Мы уже обсуждали тот факт, что если существует инъекция из конечного множе ства $A$ в конечное множество $B$, то $|A| \leq |B|$, поэтому при $m > n$ инъекций нет вовсе. В случае $m \leq n$ инъекцию, как и любое отображение, можно закодировать в виде слова над $n$-ичным алфавитом длины $m$, а условие инъективности означает, что в слове все буквы разные. Таким образом, число инъекций совпадает с числом размещений и равно $\dfrac{n!}{(n - m)!}$

        В случае конечных множеств, биекция является частным случаем инъекции, когда $[m] = [n]$. Таким образом число биекций равно $n!$ (при $m = n$) и совпадает с числом перестановок. Совпадение чисел размещений и перестановок с числом инъекций и сюръекций неслучайно. Инъекции кодируют размещения, а биекции перестановки.

    \proofitem{Формула для числа сюръекций}

        Аналогично случаю с инъекциями необходимо, чтобы m > n. Чтобы подсчитать сюръекции подсчитаем сначала отображения, не являющиеся сюръекциями и вычтем их число из количества всехотображений. Напомним, что функция является сюръекцией, если прообраз каждого из элементов $[n]$ не пуст. Таким образом, функция не является сюръекцией, если в $[n]$ есть хотя бы один элемент $y$ для которого нет подходящего $x : f(x) = y$. Обозначим через $A_i$ --- множество отображений, для которых прообраз элемента $i$ не определён, формально

        \begin{equation*}
            A_i = \{f \mid f: [m] \mapsto [n], f^{-1}(i) = \varnothing\}
        \end{equation*}

        Ясно, что все несюръекции лежат в множестве $\bigcup\limits_{i=1}^n A_i$.

        Подсчитаем мощность $\left|\bigcup\limits_{i=1}^n A_i\right|$ с помощью формулы включений-исключений. Для каждого $A_i$ число $|A_i|$ совпадает с числом отображений из $m$-элементного множества в ($n - 1$)-элементное множество ($i$-ый) элемент задействовать нельзя, а все остальные можно. Их число есть $(n - 1)^m$ . В пересечении множеств $A_i \cap A_j (i \neq j)$ содержится столько же элементов, сколько отображений $[m] \mapsto [n - 2]$ (теперь нельзя задействовать ровно два элементов), а число элементов в пересечении любых $k$ множеств из семейства $A_1, A_2, \dots, A_n$ совпадает с числом отображений $[m] \mapsto [n - k]$. Число способов выбрать $k$ множеств из семейства с $n$-множествами есть число сочетаний $\displaystyle\binom{n}{k}$, отсюда получаем по формуле включений исключений

        \begin{align*}
            \left|\bigcup_{i=1}^n A_i\right| 
            &= |A_1| + |A_2| + \dots + |A_n| - |A_1 \cap A_2| - \dots |A_{n - 1} \cap A_{n}| + \dots \\
            &= n \times (n - 1)^m - \binom{n}{2} \times (n - 2)^m + \binom{n}{3} \times (n - 3)^m - \dots \\
            &= \sum_{k = 1}^n (-1)^{k + 1} \times \binom{n}{k} \times (n - k)^m
        \end{align*}

        Таким образом число сюръекций есть

        \begin{equation*}
            n^m - \sum_{k = 1}^n (-1)^{k + 1} \times \binom{n}{k} \times (n - k)^m
            = \sum_{k = 0}^n (-1)^k \times \binom{n}{k} \times (n - k)^m
        \end{equation*}


    \proofitem{Основная теорема об отношениях эквивалентности (классы эквивалентности на множестве $A$ --- в точности разбиения множества $A$ на подмножества)}

        \begin{theorem*}
             Для любого отношения эквивалентности на множестве $A$ множество классов эквивалентности образует разбиение множества $A$. Обратно, любое разбиение множества $A$ задает на нем отношение эквивалентности, для которого классы эквивалентности совпадают с элементами разбиения.
        \end{theorem*}

        \begin{proof}
            Покажем, что отношение эквивалентности $R$ на множестве $A$ определяет некоторое разбиение этого множества. Убедимся вначале, что любые два класса эквивалентности по отношению $R$ либо не пересекаются, либо совпадают.

            Пусть два класса эквивалентности $[x]_{R}$ и $[y]_{R}$ имеют общий элемент $z \in [x]_{R} \cap [y]_{R}$. Тогда $zRx$ и $zRy$. В силу симметричности отношения $R$ имеем $xRz$, и тогда $xRz$ и $zRy$. В силу транзитивности отношения $R$ получим $xRy$. Пусть $h\in[x]_{R}$, тогда $hRx$. Так как $xRy$, то $hRy$ и, следовательно, $h\in[y]_{R}$.


            Обратно, если $h\in[y]_{R}$, то в силу симметричности $R$ получим $hRy, yRx$ и в силу транзитивности --- $hRx$, то есть $h\in[x]_{R}$. Таким образом, $[x]_{R}=[y]_{R}$.


            Итак, любые два не совпадающих класса эквивалентности не пересекаются. Так как для любого $x\in A$ справедливо $x\in[x]_{R}$ (поскольку $xRx$), т.е. каждый элемент множества $A$ принадлежит некоторому классу эквивалентности по отношению $R$, то множество всех классов эквивалентности по отношению $R$ образует разбиение исходного множества $A$. Таким образом, любое отношение эквивалентности однозначно определяет некоторое разбиение.


            Теперь пусть $(B_i)_{i\in I}$ --- некоторое разбиение множества $A$. Рассмотрим отношение $R$, такое, что $xRy$ имеет место тогда и только тогда, когда $x$ и $y$ принадлежат одному и тому же элементу $B_i$ данного разбиения:

            \begin{equation*}
                xRy \iff \exists i \in I: \; (x \in B_i) \land (y\in B_i).
            \end{equation*}

            Очевидно, что введенное отношение рефлексивно и симметрично. Если для любых $x$, $y$ и $z$ имеет место $xRy$ и $yRz$, то x,y и z в силу определения отношения $R$ принадлежат одному и тому же элементу $B_i$ разбиения. Следовательно, $xRz$ и отношение $R$ транзитивно. Таким образом, $R$ --- эквивалентность на $A$.
        \end{proof}

    \proofitem{Равносильность свойств ориентированных графов: (1) каждая компонента сильной связности состоит из одной вершины; (2) вершины графа возможно занумеровать так, чтобы каждое ребро вело из вершины с меньшим номером в вершину с большим номером; (3) в графе нет циклов длины больше 1 (граф ацикличен).}

        \begin{proof}
            Компонента является тривиальной, если состоит из 1ой вершины.

            Равносильность условий (1) и (3) вытекает из данного утверждения:

            \begin{statement}
                Вершины $u$ и $v$ лежат в одной компоненте сильной связности тогда и только тогда, когда они лежат на замкнутом маршруте.
            \end{statement}

            \begin{proof}
                Действительно, условие двусторонней достижимости влечёт существование маршрутов из $u$ в $v$ и из $v$ в $u$, склеив их получим замкнутый маршрут. С другой стороны, замкнутый маршрут, содержащий вершины $u$ и $v$ очевидно влечёт двустороннюю достижимость.
            \end{proof}

            В графе есть нетривиальная компонента сильной связности тогда и только тогда
            когда в нём есть замкнутый маршрут длины 2 или больше, а это равносильно
            существованию цикла.

            Из условия (2) очевидно следует условие (3). Действительно, если такая нумерация существует, то если был бы цикл, то в нём вершина с большим номером соединялась бы ребром с меньшим номером.

            Для завершения доказательства теоремы осталось доказать импликацию (3) $\implies$ (2). Докажем для этого вспомогательную лемму.

            \textbf{Лемма}. В ориентированном ациклическом графе G есть сток
            \begin{proof}
                Возьмём самый длинный ориентированный путь $v_1 \to v_2 \to \dots \to v_n$ в графе $G$. Такой существует, потому что множество путей конечно (вершины в пути повторяться не могут). Докажем от противного, что вершина $v_n$ является стоком (из неё нет исходящих рёбер). 

                Если в $G$ есть ребро $v_n \to u$ и $u \neq v_i$, то путь можно сделать длиннее, добавив к нему $u$; если же $u = v_i$, то в графе $G$ есть цикл (петель в $G$ по определению быть не может). Оба случая приводят нас к противоречию.
            \end{proof}

            Докажем с помощью леммы импликацию индукцией по числу вершин. 

            \begin{itemize}
                \item
                \emph{База}: для |V | = 1 очевидна: в графе из одной вершины нет рёбер, поэтому занумеровав единственную вершину единицей получим корректную нумерацию.

                \item
                \emph{Шаг}: пусть утверждение верно для любого графа на $n$ вершинах; согласно лемме в графе $G$ на $(n + 1)$-ой вершине существует сток --- занумеруем его числом $n + 1$ и рассмотрим граф $G_0$, получающийся из $G$ удалением этого стока. 

                По предложению индукции вершины $G_0$ можно занумеровать корректно; перенеся эту нумерацию на $G$ получим также корректную нумерацию: поскольку из $(n + 1)$-ой вершины не идёт ни одного ребра, испортить нумерацию она не может, а для всех остальных рёбер нумерация корректна.
            \end{itemize}
        \end{proof}

    \end{colloq}

\end{document}