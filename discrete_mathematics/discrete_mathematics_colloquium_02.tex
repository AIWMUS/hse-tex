\documentclass[a4paper]{article}
\usepackage{header}


\newcommand\enumtocitem[3]{\item\textbf{#1}\addtocounter{#2}{1}\addcontentsline{toc}{#2}{\protect{\numberline{#3}} #1}}
\newcommand\defitem[1]{\enumtocitem{#1}{subsection}{\thesubsection}}
\newcommand\proofitem[1]{\enumtocitem{#1}{subsection}{\thesubsection}}

\newlist{colloq}{enumerate}{1}
\setlist[colloq]{label=\textbf{\arabic*.}}

\graphicspath{
    {img/}
}


\title{\HugeДискретная математика, Коллоквиум 2}
\author{
	Балюк Игорь \\
	\href{https://teleg.run/lodthe}{@lodthe},
    \href{https://github.com/LoDThe/hse-tex}{GitHub} \\
}
\date{2019 --- 2020}

\begin{document}
    \maketitle

    Материалы взяты из учебника Александра Рубцова.

    \tableofcontents

    \newpage

    \section{Определения}

    Контрольный вопрос на понимание определений включает в себя формулировку одного определения из списка ниже и контрольный вопрос по этому определению. Пример: <<Определение полного прообраза. Пусть $f(x) = x^2$ --- функция из $\ZZ$ в $\ZZ$. Найдите полный прообраз множества $\{1, 2, 3, 4\}$.

    \begin{colloq}
    
    \defitem{Деление целых чисел с остатком.}
    \defitem{Сравнения по модулю. Основные свойства.}
    \defitem{Арифметика остатков (вычетов). Обратимые остатки (вычеты).}
    \defitem{Малая теорема Ферма.}
    \defitem{Функция Эйлера. Теорема Эйлера.}
    \defitem{Наибольший общий делитель. Алгоритм Евклида.}
    \defitem{Расширенный алгоритм Евклида нахождения решения линейного диофантова уравнения.}
    \defitem{Простые числа, формулировка основной теоремы арифметики.}
    \defitem{Равномощные множества.}
    \defitem{Cчётные множества.}
    \defitem{Множества мощности континуум.}
    \defitem{Основные определения элементарной теории вероятностей: исходы, события, вероятность события.}
    \defitem{Формулировка формулы включений и исключений для вероятностей.}
    \defitem{Условная вероятность.}
    \defitem{Независимые события. Основные свойства независимых событий.}
    \defitem{Формула полной вероятности.}
    \defitem{Случайная величина и математическое ожидание. Линейность математического ожидания.}
    \defitem{Формулировка неравенства Маркова.}
    \defitem{Определение схемы в некотором функциональном базисе. Представление схем графами.}
    \defitem{Полный базис. Примеры полных и неполных базисов.}
    \defitem{Полином Жегалкина (в стандартном виде).}
    \defitem{Схемная сложность функции (размер схемы).}



    \end{colloq}

    \section{Вопросы на знание доказательств}
    \begin{colloq}
    \setlength\parindent{20pt}

    \proofitem{Сравнение $ax \equiv 1 \pmod{N}$ имеет решение тогда и только тогда, когда $\text{НОД}(a, N) = 1$.}
    \proofitem{Малая теорема Ферма.}
    \proofitem{Теорема Эйлера.}
    \proofitem{Корректность алгоритма Евклида и расширенного алгоритма Евклида.}
    \proofitem{Основная теорема арифметики.}
    \proofitem{Китайская теорема об остатках.}
    \proofitem{Мультипликативность функции Эйлера. Формула для функции Эйлера.}
    \proofitem{Формула Байеса. Формула полной вероятности.}
    \proofitem{Парадокс дней рождений (математическое ожидание числа людей с совпавшими днями рождений)}
    \proofitem{Неравенство Маркова.}
    \proofitem{Нижняя оценка на максимальное количество ребер в разрезе.}
    \proofitem{Любое бесконечное множество содержит счётное подмножество. Любое подмножество счётного множества конечно или счётно.}
    \proofitem{Конечное или счётное объединение конечных или счётных множеств конечно или счётно}
    \proofitem{Счётность декартова произведения счетных множеств. Счётность множества рациональных чисел.}
    \proofitem{Равномощность отрезков, интервалов, лучей и прямых (явные биекции).}
    \proofitem{Несчетность множества бесконечных двоичных последовательностей.}
    \proofitem{Теорема Кантора-Бернштейна.}
    \proofitem{Нижняя оценка на число монотонных булевых функций: монотонных булевых функций от $2n$ переменных не меньше $2^{\frac{2^n}{2n + 1}}$}
    \proofitem{Существование и единственность полинома Жегалкина (в стандартном виде) для любой булевой функции.}
    \proofitem{Разложение в ДНФ и КНФ булевой функции.}
    \proofitem{Верхняя оценка $O(n 2^n)$ схемной сложности булевой функции от $n$ переменных.}
    \proofitem{Булевы схемы для сложения и умножения n-битовых чисел. Оценка размера.}
    \proofitem{Булева схема для задачи о связности графа. Оценка размера.}
    \proofitem{Задача об угадывании числа. Верхняя и нижняя оценки.}
    \proofitem{Задача о сортировке нижняя оценка.}
    \proofitem{Задача о нахождении самой тяжелой монеты. Верхние и нижние оценки.}

    \end{colloq}

\end{document}