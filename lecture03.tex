\section{Лекция 28.09.2017}

\subsection{Матрицы}

\vspace{\baselineskip}
\textbf{Определение.} \textit{Матрица размера $m \times n$} (или $(m \times n)$-матрица) -- это прямоугольная таблица высоты m и ширины n, заполненная числами (m строк, n столбцов).

\begin{equation*}
	\begin{pmatrix}
		a_{11} & a_{12} & \cdots & a_{1n} \\
		a_{21} & a_{22} & \cdots & a_{2n} \\
       \vdots & \vdots & \vdots& \vdots \\ 
       a_{m1} & a_{m2} & \cdots & a_{mn}
	\end{pmatrix}
    = A
\end{equation*}

\vspace{\baselineskip}
$a_{ij}$ -- элемент на пересечении i-ой строки и j-ого столбца.

Краткая запись: $A = (a_{ij})$.

\vspace{\baselineskip}
\textit{Множество всех матриц размера} $m \times n$ (с коэффициентами из $\RR$) обозначается $Mat_{m\times n}(\RR)$. 

\vspace{\baselineskip}
\textbf{Определение.} Матрицы $A \in Mat_{m\times n}$ и $B \in Mat_{k\times l}$ называются равными, если:

1) $m = k$, $n = l$ (размер один и тот же)

2) $a_{ij} = b_{ij} \ \forall i, j$

\vspace{\baselineskip}
\subsection{Операции над матрицами}

\vspace{\baselineskip}
\textbf{Сложение.} $A, B \in Mat_{m\times n}, A = (a_{ij}), B = (b_{ij})$

\begin{equation*}A + B = (a_{ij} + b_{ij}) = 
\begin{pmatrix}
		a_{11} + b_{11} & a_{12} + b_{12} & \cdots & a_{1n} + b_{1n} \\
		a_{21} + b_{21} & a_{22} + b_{22} & \cdots & a_{2n} + b_{2n} \\
       \vdots & \vdots & \vdots& \vdots \\ 
       a_{m1} + b_{m1} & a_{m2} + b_{m2} & \cdots & a_{mn} + b_{mn}
\end{pmatrix} \in Mat_{m\times n}
\end{equation*}

\textbf{Умножение на скаляр.} $A \in Mat_{m\times n}, \lambda \in \RR \Rightarrow$
\begin{equation*} \lambda A = (\lambda a_{ij}) = 
\begin{pmatrix}
		\lambda a_{11} & \lambda a_{12} & \cdots & \lambda a_{1n} \\
		\lambda a_{21} & \lambda a_{22} & \cdots & \lambda a_{2n} \\
       \vdots & \vdots & \vdots& \vdots \\ 
       \lambda a_{m1} & \lambda a_{m2} & \cdots & \lambda a_{mn}
\end{pmatrix} \in Mat_{m\times n}
\end{equation*}

\vspace{\baselineskip}
\textbf{Свойства сложения и умножения на скаляр}

$\forall A, B, C \in Mat_{m\times n}$ и $\forall \lambda, \mu \in \RR$

1) $A + B = B + A$ (коммутативность)

2) $(A + B) + C = A + (B + C)$ (ассоциативность)

3)$A + 0 = 0 + A = A$, где \begin{equation*}0 = \begin{pmatrix}
0 & 0 & \cdots & 0 \\
\cdots & \cdots & \cdots & \cdots \\
0 & 0 & \cdots & 0
\end{pmatrix} \end{equation*} - нулевая матрица

4)$A + (-A) = (-A) + A = 0$, где $-A = (-a_{ij})$ -- противоположная А матрица 

5) $(\lambda + \mu)A = \lambda A + \mu A$

6) $\lambda (A + B) = \lambda A + \lambda B$

7) $(\lambda \mu) A = \lambda (\mu A)$

8) $1 \times A = A$

\vspace{\baselineskip}
\textit{Упражнение на дом:} \textbf{\textit{Доказательство свойств.}}

\vspace{\baselineskip}
\begin{comment}
	Свойства (1)-(8) означают, что $Mat_{m\times n}$ является             векторным пространством.
\end{comment}

\vspace{\baselineskip}
\subsection{Декартово произведение множеств}

\vspace{\baselineskip}
\textbf{Стандартный способ задания множества.}

\vspace{\baselineskip}
$M = \{$ какие элементы $|$ условие, которому удовлетворяют элементы $\}$.

\vspace{\baselineskip}
$X, Y$ -- множества $\Rightarrow$ их декартово произедение есть множество $X \times Y := \{ (x, y) \ | \ x \in X, y \in Y\}$ (упорядоченные пары).

$X_1, X_2, \dots, X_n$ -- множества.

$X_1 \times X_2 \times \dots \times X_n := \{(x_1, x_2, \dots, x_n) \ | \ x_i \in X_i\}$

Пространство $\RR^n$ -- это $\RR \times \RR \times \cdots \times \RR$ (n штук).

$\RR^n = \{ (x_1, x_2, \dots, x_n) \ | \ x_i \in \RR \  \forall i = 1, 2, \dots, n \}$

\vspace{\baselineskip}
Договоримся элементы из $\RR^n$ записывать в виде столбцов (не строк):

\begin{equation*} \RR^n = \left\{ \begin{pmatrix}
    x_1 \\
    x_2 \\
    x_3 \\
    \cdots \\
    x_n
	\end{pmatrix} | \ x_i \in \RR \ \forall i = 1, 2, \dots , n \right\} = Mat_{n \times 1}
\end{equation*}

\begin{equation*}x = \begin{pmatrix}
x_1 \\
x_2 \\
\cdots \\
x_n
\end{pmatrix} \in \RR^n, \ y = \begin{pmatrix}
y_1 \\
y_2 \\
\cdots \\
y_n
\end{pmatrix} \in \RR^n \end{equation*}
\begin{equation*}x+y:= \begin{pmatrix}
x_1 + y_1 \\
x_2 + y_2 \\
\cdots \\
x_n + y_n
\end{pmatrix}
\end{equation*}

\begin{equation*}\lambda \in \RR \Rightarrow \lambda x := \begin{pmatrix}
\lambda x_1 \\
\lambda x_2 \\
\cdots \\
\lambda x_n 
\end{pmatrix}
\end{equation*}

Выполняются свойства (1)--(8).

\vspace{\baselineskip}
\subsection{Транспонирование}

$A \in Mat_{m \times n}, A = (a_{ij})$

\begin{equation*}A = 
	\begin{pmatrix}
		a_{11} & a_{12} & \cdots & a_{1n} \\
		a_{21} & a_{22} & \cdots & a_{2n} \\
       \vdots & \vdots & \vdots& \vdots \\ 
       a_{m1} & a_{m2} & \cdots & a_{mn}
	\end{pmatrix}
\end{equation*}

\begin{equation*} A^T = (a_{ji}) = 
	\begin{pmatrix}
		a_{11} & a_{21} & \cdots & a_{m1} \\
		a_{12} & a_{22} & \cdots & a_{m2} \\
       \vdots & \vdots & \vdots& \vdots \\ 
       a_{1n} & a_{2n} & \cdots & a_{mn}
	\end{pmatrix} \in Mat_{n \times m}
\end{equation*} 
-- транспонированная к А матрица.

\vspace{\baselineskip}
Примеры:

1) $(x_1 x_2 \dots x_n)^T = \begin{pmatrix} x_1 \\ x_2 \\ \dots \\ x_n \end{pmatrix}$

2) $\begin{pmatrix} x_1 \\ x_2 \\ \dots \\ x_n \end{pmatrix}^T = (x_1 x_2 \dots x_n)$

3) $\begin{pmatrix} 1 & 2 \\ 3 & 4 \\ 5 & 6 \end{pmatrix}^T = \begin{pmatrix} 1 & 3 & 5 \\ 2 & 4 & 6 \end{pmatrix}$

\vspace{\baselineskip}
\textbf{Свойства транспонирования.}

1) $(A+B)^T = A^T + B^T$

2) $(\lambda A)^T = \lambda A^T$

3) $(A^T)^T = A$

\vspace{\baselineskip}
$A = (a_{ij}) \in Mat_{m \times n}$

$A_{(i)} = (a_{i1}\ a_{i2}\ \dots a_{in})$ -- $i$-ая строка матрицы A

$A^{(j)} = \begin{pmatrix}
    a_{1j} \\
    a_{2j} \\
    \cdots \\
    a_{nj}
	\end{pmatrix}$ -- $j$-ый столбец матрицы А

\vspace{\baselineskip}
\subsection{Умножение матриц}

1) \textit{Частный случай:} умножение строки на столбец той же длины

\begin{equation*}(x_1 x_2 \dots x_n) \begin{pmatrix} y_1 \\ y_2 \\ \dots \\ y_n \end{pmatrix} = x_1 y_1 + x_2 y_2 + \dots + x_n y_n \end{equation*}

2) \textit{Общий случай:}

$A \in Mat_{m \times n}$ (размер $m \times \underline{n}$)

$B \in Mat_{n \times p}$ (размер $\underline{n} \times p$) -- соответствие размеров

\vspace{\baselineskip}
Тогда AB -- такая матрица $C \in Mat_{m \times p}$, что $c_{ij} := A_{(i)}B^{(j)} = \sum\limits_{k=1}^n a_{ik} b_{kj}$.

Примеры:

1) \begin{equation*} \begin{pmatrix} x_1 \\ x_2 \\ \dots \\ x_n \end{pmatrix} (y_1 y_2 \dots y_n) = \begin{pmatrix}]
x_1 y_1 & x_1 y_2 & \dots & x_1 y_n \\
x_2 y_1 & x_2 y_2 & \dots & x_2 y_n \\
\dots & \dots & \dots & \dots \\
x_n y_1 & x_n y_2 & \dots & x_n y_n
\end{pmatrix} = (x_i y_j) \end{equation*}

2) \begin{equation*}\begin{pmatrix}
1 & 0 & 2 \\
0 & -1 & 3
\end{pmatrix} \begin{pmatrix}
2 & -1 \\
0 & 5 \\
1 & 1
\end{pmatrix} = \begin{pmatrix}
1 \cdot 2 + 0 \cdot 0 + 2 \cdot 1 & 1 \cdot (-1) + 0 \cdot 5 + 2 \cdot 1 \\
0 \cdot 2 + (-1) \cdot 0 + 3 \cdot 1 & 0 \cdot (-1) + (-1) \cdot 5 + 3 \cdot 1 \end{pmatrix} = \begin{pmatrix}
4 & 1 \\
3 & -2 \end{pmatrix} \end{equation*}

3) Линейное уравнение $a_1 x_1 + a_2 x_2 + \dots + a_n x_n = b$

\begin{equation*}(a_1 a_2 \dots a_n) \begin{pmatrix}
    x_1 \\
    x_2 \\
    \cdots \\
   x_n
	\end{pmatrix} = b
\end{equation*}

СЛУ с расширенной матрицей (A | b), $A \in Mat_{m \times n}
\Rightarrow$ \textit{матричная форма записи}:

Ax = b, где А -- матрица коэфициентов, $x \in \RR^n$ -- столбец неизвестных, $b \in \RR^m$ -- столбец правых частей.

\vspace{\baselineskip}
\subsection{Отступление} 

$s_p, s_{p+1}, \dots, s_q$ -- последовательность чисел

$\sum\limits_{i=p}^q s_i := s_p + s_{p+1} + \dots + s_q$

\vspace{\baselineskip}
\textbf{Свойства суммирования: }

1) $\lambda \sum\limits_{i=1}^n s_i = \sum\limits_{i=1}^n \lambda s_i$

\vspace{\baselineskip}
2) $\sum\limits_{i=1}^n s_i + \sum\limits_{i=1}^n t_i = \sum\limits_{i=1}^n (s_i + t_i)$

\vspace{\baselineskip}
3) $\sum\limits_{i=1}^m ( \sum\limits_{j=1}^n s_{ij}) = \sum\limits_{j=1}^n ( \sum\limits_{i=1}^m s_{ij})$

Обе части равенства представляют собой сумму всех элементов матрицы 

$S = (s_{ij}) = \begin{pmatrix} s_{11} & s_{12} & \dots & s_{1n} \\
s_{21} & s_{22} & \dots & s_{2n} \\
\dots & \dots & \dots & \dots \\
s_{n1} & s_{n2} & \dots & s_{nn} \end{pmatrix}$ 

\vspace{\baselineskip}
В левой части мы сначала суммируем элементы в каждой строке, а потом складываем полученные суммы в столбце. В правой части мы сначала суммируем элементы в каждом столбце, а после -- полученные суммы в строке.

